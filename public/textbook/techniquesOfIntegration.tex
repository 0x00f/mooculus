\chapter{Techniques of Integration}

\section{Integration by Parts}

While integration by substitution allows us to identify and ``undo''
the chain rule, \textit{integration by parts} allows us to recogonize
the product rule.

\begin{mainTheorem}[Integration by Parts Formula] 
If $f(x)g(x)$ is differentiable on the interval $[a,b]$, then
\[
\int_a^b f(x) g'(x) \d x =f(x)g(x) \bigg|_a^b - \int_a^b f'(x) g(x) \d x.
\]
\end{mainTheorem}
\begin{proof} First note by the power rule we have
\[
\ddx f(x)g(x) = f(x)g'(x) + f'(x) g(x).
\]
Now integrate both sides of the equation above
\[
\int_a^b \ddx f(x) g(x) \d x = \int_a^b \left(f(x)g'(x) + f'(x) g(x)\right) \d x.
\]
By the Fundamental Theorem of Calculus, the left-hand side of the equation is
\[
f(x)g(x) \bigg|_a^b.
\]
However, by properties of integrals the right-hand side is equal to 
\[
\int_a^b f(x)g'(x)\d x + \int_a^b f'(x) g(x) \d x.
\]
Hence
\[
f(x)g(x) \bigg|_a^b = \int_a^b f(x)g'(x)\d x + \int_a^b f'(x) g(x) \d x.
\]
and so 
\[
 \int_a^b f(x)g'(x)\d x = f(x)g(x) \bigg|_a^b -  \int_a^b f'(x) g(x) \d x.
\]
\end{proof}


Integration by parts is often written in a more compact form
\[
\int u\d v = uv-\int v\d u,
\]
where $u=f(x)$, $v=g(x)$, $du=f'(x)\d x$ and $dv=g'(x)\d x$.  To use
this technique we need to identify likely candidates for $u=f(x)$ and
$dv=g'(x)\d x$.



\begin{example}
Compute
\[
\int \ln(x)\d x.
\]
\end{example}

\begin{solution}
Let $u=\ln(x)$ so $du=1/x\d x$. Hence,  $dv=1\d x$ so $v=x$
and so 
\begin{align*}
 \int \ln(x)\d x&=x\ln (x)-\int \frac{x}{x}\d x\\
&= x\ln (x)- x+C.\\
\end{align*}
\end{solution}

\begin{example}
Compute
\[
\int x\sin(x) \d x.
\]
\end{example}

\begin{solution} Let $u=x$ so $du=dx$. Hence, $dv=\sin(x)\d x$ so $v=-\cos(x)$ and
\begin{align*}
\int x\sin(x)\d x &=-x\cos(x)-\int -\cos(x)\d x\\
&= -x\cos(x)+\int \cos(x)\d x\\
&=-x\cos(x)+\sin x+C.
\end{align*}
\end{solution}


\begin{example}
Compute
\[
\int x^2\sin(x)\d x.
\] 
\end{example}

\begin{solution}
Let $u=x^2$, $dv=\sin(x)\d x$; then $du=2x\d x$ and $v=-\cos(x)$. 
Now 
\[
\int x^2\sin(x)\d x=-x^2\cos(x)+\int 2x\cos(x)\d x.
\] 
This is better than the original integral, but we need to do
integration by parts again. Let $u=2x$, $dv=\cos(x)\d x$; then $du=2$
and $v=\sin(x)$, and
\begin{align*}
  \int x^2\sin(x)\d x &=-x^2\cos(x)+\int 2x\cos(x)\d x \\
  &=-x^2\cos(x)+ 2x\sin(x) - \int 2\sin(x)\d x \\
  &=-x^2\cos(x)+ 2x\sin(x) + 2\cos(x) + C. 
\end{align*}
\end{solution}

Such repeated use of integration by parts is fairly common, but it can
be a bit tedious to accomplish, and it is easy to make
errors, especially sign errors involving the subtraction in the
formula. There is a nice tabular method to accomplish the calculation
that minimizes the chance for error and speeds up the whole
process. We illustrate with the previous example. Here is the
table:
\[
\begin{array}{|c|c|c|}\hline
\text{sign} & u & dv \\ \hline \hline
 & x^2 & \sin(x) \\ \hline
- & 2x & -\cos(x) \\ \hline
  & 2  & -\sin(x) \\ \hline
- & 0  & \cos(x) \\ \hline
\end{array}
\qquad\text{or}\qquad
\begin{array}{|c|c|}\hline
u & dv \\ \hline\hline
x^2 & \sin(x) \\ \hline 
-2x & -\cos(x) \\\hline
2 & -\sin(x)\\\hline
0 & \cos(x)\\\hline
\end{array}
\]

To form the first table, we start with $u$ at the top of the second
column and repeatedly compute the derivative; starting with $dv$ at
the top of the third column, we repeatedly compute the
antiderivative. In the first column, we place a ``$-$'' in every
second row. To form the second table we combine the first and second
columns by ignoring the boundary; if you do this by hand, you may
simply start with two columns and add a ``$-$'' to every second row.

To compute with this second table we begin at the top. Multiply the
first entry in column $u$ by the second entry in column $dv$ to get
$-x^2\cos(x)$, and add this to the integral of the product of the
second entry in column $u$ and second entry in column $dv$.  This
gives:
$$-x^2\cos(x)+\int 2x\cos(x)\d x,$$
or exactly the result of the first application of integration by
parts.  Since this integral is not yet easy, we return to the table.
Now we multiply twice on the diagonal, $(x^2)(-\cos(x))$ and
$(-2x)(-\sin(x))$ and then once straight across, $(2)(-\sin(x))$, and
combine these as
\[
-x^2\cos(x)+2x\sin(x)-\int 2\sin(x)\d x,
\]
giving the same result as the second application of integration by
parts. While this integral is easy, we may return yet once more to the
table. Now multiply three times on the diagonal to get
$(x^2)(-\cos(x))$, $(-2x)(-\sin(x))$, and $(2)(\cos(x))$, and once
straight across, $(0)(\cos(x))$. We combine these as before to get
\[
  -x^2\cos(x)+2x\sin(x) +2\cos(x)+\int 0\d x=
  -x^2\cos(x)+2x\sin(x) +2\cos(x)+C.
\]
Typically we would fill in the table one line at a time, until the
``straight across'' multiplication gives an easy integral. If we can
see that the $u$ column will eventually become zero, we can instead
fill in the whole table; computing the products as indicated will then
give the entire integral, including the ``$+C$'', as above.

\begin{exercises}
\noindent Compute the indefinite integrals.

\twocol

\begin{exercise} $\int x\cos x\d x$
\begin{answer} $\cos x+x\sin x+C$
\end{answer}\end{exercise}

\begin{exercise} $\int x^2\cos x\d x$
\begin{answer} $x^2\sin x-2 \sin x+2x\cos x +C$
\end{answer}\end{exercise}

\begin{exercise} $\int xe^x\d x$
\begin{answer} $(x-1)e^x +C$
\end{answer}\end{exercise}

\begin{exercise} $\int xe^{x^2}\d x$
\begin{answer} $(1/2)e^{x^2} +C$
\end{answer}\end{exercise}

\begin{exercise} $\int \sin^2 x\d x$
\begin{answer} $(x/2)-\sin(2x)/4 +C$
\end{answer}\end{exercise}

\begin{exercise} $\int \ln x\d x$
\begin{answer} $x\ln x-x +C$
\end{answer}\end{exercise}

\begin{exercise} $\int x\arctan x\d x$
\begin{answer} $(x^2\arctan x +\arctan x -x)/2+C$
\end{answer}\end{exercise}

\begin{exercise} $\int x^3\sin x\d x$
\begin{answer} $-x^3\cos x+3x^2\sin x+6x\cos x-6\sin x+C$
\end{answer}\end{exercise}

\begin{exercise} $\int x^3\cos x\d x$
\begin{answer} $x^3\sin x+3x^2\cos x-6x\sin x-6\cos x+C$
\end{answer}\end{exercise}

\begin{exercise} $\int x\sin^2 x\d x$
\begin{answer} $x^2/4-(\cos^2 x)/4-(x\sin x\cos x)/2+C$
\end{answer}\end{exercise}

\begin{exercise} $\int x\sin x\cos x\d x$
\begin{answer} $x/4-(x\cos^2 x)/2+(\cos x\sin x)/4+C$
\end{answer}\end{exercise}

\begin{exercise} $\int \arctan(\sqrt x)\d x$
\begin{answer} $x\arctan(\sqrt x)+\arctan(\sqrt x)-\sqrt{x}+C$
\end{answer}\end{exercise}

\begin{exercise} $\int \sin(\sqrt x)\d x$
\begin{answer} $2\sin(\sqrt x)-2\sqrt x\cos(\sqrt x)+C$
\end{answer}\end{exercise}

\begin{exercise} $\int\sec^2 x\csc^2 x\d x$
\begin{answer} $\sec x\csc x-2\cot x+C$
\end{answer}\end{exercise}

\endtwocol

\end{exercises}










\section{Powers of Sine and Cosine}


Functions consisting of products of the sine and cosine can be
integrated by using substitution and trigonometric identities. These
can sometimes be tedious, but the technique is straightforward. Some
examples will suffice to explain the approach.

\begin{example}
Evaluate $\int \sin^5 x\d x$.
Rewrite the function:
$$
  \int \sin^5 x\d x=\int \sin x \sin^4 x\d x=
  \int \sin x (\sin^2 x)^2\d x=
  \int \sin x (1-\cos^2 x)^2\d x.
$$
Now use $u=\cos x$, $du=-\sin x\d x$:
\begin{align*}
  \int \sin x (1-\cos^2 x)^2\d x&=\int -(1-u^2)^2\d u \\
  &=\int -(1-2u^2+u^4)\d u \\
  &=-u+{2\over3}u^3-{1\over5}u^5+C \\
  &=-\cos x+{2\over3}\cos^3 x-{1\over5}\cos^5x+C. 
\end{align*}
\end{example}

\begin{example}
Evaluate $\int \sin^6 x\d x$.
Use $\sin^2x =(1-\cos(2x))/2$ to
rewrite the function:
\begin{align*}
  \int \sin^6 x\d x=\int (\sin^2 x)^3\d x&=
  \int {(1-\cos 2x)^3\over 8}\d x \\
  &={1\over 8}\int 1-3\cos 2x+3\cos^2 2x-\cos^3 2x\d x.
\end{align*}
Now we have four integrals to evaluate:
$$\int 1\d x=x$$
and
$$\int -3\cos 2x\d x = -{3\over 2}\sin 2x$$
are easy. The $\cos^3 2x$ integral is like the previous example:
\begin{align*}
  \int -\cos^3 2x\d x&=\int -\cos 2x\cos^2 2x\d x \\
  &=\int -\cos 2x(1-\sin^2 2x)\d x \\
  &=\int -{1\over 2}(1-u^2)\d u \\
  &=-{1\over 2}\left(u-{u^3\over 3}\right) \\
  &=-{1\over 2}\left(\sin 2x-{\sin^3 2x\over 3}\right).
\end{align*}
And finally we use another trigonometric identity,
$\cos^2x=(1+\cos(2x))/2$:
$$
  \int 3\cos^2 2x\d x=3\int {1+\cos 4x\over 2}\d x=
  {3\over 2}\left(x+{\sin 4x\over 4}\right).
$$
So at long last we get
$$
  \int \sin^6 x\d x = {x\over8} -{3\over 16}\sin 2x 
  -{1\over 16}\left(\sin 2x-{\sin^3 2x\over 3}\right)
  +{3\over 16}\left(x+{\sin 4x\over 4}\right)+C.
$$
\end{example}

\begin{example}
Evaluate $\int \sin^2x\cos^2x\d x$. 
Use the formulas
$\sin^2x =(1-\cos(2x))/2$ and $\cos^2x =(1+\cos(2x))/2$ to get:
$$
  \int \sin^2x\cos^2x\d x=\int {1-\cos(2x)\over2}\cdot
  {1+\cos(2x)\over2}\d x.
$$
The remainder is left as an exercise.
\end{example}

\begin{exercises}

Find the antiderivatives.

\twocol

\begin{exercise} $\int \sin^2 x\d x$
\begin{answer} $x/2-\sin(2x)/4+C$
\end{answer}\end{exercise}

\begin{exercise} $\int \sin^3 x\d x$
\begin{answer} $-\cos x+(\cos^3x)/3+C$
\end{answer}\end{exercise}

\begin{exercise} $\int \sin^4 x\d x$
\begin{answer} $3x/8-(\sin 2x)/4+(\sin 4x)/32+C$
\end{answer}\end{exercise}

\begin{exercise} $\int \cos^2 x\sin^3 x\d x$
\begin{answer} $(\cos^5 x)/5-(\cos^3x)/3+C$
\end{answer}\end{exercise}

\begin{exercise} $\int \cos^3 x\d x$
\begin{answer} $\sin x-(\sin^3x)/3+C$
\end{answer}\end{exercise}

\begin{exercise} $\int \sin^2 x\cos^2 x\d x$
\begin{answer} $x/8-(\sin 4x)/32+C$
\end{answer}\end{exercise}

\begin{exercise} $\int \cos^3 x \sin^2 x\d x$
\begin{answer} $(\sin^3x)/3-(\sin^5x)/5+C$
\end{answer}\end{exercise}

\begin{exercise} $\int \sin x (\cos x)^{3/2}\d x$
\begin{answer} $-2(\cos x)^{5/2}/5+C$
\end{answer}\end{exercise}

\begin{exercise} $\int \sec^2 x\csc^2 x\d x$
\begin{answer} $\tan x-\cot x+C$
\end{answer}\end{exercise}

\begin{exercise} $\int \tan^3x \sec x\d x$
\begin{answer} $(\sec^3x)/3-\sec x+C$
\end{answer}\end{exercise}

\endtwocol

\end{exercises}

