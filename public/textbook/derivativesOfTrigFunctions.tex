\chapter{The Derivatives of Trigonometric Functions and their Inverses}



\section{The Derivatives of Trigonometric Functions}

Up until this point of the course we have been largely ignoring a large
class of functions---those involving $\sin(x)$ and $\cos(x)$. It is
now time to visit our two friends who concern themselves periodically
with triangles and circles.

\begin{theorem}[The Derivative of sin(\textit{x})]\index{derivative!of sine}\label{theorem:deriv sin}
\[
\ddx \sin(x) = \cos(x).
\]
\end{theorem}
\marginnote{
\begin{align*}
\lim_{h\to 0}\frac{\cos(h)-1}{h} &= \lim_{h\to 0}\left(\frac{\cos(h)-1}{h}\cdot\frac{\cos(h)+1}{\cos(h)+1}\right)\\
&=\lim_{h\to 0}\frac{\cos^2(h)-1}{h(\cos(h)+1)}\\
&=\lim_{h\to 0}\frac{-\sin^2(h)}{h(\cos(h)+1)}\\
&=-\lim_{h\to 0}\left(\frac{\sin(h)}{h}\cdot\frac{\sin(h)}{(\cos(h)+1)}\right)\\
&= -1 \cdot \frac{0}{2} = 0.
\end{align*}
}
\begin{proof}
Using the definition of the derivative, write
\begin{align*}
\ddx \sin(x) &= \lim_{h\to0} \frac{\sin(x+h)-\sin(x)}{h} \\
&= \lim_{h\to0} \frac{\sin(x)\cos(h)+\sin(h)\cos(x)-\sin(x)}{h}  & \text{Trig Identity.}\\
&= \lim_{h\to0} \left(\frac{\sin(x)\cos(h)-\sin(x)}{h} + \frac{\sin(h)\cos{x}}{h} \right)\\
&=\lim_{h\to0} \left(\sin (x)\frac{\cos(h) - 1}{h}+\cos(x)\frac{\sin(h)}{h}\right) \\
&=\sin(x) \cdot 0 + \cos(x) \cdot 1 = \cos x. & \text{See Example~\ref{example:sinx/x}.}
\end{align*}
\end{proof}

Consider the following geometric interpretation of the derivative of
$\sin(\theta)$.  
%\begin{figure}

\begin{tikzpicture}
	\begin{axis}[
            xmin=-.1,xmax=1.1,ymin=-.1,ymax=1.1,
            axis lines=center,
            ticks=none,
            width=5in,
            unit vector ratio*=1 1 1,
            xlabel=$x$, ylabel=$y$,
            every axis y label/.style={at=(current axis.above origin),anchor=south},
            every axis x label/.style={at=(current axis.right of origin),anchor=west},
          ]        
          \addplot [very thick, textColor!30!background, smooth, domain=(-.2:.2+pi/2)] ({cos(deg(x))},{sin(deg(x))});
          \addplot [textColor,very thick] plot coordinates {(0,0) (.766,.643)}; %% 40 degrees
          \addplot [textColor,very thick] plot coordinates {(0,0) (.766,0)}; %% bottom
          \addplot [very thick, penColor2!30!background] {(x-.766)*(-.766/.643)+.643};
          \addplot [textColor,dashed] plot coordinates {(0,0) (.766-.196,.643+1-.766)}; %% 40+16.98 degrees          

          %% \addplot [textColor!20!background] plot coordinates {(.766,.643) (1,.839)}; %% hyp
          %% \addplot [textColor!20!background] plot coordinates {(1,.643) (1,.839)}; %% side
          %% \addplot [textColor!20!background] plot coordinates {(.766,.643) (1,.643)}; %% bottom
          %% \addplot [textColor!20!background,smooth, domain=(0:40)] ({.05*cos(x)+.766},{.05*sin(x)+.643}); %% angle
          %% \node at (axis cs:.84,.670) [textColor!20!background] {\footnotesize$\theta$};
          
          %% \addplot [textColor!20!background] plot coordinates {(.766,.643) (.766,.839)}; %% side
          %% \addplot [textColor!20!background] plot coordinates {(.766,.839) (1,.839)}; %% bottom
          %% \addplot [textColor!20!background,smooth, domain=(180:220)] ({.05*cos(x)+1},{.05*sin(x)+.839}); %% angle
          %% \node at (axis cs:.926,.812) [textColor!20!background] {\footnotesize$\theta$};
          
          \draw[rotate around={30:(.5,.5)}] (.7,.7) rectangle (.25,.25);

          %\draw[textColor, rotate around={45:(.5,.5)}] (.5,.5) rectangle (.2,.2);

          \addplot [penColor4,very thick] plot coordinates {(.766,.643) (.766,.643+1-.766)}; %% side
          \addplot [textColor,very thick] plot coordinates {(.766,.643+1-.766) (.766-.196,.643+1-.766)}; %% top
          \addplot [textColor,smooth, domain=(90:130)] ({.05*cos(x)+.766},{.05*sin(x)+.643}); %% angle
          \addplot [very thick, textColor] plot coordinates {(.766-.196,.643+1-.766) (.766,.643)}; %% hyp
          \node at (axis cs:.739,.717) [textColor] {\footnotesize$\theta$};
          
          \node at (axis cs:.668,.877) [anchor=south] {\footnotesize$h\sin(\theta)$};
          \node at (axis cs:.766,.76) [anchor=west] {\footnotesize$h\cos(\theta)$};
          \node at (axis cs:.65,.78) [anchor=west] {\footnotesize$\approx h$};

          \addplot [very thick, penColor] plot coordinates {(.766,0) (.766,.643)}; %% sin theta          
          
          \addplot [textColor, smooth, domain=(0:40)] ({.15*cos(x)},{.15*sin(x)});
          \addplot [textColor, smooth, domain=(40:56.90)] ({.17*cos(x)},{.17*sin(x)});
          \addplot [textColor, smooth, domain=(40:56.90)] ({.185*cos(x)},{.185*sin(x)});
          \node at (axis cs:.15,.07) [anchor=west] {$\theta$};
          \node at (axis cs:.15,.17) {$h$};
          \node at (axis cs:.766,.322) [anchor=east] {$\sin(\theta)$};
          \node at (axis cs:.383,0) [anchor=north] {$\cos(\theta)$};
        \end{axis}
\end{tikzpicture}
%\label{figure:geo-interp sinx/x}
%\end{figure}

Here we see that increasing $\theta$ by a ``small amount'' $h$,
increases $\sin(\theta)$ by approximately $h\cos(\theta)$. Hence,
\[
\frac{\Delta y}{\Delta \theta}\approx \frac{h\cos(\theta)}{h} =
\cos(\theta).
\]

With this said, the derivative of a function measures the slope of the
plot of a function.  If we examine the graphs of the sine and cosine
side by side, it should be that the latter appears to accurately
describe the slope of the former, and indeed this is true, see
Figure~\ref{figure:sin/cos}.
\begin{figure*}
\begin{tikzpicture}
	\begin{axis}[
            xmin=-6.75,xmax=6.75,ymin=-1.5,ymax=1.5,
            axis lines=center,
            xtick={-6.28, -4.71, -3.14, -1.57, 0, 1.57, 3.142, 4.71, 6.28},
            xticklabels={$-2\pi$,$-3\pi/2$,$-\pi$, $-\pi/2$, $0$, $\pi/2$, $\pi$, $3\pi/2$, $2\pi$},
            ytick={-1,1},
            %ticks=none,
            width=9in,
            height=2in,
            unit vector ratio*=1 1 1,
            xlabel=$x$, ylabel=$y$,
            every axis y label/.style={at=(current axis.above origin),anchor=south},
            every axis x label/.style={at=(current axis.right of origin),anchor=west},
          ]        
          \addplot [very thick, penColor, samples=100,smooth, domain=(-6.75:6.75)] {sin(deg(x))};
          \addplot [very thick, penColor2, samples=100,smooth, domain=(-6.75:6.75)] {cos(deg(x))};
          \node at (axis cs:3.14,.75) [penColor] {$f(x)$};
          \node at (axis cs:-1.57,.75) [penColor2] {$f'(x)$};
        \end{axis}
\end{tikzpicture}
\caption{Here we see a plot of $f(x)=\sin(x)$ and its derivative
  $f'(x)=\cos(x)$. One can readily see that $\cos(x)$ is positive when
  $\sin(x)$ is increasing, and that $\cos(x)$ is negative when
  $\sin(x)$ is decreasing.}
\label{figure:sin/cos}
\end{figure*}

Of course, now that we know the derivative of the sine, we can compute
derivatives of more complicated functions involving the sine.

%\break

\begin{theorem}[The Derivative of cos(\textit{x})]\index{derivative!of cosine}
\[
\ddx \cos(x) = -\sin(x).
\]
\end{theorem}

\begin{proof}
Recall that
\begin{align*}
\cos(x) &= \sin\left(x+\frac{\pi}{2}\right), \\
\sin(x) &= -\cos\left(x+\frac{\pi}{2}\right).
\end{align*}
Now:
\begin{align*}
\ddx \cos(x) &= \ddx \sin\left(x+\frac{\pi}{2}\right)\\
&=\cos\left(x+\frac{\pi}{2}\right)\cdot 1 \\
&= -\sin(x).
\end{align*}
\end{proof}

Next we have:

\begin{theorem}[The Derivative of tan(\textit{x})]\index{derivative!of tangent}
\[
\ddx \tan(x) = \sec^2(x).
\]
\end{theorem}

\begin{proof}
We'll rewrite $\tan(x)$ as $\frac{\sin(x)}{\cos(x)}$ and use the quotient rule. Write
\begin{align*}
\ddx\tan(x) &= \ddx\frac{\sin(x)}{\cos(x)}\\
&=\frac{\cos^2(x) + \sin^2(x)}{\cos^2(x)}\\
&=\frac{1}{\cos^2(x)}\\
&=\sec^2(x).
\end{align*}
\end{proof}

Finally, we have

\begin{theorem}[The Derivative of sec(\textit{x})]\index{derivative!of secant}
\[
\ddx \sec(x) = \sec(x)\tan(x).
\]
\end{theorem}

\begin{proof}
We'll rewrite $\sec(x)$ as $(\cos(x))^{-1}$ and use the power rule and the chain rule. Write
\begin{align*}
\ddx \sec(x) &= \ddx(\cos (x))^{-1}\\
&=-1(\cos(x))^{-2}(-\sin(x)) \\
&= \frac{\sin(x)}{\cos^2(x)} \\
&= \sec(x)\tan(x).
\end{align*}
\end{proof}


The derivatives of the cotangent and cosecant are similar and left as
exercises. 

Putting this all together, we have:

\begin{mainTheorem}[The Derivatives of Trigonometric Functions] \hfil
\begin{itemize}
\item $\ddx \sin(x) = \cos(x)$.
\item $\ddx \cos(x) = -\sin(x)$.
\item $\ddx \tan(x) = \sec^2(x)$.
\item $\ddx \sec(x) = \sec(x)\tan(x)$.
\item $\ddx \csc(x) = -\csc(x)\cot(x)$.
\item $\ddx \cot(x) = -\csc^2(x)$.
\end{itemize}
\end{mainTheorem}


\begin{warning}
When working with derivatives of trigonometric functions, we suggest
you use \textbf{radians} for angle measure. For example, while
\[
\sin\left((90^\circ\right)^2) = \sin\left(\left(\frac{\pi}{2}\right)^2\right),
\]
one must be careful with derivatives as
\[
\left. \ddx \sin\left(x^2\right)\right|_{x=90^\circ} \ne \underbrace{2\cdot 90\cdot \cos(90^2)}_{\text{incorrect}}
\]
Alternatively, one could think of $x^\circ$ as meaning
$\frac{x\cdot\pi}{180}$, as then $90^\circ = \frac{90\cdot\pi}{180} =
\frac{\pi}{2}$. In this case
\[
2\cdot 90^\circ\cdot \cos((90^\circ)^2) = 2\cdot \frac{\pi}{2}\cdot\cos\left(\left(\frac{\pi}{2}\right)^2\right).
\]
\end{warning}



\begin{exercises}
Find the derivatives of the following functions.

\twocol

\begin{exercise} $\sin^2(\sqrt{x})$
\begin{answer} $\sin(\sqrt{x})\cos(\sqrt{x})/\sqrt{x}$
\end{answer}\end{exercise}

\begin{exercise} $\sqrt{x}\sin(x)$
\begin{answer} ${\sin(x)\over2\sqrt x}+\sqrt{x}\cos(x)$
\end{answer}\end{exercise}

\begin{exercise} ${1\over \sin(x)}$
\begin{answer} $ -{\cos(x)\over \sin^2(x)}$
\end{answer}\end{exercise}

\begin{exercise} ${x^2+x\over \sin(x)}$
\begin{answer} ${(2x +1)\sin(x) - (x^2+x)\cos(x) \over \sin^2 (x)}$
\end{answer}\end{exercise}

\begin{exercise} $\sqrt{1-\sin^2(x)}$
\begin{answer} ${-\sin(x)\cos(x)\over \sqrt{1-\sin^2(x)}}$
\end{answer}\end{exercise}

\begin{exercise} $\sin (x)\cos(x)$
\begin{answer} $\cos^2(x)-\sin^2(x)$
\end{answer}\end{exercise}

\begin{exercise} $\sin(\cos(x))$
\begin{answer} $-\sin(x)\cos(\cos(x))$
\end{answer}\end{exercise}

\begin{exercise} $\sqrt{x\tan(x)}$
\begin{answer} ${\tan(x)+x\sec^2(x)\over2\sqrt{x\tan(x)}}$
\end{answer}\end{exercise}

\begin{exercise} $\tan(x)/(1+\sin(x))$
\begin{answer} ${\sec^2(x)(1+\sin(x))-\tan(x) \cos(x)\over (1+\sin(x))^2}$
\end{answer}\end{exercise}

\begin{exercise} $\cot(x)$
\begin{answer} $ -\csc^2(x)$
\end{answer}\end{exercise}

\begin{exercise} $\csc(x)$
\begin{answer} $ -\csc(x)\cot(x)$
\end{answer}\end{exercise}

\begin{exercise} $x^3 \sin (23x^2 )$
\begin{answer} $3x^2\sin(23x^2)+46x^4\cos(23x^2)$
\end{answer}\end{exercise}

\begin{exercise} $\sin ^2(x) + \cos ^2(x)$
 \begin{answer} $0$
\end{answer}\end{exercise}

\begin{exercise}  $\sin (\cos (6x) )$
 \begin{answer} $-6\cos(\cos(6x))\sin(6x)$
\end{answer}\end{exercise}

\endtwocol

\begin{exercise} Compute ${d\over d\theta}{\sec(\theta)\over 1+\sec(\theta)}$.
 \begin{answer} $\sin(\theta)/(\cos(\theta)+1)^2$
\end{answer}\end{exercise}

\begin{exercise} Compute ${d\over dt}t^5 \cos (6t)$.
\begin{answer} $5t^4\cos(6t)-6t^5\sin(6t)$
\end{answer}\end{exercise}

\begin{exercise} Compute ${d\over dt}{t^3 \sin (3t)\over\cos (2t)}$.
\begin{answer} $3t^2(\sin(3t)+t\cos(3t))/\cos(2t)+2t^3\sin(3t)\sin(2t)/\cos^2(2t)$
\end{answer}\end{exercise}

\begin{exercise} Find all points on the graph of
$f(x)=\sin^2(x)$ at which the tangent line is horizontal.
\begin{answer} $n\pi/2$, any integer $n$
\end{answer}\end{exercise}

\begin{exercise} Find all points on the graph of $f(x) = 2\sin(x) -
\sin^2(x)$ at which the tangent line is horizontal.
\begin{answer} $\pi/2+n\pi$, any integer $n$
\end{answer}\end{exercise}

\begin{exercise} Find an
 equation for the tangent line to $\sin^2(x)$ at 
$x=\pi/3$.
\begin{answer} $\sqrt3x/2+3/4-\sqrt3\pi/6$
\end{answer}\end{exercise}

\begin{exercise} Find an equation for the tangent line to $\sec^2(x)$
at $x=\pi/3$.
\begin{answer} $8\sqrt3x+4-8\sqrt3\pi/3$
\end{answer}\end{exercise}

\begin{exercise} Find an equation for the tangent line to $\cos^2(x) -
\sin^2(4x)$ at $x=\pi/6$.
\begin{answer} $3\sqrt3x/2-\sqrt3\pi/4$
\end{answer}\end{exercise}

\begin{exercise} Find the points on the curve $y= x+ 2\cos(x)$ that have a
horizontal tangent line.
\begin{answer} $\pi/6+2n\pi$, $5\pi/6+2n\pi$, any integer $n$
\end{answer}\end{exercise}

\end{exercises}











\section{Inverse Trigonometric Functions}{}{}

The trigonometric functions frequently arise in problems, and often we
are interested in finding specific angles, say $\theta$ such that
\[
\sin(\theta) = .7
\]
Hence we want to be able to invert functions like $\sin(\theta)$ and
$\cos(\theta)$.  

However, since these functions are not one-to-one, meaning there are
are infinitely many angles with $\sin(\theta) = .7$, it is impossible
to find a true inverse function for $\sin(\theta)$. Nevertheless, it
is useful to have something like an inverse to the sine, however
imperfect. The usual approach is to pick out some collection of angles
that produce all possible values of the sine exactly once. If we
``discard'' all other angles, the resulting function has a proper
inverse.
\begin{figure*}
\begin{tikzpicture}
	\begin{axis}[
            xmin=-6.75,xmax=6.75,ymin=-1.5,ymax=1.5,
            axis lines=center,
            xtick={-6.28, -4.71, -3.14, -1.57, 0, 1.57, 3.142, 4.71, 6.28},
            xticklabels={$-2\pi$,$-3\pi/2$,$-\pi$, $-\pi/2$, $0$, $\pi/2$, $\pi$, $3\pi/2$, $2\pi$},
            ytick={-1,1},
            %ticks=none,
            width=9in,
            height=2in,
            unit vector ratio*=1 1 1,
            xlabel=$\theta$, ylabel=$y$,
            every axis y label/.style={at=(current axis.above origin),anchor=south},
            every axis x label/.style={at=(current axis.right of origin),anchor=west},
          ]        
          \addplot [very thick, penColor!20!background, samples=100,smooth, domain=(-6.75:-1.57)] {sin(deg(x))};
          \addplot [very thick, penColor!20!background, samples=100,smooth, domain=(1.57:6.75)] {sin(deg(x))};
          \addplot [very thick, penColor, samples=100,smooth, domain=(-1.57:1.57)] {sin(deg(x))};
          
          \addplot[color=penColor,fill=penColor,only marks,mark=*] coordinates{(-1.57,-1)};  %% closed hole          
          \addplot[color=penColor,fill=penColor,only marks,mark=*] coordinates{(1.57,1)};  %% closed hole          
          \node at (axis cs:3.14,.75) [penColor] {$\sin(\theta)$};
        \end{axis}
\end{tikzpicture}
\caption{The function $\sin(\theta)$ takes on all values between $-1$
  and $1$ exactly once on the interval $[-\pi/2,\pi/2]$. If we
  restrict $\sin(\theta)$ to this interval, then this restricted
  function has an inverse.}
\label{figure:sin-restricted}
\end{figure*}

In a similar fashion, we need to restrict cosine to be able to take an inverse.

\begin{figure*}
\begin{tikzpicture}
	\begin{axis}[
            xmin=-6.75,xmax=6.75,ymin=-1.5,ymax=1.5,
            axis lines=center,
            xtick={-6.28, -4.71, -3.14, -1.57, 0, 1.57, 3.142, 4.71, 6.28},
            xticklabels={$-2\pi$,$-3\pi/2$,$-\pi$, $-\pi/2$, $0$, $\pi/2$, $\pi$, $3\pi/2$, $2\pi$},
            ytick={-1,1},
            %ticks=none,
            width=9in,
            height=2in,
            unit vector ratio*=1 1 1,
            xlabel=$\theta$, ylabel=$y$,
            every axis y label/.style={at=(current axis.above origin),anchor=south},
            every axis x label/.style={at=(current axis.right of origin),anchor=west},
          ]        
          \addplot [very thick, penColor2!20!background, samples=100,smooth, domain=(-6.75:0)] {cos(deg(x))};
          \addplot [very thick, penColor2!20!background, samples=100,smooth, domain=(3.14:6.75)] {cos(deg(x))};
          \addplot [very thick, penColor2, samples=100,smooth, domain=(0:3.14)] {cos(deg(x))};
          
          \addplot[color=penColor2,fill=penColor2,only marks,mark=*] coordinates{(0,1)};  %% closed hole          
          \addplot[color=penColor2,fill=penColor2,only marks,mark=*] coordinates{(pi,-1)};  %% closed hole          
          \node at (axis cs:-1.57,.75) [penColor2] {$\cos(\theta)$};
        \end{axis}
\end{tikzpicture}
\caption{The function $\cos(\theta)$ takes on all values between $-1$
  and $1$ exactly once on the interval $[0,\pi]$. If we restrict
  $\cos(\theta)$ to this interval, then this restricted function has
  an inverse.}
\label{figure:cos-restricted}
\end{figure*}

By examining both sine and cosine on restricted domains, we can now produce functions arcsine and arccosine:

\begin{fullwidth}
\begin{tabular}{lll}\index{arcsine}\index{arccosine}
\begin{tikzpicture}
	\begin{axis}[
            xmin=-1.5,xmax=1.5,ymin=-1.75,ymax=1.75,
            axis lines=center,
            ytick={-1.57, 0, 1.57},
            yticklabels={$-\pi/2$, $0$, $\pi/2$},
            xtick={-1,1},
            unit vector ratio*=1 1 1,
            xlabel=$y$, ylabel=$\theta$,
            every axis y label/.style={at=(current axis.above origin),anchor=south},
            every axis x label/.style={at=(current axis.right of origin),anchor=west},
          ]        
          \addplot [very thick, penColor4, samples=100,smooth, domain=(-1:1)] {asin(x)*pi/180};
                    
          \addplot[color=penColor4,fill=penColor4,only marks,mark=*] coordinates{(-1,-pi/2)};  %% closed hole          
          \addplot[color=penColor4,fill=penColor4,only marks,mark=*] coordinates{(1,pi/2)};  %% closed hole          
        \end{axis}
\end{tikzpicture}

&

\hspace{1in}

&

\begin{tikzpicture}
	\begin{axis}[
            xmin=-1.5,xmax=1.5,ymin=-.25,ymax=3.25,
            axis lines=center,
            ytick={0, 1.57,3.14},
            yticklabels={$0$, $\pi/2$,$\pi$},
            xtick={-1,1},
            unit vector ratio*=1 1 1,
            xlabel=$y$, ylabel=$\theta$,
            every axis y label/.style={at=(current axis.above origin),anchor=south},
            every axis x label/.style={at=(current axis.right of origin),anchor=west},
          ]        
          \addplot [very thick, penColor5, samples=100,smooth, domain=(-1:1)] {acos(x)*pi/180};
                    
          \addplot[color=penColor5,fill=penColor5,only marks,mark=*] coordinates{(1,0)};  %% closed hole          
          \addplot[color=penColor5,fill=penColor5,only marks,mark=*] coordinates{(-1,pi)};  %% closed hole          
        \end{axis}
\end{tikzpicture} \\
\begin{minipage}{2.5in}
Here we see a plot of $\arcsin(y)$, the inverse function of
$\sin(\theta)$ when it is restricted to the interval $[-\pi/2,\pi/2]$.
\end{minipage}

& 

& 
\begin{minipage}{2.5in}
Here we see a plot of $\arccos(y)$, the inverse function of
$\cos(\theta)$ when it is restricted to the interval $[0,\pi]$.
\end{minipage}
\end{tabular}
\end{fullwidth}

\vspace{.5cm}

Recall that a function and its inverse undo each other in either
order, for example, 
\marginnote{Compare this with the fact that while
  $\left(\sqrt{x}\right)^2=x$, we have that $\sqrt{x^2}=|x|$.}
\[
\sqrt[3]{x^3}=x\qquad \text{and}\qquad \left(\sqrt[3]{x}\right)^3=x. 
\]
However, since arcsine is the inverse of sine restricted to the
interval $[-\pi/2,\pi/2]$, this does not work with sine and arcsine, for example
\[
\arcsin(\sin(\pi))=0.
\]
Moreover, there is a similar situation for cosine and arccosine as
\[
\arccos(\cos(2\pi))=0.
\]
Once you get a feel for how $\arcsin(y)$ and $\arccos(y)$ behave, let's examine tangent.

\newpage

\begin{fullwidth}
\begin{figure*}
\begin{tikzpicture}
	\begin{axis}[
            xmin=-6.75,xmax=6.75,ymin=-3,ymax=3,
            axis lines=center,
            width=9in,
            height=3.5in,
            xtick={-6.28, -4.71, -3.14, -1.57, 0, 1.57, 3.142, 4.71, 6.28},
            xticklabels={$-2\pi$,$-3\pi/2$,$-\pi$, $-\pi/2$, $0$, $\pi/2$, $\pi$, $3\pi/2$, $2\pi$},       
            unit vector ratio*=1 1 1,
            xlabel=$\theta$, ylabel=$y$,
            every axis y label/.style={at=(current axis.above origin),anchor=south},
            every axis x label/.style={at=(current axis.right of origin),anchor=west},
          ]        
          \addplot [very thick, penColor3, samples=100,smooth, domain=(-1.55:1.55)] {tan(deg(x))};
          \addplot [very thick, penColor3!30!background, samples=100,smooth, domain=(-4.69:-1.59)] {tan(deg(x))};
          \addplot [very thick, penColor3!30!background, samples=100,smooth, domain=(-6.75:-4.73)] {tan(deg(x))};
          \addplot [very thick, penColor3!30!background, samples=100,smooth, domain=(1.59:4.69)] {tan(deg(x))};
          \addplot [very thick, penColor3!30!background, samples=100,smooth, domain=(4.73:6.75)] {tan(deg(x))};
          
          \addplot [textColor,dashed] plot coordinates {(-4.71,-3) (-4.71,3)};
          \addplot [textColor,dashed] plot coordinates {(-1.57,-3) (-1.57,3)};
          \addplot [textColor,dashed] plot coordinates {(1.57,-3) (1.57,3)};
          \addplot [textColor,dashed] plot coordinates {(4.71,-3) (4.71,3)};
          
          \node at (axis cs:.4,1.25) [penColor3] {$\tan(\theta)$};    
        \end{axis}
\end{tikzpicture}
\caption{The function $\tan(\theta)$ takes on all values in $\R$
  exactly once on the open interval $(-\pi/2,\pi/2)$. If we restrict
  $\tan(\theta)$ to this interval, then this restricted function has
  an inverse.}
\end{figure*}
\end{fullwidth}

Again, only working on a restricted domain of tangent, we can produce an inverse function, arctangent. 
\begin{fullwidth}
\begin{figure*}[!h]
\begin{tikzpicture}
	\begin{axis}[
            xmin=-6,xmax=6,ymin=-2,ymax=2,
            axis lines=center,
            ytick={0, -1.57,1.57},
            width=9in,
            height=2.5in,
            yticklabels={$0$, $-\pi/2$,$\pi/2$},
            xtick={0},
            unit vector ratio*=1 1 1,
            xlabel=$y$, ylabel=$\theta$,
            every axis y label/.style={at=(current axis.above origin),anchor=south},
            every axis x label/.style={at=(current axis.right of origin),anchor=west},
          ]        
          \addplot [very thick, penColor3!20!penColor2, samples=100,smooth, domain=(-6:6)] {atan(x)*pi/180};
          \addplot [textColor,dashed] plot coordinates {(-6,-1.57) (6,-1.57)};
          \addplot [textColor,dashed] plot coordinates {(-6,1.57) (6,1.57)};
        \end{axis}
\end{tikzpicture}
\caption{Here we see a plot of $\arctan(y)$, the inverse function of
$\tan(\theta)$ when it is restricted to the interval $(-\pi/2,\pi/2)$.}
\end{figure*}
\index{arctangent}
\end{fullwidth}

We leave it to you, the reader, to investigate the functions
arcsecant, arccosecant, and arccotangent.


\subsection*{The Derivatives of Inverse Trigonometric Functions}

What is the derivative of the arcsine? Since this is an inverse
function, we can find its derivative by using implicit
differentiation and the Inverse Function Theorem, Theorem~\ref{theorem:IFT}.


\begin{theorem}[The Derivative of arcsin(\textit{y})]\index{derivative!of arcsine}
\[
\dd{y} \arcsin(y) = \frac{1}{\sqrt{1-y^2}}.
\]
\end{theorem}

\begin{proof} 
To start, note that the Inverse Function Theorem,
Theorem~\ref{theorem:IFT} assures us that this derivative actually
exists.  Recall
\[
\arcsin(y) = \theta \qquad\Rightarrow\qquad \sin(\theta) = y.
\]
Hence
\begin{align*}
\sin(\theta) &= y\\
\dd{y} \sin(\theta) &= \dd{y} y \\
\cos(\theta) \frac{d\theta}{dy} &= 1 \\
\frac{d\theta}{dy} &= \frac{1}{\cos(\theta)}.
\end{align*}
At this point, we would like $\cos(\theta)$ written in terms of $y$. Since
\[
\cos^2(\theta)+\sin^2(\theta) =1
\]
and $\sin(\theta) = y$, we may write
\begin{align*}
\cos^2(\theta)+y^2 &=1\\
\cos^2(\theta) &=1-y^2\\
\cos(\theta) &= \pm \sqrt{1-y^2}.
\end{align*}
Since $\theta=\arcsin(y)$ we know that $-\pi/2\le \theta\le \pi/2$, and the cosine of
an angle in this interval is always positive. Thus
$\cos(\theta)=\sqrt{1-y^2}$ and 
\[
\dd{y} \arcsin(y) = \frac{1}{\sqrt{1-y^2}}.
\]
\end{proof}

We can do something similar with arccosine. 

\begin{theorem}[The Derivative of arccos(\textit{y})]\index{derivative!of arccosine}
\[
\dd{y} \arccos(y) = \frac{-1}{\sqrt{1-y^2}}.
\]
\end{theorem}

\begin{proof} 
To start, note that the Inverse Function Theorem,
Theorem~\ref{theorem:IFT} assures us that this derivative actually
exists.  Recall
\[
\arccos(y) = \theta \qquad\Rightarrow\qquad \cos(\theta) = y.
\]
Hence
\begin{align*}
\cos(\theta) &= y\\
\dd{y} \cos(\theta) &= \dd{y} y \\
-\sin(\theta) \frac{d\theta}{dy} &= 1 \\
\frac{d\theta}{dy} &= \frac{-1}{\sin(\theta)}.
\end{align*}
At this point, we would like $\sin(\theta)$ written in terms of $y$. Since
\[
\cos^2(\theta)+\sin^2(\theta) =1
\]
and $\cos(\theta) = y$, we may write
\begin{align*}
y^2+\sin^2(\theta) &=1\\
\sin^2(\theta) &=1-y^2\\
\sin(\theta) &= \pm \sqrt{1-y^2}.
\end{align*}
Since $\theta=\arccos(y)$ we know that $0\le \theta\le \pi$, and the sine of
an angle in this interval is always positive. Thus
$\sin(\theta)=\sqrt{1-y^2}$ and 
\[
\dd{y} \arccos(y) = \frac{-1}{\sqrt{1-y^2}}.
\]
\end{proof}

Finally, let's look at arctangent.

\begin{theorem}[The Derivative of arctan(\textit{y})]\index{derivative!of arctangent}
\[
\dd{y} \arctan(y) = \frac{1}{1+y^2}.
\]
\end{theorem}

\begin{proof} 
To start, note that the Inverse Function Theorem,
Theorem~\ref{theorem:IFT} assures us that this derivative actually
exists.  Recall
\[
\arctan(y) = \theta \qquad\Rightarrow\qquad \tan(\theta) = y.
\]
Hence
\begin{align*}
\tan(\theta) &= y\\
\dd{y} \tan(\theta) &= \dd{y} y \\
\sec^2(\theta) \frac{d\theta}{dy} &= 1 \\
\frac{d\theta}{dy} &= \frac{1}{\sec^2(\theta)}.
\end{align*}
At this point, we would like $\sec^2(\theta)$ written in terms of $y$. Recall
\[
\sec^2(\theta) = 1+\tan^2(\theta)
\]
and $\tan(\theta) = y$, we may write $\sec^2(\theta)=1+y^2$. Hence
\[
\dd{y} \arctan(y) = \frac{1}{1+y^2}.
\]
\end{proof}

We leave it to you, the reader, to investigate the derivatives of
arcsecant, arccosecant, and arccotangent. However, as a gesture of
friendship, we now present you with a list of derivative formulas for
inverse trigonometric functions.

\begin{mainTheorem}[The Derivatives of Inverse Trigonometric Functions] \hfil
\begin{itemize}
\item $\dd{y} \arcsin(y) = \frac{1}{\sqrt{1-y^2}}$.
\item $\dd{y} \arccos(y) = \frac{-1}{\sqrt{1-y^2}}$.
\item $\dd{y} \arctan(y) = \frac{1}{1+y^2}$.
\item $\dd{y} \arcsec(y) = \frac{1}{|y|\sqrt{y^2-1}}$ for $|y|>1$.
\item $\dd{y} \arccsc(y) = \frac{-1}{|y|\sqrt{y^2-1}}$ for $|y|>1$.
\item $\dd{y} \arccot(y) = \frac{-1}{1+y^2}$.
\end{itemize}
\end{mainTheorem}





\begin{exercises}

\begin{exercise} The inverse of $\cot$ is usually defined so that the
 range of arccotangent is $(0, \pi)$.  Sketch the graph of
 $y=\arccot(x)$. In the process you will make it clear what the domain
 of arccotangent is. Find the derivative of the arccotangent.
\begin{answer} ${-1\over 1+x^2}$
\end{answer}\end{exercise}

%% \begin{exercise} Show that $\arccot(x) + \arctan(x) =\pi/2$.
%% \end{exercise}

\begin{exercise} Find the derivative of $\arcsin(x^2)$.
\begin{answer} ${2x\over\sqrt{1-x^4}}$
\end{answer}\end{exercise}

\begin{exercise} Find the derivative of $\arctan(e^x)$.
\begin{answer} ${e^x\over1+e^{2x}}$
\end{answer}\end{exercise}

\begin{exercise} Find the derivative of $\arccos (\sin x^3 )$
\begin{answer} $-3x^2\cos(x^3)/\sqrt{1-\sin^2(x^3)}$
\end{answer}\end{exercise}

\begin{exercise} Find the derivative of $\ln( (\arcsin(x) )^2)$
\begin{answer} ${2\over(\arcsin(x))\sqrt{1-x^2}}$
\end{answer}\end{exercise}

\begin{exercise} Find the derivative of $\arccos(e^x)$
\begin{answer} $-e^x/\sqrt{1-e^{2x}}$
\end{answer}\end{exercise}

\begin{exercise} Find the derivative of $\arcsin(x) + \arccos(x)$
\begin{answer} $0$
\end{answer}\end{exercise}

\begin{exercise} Find the derivative of $\log _5 (\arctan (x^x ) )$
\begin{answer} ${(1+\ln x)x^x\over\ln5(1+x^{2x})\arctan(x^x)}$
\end{answer}\end{exercise}

\end{exercises}
