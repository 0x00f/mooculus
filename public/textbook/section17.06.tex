\section{Second Order Linear Equations}{}{}
%\label{sec:2nd order differential equations}
\nobreak
Now we consider second order equations of the form $\ds a\ddot y+b\dot
y+cy=f(t)$, with $a$, $b$, and $c$ constant. Of course, if $a=0$ this
is really a first order equation, so we assume $a\not=0$.
Also, much
as in exercise~\xrefn{exer:second order really first order} of
section~\xrefn{sec:second order homogeneous}, if $c=0$ we can solve
the related first order equation $\ds a\dot h+bh=f(t)$, and then solve
$\ds h=\dot y$ for $y$. So we will only examine examples in which
$c\not=0$.

Suppose that
$\ds y_1(t)$ and $\ds y_2(t)$ are solutions to $\ds a\ddot y+b\dot
y+cy=f(t)$, and consider the function $\ds h=y_1-y_2$. We substitute
this function into the left hand side of the differential equation and
simplify: 
$$
a(y_1-y_2)''+b(y_1-y_2)'+c(y_1-y_2)=ay_1''+by_1'+cy_1 -
(ay_2''+by_2'+cy_2)=f(t)-f(t)=0.
$$ 
So $h$ is a solution to the homogeneous equation $\ds a\ddot
y+b\dot y+cy=0$. Since we know how to find all such $h$, then with
just one particular solution $\ds y_2$ we can express all possible
solutions $\ds y_1$, namely, $\ds y_1=h+y_2$, where now $h$ is the
general solution to the homogeneous equation. Of course, this is
exactly how we approached the first order linear equation.

To make use of this observation we need a method to find a single
solution $y_2$. This turns out to be somewhat more difficult than the
first order case, but if $f(t)$ is of a certain simple form, we can
find a solution using the {\dfont method of undetermined
  coefficients\index{undetermined coefficients}}, sometimes 
more whimsically called the
{\dfont method of judicious guessing\index{judicious guessing}}.

\begin{example} Solve the differential equation $\ds \ddot y-\dot
y-6y=18t^2+5$. The general solution of the homogeneous equation is
$\ds Ae^{3t}+Be^{-2t}$. We guess that a solution to the
non-homogeneous equation might look like $f(t)$ itself, namely,
a quadratic $\ds y=at^2+bt+c$. Substituting this guess into the
differential equation we get
$$
\ddot y-\dot y-6y = 2a-(2at+b)-6(at^2+bt+c) = -6at^2+(-2a-6b)t+(2a-b-6c).
$$
We want this to equal $18t^2+5$, so we need 
$$\eqalign{
-6a&=18 \\
-2a-6b&=0 \\
2a-b-6c&=5 \\}
$$
This is a system of three equations in three unknowns and is not hard
to solve: $a=-3$, $b=1$, $c=-2$. Thus the general solution to the
differential equation is $\ds Ae^{3t}+Be^{-2t}-3t^2+t-2$.
\end{example}

So the ``judicious guess'' is a function with the same form as $f(t)$
but with undetermined (or better, yet to be determined)
coefficients. This works whenever $f(t)$ is a polynomial.

\begin{example} Consider the initial value problem $\ds m\ddot y +ky=-mg$,
$y(0)=2$, $\ds\dot y(0)=50$. The left hand side represents a mass-spring
system with no damping, i.e., $b=0$. Unlike the homogeneous case, we
now consider the force due to gravity, $-mg$, assuming the spring is
vertical at the surface of the earth, so that $g=980$. To be specific,
let us take $m=1$ and $k=100$. The general solution to the homogeneous
equation is $\ds A\cos(10t)+B\sin(10t)$. For the solution to the 
non-homogeneous equation we guess simply a constant $y=a$, since $-mg=-980$
is a constant. Then $\ds \ddot y+100y= 100a$ so $a=-980/100=-9.8$. The
desired general solution is then $\ds A\cos(10t)+B\sin(10t)-9.8$.
Substituting the initial conditions we get
$$\eqalign{
2&=A-9.8 \\
50&=10B \\}
$$
so $A=11.8$ and $B=5$ and the solution is $\ds 11.8\cos(10t)+5\sin(10t)-9.8$.
\end{example}

More generally, this method can be used when a function similar to
$f(t)$ has derivatives that are also similar to $f(t)$; in the
examples so far, since $f(t)$ was a polynomial, so were its derivatives.
The method will work if $f(t)$ has the form $p(t)e^{\alpha t}\cos(\beta t)+
q(t)e^{\alpha t}\sin(\beta t)$, where $p(t)$ and $q(t)$ are
polynomials; when $\alpha=\beta=0$ this is simply $p(t)$, a
polynomial. In the most general form it is not simple to describe the
appropriate judicious guess; we content ourselves with some examples
to illustrate the process.

\begin{example} Find the general solution to $\ds\ddot y+7\dot
y+10y=e^{3t}$. The characteristic equation is $r^2+7r+10=(r+5)(r+2)$,
so the solution to the homogeneous equation is
$Ae^{-5t}+Be^{-2t}$. For a particular solution to the inhomogeneous
equation we guess $Ce^{3t}$. Substituting we get
$$
9Ce^{3t}+21Ce^{3t}+10Ce^{3t}=e^{3t}40C.
$$
When $C=1/40$ this is equal to $f(t)=e^{3t}$, so the solution is
$Ae^{-5t}+Be^{-2t}+(1/40)e^{3t}$.
\end{example}

\begin{example} Find the general solution to $\ds\ddot y+7\dot
y+10y=e^{-2t}$. Following the last example we might guess
$Ce^{-2t}$, but since this is a solution to the homogeneous equation
it cannot work. Instead we guess $Cte^{-2t}$. Then
$$
(-2Ce^{-2t}-2Ce^{-2t}+4Cte^{-2t})+7(Ce^{-2t}-2Cte^{-2t})+10Cte^{-2t}
=e^{-2t}(-3C).
$$
Then $C=-1/3$ and the solution is $Ae^{-5t}+Be^{-2t}-(1/3)te^{-2t}$.
\end{example}

In general, if $f(t)=e^{kt}$ and $k$ is one of the roots of the
characteristic equation, then we guess $Cte^{kt}$ instead of
$Ce^{kt}$. If $k$ is the only root of the characteristic equation,
then $Cte^{kt}$ will not work, and we must guess $Ct^2e^{kt}$.

\begin{example} Find the general solution to $\ds\ddot y-6\dot
y+9y=e^{3t}$. The characteristic equation is 
$\ds r^2-6r+9=(r-3)^2$, so the general solution to the homogeneous
equation is $Ae^{3t}+Bte^{3t}$. Guessing $Ct^2e^{3t}$ for the
particular solution, we get
$$
(9Ct^2e^{3t}+6Cte^{3t}+6Cte^{3t}+2Ce^{3t})-6(3Ct^2e^{3t}+2Cte^{3t})+9Ct^2e^{3t}
=e^{3t}2C.
$$
The solution is thus $\ds Ae^{3t}+Bte^{3t}+(1/2)t^2e^{3t}$.
\end{example}

It is common in various physical systems to encounter an $f(t)$ of the
form $\ds a\cos(\omega t)+b\sin(\omega t)$.

\begin{example} Find the general solution to $\ds\ddot y+6\dot
y+25y=\cos(4t)$. The roots of the characteristic equation are
$-3\pm 4i$, so the solution to the homogeneous equation is
$\ds e^{-3t}(A\cos(4t)+B\sin(4t))$. For a particular solution, we
guess $C\cos(4t)+D\sin(4t)$. Substituting as usual:
$$\eqalign{
(-16C\cos(4t)&+-16D\sin(4t))+6(-4C\sin(4t)+4D\cos(4t))+25(C\cos(4t)+D\sin(4t)) \\
&=(24D+9C)\cos(4t)+(-24C+9D)\sin(4t). \\}
$$
To make this equal to $\cos(4t)$ we need
$$\eqalign{
24D+9C&=1 \\
9D-24C&=0 \\}
$$
which gives $C=1/73$ and $D=8/219$. The full solution is then
$\ds e^{-3t}(A\cos(4t)+B\sin(4t))+(1/73)\cos(4t)+(8/219)\sin(4t)$.

The function $\ds e^{-3t}(A\cos(4t)+B\sin(4t))$ is a damped
oscillation as in example~\xrefn{example:damped spring oscillation},
while $\ds(1/73)\cos(4t)+(8/219)\sin(4t)$ is a simple undamped
oscillation. As $t$ increases, the sum $\ds
e^{-3t}(A\cos(4t)+B\sin(4t))$ approaches zero, so the solution
$$e^{-3t}(A\cos(4t)+B\sin(4t))+(1/73)\cos(4t)+(8/219)\sin(4t)$$
becomes more and more like the simple oscillation
$\ds(1/73)\cos(4t)+(8/219)\sin(4t)$---notice that the initial
conditions don't matter to this long term behavior. The damped portion
is called the 
{\dfont transient\index{transient part of solution to d.e.}\/} part of the
solution, and the simple oscillation is called the {\dfont
  steady state\index{steady state part of solution to d.e.}\/} 
part of the solution. 
A physical example is a mass-spring system. If the only force on the
mass is due to the spring, then the behavior of the system is a damped
oscillation. If in addition an external force is applied to the mass,
and if the force varies according to a function of the form
 $\ds a\cos(\omega t)+b\sin(\omega t)$, then the long term behavior
will be a simple oscillation determined by the steady state part of the
general solution; the initial position of the mass will not matter.
\end{example}

As with the exponential form, such a simple guess may not work.

\begin{example} Find the general solution to $\ds\ddot y+16y=-\sin(4t)$. 
The roots of the characteristic equation are $\pm4i$, so the
solution to the homogeneous equation is $A\cos(4t)+B\sin(4t)$. Since
both $\cos(4t)$ and $\sin(4t)$ are solutions to the homogeneous
equation,  $C\cos(4t)+D\sin(4t)$ is also, so it cannot be a solution
to the non-homogeneous equation. Instead, we guess
$Ct\cos(4t)+Dt\sin(4t)$. Then substituting:
$$\eqalign{
(-16Ct\cos(4t)&-16D\sin(4t)+8D\cos(4t)-8C\sin(4t)))+16(Ct\cos(4t)+Dt\sin(4t)) \\
&=8D\cos(4t)-8C\sin(4t). \\}
$$
Thus $C=1/8$, $D=0$, and the solution is
$\ds C\cos(4t)+D\sin(4t)+(1/8)t\cos(4t)$.
\end{example}

In general, if $f(t)=a\cos(\omega t)+b\sin(\omega t)$, and $\pm \omega
i$ are the roots of the characteristic equation, then instead of 
$C\cos(\omega t)+D\sin(\omega t)$ we guess $Ct\cos(\omega t)+Dt\sin(\omega t)$.

\begin{exercises}

Find the general solution to the differential equation.

\exer $\ds\ddot y -10\dot y+25y=\cos t$
\begin{answer} $Ae^{5t}+Bte^{5t}+(6/169)\cos t-(5/338)\sin t$
\end{answer}\end{exercise}

\exer $\ds\ddot y+2\sqrt2\dot y+2y=10$
\begin{answer} $\ds Ae^{-\sqrt2t}+Bte^{-\sqrt2t}+5$
\end{answer}\end{exercise}

\exer $\ds\ddot y+16y=8t^2+3t-4$
\begin{answer} $\ds A\cos(4t)+B\sin(4t)+ (1/2)t^2+(3/16)t-5/16$
\end{answer}\end{exercise}

\exer $\ds\ddot y+2y=\cos(5t)+\sin(5t)$
\begin{answer} $\ds A\cos(\sqrt2t)+B\sin(\sqrt2t)-(\cos(5t)+\sin(5t))/23$
\end{answer}\end{exercise}

\exer $\ds\ddot y-2\dot y+2y=e^{2t}$
\begin{answer} $\ds e^{t}(A\cos t+B\sin t)+e^{2t}/2$
\end{answer}\end{exercise}

\exer $\ds\ddot y-6y+13=1+2t+e^{-t}$
\begin{answer} $\ds Ae^{\sqrt6t}+Be^{-\sqrt6t}+2-t/3-e^{-t}/5$
\end{answer}\end{exercise}

\exer $\ds\ddot y+\dot y-6y=e^{-3t}$
\begin{answer} $\ds Ae^{-3t}+Be^{2t}-(1/5)te^{-3t}$
\end{answer}\end{exercise}

\exer $\ds\ddot y-4\dot y+3y=e^{3t}$
\begin{answer} $\ds Ae^t+Be^{3t}+(1/2)te^{3t}$
\end{answer}\end{exercise}

\exer $\ds\ddot y+16y=\cos(4t)$
\begin{answer} $\ds A\cos(4t)+B\sin(4t)+(1/8)t\sin(4t)$
\end{answer}\end{exercise}

\exer $\ds\ddot y +9y=3\sin(3t)$
\begin{answer} $\ds A\cos(3t)+B\sin(3t)-(1/2)t\cos(3t)$
\end{answer}\end{exercise}

\exer $\ds\ddot y+12\dot y+36y=6e^{-6t}$
\begin{answer} $\ds Ae^{-6t}+Bte^{-6t}+3t^2e^{-6t}$
\end{answer}\end{exercise}

\exer $\ds\ddot y-8\dot y+16y=-2e^{4t}$
\begin{answer} $\ds Ae^{4t}+Bte^{4t}-t^2e^{4t}$
\end{answer}\end{exercise}

\exer $\ds\ddot y+6\dot y+5y=4$
\begin{answer} $\ds Ae^{-t}+Be^{-5t}+(4/5)$
\end{answer}\end{exercise}

\exer $\ds\ddot y-\dot y-12y=t$
\begin{answer} $\ds Ae^{4t}+Be^{-3t}+(1/144)-(t/12)$
\end{answer}\end{exercise}

\exer $\ds\ddot y+5y=8\sin(2t)$
\begin{answer} $\ds A\cos(\sqrt5t)+B\sin(\sqrt5t)+8\sin(2t)$
\end{answer}\end{exercise}

\exer $\ds\ddot y-4y=4e^{2t}$
\begin{answer} $\ds Ae^{2t}+Be^{-2t}+te^{2t}$
\end{answer}\end{exercise}

Solve the initial value problem.

\exer $\ds\ddot y-y=3t+5$, $y(0)=0$, $\ds\dot y(0)=0$
\begin{answer} $\ds 4e^{t}+e^{-t}-3t-5$
\end{answer}\end{exercise}

\exer $\ds\ddot y+9y=4t$, $y(0)=0$, $\ds\dot y(0)=0$
\begin{answer} $\ds -(4/27)\sin(3t)+(4/9)t$
\end{answer}\end{exercise}

\exer $\ds\ddot y +12\dot y +37y=10e^{-4t}$, $y(0)=4$, $\ds\dot y(0)=0$
\begin{answer} $\ds e^{-6t}(2\cos t+20\sin t)+2e^{-4t}$
\end{answer}\end{exercise}

\exer $\ds\ddot y+6\dot y+18y=\cos t-\sin t$, $y(0)=0$, $\ds\dot
y(0)=2$ 
\begin{answer} $\ds
\left(-{23\over 325}\cos(3t)+{592\over 975}\sin(3t)\right)+
{23\over325}\cos t-{11\over325}\sin t$
\end{answer}\end{exercise}

\exer Find the solution for the mass-spring equation
$\ds\ddot y+4\dot y+29y=689\cos(2t)$.
\begin{answer} $\ds e^{-2t}(A\sin(5t)+B\cos(5t))+8\sin(2t)+25\cos(2t)$
\end{answer}\end{exercise}

\exer Find the solution for the mass-spring equation
$\ds3\ddot y+12\dot y+24y=2\sin t$.
\begin{answer} $\ds e^{-2t}(A\sin(2t)+B\cos(2t))+(14/195)\sin t-(8/195)\cos t$
\end{answer}\end{exercise}

\exer Consider the differential 
equation $\ds m\ddot y+b\dot y+ky=\cos(\omega t)$,
with $m$, $b$, and $k$ all positive and $\ds b^2<2mk$; this equation
is a model for a damped mass-spring system with external 
driving force $\cos(\omega t)$.
Show that the steady state part of the solution has amplitude
$${1\over \sqrt{(k-m\omega^2)^2+\omega^2b^2}}.$$
Show that this amplitude is largest when 
$\ds \omega={\sqrt{4mk-2b^2}\over 2m}$. This is the 
{\dfont resonant frequency\index{resonant frequency}} of the system.

\end{exercises}
