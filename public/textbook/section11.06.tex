\section{Absolute Convergence}{}{}
\nobreak
Roughly speaking there are two ways for a series to converge: As in
the case of $\sum 1/n^2$, the individual terms get small very quickly,
so that the sum of all of them stays finite, or, as in the case of
$\ds \sum (-1)^{n-1}/n$, the terms don't get small fast enough ($\sum 1/n$
diverges), but a mixture of positive and negative terms provides
enough cancellation to keep the sum finite. You might guess from what
we've seen that if the terms get small fast enough to do the job, then
whether or not some terms are negative and some positive the series
converges. 

\begin{theorem} If $\ds\sum_{n=0}^\infty |a_n|$ converges, then 
$\ds\sum_{n=0}^\infty a_n$ converges.
\begin{proof}
Note that $\ds 0\le a_n+|a_n|\le 2|a_n|$ so by the comparison test
$\ds\sum_{n=0}^\infty (a_n+|a_n|)$ converges. Now
$$
  \sum_{n=0}^\infty (a_n+|a_n|) -\sum_{n=0}^\infty |a_n|
  = \sum_{n=0}^\infty a_n+|a_n|-|a_n| = \sum_{n=0}^\infty a_n 
$$
converges by theorem~\xrefn{thm:series are linear}.
\end{proof}

So given a series $\sum a_n$ with both positive and negative terms,
you should first ask whether $\sum |a_n|$ converges. This may be an
easier question to answer, because we have tests that apply
specifically to terms with non-negative terms. If $\sum |a_n|$
converges then you know that $\sum a_n$ converges as well. If $\sum
|a_n|$ diverges then it still may be true that $\sum a_n$
converges---you will have to do more work to decide the question.
Another way to think of this result is: it is (potentially) easier for
$\sum a_n$ to converge than for $\sum |a_n|$ to converge, because the
latter series cannot take advantage of cancellation. 

If $\sum |a_n|$ converges we say that $\sum a_n$ converges {\dfont
absolutely\index{series!absolute convergence}\/}; to say that $\sum
a_n$ converges absolutely is to say that any cancellation that happens
to come along is not really needed, as the terms already get small so
fast that convergence is guaranteed by that alone. If $\sum a_n$
converges but $\sum |a_n|$ does not, we say that $\sum a_n$ converges
{\dfont conditionally\index{series!conditional convergence}}. For
example $\ds\sum_{n=1}^\infty (-1)^{n-1} {1\over n^2}$ converges
absolutely, while $\ds\sum_{n=1}^\infty (-1)^{n-1} {1\over n}$
converges conditionally.

\begin{example} Does $\ds\sum_{n=2}^\infty {\sin n\over n^2}$ converge?

\ssk\noindent
In example~\xrefn{example:absolute sine over n squared} we saw that 
$\ds\sum_{n=2}^\infty {|\sin n|\over n^2}$ converges, so the given
series converges absolutely.
\end{example}

\begin{example} Does $\ds\sum_{n=0}^\infty (-1)^{n}{3n+4\over 2n^2+3n+5}$ converge?

\ssk\noindent
Taking the absolute value, $\ds\sum_{n=0}^\infty {3n+4\over 2n^2+3n+5}$
diverges by comparison to $\ds\sum_{n=1}^\infty {3\over 10n}$, so if
the series converges it does so conditionally. It is true that
$\ds\lim_{n\to\infty}(3n+4)/(2n^2+3n+5)=0$, so to apply the
alternating series test we need to know whether the terms are
decreasing.
If we let $\ds f(x)=(3x+4)/(2x^2+3x+5)$ then 
$\ds f'(x)=-(6x^2+16x-3)/(2x^2+3x+5)^2$, and it is not hard to see that
this is negative for $x\ge1$, so the series is decreasing and by the
alternating series test it converges.
\end{example}

\begin{exercises}

Determine whether each series converges absolutely, converges
conditionally, or diverges.

\twocol

\begin{exercise} $\ds\sum_{n=1}^\infty (-1)^{n-1}{1\over 2n^2+3n+5}$
\begin{answer} converges absolutely
\end{answer}\end{exercise}

\begin{exercise} $\ds\sum_{n=1}^\infty (-1)^{n-1}{3n^2+4\over 2n^2+3n+5}$
\begin{answer} diverges
\end{answer}\end{exercise}

\begin{exercise} $\ds\sum_{n=1}^\infty (-1)^{n-1}{\ln n\over n}$
\begin{answer} converges conditionally
\end{answer}\end{exercise}

\begin{exercise} $\ds\sum_{n=1}^\infty (-1)^{n-1} {\ln n\over n^3}$
\begin{answer} converges absolutely
\end{answer}\end{exercise}

\begin{exercise} $\ds\sum_{n=2}^\infty (-1)^n{1\over \ln n}$
\begin{answer} converges conditionally
\end{answer}\end{exercise}

\begin{exercise} $\ds\sum_{n=0}^\infty (-1)^{n} {3^n\over 2^n+5^n}$
\begin{answer} converges absolutely
\end{answer}\end{exercise}

\begin{exercise} $\ds\sum_{n=0}^\infty (-1)^{n} {3^n\over 2^n+3^n}$
\begin{answer} diverges
\end{answer}\end{exercise}

\begin{exercise} $\ds\sum_{n=1}^\infty (-1)^{n-1} {\arctan n\over n}$
\begin{answer} converges conditionally
\end{answer}\end{exercise}

\endtwocol

\end{exercises}


