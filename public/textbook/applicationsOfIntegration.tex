\chapter{Applications of Integration}


\section{Volume}

We have seen how to compute certain areas by using integration. We can
do more, some volumes may also be computed by evaluating an
integral. 


\subsection{The Slab Method}

Generally, the volumes that we can compute this way have
cross-sections that are easy to describe. Sometimes we think of these
cross-sections as being ``slabs'' that we are layering to create a
volume.


\begin{example}
Find the volume of a pyramid with a square base that is 20 meters tall
and 20 meters on a side at the base. 
\end{example}

\begin{marginfigure}
%pyramid
\begin{tikzpicture}
 \begin{axis}[view={30}{30},colormap/\surfaceColor,
   xlabel=$x$, ylabel=$z$, zlabel=$y$,]

  \addplot3[surf,shader=faceted,opacity=.3,
  samples=12,
  samples y = 6,
  domain=-10:10,y domain=0:.5,
  z buffer=sort]
  ({10*(1-y)},{(1-y)*(x)}, {y*20});

  \addplot3[surf,shader=faceted,opacity=.3,
  samples=12,
  samples y = 6,
  domain=-10:10,y domain=0:.5,
  z buffer=sort]
  ({x*(1-y)},{(1-y)*(10)}, {y*20}); 

  \addplot3[surf,shader=faceted,opacity=.7, %front
  samples=12,
  samples y = 6,
  domain=-10:10,y domain=0:.5,
  z buffer=sort]
  ({x*(1-y)},{(1-y)*(-10)}, {y*20});
 
 \addplot3[surf,shader=faceted,opacity=.7, %front
  samples=12,
  samples y = 6,
  domain=-10:10,y domain=0:.5,
  z buffer=sort]
  ({10*(1-y)},{(1-y)*(x)}, {y*20});



  %square
  \addplot3[surf,colormap/\sliceColor,shader=flat,
  samples=4,
  samples y = 4,
  domain=-1:1,y domain=-1:1,
  z buffer=sort]
  ({10*(.5)*y},{(.5)*(x)*10}, {.5*20});


  \addplot3[surf,shader=faceted,opacity=.3,
  samples=12,
  samples y = 6,
  domain=-10:10,y domain=.5:1,
  z buffer=sort]
  ({10*(1-y)},{(1-y)*(x)}, {y*20});

  \addplot3[surf,shader=faceted,opacity=.3,
  samples=12,
  samples y = 6,
  domain=-10:10,y domain=.5:1,
  z buffer=sort]
  ({x*(1-y)},{(1-y)*(10)}, {y*20}); 

  \addplot3[surf,shader=faceted,opacity=.7, %front
  samples=12,
  samples y = 6,
  domain=-10:10,y domain=.5:1,
  z buffer=sort]
  ({x*(1-y)},{(1-y)*(-10)}, {y*20});
 
 \addplot3[surf,shader=faceted,opacity=.7, %front
  samples=12,
  samples y = 6,
  domain=-10:10,y domain=.5:1,
  z buffer=sort]
  ({10*(1-y)},{(1-y)*(x)}, {y*20});
 \end{axis}
\end{tikzpicture}
\caption{A pyramid with a 20 meter square base. Here we see the thin
  slab used to generate the volume.}
\label{fig:pyramid}
\end{marginfigure}


\begin{solution}
As with most of our applications of integration, we begin by asking
how we might approximate the volume. Since we can easily compute the
volume of a box, we will use some ``thin'' boxes to approximate
the volume of the pyramid, as shown in Figure~\ref{fig:pyramid}.

Centering our pyramid at the origin, each box has volume of the form
\[
\text{width}\cdot\text{length}\cdot\text{height} = (2x)(2x)h
\]
where $h$ is understood to be a value close to zero. Write $x$ in
terms of $y$, $x=10-y/2$. If we really were adding together many small
rectangles, we might say $x_i=10-y_i/2$. In this case the total volume
is approximately
\[
\sum_{i=0}^{n-1} 4(10-y_i/2)^2 h.
\]
This is a Riemann sum! If we take the limit as the number of slabs
goes to infinity and the thickness of these slabs goes to zero, we
obtain the following integral:
\begin{align*}
  \int_0^{20} 4(10-y/2)^2\d y &=\int_0^{20} (20-y)^2\d y\\
  &=\left.-{(20-y)^3\over3}\right|_0^{20}\\
  &=-{0^3\over3}-\left(-{20^3\over3}\right)={8000\over3}.
\end{align*}
As you may know, the volume of a pyramid is 
$(1/3)(\hbox{height})(\hbox{area of base})=(1/3)(20)(400)$, which
agrees with our answer.
\end{solution}



\begin{marginfigure}[-2.5in]
\begin{tikzpicture}
  \begin{axis}[
      xmin=-1, xmax=1,domain=-1:1,
      clip=false,
      axis lines =center, xlabel=$x$, ylabel=$z$,
      every axis y label/.style={at=(current axis.above origin),anchor=south},
      every axis x label/.style={at=(current axis.right of origin),anchor=west},
      axis on top,
    ] 
    \addplot [penColor2,very thick] {x^2-1};
    \addplot [penColor,very thick] {-x^2+1};
  \end{axis}
\end{tikzpicture}
\caption{A plot of $f(x) = x^2-1$ and $g(x) = -x^2+1$.}
\label{figure:regionApI}
\end{marginfigure}


\begin{example}
The base of a solid is the region between $f(x)=x^2-1$ and
$g(x)=-x^2+1$, see Figure~\ref{figure:regionApI}. Its cross-sections
perpendicular to the $x$-axis are equilateral triangles. See
Figure~\ref{fig:triangular cross-sections}. Find the volume of the
solid.
\end{example}

\begin{marginfigure}
% triangleRegion
\begin{tikzpicture}
 \begin{axis}[view={30}{30},colormap/\surfaceColor,
   xlabel=$x$, ylabel=$z$, zlabel=$y$,]

  \addplot3[surf,shader=faceted,opacity=.3,
  samples=12,
  samples y = 6,
  domain=-1:.5,y domain=0:1,
  z buffer=sort]
  (x,{(1-y)*(-x^2+1)}, {y*sqrt(3)*(1-x^2)});

  \addplot3[surf,shader=faceted,opacity=.7,%front
  samples=12,
  samples y = 6,
  domain=-1:.5,y domain=0:1,
  z buffer=sort]
  (x,{(1-y)*(x^2-1)}, {y*sqrt(3)*(1-x^2)});

  %tri
  \addplot3[surf,colormap/\sliceColor,shader=flat,
  samples=6,
  samples y = 6,
  domain=-1:1,y domain=0:1,
  z buffer=sort]
  (.5,{x*(1-y)*(.5^2-1)}, {abs(x)*y*sqrt(3)*(1-.5^2)});

  \addplot3[surf,shader=faceted,opacity=.3, 
  samples=4,
  samples y = 6,
  domain=.5:1,y domain=0:1,
  z buffer=sort]
  (x,{(1-y)*(-x^2+1)}, {y*sqrt(3)*(1-x^2)});

  \addplot3[surf,shader=faceted,opacity=.7,%front
  samples=4,
  samples y = 6,
  domain=.5:1,y domain=0:1,
  z buffer=sort]
  (x,{(1-y)*(x^2-1)}, {y*sqrt(3)*(1-x^2)});

 \end{axis}
\end{tikzpicture}
\caption{A solid with equilateral triangle cross-sections bounded by
  the region between $f(x)=x^2-1$ and $g(x)=-x^2+1$.}
\label{fig:triangular cross-sections}
\end{marginfigure}



\begin{solution}
For any value of $x$, a cross-section is a triangle with base
$2(1-x^2)$ and height $\sqrt3(1-x^2)$, so the area of the
cross-section is
\[
  {1\over2}(\hbox{base})(\hbox{height})=
  (1-x^2)\sqrt3(1-x^2),
\]
Thus the total volume is
\[
\int_{-1}^1 \sqrt3(1-x^2)^2\d x={16\over15}\sqrt3.
\]
\end{solution}

One easy way to get ``nice'' cross-sections is by rotating a plane
figure around a line. For example, in Figure~\ref{figure:regionCurve}
we see $f(x)$ bounded by two vertical lines.  Rotating $f(x)$ around the
$x$-axis will generate a figure whose volume we can compute.








%disk
\begin{tikzpicture}
 \begin{axis}[view={30}{30},colormap/\surfaceColor,
   xlabel=$x$, ylabel=$z$, zlabel=$y$,
   ]

  \addplot3[surf,shader=faceted,opacity=.3,
  samples=10,
  samples y =20,
  domain=1:2.7,y domain=0:2*pi,
  z buffer=sort]
  (x,{(16-19*x+8*x^2-x^3) * cos(deg(y))}, {(16-19*x+8*x^2-x^3) * sin(deg(y))});

  \addplot3[surf,colormap/\sliceColor,shader=flat,%disk
  samples=5,
  samples y = 20,
  domain=0:1,y domain=0:2*pi,
  z buffer=sort
]
  (2.7,{3.337*x*cos(deg(y))}, {3.337*x*sin(deg(y))});


  \addplot3[surf,shader=faceted,opacity=.3,
  samples=10,
  samples y=20,
  domain=2.7:4,y domain=0:2*pi,
  z buffer=sort]
  (x,{(16-19*x+8*x^2-x^3) * cos(deg(y))}, {(16-19*x+8*x^2-x^3) * sin(deg(y))});


  \addplot3[surf,shader=faceted,opacity=.7,
  samples=5,
  samples y = 20,
  domain=0:1,y domain=0:2*pi,
  z buffer=sort
]
  (4,{4*x*cos(deg(y))}, {4*x*sin(deg(y))});

 \end{axis}
\end{tikzpicture}


\begin{marginfigure}[0in]
\begin{tikzpicture}
  \begin{axis}[
      xmin=0, xmax=5,domain=0:5, clip=false, axis lines =center,
      xlabel=$x$, ylabel=$z$, every axis y label/.style={at=(current
        axis.above origin),anchor=south}, every axis x
      label/.style={at=(current axis.right of origin),anchor=west},
      axis on top, ] \addplot [penColor,very thick,smooth]
    {16-19*x+8*x^2-x^3}; \addplot [textColor,dashed] plot coordinates
    {(1,0) (1,4)}; \addplot [textColor,dashed] plot coordinates {(4,0)
      (4,4)};
  \end{axis}
\end{tikzpicture}
\caption{A plot of $f(x)$.}
\label{figure:regionCurve}
\end{marginfigure}


The volume of each disk will have the form $\pi r^2 h$. As long as we
can write $r$ in terms of $x$ we can compute the volume by an
integral.




\begin{example}
Find the volume of a right circular cone with base radius 10 and
height 20. Here, a right circular cone is one with a circular base and
with the tip of the cone directly over the center of the base.
\end{example}

\begin{marginfigure}
%cone
\begin{tikzpicture}
 \begin{axis}[
     view={30}{30},colormap/\surfaceColor,
     xlabel=$x$, ylabel=$z$, zlabel=$y$,
   ]

  \addplot3[surf,shader=faceted,opacity=.3,
  samples=10,
  samples y =20,
  domain=0:12,y domain=0:2*pi,
  z buffer=sort]
  (x,{(x/2) * cos(deg(y))}, {(x/2) * sin(deg(y))});

  \addplot3[surf,colormap/\sliceColor,shader=flat,%disk
  samples=5,
  samples y = 20,
  domain=0:1,y domain=0:2*pi,
  z buffer=sort
]
  (12,{6*x*cos(deg(y))}, {6*x*sin(deg(y))});


  \addplot3[surf,shader=faceted,opacity=.3,
  samples=10,
  samples y=20,
  domain=12:20,y domain=0:2*pi,
  z buffer=sort]
  (x,{(x/2) * cos(deg(y))}, {(x/2) * sin(deg(y))});


  \addplot3[surf,shader=faceted,opacity=.7,
  samples=5,
  samples y = 20,
  domain=0:1,y domain=0:2*pi,
  z buffer=sort
]
  (20,{10*x*cos(deg(y))}, {10*x*sin(deg(y))});

 \end{axis}
\end{tikzpicture}
\caption{A right circular cone with base radius 10 and height 20.}
\label{fig:line to cone}
\end{marginfigure}


\begin{solution}
We can view this cone as produced by the rotation of the line
$y=x/2$ rotated about the $x$-axis, as indicated in
Figure~\ref{fig:line to cone}.

At a particular point on the $x$-axis, the radius of the resulting
cone is the $y$-coordinate of the corresponding point on the line
$y=x/2$. The area of the cross section is given by
\[
\pi \cdot \text{radius}^2 = \pi \left(\frac{x}{2}\right)^2
\]
so the volume is given by
$$
  \int_0^{20} \pi
  {x^2\over4}\d x={\pi\over4}{20^3\over3}={2000\pi\over3}.
$$ 
Note that we can instead do the calculation with a generic height and
radius: 
$$
  \int_0^{h} \pi{r^2\over h^2}x^2\d x
  ={\pi r^2\over h^2}{h^3\over3}={\pi r^2h\over3},
$$ 
giving us the usual formula for the volume of a cone.
\end{solution}








\subsection{The Washer Method}

Sometimes the ``slabs'' look like disks with holes in them, or
``washers.'' Let's see an example of this.


\begin{marginfigure}[0in]
\begin{tikzpicture}
  \begin{axis}[
      xmin=0, xmax=1,domain=0:1,
      clip=false,
      axis lines =center, xlabel=$x$, ylabel=$y$,
      every axis y label/.style={at=(current axis.above origin),anchor=south},
      every axis x label/.style={at=(current axis.right of origin),anchor=west},
      axis on top,
    ] 
    \addplot [draw=none,fill=fill1,very thick] {x} \closedcycle;
    \addplot [draw=none, fill=background,very thick] {x^2}\closedcycle;

    \addplot [penColor,very thick] {x};
    \addplot [penColor2,very thick] {x^2};

    \node at (axis cs:.5,.58) [penColor] {$f(x)$};
    \node at (axis cs:.5,.2) [penColor2] {$g(x)$};
    \end{axis}
\end{tikzpicture}
\caption{A plot of $f(x) = x$ and $g(x) = x^2$.}
\label{figure:regionHole}
\end{marginfigure}

\begin{example}
Find the volume of the object generated when the area between $f(x) =
x$ and $g(x)=x^2$ is rotated around the $x$-axis, see
Figure~\ref{figure:regionHole}.
\end{example}

\begin{marginfigure}
%horn
\begin{tikzpicture}
 \begin{axis}[
     view={30}{30},colormap/\surfaceColor,
     xlabel=$x$, ylabel=$z$, zlabel=$y$,
   ]

  \addplot3[surf,shader=faceted,opacity=.7,
  samples=10,
  samples y =20,
  domain=0:.5,y domain=0:2*pi,
  z buffer=sort]
  (x,{(x) * cos(deg(y))}, {(x) * sin(deg(y))});

  \addplot3[surf,colormap/\surfaceColorTwo,shader=faceted,opacity=.1, %inside
  samples=10,
  samples y =20,
  domain=0:.5,y domain=0:2*pi,
  z buffer=sort]
  (x,{(x^2) * cos(deg(y))}, {(x^2) * sin(deg(y))});

  \addplot3[surf,colormap/\sliceColor,shader=flat,%disk
  samples=5,
  samples y = 20,
  domain=.25:.5,y domain=0:2*pi,
  z buffer=sort
]
  (.5,{x*cos(deg(y))}, {x*sin(deg(y))});


  \addplot3[surf,shader=faceted,opacity=.7,
  samples=10,
  samples y=20,
  domain=.5:1,y domain=0:2*pi,
  z buffer=sort]
  (x,{(x) * cos(deg(y))}, {(x) * sin(deg(y))});

  \addplot3[surf,colormap/\surfaceColorTwo,shader=faceted,opacity=.1,
  samples=10,
  samples y=20,
  domain=.5:1,y domain=0:2*pi,
  z buffer=sort]
  (x,{(x^2) * cos(deg(y))}, {(x^2) * sin(deg(y))});

 \end{axis}
\end{tikzpicture}
\caption{A solid generated by revolving $f(x) = x$ around the $x$-axis
  and then removing the volume generated by revolving $g(x) = x^2$
  around the $x$-axis.}
\label{fig:solid with hole}
\end{marginfigure}


\begin{solution}
This solid has a ``hole'' in the middle. We can compute the volume by
subtracting the volume of the hole from the volume enclosed by the
outer surface of the solid. In Figure~\ref{fig:solid with hole} we
show the region that is rotated, the resulting solid with the front
half cut away, the cone that forms the outer surface, the horn-shaped
hole, and a cross-section perpendicular to the $x$-axis.

We can compute the desired volume ``all at once'' by approximating the
volume of the actual solid. We can approximate the volume of a slice
of the solid with a washer-shaped volume, as indicated in
Figure~\ref{fig:solid with hole}. The area of the face is the area of
the outer circle minus the area of the inner circle, say
\[
\pi R^2-\pi r^2.
\] 
In the present example, we have
\[
\pi x^2 - \pi x^4. 
\]
Hence, the whole volume is
\begin{align*}
  \int_0^1 \pi x^2-\pi x^4\d x &= \left.\pi\left({x^3\over3}-{x^5\over5}\right)\right|_0^1\\
&=\pi\left({1\over3}-{1\over5}\right)={2\pi\over15}.
\end{align*}
\end{solution}

\break

\subsection{The Shell Method}

\begin{marginfigure}[0in]
\begin{tikzpicture}
  \begin{axis}[
      xmin=0, xmax=3,domain=0:3,
      clip=false,
      axis lines =center, xlabel=$x$, ylabel=$y$,
      every axis y label/.style={at=(current axis.above origin),anchor=south},
      every axis x label/.style={at=(current axis.right of origin),anchor=west},
      axis on top,
    ] 
    \addplot [penColor2,very thick] {x+1};
    \addplot [penColor,very thick] {(x-1)^2};

    \node at (axis cs:.5,1.7) [penColor2] {$f(x)$};
    \node at (axis cs:2.5,1.7) [penColor] {$g(x)$};
    \addplot [very thick, penColor4] plot coordinates {(.25,.56) (1.75,.56)};
    \addplot [very thick, penColor4] plot coordinates {(1,2) (2.41,2)};
    \end{axis}
\end{tikzpicture}
\caption{A plot of $f(x) = x+1$ and $g(x) = (x-1)^2$ with the two
  types of ``washers'' indicated.}
\label{figure:washerHARD}
\end{marginfigure}

Suppose the region between $f(x)=x+1$ and $g(x)=(x-1)^2$ is rotated
around the $y$-axis. It is possible, but inconvenient, to compute the
volume of the resulting solid by the method we have used so far. The
problem is that there are two ``kinds'' of typical rectangles: those
that go from the line to the parabola and those that touch the
parabola on both ends, see Figure~\ref{figure:washerHARD}. To compute
the volume using this approach, we need to break the problem into two
parts and compute two integrals:
$$
  \pi\int_0^1 (1+\sqrt{y})^2-(1-\sqrt{y})^2\d y+
  \pi\int_1^4  (1+\sqrt{y})^2-(y-1)^2\d y={8\over3}\pi + {65\over6}\pi
  ={27\over2}\pi.
$$

\begin{marginfigure}[0in]
\begin{tikzpicture}
  \begin{axis}[
      xmin=0, xmax=3,domain=0:3,
      clip=false,
      axis lines =center, xlabel=$x$, ylabel=$y$,
      every axis y label/.style={at=(current axis.above origin),anchor=south},
      every axis x label/.style={at=(current axis.right of origin),anchor=west},
      axis on top,
    ] 
    \addplot [penColor2,very thick] {x+1};
    \addplot [penColor,very thick] {(x-1)^2};

    \node at (axis cs:.5,1.7) [penColor2] {$f(x)$};
    \node at (axis cs:2.5,1.7) [penColor] {$g(x)$};
    \addplot [very thick, penColor4] plot coordinates {(1.5,2.5) (1.5,.25)};
  \end{axis}
\end{tikzpicture}
\caption{A plot of $f(x) = x+1$ and $g(x) = (x-1)^2$ with the
  ``shell'' indicated.}
\label{figure:shellIndicated}
\end{marginfigure}


If instead we consider a typical vertical rectangle, but still rotate
around the $y$-axis, we get a thin ``shell'' instead of a thin
``washer,'' see Figure~\ref{figure:shellIndicated}. If we add up the
volume of such thin shells we will get an approximation to the true
volume. 


\begin{tikzpicture}
 \begin{axis}[
     view={30}{30},colormap/\surfaceColor,
     xlabel=$x$, ylabel=$z$, zlabel=$y$,
   ]

  \addplot3[surf,colormap/\surfaceColorTwo,shader=faceted,opacity=.7, %inside
  samples=10,
  samples y =20,
  domain=0:3,y domain=0:2*pi,
  z buffer=sort]
  (x* cos(deg(y)), {(x) * sin(deg(y))},{x+1});

  \addplot3[surf,colormap/\sliceColor,shader=interp,%shell
  samples=5,
  samples y = 20,
  domain=.25:2.5,y domain=0:2*pi,
  z buffer=sort
]
  ({1.5*cos(deg(y))}, {1.5*sin(deg(y))},x);


  \addplot3[surf,shader=faceted,opacity=.3, %outside
  samples=10,
  samples y =20,
  domain=0:3,y domain=0:2*pi,
  z buffer=sort]
  (x*cos(deg(y)), {(x) * sin(deg(y))},{(x-1)^2});


 \end{axis}
\end{tikzpicture}




What is the volume of such a shell?  Consider the shell at
$x$.  Imagine that we cut the shell vertically in one place and
``unroll'' it into a thin, flat sheet. This sheet will be $f(x)-g(x)$
tall, and $2\pi x$ wide namely, the circumference of the shell before
it was unrolled.  We may now write the integral
$$
  \int_0^3 2\pi x(f(x)-g(x))\d x=
  \int_0^3 2\pi x(x+1-(x-1)^2)\d x={27\over2}\pi.
$$
Not only does this accomplish the task with only one integral, the
integral is somewhat easier than those in the previous
calculation. Things are not always so neat, but it is often the case
that one of the two methods will be simpler than the other, so it is
worth considering both before starting to do calculations.

\begin{example} 
Suppose the area under $y=-x^2+1$ between $x=0$ and $x=1$ is rotated
around the $x$-axis.
\end{example}

\begin{solution}
We'll just set up integrals for each method.

Disk method: $\int_0^1 \pi(1-x^2)^2\d x={8\over15}\pi$.


Shell method: $\int_0^1 2\pi y \sqrt{1-y}\d y={8\over15}\pi$.
\end{solution}

\begin{exercises}

%% \begin{exercise}
%% Verify that $\pi\int_0^1 (1+\sqrt{y})^2-(1-\sqrt{y})^2\d y+
%% \pi\int_1^4  (1+\sqrt{y})^2-(y-1)^2={8\over3}\pi + {65\over6}\pi
%% ={27\over2}\pi$.
%% \end{exercise}

%% \begin{exercise} Verify that $\int_0^3 2\pi x(x+1-(x-1)^2)\d x={27\over2}\pi$.
%% \end{exercise}

%% \begin{exercise} Verify that $\int_0^1 \pi(1-x^2)^2\d x={8\over15}\pi$.
%% \end{exercise}

%% \begin{exercise} Verify that $\int_0^1 2\pi y \sqrt{1-y}\d y={8\over15}\pi$.
%% \end{exercise}

\begin{exercise}
Use integration to find the volume of the solid obtained by revolving 
the region bounded by $x+y=2$ and the $x$ and $y$ axes around the
$x$-axis. 
\begin{answer} $8\pi/3$
\end{answer}\end{exercise}

\begin{exercise}
Find the volume of the solid obtained by revolving 
the region bounded by $y=x-x^2$
and the $x$-axis around the
$x$-axis. 
\begin{answer} $\pi/30$
\end{answer}\end{exercise}

\begin{exercise}
Find the volume of the solid obtained by revolving 
the region bounded by $y=\sqrt{\sin x}$ between $x=0$ and
$x=\pi/2$, the $y$-axis, and the line
$y=1$ around the
$x$-axis. 
\begin{answer} $\pi(\pi/2-1)$
\end{answer}\end{exercise}

\begin{exercise}
Let $S$ be the region of the $xy$-plane bounded above by the curve
$x^3y=64$, below by the line $y=1$, on the left by  the line $x=2$, and
on the right by the line $x=4$.  Find
the volume of the solid obtained by rotating $S$ around (a) the $x$-axis,
(b) the line $y=1$, (c) the $y$-axis, (d) the line $x=2$.
\begin{answer} (a) $114\pi/5$ (b) $74\pi/5$ (c) $20\pi$\hfill\break
(d) $4\pi$
\end{answer}\end{exercise}

\begin{exercise} The equation $x^2/9+y^2/4=1$ describes an ellipse.  Find the
volume of the solid obtained by rotating the ellipse around the
$x$-axis and also around the $y$-axis. These solids are
called {\dfont ellipsoids\/}; one is vaguely rugby-ball shaped, one is
sort of flying-saucer shaped, or perhaps squished-beach-ball-shaped.
\begin{answer} $16\pi$, $24\pi$
\end{answer}\end{exercise}

%% AN IMAGE WAS HERE!!!!


\begin{exercise} Use integration to compute the volume of a sphere of radius
$r$. 
\begin{answer}
$4\pi r^3/3$
\end{answer}
\end{exercise}

\begin{exercise}
A hemispheric bowl of radius $r$ contains water to a depth $h$.  Find
the volume of water in the bowl.
\begin{answer} $\pi h^2(3r-h)/3$
\end{answer}\end{exercise}

\begin{exercise} The base of a tetrahedron (a triangular pyramid) of height $h$
is an equilateral triangle of side $s$.  Its cross-sections
perpendicular to an altitude are equilateral triangles.  Express its
volume $V$ as an integral, and find a formula for $V$ in terms of $h$
and $s$. 
\begin{answer}
$(1/3)(\hbox{area of base})(\hbox{height})$
\end{answer}
\end{exercise}

%% fixme: include picture? see exercise_9.3.13.mw
\begin{exercise}
The base of a solid is the region between $f(x)=\cos x$ and
$g(x)=-\cos x$, $-\pi/2\le x\le\pi/2$,
and its cross-sections perpendicular to the $x$-axis 
are squares.
Find the volume of the solid.
\begin{answer} $2\pi$
\end{answer}\end{exercise}

\end{exercises}

















\section{Arc Length}

Here is another geometric application of the integral, finding the
length of a portion of a curve. As usual, we need to think about how
we might approximate the length, and turn the approximation into an
integral.

\begin{marginfigure}[0in]
\begin{tikzpicture}
  \begin{axis}[
      xmin=0, xmax=7, ymin=0, ymax=5,
      clip=false,
      ticks=none,
      axis lines =center, xlabel=$x$, ylabel=$y$,
      every axis y label/.style={at=(current axis.above origin),anchor=south},
      every axis x label/.style={at=(current axis.right of origin),anchor=west},
      axis on top,
    ] 
    \addplot [very thick, penColor] plot coordinates {(3,2) (6,4)};
    \addplot [dashed, textColor] plot coordinates {(3,2) (6,2)};
    \addplot [dashed, textColor] plot coordinates {(6,4) (6,2)};
    \addplot[color=penColor,fill=penColor,only marks,mark=*] coordinates{(3,2)};  %% closed hole
    \addplot[color=penColor,fill=penColor,only marks,mark=*] coordinates{(6,4)};  %% closed hole
    \node at (axis cs:3,2) [anchor=east] {$(x_0,y_0)$};
    \node at (axis cs:6,4) [anchor=west] {$(x_1,y_1)$};
    \node at (axis cs:4.5,3.2) [penColor,anchor=east] {$\sqrt{(x_1-x_0)^2+(y_1-y_0)^2}$};
  \end{axis}
\end{tikzpicture}
\caption{The length of a line segment.}
\label{fig:length of a line segment}
\end{marginfigure}


We already know how to compute one simple arc length, that of a line
segment. If the endpoints are $(x_0,y_0)$ and $(x_1,y_1)$ then the
length of the segment is the distance between the points,
$\sqrt{(x_1-x_0)^2+(y_1-y_0)^2}$, see Figure~\ref{fig:length of a line segment}.

Now if $f(x)$ is ``nice'' (say, differentiable) it appears that we can
approximate the length of a portion of the curve with line segments,
and that as the number of segments increases, and their lengths
decrease, the sum of the lengths of the line segments will approach
the true arc length, see Figure~\ref{fig:approximating arc length}.

\begin{marginfigure}[0in]
\begin{tikzpicture}
	\begin{axis}[
            domain=-3:2,
            ymax=5,
            ymin=-5,
            %samples=100,
            axis lines =middle, xlabel=$x$, ylabel=$y$,
            every axis y label/.style={at=(current axis.above origin),anchor=south},
            every axis x label/.style={at=(current axis.right of origin),anchor=west}
          ]
          \addplot [dashed, textColor, smooth] {-(x^4)/4 - (x^3)/3 +x^2};
          \addplot [very thick, penColor] plot coordinates {
            (-3,-9/4) (-2,8/3) (-1,13/12) (0,0) (1,5/12) (2,-8/3)
          };
          \addplot[color=penColor,fill=penColor,only marks,mark=*] coordinates{(-3,-9/4) (-2,8/3) (-1,13/12) (0,0) (1,5/12) (2,-8/3)};  %% closed hole

          \node at (axis cs:-1.3,2) [anchor=west,textColor] {$f(x)$};  
        \end{axis}
\end{tikzpicture}
\caption{Approximating the arc length of the curve defined by $f(x)$.}
\label{fig:approximating arc length}
\end{marginfigure}

Now we need to write a formula for the sum of the lengths of the line
segments, in a form that we know becomes an integral in the limit.  So
we suppose we have divided the interval $[a,b]$ into $n$ subintervals
as usual, each with length $h=(b-a)/n$, and endpoints 
\[
a=x_0, x_1, x_2, \dots, x_n=b.
\]
The length of a typical line segment, joining $(x_i,f(x_i))$ to $
(x_{i+1},f(x_{i+1}))$, is 
\[
\sqrt{h^2 +(f(x_{i+1})-f(x_i))^2}.
\]
By the Mean Value Theorem, Theorem~\ref{thm:mvt}, there is a number $c_i$
in $(x_i,x_{i+1})$ such that 
\[
f'(c_i)=\frac{f(x_{i+1})-f(x_i)}{x_{i+1}-x_i} = \frac{f(x_{i+1})-f(x_i)}{h}.
\]
so $f'(c_i)h=f(x_{i+1})-f(x_i).$ Hence, the length of the line segment
can be written as
$$
  \sqrt{h^2 + (f'(c_i))^2h^2}=
  h\sqrt{1+(f'(c_i))^2}.
$$ 
The arc length is then
\[
\lim_{n\to\infty}\sum_{i=0}^{n-1} h\sqrt{1+(f'(c_i))^2} 
\]
This is a Riemann sum! Now we may take the limit as the number of
$x_i$'s chosen goes to infinity and $h$ goes to zero to obtain the
integral
\[
  \int_a^b \sqrt{1+(f'(x))^2}\d x.
\]
Note that the sum looks a bit different than others we have
encountered, because the approximation contains a $c_i$ instead of an
$x_i$. In the past we have always used left endpoints (namely, $x_i$)
to get a representative value of $f$ on $[x_i,x_{i+1}]$; now we are
using a different point, but the principle is the same.

To summarize, to compute the length of a curve on the interval
$[a,b]$, we compute the integral
$$\int_a^b \sqrt{1+(f'(x))^2}\d x.$$ 
Unfortunately, integrals of this form are typically difficult or
impossible to compute exactly, because usually none of our methods for
finding antiderivatives will work. In practice this means that the
integral will usually have to be approximated.


\begin{example} Let $f(x) = \sqrt{r^2-x^2}$, the upper half circle of radius
$r$. The length of this curve is half the circumference, namely $\pi
r$. Let's compute this with the arc length formula.
The derivative $f'$ is $-x/\sqrt{r^2-x^2}$ so the integral is
$$
  \int_{-r}^r \sqrt{1+{x^2\over r^2-x^2}}\d x
  =\int_{-r}^r \sqrt{r^2\over r^2-x^2}\d x
  =r\int_{-r}^r \sqrt{1\over r^2-x^2}\d x.
$$
Using a trigonometric substitution, we find the antiderivative, namely
$\arcsin(x/r)$. Notice that the integral is improper at both
endpoints, as the function $\sqrt{1/(r^2-x^2)}$ is undefined when
$x=\pm r$. So we need to compute
$$
  \lim_{D\to-r^+}\int_D^0  \sqrt{1\over r^2-x^2}\d x +
  \lim_{D\to r^-}\int_0^D  \sqrt{1\over r^2-x^2}\d x.
$$
This is not difficult, and has value $\pi$, so the original integral,
with the extra $r$ in front, has value $\pi r$ as expected.
\end{example}

\begin{exercises}

\begin{exercise} Find the arc length of $f(x)=x^{3/2}$ on $[0,2]$.
\begin{answer} $(22\sqrt{22}-8)/27$
\end{answer}\end{exercise}

\begin{exercise} Find the arc length of $f(x) = x^2/8-\ln x$
on $[1,2]$.
\begin{answer} $\ln(2)+3/8$
\end{answer}\end{exercise}

\begin{exercise}
Find the arc length of $f(x) = (1/3)(x^2 +2)^{3/2}$
on the interval $[0,a]$.
\begin{answer} $a+a^3/3$
\end{answer}\end{exercise}

\begin{exercise} Find the arc length of $f(x)=\ln(\sin x)$ on the
interval $[\pi/4,\pi/3]$.
\begin{answer} $\ln((\sqrt2+1)/\sqrt3)$
\end{answer}\end{exercise}


\begin{exercise} Set up the integral to find the arc length of $\sin x$ 
on the interval $[0,\pi]$; do not evaluate the integral. If you have
access to appropriate software, approximate the value of the integral.
\begin{answer} $\approx 3.82$
\end{answer}\end{exercise}

\begin{exercise} Set up the integral to find the arc length of $y=x e^{-x}$
on the interval $[2,3]$; do not evaluate the integral. If you have
access to appropriate software, approximate the value of the integral.
\begin{answer} $\approx 1.01$
\end{answer}\end{exercise}

\begin{exercise} Find the arc length of $y=e^x$ on the interval $[0,1]$.
(This can be done exactly; it is a bit tricky and a bit long.)
\begin{answer} $\sqrt{1+e^2}-\sqrt2+
{1\over2}\ln\left({\sqrt{1+e^2}-1\over\sqrt{1+e^2}+1}\right)+
{1\over2}\ln(3+2\sqrt2)$
\end{answer}\end{exercise}

\end{exercises}

