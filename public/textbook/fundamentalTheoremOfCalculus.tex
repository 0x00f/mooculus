\chapter{The Fundamental Theorem of Calculus}

\section{The Fundamental Theorem}



Let's rewrite this slightly: 
\[
  \int_a^x f(t)\d t = F(x)-F(a).
\]
We've replaced the variable $x$ by $t$ and $b$ by $x$. These are just
different names for quantities, so the substitution doesn't change the
meaning. It does make it easier to think of the two sides of the
equation as functions. The expression
\[
  \int_a^x f(t)\d t
\]
is a function: plug in a value for $x$, get out some other value. The
expression $F(x)-F(a)$ is of course also a function, and it has a nice
property: 
\[
\ddx (F(x)-F(a)) = F'(x) = f(x),
\]
since $F(a)$ is a constant and has derivative zero. In other words, by
shifting our point of view slightly, we see that the odd looking
function
\[
  G(x)=\int_a^x f(t)\d t
\]
has a derivative, and that in fact $G'(x)=f(x)$. This is really just a
restatement of the Fundamental Theorem of Calculus, and indeed is
often called the Fundamental Theorem of Calculus. To avoid confusion,
some people call the two versions of the theorem ``The Fundamental
Theorem of Calculus, part I'' and ``The Fundamental
Theorem of Calculus, part II'', although unfortunately there is no
universal agreement as to which is part I and which part II. Since it
really is the same theorem, differently stated, some people simply
call them both ``The Fundamental
Theorem of Calculus.''

\begin{mainTheorem}[Fundamental Theorem of Calculus---Version I]
\index{fundamental theorem of calculus---version 1}
\label{thm:fundamental_theorem_I}
Suppose that $f(x)$ is continuous on the interval $[a,b]$ and let
\[
  F(x)=\int_a^x f(t)\d t.
\]
Then $F'(x)=f(x)$.
\end{proof}

We have not really proved the Fundamental Theorem. In a nutshell, we
gave the following argument to justify it: Suppose we want to know the
value of 
$$
  \int_a^b f(t)\d t = \lim_{n\to\infty}\sum_{i=0}^{n-1} f(t_i)\Delta t.
$$
We can interpret the right hand side as the distance traveled by an
object whose speed is given by $f(t)$. We know another way to compute
the answer to such a problem: find the position of the object by
finding an antiderivative of $f(t)$, then substitute $t=a$ and $t=b$
and subtract to find the distance traveled. This must be the answer to
the original problem as well, even if $f(t)$ does not represent a
speed.

What's wrong with this? In some sense, nothing. As a practical matter
it is a very convincing argument, because our understanding of the
relationship between speed and distance seems to be quite solid. From
the point of view of mathematics, however, it is unsatisfactory to
justify a purely mathematical relationship by appealing to our
understanding of the physical universe, which could, however unlikely
it is in this case, be wrong.

A complete proof is a bit too involved to include here, but we will
indicate how it goes. First, if we can prove the second version of the
Fundamental Theorem, theorem~\xrefn{thm:fundamental_theorem_II}, then
we can prove the first version from that:

\vskip6pt\noindent{\begin{proof}font Proof of 
Theorem~\xrefn{thm:fundamental_theorem_I}.}\kern1pc\bgroup
We know from theorem~\xrefn{thm:fundamental_theorem_II} that 
$$
  G(x)=\int_a^x f(t)\d t
$$
is an antiderivative of $f(x)$, and therefore any antiderivative
$F(x)$ of $f(x)$ is of the form $F(x)=G(x)+k$. Then 
$$\eqalign{
  F(b)-F(a)=G(b)+k-(G(a)+k) &= G(b)-G(a) \\
  &=\int_a^b f(t)\d t-\int_a^a f(t)\d t. \\
}$$
It is not hard to see that $\int_a^a f(t)\d t=0$, so this means that
$$
  F(b)-F(a)=\int_a^b f(t)\d t,
$$
which is exactly what theorem~\xrefn{thm:fundamental_theorem_I} says.
\end{proof}

So the real job is to prove
theorem~\xrefn{thm:fundamental_theorem_II}. We will sketch the proof,
using some facts that we do not prove. First, the following identity
is true of integrals:
$$
  \int_a^b f(t)\d t = \int_a^c f(t)\d t + \int_c^b f(t)\d t.
$$
This can be proved directly from the definition of the integral, that
is, using the limits of sums. It is quite easy to see that it must be
true by thinking of either of the two applications of integrals that
we have seen. It turns out that the identity is true no matter what
$c$ is, but it is easiest to think about the meaning when 
$a\le c\le b$.

First, if $f(t)$ represents a speed, then we know that the three
integrals represent the distance traveled between time $a$ and time $b$;
the distance traveled between time $a$ and time $c$; and 
the distance traveled between time $c$ and time $b$. Clearly the sum of
the latter two is equal to the first of these.

Second, if $f(t)$ represents the height of a curve, the three
integrals represent the area under the curve between $a$ and $b$;
the area under the curve between $a$ and $c$;
and the area under the curve between $c$ and $b$. Again it is clear
from the geometry that the first is equal to the sum of the second and
third. 


\vskip6pt\noindent{\begin{proof}font Proof sketch for
Theorem~\xrefn{thm:fundamental_theorem_II}.}\kern1pc\bgroup
We want to compute $G'(x)$, so we start with the definition of the
derivative in terms of a limit:
$$\eqalign{
  G'(x)&=\lim_{\Delta x\to0}{G(x+\Delta x)-G(x)\over\Delta x} \\
  &=\lim_{\Delta x\to0}{1\over \Delta x}\left(
  \int_a^{x+\Delta x} f(t)\d t - \int_a^x f(t)\d t\right) \\
  &=\lim_{\Delta x\to0}{1\over \Delta x}\left(
  \int_a^{x} f(t)\d t + \int_x^{x+\Delta x} f(t)\d t - 
  \int_a^x f(t)\d t\right) \\
  &=\lim_{\Delta x\to0}{1\over \Delta x}\int_x^{x+\Delta x} f(t)\d t. \\
}$$
Now we need to know something about 
$$
  \int_x^{x+\Delta x} f(t)\d t
$$
when $\Delta x$ is small; in fact, it is very close to 
$\Delta x f(x)$, but we will not prove this. Once again, it is easy to
believe this is true by thinking of our two applications:
The integral 
$$
  \int_x^{x+\Delta x} f(t)\d t
$$
can be interpreted as the distance traveled by an object over a very
short interval of time. Over a sufficiently short period of time, the
speed of the object will not change very much, so the distance
traveled will be approximately the length of time multiplied by the
speed at the beginning of the interval, namely, $\Delta x
f(x)$. Alternately, the integral may be interpreted as the area under
the curve between $x$ and $x+\Delta x$. When $\Delta x$ is very small,
this will be very close to the area of the rectangle with base $\Delta
x$ and height $f(x)$; again this is $\Delta x
f(x)$. If we accept this, we may proceed:
$$
  \lim_{\Delta x\to0}{1\over \Delta x}\int_x^{x+\Delta x} f(t)\d t
  =\lim_{\Delta x\to0}{\Delta x f(x)\over \Delta x}=f(x),
$$
which is what we wanted to show.
\end{proof}

It is still true that we are depending on an interpretation of the
integral to justify the argument, but we have isolated this part of
the argument into two facts that are not too hard to prove. Once the
last reference to interpretation has been removed from the proofs of
these facts, we will have a real proof of the Fundamental Theorem.

Now we know that to solve certain kinds of problems, those that lead
to a sum of a certain form, we ``merely'' find an antiderivative and
substitute two values and subtract. Unfortunately, finding
antiderivatives can be quite difficult. While there are a small number
of rules that allow us to compute the derivative of any common
function, there are no such rules for antiderivatives. There are some
techniques that frequently prove useful, but we will never be able to
reduce the problem to a completely mechanical process.




Because of the close relationship between an integral and an
antiderivative, the integral sign is also used to mean
``antiderivative''. You can tell which is intended by whether the
limits of integration are included:
$$
  \int_1^2 x^2\d x
$$
is an ordinary integral, also called a 
{\dfont definite integral\/}\index{definite integral},
because it has a definite value, namely
$$
  \int_1^2 x^2\d x={2^3\over3}-{1^3\over3}={7\over3}.
$$
We use
$$
  \int x^2\d x
$$
to denote the antiderivative of $x^2$, also called an
{\dfont indefinite integral\/}\index{indefinite integral}.
So this is evaluated as
$$
  \int x^2\d x = {x^3\over 3}+C.
$$
It is customary to include the constant $C$ to indicate that there are
really an infinite number of antiderivatives. We do not need this $C$
to compute definite integrals, but in other circumstances we will need
to remember that the $C$ is there, so it is best to get into the habit
of writing the $C$.



Suppose you wish to know the signed area between the plot of the
function $f(x)$ and the $x$-axis on the interval $[a,b]$. This signed area is represented by
\[
\int_a^b f(x) \d x= \lim_{n\to \infty}\sum_{i=0}^{n-1} f(x_i) (x_{i+1}-x_i).
\]
However, Riemann sums can be difficult to compute. \textit{The Fundamental
Theorem of Calculus} is what we need:

\begin{mainTheorem}[Fundamental Theorem of Calculus---Version II]\index{fundamental theorem of calculus---version 2}
\label{thm:fundamental_theorem_II}
Suppose that $f(x)$ is continuous on the interval $[a,b]$. If $F(x)$
is any antiderivative of $f(x)$, then
\[
  \left.\int_a^b f(x)\d x = F(x) \right|_a^b = F(b)-F(a).
\]
\end{mainTheorem}


Let's see an example of the the fundamental theorem in action.

\begin{example}
BLAH
\end{example}



When we compute a definite integral, we first find an antiderivative
and then substitute. It is convenient to first display the
antiderivative and then do the substitution; we need a notation
indicating that the substitution is yet to be done. A typical solution
would look like this:
$$
  \int_1^2 x^2\d x=\left.{x^3\over 3}\right|_1^2 = 
  {2^3\over3}-{1^3\over3}={7\over3}.
$$
The vertical line with subscript and superscript is used to indicate
the operation ``substitute and subtract'' that is needed to finish the
evaluation. 

\begin{exercises}

Find the antiderivatives of the functions:

\twocol

\begin{exercise} $8\sqrt{x}$
\begin{answer} $(16/3)x^{3/2}+C$
\end{answer}\end{exercise}

\begin{exercise} $3t^2+1$
\begin{answer} $t^3+t+C$
\end{answer}\end{exercise}

\begin{exercise} $4/\sqrt{x}$
\begin{answer} $8\sqrt{x}+C$
\end{answer}\end{exercise}

\begin{exercise} $2/z^2$
\begin{answer} $-2/z+C$
\end{answer}\end{exercise}

\iflatetranscendentals
\else
\begin{exercise} $7s^{-1}$
\begin{answer} $7\ln s+C$
\end{answer}\end{exercise}
\fi

\begin{exercise} $(5x+1)^2$
\begin{answer} $(5x+1)^3/15+C$
\end{answer}\end{exercise}

\begin{exercise} $(x-6)^2$
\begin{answer} $(x-6)^3/3+C$
\end{answer}\end{exercise}

\begin{exercise} $x^{3/2}$
\begin{answer} $2x^{5/2}/5+C$
\end{answer}\end{exercise}

\begin{exercise} ${2\over x\sqrt x}$
\begin{answer} $-4/\sqrt{x}+C$
\end{answer}\end{exercise}

\begin{exercise} $|2t-4|$
\begin{answer} $4t-t^2+C$, $t<2$; $t^2-4t+8+C$, $t\ge 2$
\end{answer}\end{exercise}

\endtwocol

\noindent
Compute the values of the integrals:

\twocol

\begin{exercise} $\int_1^4 t^2+3t\d t$
\begin{answer} $87/2$
\end{answer}\end{exercise}

\begin{exercise} $\int_0^\pi \sin t\d t$
\begin{answer} $2$
\end{answer}\end{exercise}

\iflatetranscendentals
\else
\begin{exercise} $\int_1^{10} {1\over x}\d x$
\begin{answer} $\ln(10)$
\end{answer}\end{exercise}

\begin{exercise} $\int_0^5 e^x\d x$
\begin{answer} $e^5-1$
\end{answer}\end{exercise}
\fi

\begin{exercise} $\int_0^3 x^3\d x$
\begin{answer} $3^4/4$
\end{answer}\end{exercise}

\begin{exercise} $\int_1^2 x^5\d x$
\begin{answer} $2^6/6 -1/6$
\end{answer}\end{exercise}

\endtwocol

\msk
\begin{exercise} Find the derivative of $G(x)=\int_1^x t^2-3t\d t$
\begin{answer} $x^2-3x$
\end{answer}\end{exercise}

\begin{exercise} Find the derivative of $G(x)=\int_1^{x^2} t^2-3t\d t$
\begin{answer} $2x(x^4-3x^2)$
\end{answer}\end{exercise}

\iflatetranscendentals
\else
\begin{exercise} Find the derivative of $G(x)=\int_1^x e^{t^2}\d t$
\begin{answer} $e^{x^2}$
\end{answer}\end{exercise}

\begin{exercise} Find the derivative of $G(x)=\int_1^{x^2} e^{t^2}\d t$
\begin{answer} $2xe^{x^4}$
\end{answer}\end{exercise}
\fi

\begin{exercise} Find the derivative of $G(x)=\int_1^x \tan(t^2)\d t$
\begin{answer} $\tan(x^2)$
\end{answer}\end{exercise}

\begin{exercise} Find the derivative of $G(x)=\int_1^{x^2} \tan(t^2)\d t$
\begin{answer} $2x\tan(x^4)$
\end{answer}\end{exercise}

\end{exercises}





\section{The Accumulation Formula}


\section{Euler's Method}
