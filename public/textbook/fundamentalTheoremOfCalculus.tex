\chapter{The Fundamental Theorem of Calculus}

\section{The Fundamental Theorem}

Let $f(x)$ be continuous on the real numbers and consider
\[
  F(x) = \int_a^x f(t)\d t.
\]
From our previous work we know that $F(x)$ is increasing when $f(x)$
is positive and $F(x)$ is decreasing when $f(x)$ is negative. Moreover,
with careful observation, we can even see that $F(x)$ is concave up
when $f'(x)$ is positive and that $F(x)$ is concave down when $f'(x)$
is negative. Thinking about what we have learned about the
relationship of a function to its first and second derivatives, it is
not too hard to guess that there must be a connection between $F'(x)$
and the function $f(x)$. This is a good guess, check out our next
theorem:


\begin{mainTheorem}[Fundamental Theorem of Calculus---Version I]
\index{fundamental theorem of calculus---version 1}
\label{thm:fundamental_theorem_I}\hfil

\noindent Suppose that $f(x)$ is continuous on the real numbers and let
\[
  F(x)=\int_a^x f(t)\d t.
\]
Then $F'(x)=f(x)$.
\end{mainTheorem}

\begin{proof}
Using the limit definition of the derivative we'll compute $F'(x)$.
Write
\begin{align*}
F'(x) &= \lim_{h\to 0}\frac{F(x+h)-F(x)}{h}\\ 
&=\lim_{h\to 0}\frac{1}{h}\left( \int_a^{x+h} f(t)\d t - \int_a^x f(t)\d t\right).
\end{align*}
Recall that if the limits of integration are swapped, then the sign
of the integral is swapped, so we have
\[
F'(x) =\lim_{h\to 0} \frac{1}{h}\left( \int_a^{x+h} f(t)\d t + \int_x^a f(t)\d t\right)
\]
At this point, we can combine the integrals, as we are just ``connecting'' adjacent signed areas to find
\begin{equation}\label{ftc:eqn1}
F'(x)=\lim_{h\to 0} \frac{1}{h}\int_x^{x+h} f(t)\d t.
\end{equation}
Since $f(x)$ is continuous on the interval $[x,x+h]$, and $h$ is
approaching zero, there is an $\epsilon$ that goes to zero as $h$ goes
to zero such that
\[
f(x)-\epsilon < f(x^*) < f(x) + \epsilon \qquad \text{for all }x^*\in[x,x+h],
\]
see Figure~\ref{F:fun diagram}. This means that 
\[
(f(x) - \epsilon)h < \int_x^{x+h} f(t)\d t < (f(x) + \epsilon)h
\]
Dividing all sides by $h$ we find
\[
f(x) - \epsilon < \frac{1}{h}\int_x^{x+h} f(t)\d t < f(x) + \epsilon.
\]
Comparing this to Equation~\ref{ftc:eqn1}, and taking the limit as $h$
goes to zero (remembering that this also means that $\epsilon$ goes to
zero) we see that $F'(x) = f(x)$.
\end{proof}

\begin{marginfigure}[-6in]
\begin{tikzpicture}
  \begin{axis}[
      xmin=0, xmax=2,ymin=0,ymax=2.3,domain=0:2,
      axis lines =center, xlabel=$x$, ylabel=$y$,
      every axis y label/.style={at=(current axis.above origin),anchor=south},
      every axis x label/.style={at=(current axis.right of origin),anchor=west},
      axis on top,
      xtick={.5,1.3,1.453,1.7}, 
      ytickmin=4, ytickmax=1,
      xticklabels={$a$,$x$,$x^*$,$x+h$}, 
    ] 
    \addplot [draw=none, fill=fill1,domain=1.3:1.7] {1+sin(deg(x))*sin(deg(x^2/1.3))} \closedcycle;
    \addplot [textColor,dashed] plot coordinates {(1.453,0) (1.453,1.99)};
    \addplot [penColor,very thick,smooth] {1+sin(deg(x))*sin(deg(x^2/1.3))};
    
    \node at (axis cs:.9,1.7) [penColor] {$f(x)$};
  \end{axis}
\end{tikzpicture}
\caption{Here we see $f(x)$ along with $a$, $x$, $x^*$ and $x+h$.}
\label{F:fun diagram}
\end{marginfigure}


The Fundamental Theorem of Calculus says that an accumulation function
of $f(x)$ is an antiderivative of $f(x)$.  Because of the close
relationship between an integral and an antiderivative, the integral
sign is also used to mean ``antiderivative.'' You can tell which is
intended by whether the limits of integration are included. Hence
\[
  \int_a^b f(x)\d x
\] 
is a definite integral, because it has a definite value---the signed
area between $f(x)$ and the $x$-axis.  On the other hand, we use
\[
  \int f(x)\d x
\]
to denote the antiderivative of $f(x)$, also called an
\textit{indefinite integral}\index{indefinite integral}.
This is evaluated as
\[
  \int f(x)\d x = F(x)+C.
\]
Where $F'(x) = f(x)$ and the constant $C$ indicates that there are
really an infinite number of antiderivatives. We do not need to add
this $C$ to compute definite integrals, but in other circumstances we
will need to remember that the $C$ is there, so it is best to get into
the habit of writing the $C$.

There is a another common form of the Fundamental Theorem of Calculus:

\marginnote[.4in]{
Here the notation
\[
F(x) \Biggr|_a^b
\]
means that one should evaluate $F(x)$ at $b$ and then subtract from this $F(x)$ evaluated at $a$. Hence
\[
F(x) \Biggr|_a^b = F(b)-F(a).
\]  
}
\begin{mainTheorem}[Fundamental Theorem of Calculus---Version II]\index{fundamental theorem of calculus---version 2}
\label{thm:fundamental_theorem_II}\hfil

\noindent Suppose that $f(x)$ is continuous on the interval $[a,b]$. If $F(x)$
is any antiderivative of $f(x)$, then
\[
  \left.\int_a^b f(x)\d x = F(x) \right|_a^b = F(b)-F(a).
\]
\end{mainTheorem}

\begin{proof}
We know from Theorem~\ref{thm:fundamental_theorem_I} 
\[
  G(x)=\int_a^x f(t)\d t
\]
is an antiderivative of $f(x)$, and therefore any antiderivative
$F(x)$ of $f(x)$ is of the form $F(x)=G(x)+k$. Then 
\begin{align*}
  F(b)-F(a) &=G(b)+k-(G(a)+k) 
  &= G(b)-G(a) \\
  &=\int_a^b f(t)\d t-\int_a^a f(t)\d t.
\end{align*}
It is not hard to see that $\int_a^a f(t)\d t=0$, so this means that
\[
  F(b)-F(a)=\int_a^b f(t)\d t,
\]
which is exactly what Theorem~\ref{thm:fundamental_theorem_II} says.
\end{proof}

From this you should see that the two versions of the Fundamental
Theorem are very closely related. To avoid confusion, some people call
the two versions of the theorem ``The Fundamental Theorem of
Calculus---Version I'' and ``The Fundamental Theorem of
Calculus---Version II'', although unfortunately there is no universal
agreement as to which is ``Version I'' and which ``Version II''. Since
it really is the same theorem, differently stated, people often simply
call them both ``The Fundamental Theorem of Calculus.''

Let's see an example of the fundamental theorem in action.

\begin{example}
Compute
\[
\int_1^2\left(x^9 + \frac{1}{x}\right) \d x
\]
\end{example}

\begin{solution}
Here we start by finding an antiderivative of 
\[
x^9 + \frac{1}{x}.
\]
The correct choice is $\frac{x^{10}}{10} + \ln(x)$, one could verify this by
taking the derivative. Hence
\begin{align*}
\int_1^2\left(x^9 + \frac{1}{x}\right) \d x &= \left(\frac{x^{10}}{10} + \ln(x)\right)\Bigg|_1^2 \\
&= \frac{2^{10}}{10} \ln(2) - \frac{1}{10}.
\end{align*}
\end{solution}


When we compute a definite integral, we first find an antiderivative
and then substitute. It is convenient to first display the
antiderivative and then do the substitution; we need a notation
indicating that the substitution is yet to be done. A typical solution
would look like this:
\[
  \left.\int_1^2 x^2\d x={x^3\over 3}\right|_1^2 = 
  {2^3\over3}-{1^3\over3}={7\over3}.
\]
The vertical line with subscript and superscript is used to indicate
the operation ``substitute and subtract'' that is needed to finish the
evaluation. 

Now we know that to solve certain kinds of problems, those that lead
to a sum of a certain form, we ``merely'' find an antiderivative and
substitute two values and subtract. Unfortunately, finding
antiderivatives can be quite difficult. While there are a small number
of rules that allow us to compute the derivative of any common
function, there are no such rules for antiderivatives. There are some
techniques that frequently prove useful, but we will never be able to
reduce the problem to a completely mechanical process.






\subsection*{Euler's Method}

We have given a proof of the Fundamental Theorem of Calculus,
nevertheless it is good to give intuition as to why it is
true. Consider the following example:

\begin{example}
Suppose that the velocity in meters per second of a ball tossed from a
height of 1 meter is given by
\[
v(t) = -9.8t + 6.
\]
What is the height of the ball after 1 second?
\end{example}

\begin{solution}
Since the derivative of position is velocity, and we want to know the
height (position) after one second, we need to compute
\begin{align*}
\int_0^1 -9.8t + 6 \d t &= (-4.9t^2 + 6t)\Bigg|_0^1\\
&= -4.9 + 6 - 0\\
&= 1.1.
\end{align*}
However, since the ball was tossed at an initial height of $1$ meter,
the ball is at a height of $2.1$ meters.
\end{solution}

We did this example before in
Example~\ref{example:eulersMethodBall}. At that time we used Euler's
Method to give an approximate solution. Recall, the basic idea is to
break the time interval between 0 and 1 seconds into many small
partitions. Then at each step multiply the time duration by the
velocity of the ball. In essence you are computing a Riemann
sum. Hence, Euler's method gives some rational as to why the area
under the curve that gives the velocity should give us the position of
the ball.

What's wrong with this? In some sense, nothing. As a practical matter
it is a very convincing argument, because our understanding of the
relationship between velocity and position seems to be quite
solid. From the point of view of mathematics, however, it is
unsatisfactory to justify a purely mathematical relationship by
appealing to our understanding of the physical universe, which could,
however unlikely it is in this case, be wrong.














\begin{exercises}
\noindent Compute the following definite integrals:

\twocol

\begin{exercise} $\int_1^4 t^2+3t\d t$
\begin{answer} $87/2$
\end{answer}\end{exercise}

\begin{exercise} $\int_0^\pi \sin t\d t$
\begin{answer} $2$
\end{answer}\end{exercise}

\begin{exercise} $\int_1^{10} {1\over x}\d x$
\begin{answer} $\ln(10)$
\end{answer}\end{exercise}

\begin{exercise} $\int_0^5 e^x\d x$
\begin{answer} $e^5-1$
\end{answer}\end{exercise}

\begin{exercise} $\int_0^3 x^3\d x$
\begin{answer} $3^4/4$
\end{answer}\end{exercise}

\begin{exercise} $\int_1^2 x^5\d x$
\begin{answer} $2^6/6 -1/6$
\end{answer}\end{exercise}


\begin{exercise} $\int_1^9 8\sqrt{x}\d x$
\begin{answer} $416/3$
\end{answer}\end{exercise}

\begin{exercise} $\int_1^4 \frac{4}{\sqrt{x}}\d x$
\begin{answer} $8$
\end{answer}\end{exercise}


\begin{exercise} $\int_{-2}^{-1}7x^{-1}\d x$
\begin{answer} $-7\ln(2)$
\end{answer}\end{exercise}


\begin{exercise} $\int_{-2}^3 (5x+1)^2\d x$
\begin{answer} $965/3$
\end{answer}\end{exercise}

\begin{exercise} $\int_{-7}^4(x-6)^2 \d x$
\begin{answer} $2189/3$
\end{answer}\end{exercise}

\begin{exercise} $\int_3^{27}x^{3/2}\d x$
\begin{answer} $4356 \sqrt{3}/5$
\end{answer}\end{exercise}

\begin{exercise} $\int_4^9\frac{2}{x\sqrt x}\d x$
\begin{answer} $2/3$
\end{answer}\end{exercise}

\begin{exercise} $\int_{-4}^1|2x-4|\d x$
\begin{answer} $35$
\end{answer}\end{exercise}

\endtwocol

\begin{exercise} Find the derivative of $F(x)=\int_1^x \left(t^2-3t\right)\d t$
\begin{answer} $x^2-3x$
\end{answer}\end{exercise}

\begin{exercise} Find the derivative of $F(x)=\int_1^{x^2} \left(t^2-3t\right)\d t$
\begin{answer} $2x(x^4-3x^2)$
\end{answer}\end{exercise}

\begin{exercise} Find the derivative of $F(x)=\int_1^x e^{\left(t^2\right)}\d t$
\begin{answer} $e^{\left(x^2\right)}$
\end{answer}\end{exercise}

\begin{exercise} Find the derivative of $F(x)=\int_1^{x^2} e^{\left(t^2\right)}\d t$
\begin{answer} $2xe^{\left(x^4\right)}$
\end{answer}\end{exercise}


\begin{exercise} Find the derivative of $F(x)=\int_1^x \tan(t^2)\d t$
\begin{answer} $\tan(x^2)$
\end{answer}\end{exercise}

\begin{exercise} Find the derivative of $F(x)=\int_1^{x^2} \tan(t^2)\d t$
\begin{answer} $2x\tan(x^4)$
\end{answer}\end{exercise}

\end{exercises}













\section{Area Between Curves}

We have seen how integration can be used to find signed area between a
curve and the $x$-axis. With very little change we can find some areas
between curves. Let's see an example:

\begin{example} Find the area below $f(x)= -x^2+4x+3$ and above
$g(x)=-x^3+7x^2-10x+5$ over the interval $1\le x\le2$. 
\end{example}

\begin{marginfigure}
\begin{tikzpicture}
	\begin{axis}[
            domain=0:3, ymax=14,xmax=3,ymin=0, xmin=0,
            axis lines =left, xlabel=$x$, ylabel=$y$,
            every axis y label/.style={at=(current axis.above origin),anchor=south},
            every axis x label/.style={at=(current axis.right of origin),anchor=west},
            axis on top,
          ]
          \addplot [draw=none,fill=fillp,domain=1:2] {-x^2+4*x+3} \closedcycle;
          \addplot [draw=none,fill=background,domain=1:2] {-x^3 + 7*x^2-10*x+5} \closedcycle;
          \addplot [draw=penColor,very thick] {-x^2+4*x+3};
          \addplot [draw=penColor2,very thick] {-x^3 + 7*x^2-10*x+5};
          \node at (axis cs:1,6.7) [penColor] {$f(x)$};
          \node at (axis cs:2,4) [penColor2] {$g(x)$};
        \end{axis}
\end{tikzpicture}
\caption{The area below $f(x)= -x^2+4x+3$ and above
$g(x)=-x^3+7x^2-10x+5$ over the interval $1\le x\le2$. }
\label{fig:area between curves}
\end{marginfigure}

\begin{solution}
In Figure~\ref{fig:area between curves} we show the two curves
together, with the desired area shaded.

It is clear from the figure that the area we want is the area under
$f(x)$ minus the area under $g(x)$, which is to say
\[
\int_1^2 f(x)\d x-\int_1^2 g(x)\d x = \int_1^2 \left(f(x)-g(x)\right)\d x.
\]
It doesn't matter whether we compute the two integrals on the left and
then subtract or compute the single integral on the right. In this
case, the latter is perhaps a bit easier:
\begin{align*}
  \int_1^2 f(x)-g(x)\d x&=\int_1^2 -x^2+4x+3-(-x^3+7x^2-10x+5)\d x \\
  &=\int_1^2 x^3-8x^2+14x-2\d x \\
  &=\left.{x^4\over4}-{8x^3\over3}+7x^2-2x\right|_1^2 \\
  &={16\over4}-{64\over3}+28-4-({1\over4}-{8\over3}+7-2) \\
  &=23-{56\over3}-{1\over4}={49\over12}.
\end{align*}
\end{solution}

In our first example, one curve was higher than the other over the
entire interval. This does not always happen.


\begin{example} Find the area between $f(x)= -x^2+4x$ and
$g(x)=x^2-6x+5$ over the interval $0\le x\le 1$.
\end{example}

\begin{marginfigure}
\begin{tikzpicture}
	\begin{axis}[
            domain=-1:2, ymax=6,xmax=1.5,ymin=0, xmin=-.5,
            axis lines =center, xlabel=$x$, ylabel=$y$,
            every axis y label/.style={at=(current axis.above origin),anchor=south},
            every axis x label/.style={at=(current axis.right of origin),anchor=west},
            axis on top,
          ]
          \addplot [draw=none,fill=fillp,domain=0:0.56] {x^2-6*x+5} \closedcycle;
          \addplot [draw=none,fill=background,domain=0:0.56] {-x^2+4*x} \closedcycle;
          \addplot [draw=none,fill=fillp,domain=0.56:1] {-x^2+4*x} \closedcycle;
          \addplot [draw=none,fill=background,domain=0.56:1] {x^2-6*x+5} \closedcycle;
          \addplot [draw=penColor,very thick] {-x^2+4*x};
          \addplot [draw=penColor2,very thick] {x^2 - 6*x+5};
          \node at (axis cs:1.25,3.1) [penColor] {$f(x)$};
          \node at (axis cs:.25,4.3) [penColor2] {$g(x)$};
        \end{axis}
\end{tikzpicture}
\caption{The area between $f(x)= -x^2+4x$ and
$g(x)=x^2-6x+5$ over the interval $0\le x\le 1$.}
\label{fig:curves cross}
\end{marginfigure}


\begin{solution}
The curves are shown in Figure~\ref{fig:curves cross}. Generally we
should interpret ``area'' in the usual sense, as a necessarily
positive quantity. Since the two curves cross, we need to compute two
areas and add them. First we find the intersection point of the
curves:
\begin{align*}
  -x^2+4x&=x^2-6x+5 \\
  0&=2x^2-10x+5 \\
  x&={10\pm\sqrt{100-40}\over4}={5\pm\sqrt{15}\over2}.
\end{align*}
The intersection point we want is $x=a=(5-\sqrt{15})/2$. Then
the total area is 
\begin{align*}
  \int_0^a x^2-6x+5-(-x^2+4x)\d x&+\int_a^1 -x^2+4x-(x^2-6x+5)\d x \\
  &=\int_0^a 2x^2-10x+5\d x+\int_a^1 -2x^2+10x-5\d x \\
  &=\left.{2x^3\over3}-5x^2+5x\right|_0^a + 
    \left.-{2x^3\over3}+5x^2-5x\right|_a^1 \\
  &=-{52\over3}+5\sqrt{15},
\end{align*}
after a bit of simplification.
\end{solution}


In both of our examples above, we gave you the limits of integration
by bounding the $x$-values between $0$ and $1$. However, some problems
are not so simple.

\begin{example} Find the area between $f(x)= -x^2+4x$ and
$g(x)=x^2-6x+5$.
\end{example}

\begin{marginfigure}
\begin{tikzpicture}
	\begin{axis}[
            domain=0:5, ymax=5,xmax=5,ymin=-5, xmin=0,
            axis lines =center, xlabel=$x$, ylabel=$y$,
            every axis y label/.style={at=(current axis.above origin),anchor=south},
            every axis x label/.style={at=(current axis.right of origin),anchor=west},
            axis on top,
          ]
          \addplot [draw=none,fill=fillp,domain=.56:4] {-x^2+4*x} \closedcycle;
          \addplot [draw=none,fill=fillp,domain=.56:4.44] {x^2-6*x+5} \closedcycle;
          \addplot [draw=none,fill=background,domain=4:5] {-x^2+4*x} \closedcycle;
          \addplot [draw=none,fill=background,domain=0:1] {x^2-6*x+5} \closedcycle;
          %\addplot [draw=none,fill=fillp,domain=.56:4] {-x^2+4*x} \closedcycle;       
          \addplot [draw=penColor,very thick,smooth] {-x^2+4*x};
          \addplot [draw=penColor2,very thick,smooth] {x^2-6*x+5};
          
          \node at (axis cs:2,4.4) [penColor] {$f(x)$};
          \node at (axis cs:1,-1) [penColor2] {$g(x)$};
        \end{axis}
\end{tikzpicture}
\caption{The area between $f(x)= -x^2+4x$ and $g(x)=x^2-6x+5$.}
\label{fig:area bounded by curves}
\end{marginfigure}

\begin{solution}
The curves are shown in Figure~\ref{fig:area bounded by curves}. Here
we are not given a specific interval, so it must be the case that
there is a ``natural'' region involved. Since the curves are both
parabolas, the only reasonable interpretation is the region between
the two intersection points, which we found in the previous example:
$${5\pm\sqrt{15}\over2}.$$
If we let $a=(5-\sqrt{15})/2$ and $b=(5+\sqrt{15})/2$,
the total area is 
\begin{align*}
  \int_a^b -x^2+4x-(x^2-6x+5)\d x
  &=\int_a^b -2x^2+10x-5\d x \\
  &=\left.-{2x^3\over3}+5x^2-5x\right|_a^b \\
  &=5\sqrt{15},
\end{align*}
after a bit of simplification.
\end{solution}



\begin{exercises}

\noindent Find the area bounded by the curves.

\begin{exercise} $y=x^4-x^2$ and $y=x^2$ (the part to the right of the $y$-axis)
\begin{answer} $8\sqrt2/15$
\end{answer}\end{exercise}

\begin{exercise} $x=y^3$ and $x=y^2$
\begin{answer} $1/12$
\end{answer}\end{exercise}

\begin{exercise} $x=1-y^2$ and $y=-x-1$
\begin{answer} $9/2$
\end{answer}\end{exercise}

\begin{exercise} $x=3y-y^2$ and $x+y=3$
\begin{answer} $4/3$
\end{answer}\end{exercise}

\begin{exercise} $y=\cos(\pi x/2)$ and $y=1- x^2$ (in the first quadrant)
\begin{answer} $2/3-2/\pi$
\end{answer}\end{exercise}

\begin{exercise} $y=\sin(\pi x/3)$ and $y=x$ (in the first quadrant)
\begin{answer} $3/\pi - 3\sqrt3/(2\pi)-1/8$
\end{answer}\end{exercise}

\begin{exercise} $y=\sqrt{x}$ and $y=x^2$
\begin{answer} $1/3$
\end{answer}\end{exercise}

\begin{exercise} $y=\sqrt x$ and $y=\sqrt{x+1}$, $0\le x\le 4$
\begin{answer} $10\sqrt{5}/3-6$
\end{answer}\end{exercise}

\begin{exercise} $x=0$ and $x=25-y^2$
\begin{answer} $500/3$
\end{answer}\end{exercise}

\begin{exercise} $y=\sin x\cos x$ and $y=\sin x$, $0\le x\le \pi$
\begin{answer} $2$
\end{answer}\end{exercise}

\begin{exercise} $y=x^{3/2}$ and $y=x^{2/3}$
\begin{answer} $1/5$
\end{answer}\end{exercise}

\begin{exercise} $y=x^2-2x$ and $y=x-2$
\begin{answer} $1/6$
\end{answer}\end{exercise}



\end{exercises}

