\chapter{The Fundamental Theorem of Calculus}

\section{The Fundamental Theorem}

Let $f(x)$ be continuous on the real numbers and consider
\[
  F(x) = \int_a^x f(t)\d t.
\]
From our previous work we know that $F(x)$ is increasing when $f(x)$
is postive and $F(x)$ is decreasing when $f(x)$ is negative. Moreover,
with careful observation, we can even see that $F(x)$ is concave up
when $f'(x)$ is positive and that $F(x)$ is concave down when $f'(x)$
is negative. Thinking about what we have learned about the
relationship of a function to its first and second derivatives, it is
not too hard to guess that there must be a connection between $F'(x)$
and the function $f(x)$. This is a good guess, check out our next
theorem:


\begin{mainTheorem}[Fundamental Theorem of Calculus---Version I]
\index{fundamental theorem of calculus---version 1}
\label{thm:fundamental_theorem_I}\hfil

\noindent Suppose that $f(x)$ is continuous on the real numbers and let
\[
  F(x)=\int_a^x f(t)\d t.
\]
Then $F'(x)=f(x)$.
\end{mainTheorem}

\begin{proof}
Using the limit definition of the derivative we'll compute $F'(x)$.
Write
\begin{align*}
F'(x) &= \lim_{h\to 0}\frac{F(x+h)-F(x)}{h}\\ 
&=\lim_{h\to 0}\frac{1}{h}\left( \int_a^{x+h} f(t)\d t - \int_a^x f(t)\d t\right).
\end{align*}
Recall that if the limits of intergration are swapped, then the sign
of the integral is swapped, so we have
\[
F'(x) =\lim_{h\to 0} \frac{1}{h}\left( \int_a^{x+h} f(t)\d t + \int_x^a f(t)\d t\right)
\]
At this point, we can combine the integrals, as we are just ``connecting'' adjacent signed areas to find
\begin{equation}\label{ftc:eqn1}
F'(x)=\lim_{h\to 0} \frac{1}{h}\int_x^{x+h} f(t)\d t.
\end{equation}
Since $f(x)$ is continuous on the interval $[x,x+h]$, and $h$ is
approaching zero, there is an $\epsilon$ that goes to zero as $h$ goes
to zero such that
\[
f(x)-\epsilon < f(x^*) < f(x) + \epsilon \qquad \text{for all }x^*\in[x,x+h].
\]
This means that 
\[
(f(x) - \epsilon)h < \int_x^{x+h} f(t)\d t < (f(x) + \epsilon)h
\]
Dividing all sides by $h$ we find
\[
f(x) - \epsilon < \frac{1}{h}\int_x^{x+h} f(t)\d t < f(x) + \epsilon.
\]
Comparing this to Equation~\ref{ftc:eqn1}, and taking the limit as $h$
goes to zero (remembering that this also means that $\epsilon$ goes to
zero) we see that $F'(x) = f(x)$.
\end{proof}

The Fundamental Theorem of Calculus says that an accumulation function
of $f(x)$ is an antiderivative of $f(x)$.  Because of the close
relationship between an integral and an antiderivative, the integral
sign is also used to mean ``antiderivative.'' You can tell which is
intended by whether the limits of integration are included. Hence
\[
  \int_a^b f(x)\d x
\] 
is a definite integral, because it has a definite value---the signed
area between $f(x)$ and the $x$-axis.  On the other hand, we use
\[
  \int f(x)\d x
\]
to denote the antiderivative of $f(x)$, also called an
\textit{indefinite integral}\index{indefinite integral}.
This is evaluated as
\[
  \int f(x)\d x = F(x)+C.
\]
Where the constant $C$ indicates that there are really an infinite
number of antiderivatives. We do not need this $C$ to compute definite
integrals, but in other circumstances we will need to remember that
the $C$ is there, so it is best to get into the habit of writing the
$C$.

There is a another common form of the Fundamental Theorem of Calculus:

\marginnote[.4in]{
Here the notation
\[
F(x) \Biggr|_a^b
\]
means that one should evaluate $F(x)$ at $b$ and then subtract from this $F(x)$ evaluated at $a$. Hence
\[
F(x) \Biggr|_a^b = F(b)-F(a).
\]  
}
\begin{mainTheorem}[Fundamental Theorem of Calculus---Version II]\index{fundamental theorem of calculus---version 2}
\label{thm:fundamental_theorem_II}\hfil

\noindent Suppose that $f(x)$ is continuous on the interval $[a,b]$. If $F(x)$
is any antiderivative of $f(x)$, then
\[
  \left.\int_a^b f(x)\d x = F(x) \right|_a^b = F(b)-F(a).
\]
\end{mainTheorem}

\begin{proof}
We know from Theorem~\ref{thm:fundamental_theorem_I} 
\[
  G(x)=\int_a^x f(t)\d t
\]
is an antiderivative of $f(x)$, and therefore any antiderivative
$F(x)$ of $f(x)$ is of the form $F(x)=G(x)+k$. Then 
\begin{align*}
  F(b)-F(a) &=G(b)+k-(G(a)+k) 
  &= G(b)-G(a) \\
  &=\int_a^b f(t)\d t-\int_a^a f(t)\d t.
\end{align*}
It is not hard to see that $\int_a^a f(t)\d t=0$, so this means that
\[
  F(b)-F(a)=\int_a^b f(t)\d t,
\]
which is exactly what Theorem~\ref{thm:fundamental_theorem_II} says.
\end{proof}

From this you should see that the two versions of the Fundamental
Theorem are very closely related. To avoid confusion, some people call
the two versions of the theorem ``The Fundamental Theorem of
Calculus---Version I'' and ``The Fundamental Theorem of
Calculus---Version II'', although unfortunately there is no universal
agreement as to which is ``Verion I'' and which ``Version II''. Since
it really is the same theorem, differently stated, people often simply
call them both ``The Fundamental Theorem of Calculus.''

Let's see an example of the the fundamental theorem in action.

\begin{example}
Compute
\[
\int_1^2\left(x^9 + \frac{1}{x}\right) \d x
\]
\end{example}

\begin{solution}
Here we start by finding an antiderivative of 
\[
x^9 + \frac{1}{x}.
\]
The correct choice is $10x^{10} + \ln(x)$, one could verify this by
taking the derivative. Hence
\begin{align*}
\int_1^2\left(x^9 + \frac{1}{x}\right) \d x &= \left(10x^{10} + \ln(x)\right)\Bigg|_1^2 \\
&= 10\cdot 2^{10} \ln(2) - 10.
\end{align*}
\end{solution}


When we compute a definite integral, we first find an antiderivative
and then substitute. It is convenient to first display the
antiderivative and then do the substitution; we need a notation
indicating that the substitution is yet to be done. A typical solution
would look like this:
\[
  \left.\int_1^2 x^2\d x={x^3\over 3}\right|_1^2 = 
  {2^3\over3}-{1^3\over3}={7\over3}.
\]
The vertical line with subscript and superscript is used to indicate
the operation ``substitute and subtract'' that is needed to finish the
evaluation. 

Now we know that to solve certain kinds of problems, those that lead
to a sum of a certain form, we ``merely'' find an antiderivative and
substitute two values and subtract. Unfortunately, finding
antiderivatives can be quite difficult. While there are a small number
of rules that allow us to compute the derivative of any common
function, there are no such rules for antiderivatives. There are some
techniques that frequently prove useful, but we will never be able to
reduce the problem to a completely mechanical process.






\subsection*{Euler's Method}

We have given a proof of the Fundamental Theorem of Calculus,
nevertheless it is good to give intution as to why it is
true. Consider the following example:

\begin{example}
Suppose that the velocity in meters per second of a ball tossed from a
height of 1 meter is given by
\[
v(t) = -9.8t + 6.
\]
What is the height of the ball after 1 second?
\end{example}

\begin{solution}
Since the derivative of position is velocity, and we want to know the
height (position) after one second, we need to compute
\begin{align*}
\int_0^1 -9.8t + 6 \d t &= (-4.9t^2 + 6t)\Bigg|_0^1\\
&= -4.9 + 6 - 0\\
&= 2.1.
\end{align*}
Hence the ball is at a height of $2.1$ meters. 
\end{solution}

We did this example before in
Example~\ref{example:eulersMethodBall}. At that time we used Euler's
Method to give an approximate solution.


Suppose you have a function, 



What's wrong with this? In some sense, nothing. As a practical matter
it is a very convincing argument, because our understanding of the
relationship between speed and distance seems to be quite solid. From
the point of view of mathematics, however, it is unsatisfactory to
justify a purely mathematical relationship by appealing to our
understanding of the physical universe, which could, however unlikely
it is in this case, be wrong.














\begin{exercises}
\noindent Compute the following definite integrals:

\twocol

\begin{exercise} $\int_1^4 t^2+3t\d t$
\begin{answer} $87/2$
\end{answer}\end{exercise}

\begin{exercise} $\int_0^\pi \sin t\d t$
\begin{answer} $2$
\end{answer}\end{exercise}

\begin{exercise} $\int_1^{10} {1\over x}\d x$
\begin{answer} $\ln(10)$
\end{answer}\end{exercise}

\begin{exercise} $\int_0^5 e^x\d x$
\begin{answer} $e^5-1$
\end{answer}\end{exercise}

\begin{exercise} $\int_0^3 x^3\d x$
\begin{answer} $3^4/4$
\end{answer}\end{exercise}

\begin{exercise} $\int_1^2 x^5\d x$
\begin{answer} $2^6/6 -1/6$
\end{answer}\end{exercise}


\begin{exercise} $\int_1^9 8\sqrt{x}\d x$
\begin{answer} $416/3$
\end{answer}\end{exercise}

\begin{exercise} $\int_1^4 \frac{4}{\sqrt{x}}\d x$
\begin{answer} $8$
\end{answer}\end{exercise}


\begin{exercise} $\int_{-2}^{-1}7x^{-1}\d x$
\begin{answer} $-7\ln(2)$
\end{answer}\end{exercise}


\begin{exercise} $\int_{-2}^3 (5x+1)^2\d x$
\begin{answer} $965/3$
\end{answer}\end{exercise}

\begin{exercise} $\int_{-7}^4(x-6)^2 \d x$
\begin{answer} $2189/3$
\end{answer}\end{exercise}

\begin{exercise} $\int_3^{27}x^{3/2}\d x$
\begin{answer} $4356 \sqrt{3}/5$
\end{answer}\end{exercise}

\begin{exercise} $\int_4^9\frac{2}{x\sqrt x}\d x$
\begin{answer} $2/3$
\end{answer}\end{exercise}

\begin{exercise} $\int_{-4}^1|2x-4|\d x$
\begin{answer} $35$
\end{answer}\end{exercise}

\endtwocol

\begin{exercise} Find the derivative of $F(x)=\int_1^x t^2-3t\d t$
\begin{answer} $x^2-3x$
\end{answer}\end{exercise}

\begin{exercise} Find the derivative of $F(x)=\int_1^{x^2} t^2-3t\d t$
\begin{answer} $2x(x^4-3x^2)$
\end{answer}\end{exercise}

\begin{exercise} Find the derivative of $F(x)=\int_1^x e^{\left(t^2\right)}\d t$
\begin{answer} $e^{\left(x^2\right)}$
\end{answer}\end{exercise}

\begin{exercise} Find the derivative of $F(x)=\int_1^{x^2} e^{\left(t^2\right)}\d t$
\begin{answer} $2xe^{\left(x^4\right)}$
\end{answer}\end{exercise}


\begin{exercise} Find the derivative of $F(x)=\int_1^x \tan(t^2)\d t$
\begin{answer} $\tan(x^2)$
\end{answer}\end{exercise}

\begin{exercise} Find the derivative of $F(x)=\int_1^{x^2} \tan(t^2)\d t$
\begin{answer} $2x\tan(x^4)$
\end{answer}\end{exercise}

\end{exercises}













\section{Area between curves}

We have seen how integration can be used to find an area between a
curve and the $x$-axis. With very little change we can find some areas
between curves; indeed, the area between a curve and the $x$-axis may
be interpreted as the area between the curve and a second ``curve''
with equation $y=0$. In the simplest of cases, the idea is quite easy
to understand.

\begin{example} Find the area below $\ds f(x)= -x^2+4x+3$ and above
$\ds g(x)=-x^3+7x^2-10x+5$ over the interval $1\le x\le2$. In
figure~\xrefn{fig:area between curves} we show the two curves together, with
the desired area shaded, then $f$ alone with the area under $f$
shaded, and then $g$ alone with the area under $g$ shaded. 
%% \figure
%% \hbox to \hsize{\hfill
%% \def\yarrow{-- +(-1.5pt,-3pt) +(0pt,0pt) -- +(1.5pt,-3pt) +(0pt,0pt)}
%% \def\xarrow{-- +(-3pt,-1.5pt) +(0pt,0pt) -- +(-3pt,1.5pt) +(0pt,0pt) }
%% \tikzpicture[domain=0:3,y=3mm]
%% \draw[->] (0,0) -- (3.2,0) \xarrow node [right] {$x$};
%% \draw[->] (0,0) -- (0,11.2) \yarrow node [above] {$y$};
%% \gpad
%% \draw[color=black] plot[id=\the\gpnum,domain=0:3] function{-x**3+7*x**2-10*x+5};
%% \gpad
%% \draw[color=black] plot[id=\the\gpnum,domain=0:3] function{-x**2+4*x+3};
%% \draw[dashed] (1,0) -- (1,6);
%% \draw[dashed] (2,0) -- (2,7);
%% \foreach \x in {0,1,2,3} \draw (\x,0) -- (\x,-2pt) node[anchor=north] {\eightpoint $\x$};
%% \foreach \y in {0,5,10} \draw (0,\y) -- (-2pt,\y) node[anchor=east]
%%          {\eightpoint $\y$};
%% \gpad
%% \fill[opacity=0.5,fill=red!20] plot[id=\the\gpnum,domain=1:2]
%% function{-x**3+7*x**2-10*x+5} -- (2,7) node {\gpad}
%% plot[parametric,id=\the\gpnum,domain=0:1] function{2-t,-t**2+7} -- (1,1);
%% \endtikzpicture

%% \tikzpicture[domain=0:3,y=3mm]
%% \draw[->] (0,0) -- (3.2,0) \xarrow node [right] {$x$};
%% \draw[->] (0,0) -- (0,11.2) \yarrow node [above] {$y$};
%% \gpad
%% \draw[color=black] plot[id=\the\gpnum,domain=0:3] function{-x**2+4*x+3};
%% \draw[dashed] (1,0) -- (1,6);
%% \draw[dashed] (2,0) -- (2,7);
%% \foreach \x in {0,1,2,3} \draw (\x,0) -- (\x,-2pt) node[anchor=north] {\eightpoint $\x$};
%% \foreach \y in {0,5,10} \draw (0,\y) -- (-2pt,\y) node[anchor=east] {\eightpoint $\y$};
%% \gpad\fill[opacity=0.5,fill=red!20] (2,0) -- (2,7)
%% plot[parametric,id=\the\gpnum,domain=0:1] function{2-t,-t**2+7} -- (1,0) -- (2,0);
%% \endtikzpicture

%% \tikzpicture[domain=0:3,y=3mm]
%% \draw[angle 90] (0,0) -- (3.2,0) \xarrow node [right] {$x$};
%% \draw[->] (0,0) -- (0,11.2) \yarrow node [above] {$y$};
%% \gpad\draw[color=black] plot[id=\the\gpnum,domain=0:3] function{-x**3+7*x**2-10*x+5};
%% \draw[dashed] (1,0) -- (1,1);
%% \draw[dashed] (2,0) -- (2,5);
%% \foreach \x in {0,1,2,3} \draw (\x,0) -- (\x,-2pt) node[anchor=north] {\eightpoint $\x$};
%% \foreach \y in {0,5,10} \draw (0,\y) -- (-2pt,\y) node[anchor=east]
%%          {\eightpoint $\y$};
%% \gpad
%% \fill[opacity=0.5,fill=red!20] (1,0) -- (1,1) plot[id=\the\gpnum,domain=1:2]
%% function{-x**3+7*x**2-10*x+5} -- (2,0) -- (1,0);
%% \endtikzpicture
%% \hfill}
%% \figrdef{fig:area between curves}
%% \endfigure{Area between curves as a difference of areas.}

It is clear
from the figure that the area we want is the area under $f$ minus the
area under $g$, which is to say
$$\int_1^2 f(x)\,dx-\int_1^2 g(x)\,dx = \int_1^2 f(x)-g(x)\,dx.$$
It doesn't matter whether we compute the two integrals on the left and
then subtract or compute the single integral on the right. In this
case, the latter is perhaps a bit easier:
\begin{align*}
  \int_1^2 f(x)-g(x)\,dx&=\int_1^2 -x^2+4x+3-(-x^3+7x^2-10x+5)\,dx \\
  &=\int_1^2 x^3-8x^2+14x-2\,dx \\
  &=\left.{x^4\over4}-{8x^3\over3}+7x^2-2x\right|_1^2 \\
  &={16\over4}-{64\over3}+28-4-({1\over4}-{8\over3}+7-2) \\
  &=23-{56\over3}-{1\over4}={49\over12}. \\
\end{align*}
\vskip-10pt\end{example}

It is worth examining this problem a bit more. We have seen one way to
look at it, by viewing the desired area as a big area minus a small
area, which leads naturally to the difference between two
integrals. But it is instructive to consider how we might find the
desired area directly. We can approximate the area by dividing the
area into thin sections and approximating the area of each section by
a rectangle, as indicated in 
figure~\xrefn{fig:rectangles between curves}. 
The area of a typical rectangle is 
$\Delta x(f(x_i)-g(x_i))$, so the total area is approximately
$$\sum_{i=0}^{n-1} (f(x_i)-g(x_i))\Delta x.$$
This is exactly the sort of sum that turns into an integral in the
limit, namely the integral
$$\int_1^2 f(x)-g(x)\,dx.$$
Of course, this is the integral we actually computed above, but we
have now arrived at it directly rather than as a modification of the
difference between two other integrals. In that example it really
doesn't matter which approach we take, but in some cases this second
approach is better.

%% \figure
%% \vbox{\beginpicture
%% \normalgraphs
%% \ninepoint
%% \setcoordinatesystem units <3truecm,0.4truecm>
%% \setplotarea x from 0 to 3, y from 0 to 10
%% \axis left ticks numbered from 0 to 10 by 5 /
%% \axis bottom  ticks numbered from 0 to 3 by 1 /
%% \setquadratic
%% %\putrule from 1.52 2.707 to 1.52 6.819
%% %\putrule from 1.63 2.707 to 1.63 6.819
%% %\putrule from 1.52 2.707 to 1.63 2.707
%% %\putrule from 1.52 6.819 to 1.63 6.819
%% \putrule from 1.465 2.229 to 1.465 6.714
%% \putrule from 1.575 2.229 to 1.575 6.714
%% \putrule from 1.575 2.229 to 1.465 2.229
%% \putrule from 1.575 6.714 to 1.465 6.714
%% %
%% \putrule from 1.575 2.707 to 1.575 6.819
%% \putrule from 1.685 2.707 to 1.685 6.819
%% \putrule from 1.575 2.707 to 1.685 2.707
%% \putrule from 1.575 6.819 to 1.685 6.819
%% \plot
%% 0.000 3.000 0.075 3.294 0.150 3.578 0.225 3.849 0.300 4.110 
%% 0.375 4.359 0.450 4.598 0.525 4.824 0.600 5.040 0.675 5.244 
%% 0.750 5.438 0.825 5.619 0.900 5.790 0.975 5.949 1.050 6.098 
%% 1.125 6.234 1.200 6.360 1.275 6.474 1.350 6.578 1.425 6.669 
%% 1.500 6.750 1.575 6.819 1.650 6.878 1.725 6.924 1.800 6.960 
%% 1.875 6.984 1.950 6.998 2.025 6.999 2.100 6.990 2.175 6.969 
%% 2.250 6.938 2.325 6.894 2.400 6.840 2.475 6.774 2.550 6.698 
%% 2.625 6.609 2.700 6.510 2.775 6.399 2.850 6.278 2.925 6.144 
%% 3.000 6.000 /
%% \plot
%% 0.000 5.000 0.075 4.289 0.150 3.654 0.225 3.093 0.300 2.603 
%% 0.375 2.182 0.450 1.826 0.525 1.535 0.600 1.304 0.675 1.132 
%% 0.750 1.016 0.825 0.953 0.900 0.941 0.975 0.978 1.050 1.060 
%% 1.125 1.186 1.200 1.352 1.275 1.557 1.350 1.797 1.425 2.071 
%% 1.500 2.375 1.575 2.707 1.650 3.065 1.725 3.446 1.800 3.848 
%% 1.875 4.268 1.950 4.703 2.025 5.151 2.100 5.609 2.175 6.075 
%% 2.250 6.547 2.325 7.021 2.400 7.496 2.475 7.968 2.550 8.436 
%% 2.625 8.896 2.700 9.347 2.775 9.785 2.850 10.208 2.925 10.614 
%% 3.000 11.000 /
%% \setdashes
%% \putrule from 1 0 to 1 6
%% \putrule from 2 0 to 2 7
%% \setsolid
%% \endpicture}
%% \figrdef{fig:rectangles between curves}
%% \endfigure{Approximating area between curves with rectangles.}

\begin{example} Find the area below $\ds f(x)= -x^2+4x+1$ and above
$\ds g(x)=-x^3+7x^2-10x+3$ over the interval $1\le x\le2$; these are the
same curves as before but lowered by 2. In
figure~\xrefn{fig:area between curves too} we show the two curves
together. Note that the lower curve now dips below the $x$-axis. This
makes it somewhat tricky to view the desired area as a big area minus
a smaller area, but it is just as easy as before to think of
approximating the area by rectangles. The height of a typical
rectangle will still be $f(x_i)-g(x_i)$, even if $g(x_i)$ is
negative. Thus the area is 
$$
  \int_1^2 -x^2+4x+1-(-x^3+7x^2-10x+3)\,dx
  =\int_1^2 x^3-8x^2+14x-2\,dx.
$$
This is of course the same integral as before, because the region
between the curves is identical to the former region---it has just
been moved down by 2.
\end{example}

%% \figure
%% \hbox{\hfill
%% \tikzpicture[domain=0:3,y=3mm]
%% \draw[->] (0,0) -- (3.2,0) ;
%% \draw[->] (0,0) -- (0,10.2) ;
%% \gpad
%% \draw[color=black] plot[id=\the\gpnum,domain=0:3] function{-x**3+7*x**2-10*x+3};
%% \gpad
%% \draw[color=black] plot[id=\the\gpnum,domain=0:3] function{-x**2+4*x+1};
%% \draw[dashed] (1,0) -- (1,4);
%% \draw[dashed] (2,0) -- (2,5);
%% \foreach \x in {0,1,2,3} \draw (\x,0) -- (\x,-2pt) node[anchor=north] {\eightpoint $\x$};
%% \foreach \y in {0,5,10} \draw (0,\y) -- (-2pt,\y) node[anchor=east] {\eightpoint $\y$};
%% \gpad
%% \fill[opacity=0.5,fill=red!20] plot[id=\the\gpnum,domain=1:2]
%% function{-x**3+7*x**2-10*x+3} -- (2,5) node {\gpad}
%% plot[parametric,id=\the\gpnum,domain=0:1] function{2-t,-t**2+5} -- (1,-1);
%% \endtikzpicture
%% \hfill}
%% \figrdef{fig:area between curves too}
%% \endfigure{Area between curves.}

\begin{example} Find the area between $\ds f(x)= -x^2+4x$ and
$\ds g(x)=x^2-6x+5$ over the interval $0\le x\le 1$; the
curves are shown in figure~\xrefn{fig:curves cross}. Generally we
should interpret ``area'' in the usual sense, as a necessarily
positive quantity. Since the two curves cross, we need to compute two 
areas and add them. First we find the intersection point of the
curves:
\begin{align*}
  -x^2+4x&=x^2-6x+5 \\
  0&=2x^2-10x+5 \\
  x&={10\pm\sqrt{100-40}\over4}={5\pm\sqrt{15}\over2}. \\
\end{align*}
The intersection point we want is $\ds x=a=(5-\sqrt{15})/2$. Then
the total area is 
\begin{align*}
  \int_0^a x^2-6x+5-(-x^2+4x)\,dx&+\int_a^1 -x^2+4x-(x^2-6x+5)\,dx \\
  &=\int_0^a 2x^2-10x+5\,dx+\int_a^1 -2x^2+10x-5\,dx \\
  &=\left.{2x^3\over3}-5x^2+5x\right|_0^a + 
    \left.-{2x^3\over3}+5x^2-5x\right|_a^1 \\
  &=-{52\over3}+5\sqrt{15},
\end{align*}
after a bit of simplification.
\end{example}

%% \figure
%% \hbox{\hfill\tikzpicture[domain=0:1,x=3cm,y=1cm]
%% \draw[->] (0,0) -- (1.1,0) ;
%% \draw[->] (0,0) -- (0,5.2) ;
%% \gpad
%% \draw[color=black] plot[id=\the\gpnum,domain=0:1] function{x**2-6*x+5};
%% \gpad
%% \draw[color=black] plot[id=\the\gpnum,domain=0:1] function{-x**2+4*x};
%% \draw[dashed] (1,0) -- (1,3);
%% \foreach \x in {0,1} \draw (\x,0) -- (\x,-2pt) node[anchor=north] {\eightpoint $\x$};
%% \foreach \y in {0,1,2,3,4,5} \draw (0,\y) -- (-2pt,\y)
%% node[anchor=east] {\eightpoint $\y$};
%% \gpad
%% \fill[opacity=0.5,fill=red!20] plot[id=\the\gpnum,domain=0:0.564]
%% function{-x**2+4*x} node {\gpad}
%% plot[parametric,id=\the\gpnum,domain=0.564:0]
%% function{t,(t)**2-6*(t)+5} -- (0,0);
%% \gpad
%% \fill[opacity=0.5,fill=red!20] plot[id=\the\gpnum,domain=0.564:1]
%% function{-x**2+4*x} node {\gpad}
%% -- (1,0) plot[parametric,id=\the\gpnum,domain=1:0.564] function{t,(t)**2-6*(t)+5};
%% \endtikzpicture\hfill}
%% \figrdef{fig:curves cross}
%% \endfigure{Area between curves that cross.}

\begin{example} Find the area between $\ds f(x)= -x^2+4x$ and
$\ds g(x)=x^2-6x+5$; the
curves are shown in figure~\xrefn{fig:area bounded by curves}. Here we
are not given a specific interval, so it must be the case that there
is a ``natural'' region involved. Since the curves are both parabolas,
the only reasonable interpretation is the region between the two
intersection points, which we found in the previous example:
$${5\pm\sqrt{15}\over2}.$$
If we let $\ds a=(5-\sqrt{15})/2$ and $\ds b=(5+\sqrt{15})/2$,
the total area is 
\begin{align*}
  \int_a^b -x^2+4x-(x^2-6x+5)\,dx
  &=\int_a^b -2x^2+10x-5\,dx \\
  &=\left.-{2x^3\over3}+5x^2-5x\right|_a^b \\
  &=5\sqrt{15}. \\
\end{align*}
after a bit of simplification.
\end{example}

%% \figure
%% \vbox{\beginpicture
%% \normalgraphs
%% \ninepoint
%% \setcoordinatesystem units <1.7truecm,0.4truecm>
%% \setplotarea x from 0 to 5, y from -5 to 5
%% \axis left ticks numbered from -5 to 5 by 5 /
%% \axis bottom shiftedto y=0 ticks numbered from 1 to 5 by 1 /
%% \setquadratic
%% \plot
%% 0.000 0.000 0.250 0.938 0.500 1.750 0.750 2.438 1.000 3.000 
%% 1.250 3.438 1.500 3.750 1.750 3.938 2.000 4.000 2.250 3.938 
%% 2.500 3.750 2.750 3.438 3.000 3.000 3.250 2.438 3.500 1.750 
%% 3.750 0.938 4.000 0.000 4.250 -1.062 4.500 -2.250 4.750 -3.562 
%% 5.000 -5.000 /
%% \plot
%% 0.000 5.000 0.250 3.562 0.500 2.250 0.750 1.062 1.000 0.000 
%% 1.250 -0.938 1.500 -1.750 1.750 -2.438 2.000 -3.000 2.250 -3.438 
%% 2.500 -3.750 2.750 -3.938 3.000 -4.000 3.250 -3.938 3.500 -3.750 
%% 3.750 -3.438 4.000 -3.000 4.250 -2.438 4.500 -1.750 4.750 -0.938 
%% 5.000 0.000 /
%% \endpicture}
%% \figrdef{fig:area bounded by curves}
%% \endfigure{Area bounded by two curves.}

\begin{exercises}

\noindent Find the area bounded by the curves.

\begin{exercise} $\ds y=x^4-x^2$ and $\ds y=x^2$ (the part to the right of the $y$-axis)
\begin{answer} $\ds 8\sqrt2/15$
\end{answer}\end{exercise}

\begin{exercise} $\ds x=y^3$ and $\ds x=y^2$
\begin{answer} $1/12$
\end{answer}\end{exercise}

\begin{exercise} $\ds x=1-y^2$ and $y=-x-1$
\begin{answer} $9/2$
\end{answer}\end{exercise}

\begin{exercise} $\ds x=3y-y^2$ and $x+y=3$
\begin{answer} $4/3$
\end{answer}\end{exercise}

\begin{exercise} $y=\cos(\pi x/2)$ and $\ds y=1- x^2$ (in the first quadrant)
\begin{answer} $2/3-2/\pi$
\end{answer}\end{exercise}

\begin{exercise} $y=\sin(\pi x/3)$ and $y=x$ (in the first quadrant)
\begin{answer} $\ds 3/\pi - 3\sqrt3/(2\pi)-1/8$
\end{answer}\end{exercise}

\begin{exercise} $\ds y=\sqrt{x}$ and $\ds y=x^2$
\begin{answer} $1/3$
\end{answer}\end{exercise}

\begin{exercise} $\ds y=\sqrt x$ and $\ds y=\sqrt{x+1}$, $0\le x\le 4$
\begin{answer} $\ds 10\sqrt{5}/3-6$
\end{answer}\end{exercise}

\begin{exercise} $x=0$ and $\ds x=25-y^2$
\begin{answer} $500/3$
\end{answer}\end{exercise}

\begin{exercise} $y=\sin x\cos x$ and $y=\sin x$, $0\le x\le \pi$
\begin{answer} $2$
\end{answer}\end{exercise}

\begin{exercise} $\ds y=x^{3/2}$ and $\ds y=x^{2/3}$
\begin{answer} $1/5$
\end{answer}\end{exercise}

\begin{exercise} $\ds y=x^2-2x$ and $y=x-2$
\begin{answer} $1/6$
\end{answer}\end{exercise}



\end{exercises}

