\section{Line Integrals}{}{}

We have so far integrated ``over'' intervals, areas, and volumes with
single, double, and triple integrals. We now investigate integration
over or ``along'' a curve---``line integrals''\index{line integral}
are really ``curve integrals''.

As with other integrals, a geometric example may be easiest to
understand. Consider the function $f=x+y$ and the parabola $y=x^2$ in
the $x$-$y$ plane, for $0\le x\le 2$. Imagine that we extend the
parabola up to the surface $f$, to form a curved wall or curtain, as
in figure~\xrefn{fig:curtain}.  What is the area of the surface thus
formed? We already know one way to compute surface area, but here we
take a different approach that is more useful for the
problems to come.

\figure
\vbox{\beginpicture
\normalgraphs
\ninepoint
\setcoordinatesystem units <3truecm,3truecm>
\setplotarea x from -1 to 1, y from 0 to 1
\put {\hbox{\epsfxsize8cm\epsfbox{parabolic_curtain.eps}}} at 0 -0.15
\put {\hbox{\epsfxsize5cm\epsfbox{curtain_approx.eps}}} at 2.5 -0.15
\endpicture}
\figrdef{fig:curtain}
\endfigure{Approximating the area under a curve.
(\expandafter\url\expandafter{\liveurl parabolic_curtain.html}%
AP\endurl)}

As usual, we start by thinking about how to approximate the area. We
pick some points along the part of the parabola we're interested in,
and connect adjacent points by straight lines; when the points are
close together, the length of each line segment will be close to the
length along the parabola. Using each line segment as the base of a
rectangle, we choose the height to be the height of the surface $f$
above the line segment. If we add up the areas of these rectangles, we
get an approximation to the desired area, and in the limit this sum
turns into an integral.

Typically the curve is in vector form, or can easily be put in vector
form; in this example we have ${\bf v}(t)=\langle t,t^2\rangle$. Then as
we have seen in section~\xrefn{sec:arc length 3D} on arc length, the
length of one of the straight line segments in the approximation is
approximately $ds=|{\bf v}'|\,dt=\sqrt{1+4t^2}\,dt$, so the integral is
$$\int_0^2 f(t,t^2)\sqrt{1+4t^2}\,dt
=\int_0^2 (t+t^2)\sqrt{1+4t^2}\,dt=
{167\over48}\sqrt{17}-{1\over12}-{1\over64}\ln(4+\sqrt{17}).
$$
This integral of a function along a curve $C$ is often written in
abbreviated form as
$$\int_C f(x,y)\,ds.$$

\begin{example} Compute $\ds\int_C ye^x\,ds$ where $C$ is the line segment from
$(1,2)$ to $(4,7)$.

We write the line segment as a vector function: ${\bf v}=\langle
1,2\rangle + t\langle 3,5\rangle$, $0\le t\le 1$, or in parametric
form $x=1+3t$, $y=2+5t$. Then
$$\int_C ye^x\,ds = \int_0^1 (2+5t)e^{1+3t}\sqrt{3^2+5^2}\,dt
={16\over 9}\sqrt{34}e^4-{1\over9}\sqrt{34}\,e.
$$
\end{example}

All of these ideas extend to three dimensions in the obvious way.

\begin{example} Compute $\ds\int_C x^2  z \,ds$ where $C$ is the line segment
from $(0,6,-1)$ to $(4,1,5)$.

We write the line segment as a vector function: ${\bf v}=\langle
0,6,-1\rangle + t\langle 4,-5,6\rangle$, $0\le t\le 1$, or in parametric
form $x=4t$, $y=6-5t$, $z=-1+6t$. Then
$$\int_C x^2 z\,ds = \int_0^1 (4t)^2 (-1+6t)\sqrt{16+25+36}\,dt
=16\sqrt{77}\int_0^1 -t^2+6t^3\,dt={56\over3}\sqrt{77}.
$$
\end{example}

Now we turn to a perhaps more interesting example. Recall that in the
simplest case, the work done by a force on an object is equal to the
magnitude of the force times the distance the object moves; this
assumes that the force is constant and in the direction of motion. We
have already dealt with examples in which the force is not constant;
now we are prepared to examine what happens when the force is not
parallel to the direction of motion.

We have already examined the idea of components of force, in
example~\xrefn{example:components of force vector}: the component of a
force $\bf F$ in the direction of a vector $\bf v$ is 
$${{\bf F}\cdot {\bf v}\over|{\bf v}|^2}{\bf v},$$
the projection\index{projection} of $\bf F$ onto $\bf v$.
The length of this vector, that is, the magnitude of the force in the
direction of $\bf v$, is 
$${{\bf F}\cdot {\bf v}\over|{\bf v}|},$$
the scalar projection\index{scalar projection}\index{projection!scalar}
 of $\bf F$ onto $\bf v$.
If an object moves subject to this (constant) force, in the direction
of $\bf v$, over a distance equal to the length of $\bf v$, the work
done is
$${{\bf F}\cdot {\bf v}\over|{\bf v}|}|{\bf v}|={\bf F}\cdot {\bf v}.$$
Thus, work in the vector setting is still ``force times distance'',
except that ``times'' means ``dot product''.

If the force varies from point to point, it is represented by a vector
field $\bf F$; the displacement vector $\bf v$ may also change, as an object
may follow a curving path in two or three dimensions. Suppose that the
path of an object is given by a vector function ${\bf r}(t)$; at any
point along the path, the (small) tangent vector ${\bf r}'\,\Delta t$ 
gives an approximation to its motion over a short time $\Delta t$, so
the work done during that time is approximately 
${\bf F}\cdot{\bf r}'\,\Delta t$; the total work over some time period
is then
$$\int_{t_0}^{t_1} {\bf F}\cdot{\bf r}'\,dt.$$
It is useful to rewrite this in various ways at different times. 
We  start with 
$$\int_{t_0}^{t_1} {\bf F}\cdot{\bf r}'\,dt=\int_C {\bf F}\cdot\,d{\bf
  r},$$
abbreviating ${\bf r}'\,dt$ by $d{\bf r}$. Or we can write
$$\int_{t_0}^{t_1} {\bf F}\cdot{\bf r}'\,dt=
\int_{t_0}^{t_1} {\bf F}\cdot{{\bf r}'\over|{\bf r}'|}|{\bf r}'|\,dt=
\int_{t_0}^{t_1} {\bf F}\cdot{\bf T}\,|{\bf r}'|\,dt=
\int_{C} {\bf F}\cdot{\bf T}\,ds,$$
using the unit tangent vector $\bf T$, abbreviating 
$|{\bf r}'|\,dt$ as $ds$, and indicating the path of the object by
$C$. In other words, work is computed using a particular line integral
of the form we have considered.
Alternately, we sometimes write
$$\eqalign{
\int_C {\bf F}\cdot{\bf r}'\,dt&=
\int_C \langle f,g,h\rangle\cdot\langle x',y',z'\rangle\,dt=
\int_C \left(f{dx\over dt}+g{dy\over dt}+h{dz\over dt}\right)\,dt \\
&=
\int_C f\,dx+g\,dy+h\,dz=
\int_C f\,dx+\int_C g\,dy+\int_C h\,dz, \\}$$
and similarly for two dimensions, leaving out references to $z$.

\begin{example} Suppose an object moves from $(-1,1)$ to
$(2,4)$ along the path ${\bf r}(t)=\langle t,t^2\rangle$,
subject to the force ${\bf F}=\langle x\sin y,y\rangle$. Find the work
done. 

We can write the force in terms of $t$ as $\langle t\sin(t^2),t^2\rangle$,
and compute ${\bf r}'(t)=\langle 1,2t\rangle$, and then the work is
$$\int_{-1}^2 \langle t\sin(t^2),t^2\rangle\cdot\langle
1,2t\rangle\,dt
= \int_{-1}^2 t\sin(t^2) + 2t^3\,dt={15\over2}+{\cos(1)-\cos(4)\over2}.
$$

Alternately, we might write
$$\int_C x\sin y\,dx+\int_C y\,dy=
\int_{-1}^2 x\sin(x^2)\,dx + \int_1^4 y\,dy=
-{\cos(4)\over 2}+{\cos(1)\over 2}+{16\over2}-{1\over2}$$
getting the same answer.
\end{example}

\begin{exercises}

\begin{exercise} Compute $\ds\int_C xy^2\,ds$ along the line segment from
$(1,2,0)$ to $(2,1,3)$.
\begin{answer} $13\sqrt{11}/4$
\end{answer}\end{exercise}

\begin{exercise} Compute $\ds\int_C \sin x\,ds$ along the line segment from
$(-1,2,1)$ to $(1,2,5)$.
\begin{answer} $0$
\end{answer}\end{exercise}

\begin{exercise} Compute $\ds\int_C z\cos(xy)\,ds$ along the line segment from
$(1,0,1)$ to $(2,2,3)$.
\begin{answer} $3\sin(4)/2$
\end{answer}\end{exercise}

\begin{exercise} Compute $\ds\int_C \sin x\,dx+\cos y\,dy$ along the top half
of the unit circle, from $(1,0)$ to $(-1,0)$.
\begin{answer} $0$
\end{answer}\end{exercise}

\begin{exercise} Compute $\ds\int_C xe^y\,dx+x^2y\,dy$ along the line segment
$y=3$, $0\le x\le 2$.
\begin{answer} $2e^3$
\end{answer}\end{exercise}

\begin{exercise} Compute $\ds\int_C xe^y\,dx+x^2y\,dy$ along the line segment
$x=4$, $0\le y\le 4$.
\begin{answer} $128$
\end{answer}\end{exercise}

\begin{exercise} Compute $\ds\int_C xe^y\,dx+x^2y\,dy$ along the curve
$x=3t$, $y=t^2$, $0\le t\le1$.
\begin{answer} $(9e-3)/2$
\end{answer}\end{exercise}

\begin{exercise} Compute $\ds\int_C xe^y\,dx+x^2y\,dy$ along the 
curve $\langle e^t,e^t\rangle$, $-1\le t\le1$.
\begin{answer} $e^{e+1}-e^e-e^{1/e-1}+e^{1/e}+e^4/4-e^{-4}/4$
\end{answer}\end{exercise}

\begin{exercise} Compute $\ds\int_C \langle \cos x,\sin y\rangle\cdot 
d{\bf r}$ along the 
curve $\langle t,t\rangle$, $0\le t\le1$.
\begin{answer} $1+\sin(1)-\cos(1)$
\end{answer}\end{exercise}

\begin{exercise} Compute $\ds\int_C \langle 1/xy,1/(x+y)\rangle\cdot
d{\bf r}$ along the 
path from $(1,1)$ to $(3,1)$ to $(3,6)$ using straight line segments.
\begin{answer} $3\ln3-2\ln2$
\end{answer}\end{exercise}

\begin{exercise} Compute $\ds\int_C \langle 1/xy,1/(x+y)\rangle\cdot
d{\bf r}$ along the 
curve $\langle 2t,5t\rangle$, $1\le t\le 4$.
\begin{answer} $3/20+10\ln(2)/7$
\end{answer}\end{exercise}

\begin{exercise} Compute $\ds\int_C \langle 1/xy,1/(x+y)\rangle\cdot 
d{\bf r}$ along the 
curve $\langle t,t^2\rangle$, $1\le t\le 4$.
\begin{answer} $2\ln5-2\ln2+15/32$
\end{answer}\end{exercise}

\begin{exercise} Compute $\ds\int_C yz\,dx+xz\,dy+xy\,dz$ along the curve
$\langle t,t^2,t^3\rangle$, $0\le t\le1$.
\begin{answer} $1$
\end{answer}\end{exercise}

\begin{exercise} Compute $\ds\int_C yz\,dx+xz\,dy+xy\,dz$ along the curve
$\langle \cos t,\sin t,\tan t\rangle$, $0\le t\le\pi$.
\begin{answer} $0$
\end{answer}\end{exercise}

\begin{exercise} An object moves from $(1,1)$ to
$(4,8)$ along the path ${\bf r}(t)=\langle t^2,t^3\rangle$,
subject to the force ${\bf F}=\langle x^2,\sin y\rangle$. Find the work
done. 
\begin{answer} $21+\cos(1)-\cos(8)$
\end{answer}\end{exercise}

\begin{exercise} An object moves along the line segment from $(1,1)$ to $(2,5)$,
subject to the force ${\bf F}=\langle
x/(x^2+y^2),y/(x^2+y^2)\rangle$. Find the work done.
\begin{answer} $(\ln29-\ln2)/2$
\end{answer}\end{exercise}

\begin{exercise} An object moves along the parabola ${\bf r}(t)=\langle
t,t^2\rangle$, $0\le t\le1$, subject to the force ${\bf F}=\langle
1/(y+1),-1/(x+1)\rangle$. Find the work done.
\begin{answer} $2\ln2+\pi/4-2$
\end{answer}\end{exercise}

\begin{exercise} An object moves along the line segment from $(0,0,0)$ to
$(3,6,10)$,
subject to the force ${\bf F}=\langle x^2,y^2,z^2\rangle$. 
Find the work
done. 
\begin{answer} $1243/3$
\end{answer}\end{exercise}

\begin{exercise} An object moves along the curve ${\bf r}(t)=\langle
\sqrt{t},1/\sqrt{t},t\rangle$ $1\le t\le4$, subject to the force ${\bf
  F}=\langle y,z,x\rangle$.  Find the work done.
\begin{answer} $\ln2+11/3$
\end{answer}\end{exercise}

\begin{exercise} An object moves from $(1,1,1)$ to
$(2,4,8)$ along the path ${\bf r}(t)=\langle t,t^2,t^3\rangle$,
subject to the force ${\bf F}=\langle \sin x,\sin y,\sin z\rangle$. 
Find the work
done. 
\begin{answer} $3\cos(1)-\cos(2)-\cos(4)-\cos(8)$
\end{answer}\end{exercise}

\begin{exercise} An object moves from $(1,0,0)$ to
$(-1,0,\pi)$ along the path ${\bf r}(t)=\langle \cos t,\sin t, t\rangle$,
subject to the force ${\bf F}=\langle y^2,y^2,xz\rangle$. 
Find the work
done. 
\begin{answer} $-10/3$
\end{answer}\end{exercise}

\begin{exercise} Give an example of a non-trivial force field ${\bf F}$ and
non-trivial path ${\bf r}(t)$ for which the total work done moving along
the path is zero.

\end{exercises}

