\documentclass[justified,openany]{tufte-book}
\usepackage{mooculus}



% BADBAD should include cents symbol?
\newcommand\cents{cents}

\setcounter{secnumdepth}{2}


\DeclareMathOperator{\arccot}{arccot}
\DeclareMathOperator{\sech}{sech}
\DeclareMathOperator{\csch}{csch}
\DeclareMathOperator{\arcsinh}{arcsinh}
\DeclareMathOperator{\arcsech}{arcsech}
\DeclareMathOperator{\arccosh}{arccosh}


% For nicely typeset tabular material
\usepackage{booktabs}

% Prints the month name (e.g., January) and the year (e.g., 2008)
\newcommand{\monthyear}{%
  \ifcase\month\or January\or February\or March\or April\or May\or June\or
  July\or August\or September\or October\or November\or
  December\fi\space\number\year
}

% Generates the index
\usepackage{makeidx}
\makeindex

\newcommand{\xrefn}[1]{\ref{#1}}

%%%%%%%%%%%%%%%%%%%%%%%%%%%%%%%%%%%%%%%%%%%%%%%%%%%%%%%%%%%%%%%%
\makeatletter
\newwrite\answer@stream
\immediate\openout\answer@stream=\jobname.ans

\def\dumpanswer[#1]{\immediate\write\answer@stream{#1}}

\newtoks{\answercontent}
\usepackage{environ}
 \NewEnviron{answer}{%
   \answercontent=\expandafter{\BODY}
   \dumpanswer[\arabic{exercise}. \the\answercontent]
 }
\makeatother
%%%%%%%%%%%%%%%%%%%%%%%%%%%%%%%%%%%%%%%%%%%%%%%%%%%%%%%%%%%%%%%%

\newenvironment{lemma}{\subsection*{Lemma}}{}
\newenvironment{remark}[1]{\subsection*{Remark: #1}}{}

%\def\beginpicture{\null}

\def\exam{\null}
\def\pagerdef{\null}

\def\dfont{\bf}
\def\em{\it}           % for emphasis

\newcommand{\ds}{\displaystyle}

\let\ssk\smallskip \let\msk\medskip \let\bsk\bigskip

\usepackage{multicol}
\def\twocol{\begin{multicols}{2}}
\def\endtwocol{\end{multicols}}

\title{Calculus}
%\author{Jim Fowler and Bart Snapp}
\publisher{This document was typeset on \today.}
%\newcommand{\blankpage}{\newpage\hbox{}\thispagestyle{empty}\newpage}

%% % Prints an epigraph and speaker in sans serif, all-caps type.
%% \newcommand{\openepigraph}[2]{%
%%   %\sffamily\fontsize{14}{16}\selectfont
%%   \begin{fullwidth}
%%   \sffamily\large
%%   \begin{doublespace}
%%   \noindent\allcaps{#1}\\% epigraph
%%   \noindent\allcaps{#2}% author
%%   \end{doublespace}
%%   \end{fullwidth}
%% }




\begin{document}
\maketitle

% v.4 copyright page


\begin{fullwidth}
~\vfill
\thispagestyle{empty}
\setlength{\parindent}{0pt}
\setlength{\parskip}{\baselineskip}
Copyright \copyright\ \the\year\ Jim Fowler and Bart Snapp

This work is licensed under the Creative Commons
Attribution-NonCommercial-ShareAlike License. To view a copy of this
license, visit
\url{http://creativecommons.org/licenses/by-nc-sa/3.0/}~or send a
letter to Creative Commons, 543 Howard Street, 5th Floor, San
Francisco, California, 94105, USA. If you distribute this work or a
derivative, include the history of the document.

\noindent
This text is based on David Guichard's open-source calculus text which
in turn is a modification and expansion of notes written by Neal
Koblitz at the University of Washington. New material has been added,
and old material has been modified, so some portions now bear little
resemblance to the original.

\noindent The book includes some exercises and examples from {\it
  Elementary Calculus: An Approach Using Infinitesimals}, by H.~Jerome
Keisler, available at
\url{http://www.math.wisc.edu/~keisler/calc.html}~under a Creative
Commons license. In addition, the chapter on differential equations
and the section on numerical integration are largely derived from the
corresponding portions of Keisler's book.  Albert Schueller, Barry
Balof, and Mike Wills have contributed additional material.

\noindent This book is typeset in the Kerkis font, 
Kerkis \copyright~Department of Mathematics, University of the Aegean.
% BADBAD
%\msk\noindent
%This copy of the text was compiled from source at 
%\the\bighand:\ifnum\littlehand<10{0}\fi
%        \the\littlehand\ on \the\month/\the\day/\the\year.


\noindent We will be glad to receive corrections and suggestions for
improvement at \texttt{fowler@math.osu.edu} or
\texttt{snapp@math.osu.edu}.

\end{fullwidth}

% r.5 contents
\tableofcontents


%\listoftheorems[ignoreall,show={mainTheorem}]

\renewcommand{\listtheoremname}{List of Main Theorems}
\setcounter{tocdepth}{1}
\listoftheorems[numwidth=3em,ignoreall,show={mainTheorem}]


%\listoffigures

%\listoftables

% r.7 dedication
%\cleardoublepage
%~\vfill
%\begin{doublespace}
%\thispagestyle{empty}
%\noindent\fontsize{18}{22}\selectfont\itshape
%\nohyphenation
%\centerline{\it For Kathleen,\/}
%\centerline{\it without whose encouragement\/}
%\centerline{\it this book would not have\/}
%\centerline{\it been written.\/}
%\end{doublespace}
%\vfill
%\vfill

% r.9 introduction
%\cleardoublepage

\chapter*{How to Read Mathematics}

Reading mathematics is \textbf{not} the same as reading a
novel. To read mathematics you need: 
\begin{enumerate}
\item A pen. 
\item Plenty of blank paper. 
\item A willingness to write things down.
\end{enumerate}
As you read mathematics, you must work along side of the text
itself. You must \textbf{write} down each expression, \textbf{sketch}
each graph, and \textbf{think} about what you are doing. You should
work examples and fill-in the details. This is not an easy task, it is
in fact \textbf{hard} work. However, mathematics is not a passive
endeavor. You, the reader, must become a doer of mathematics.





%%
% Start the main matter (normal chapters)
%\mainmatter

\input limits

\input derivatives

%\input chapter01
%\input chapter02
%\input chapter03
%\input chapter04
%\input chapter05
%\input chapter06

%\input chapter07
%\input chapter08
%\input chapter09
%\input chapter10
%\input chapter11




%\bibliography{sample-handout}
%\bibliographystyle{plainnat}

%\chapter*{Answers to selected exercises}
%\input{\jobname.ans}
\backmatter
%\addcontentsline{toc}{chapter}{Index}
\printindex


\end{document}


