\section{The first derivative test} {}{}
\nobreak
The method of the previous section for deciding whether there is a
local maximum or minimum at a critical value is not always
convenient. We can instead use information about the derivative
$f'(x)$ to decide; since we have already had to compute the derivative
to find the critical values, there is often relatively little extra
work involved in this method.

How can the derivative tell us whether there is a maximum, minimum, or
neither at a point? Suppose that $f'(a)=0$. If there is a local
maximum when $x=a$, the function must be lower near $x=a$ than it is
right at $x=a$. If the derivative exists near $x=a$, this means
$f'(x)>0$ when $x$ is near $a$ and $x<a$, because the function must
``slope up'' just to the left of $a$. Similarly, $f'(x)<0$ when $x$ is
near $a$ and $x>a$, because $f$ slopes down from the local maximum as
we move to the right. Using the same reasoning, if there is a local
minimum at $x=a$, the derivative of $f$ must be negative just to the
left of $a$ and positive just to the right. If the derivative exists
near $a$ but does not change from positive to negative or negative to
positive, that is, it is positive on both sides or negative on both
sides, then there is neither a maximum nor minimum when $x=a$.
See the first graph in figure~\xrefn{fig:max and min points}
and the graph in figure~\xrefn{fig:non extremum}
for examples.

\begin{example} Find all local maximum and minimum points for $f(x)=\sin x+\cos
x$ using the first derivative test.  The derivative is $f'(x)=\cos
x-\sin x$ and from example~\xrefn{example:max and min} the critical
values we need to consider are $\pi/4$ and $5\pi/4$.

The graphs of $\sin x$ and $\cos x$ are shown in figure~\xrefn{fig:sin
  and cos}. Just to the left of $\pi/4$ the cosine is larger than the
  sine, so $f'(x)$ is positive; just to the right the cosine is
  smaller than the sine, so $f'(x)$ is negative. This means there is a
  local maximum at $\pi/4$. Just to the left of $5\pi/4$ the cosine is
  smaller than the sine, and to the right the cosine is larger than
  the sine. This means that the derivative $f'(x)$ is negative to the
  left and positive to the right, so $f$ has a local minimum at
  $5\pi/4$.
\end{example}

% BADBAD
% \figure
% \vbox{\beginpicture
% \normalgraphs
% \ninepoint
% \setcoordinatesystem units <1.5truecm,1.5truecm>
% \setplotarea x from 0 to 6.28, y from -1 to 1
% \axis left shiftedto x=0 /
% \axis bottom shiftedto y=0 ticks withvalues {${\pi\over4}$}
%       {$5\pi\over4$} / at 0.7853981635 3.926990818 / /
% \setquadratic
% \plot
% 0.000 0.000 
% 0.063 0.063 0.126 0.125 0.188 0.187 0.251 0.249 
% 0.314 0.309 0.377 0.368 0.440 0.426 0.503 0.482 0.565 0.536 
% 0.628 0.588 0.691 0.637 0.754 0.685 0.817 0.729 0.880 0.771 
% 0.942 0.809 1.005 0.844 1.068 0.876 1.131 0.905 1.194 0.930 
% 1.257 0.951 1.319 0.969 1.382 0.982 1.445 0.992 1.508 0.998 
% 1.571 1.000 1.634 0.998 1.696 0.992 1.759 0.982 1.822 0.969 
% 1.885 0.951 1.948 0.930 2.011 0.905 2.073 0.876 2.136 0.844 
% 2.199 0.809 2.262 0.771 2.325 0.729 2.388 0.685 2.450 0.637 
% 2.513 0.588 2.576 0.536 2.639 0.482 2.702 0.426 2.765 0.368 
% 2.827 0.309 2.890 0.249 2.953 0.187 3.016 0.125 3.079 0.063 
% 3.142 0.000 3.204 -0.063 3.267 -0.125 3.330 -0.187 3.393 -0.249 
% 3.456 -0.309 3.519 -0.368 3.581 -0.426 3.644 -0.482 3.707 -0.536 
% 3.770 -0.588 3.833 -0.637 3.896 -0.685 3.958 -0.729 4.021 -0.771 
% 4.084 -0.809 4.147 -0.844 4.210 -0.876 4.273 -0.905 4.335 -0.930 
% 4.398 -0.951 4.461 -0.969 4.524 -0.982 4.587 -0.992 4.650 -0.998 
% 4.712 -1.000 4.775 -0.998 4.838 -0.992 4.901 -0.982 4.964 -0.969 
% 5.027 -0.951 5.089 -0.930 5.152 -0.905 5.215 -0.876 5.278 -0.844 
% 5.341 -0.809 5.404 -0.771 5.466 -0.729 5.529 -0.685 5.592 -0.637 
% 5.655 -0.588 5.718 -0.536 5.781 -0.482 5.843 -0.426 5.906 -0.368 
% 5.969 -0.309 6.032 -0.249 6.095 -0.187 6.158 -0.125 6.220 -0.063 
% 6.283 0.000 /
% \plot
% 0.000 1.000 0.063 0.998 0.126 0.992 0.188 0.982 0.251 0.969 
% 0.314 0.951 0.377 0.930 0.440 0.905 0.503 0.876 0.565 0.844 
% 0.628 0.809 0.691 0.771 0.754 0.729 0.817 0.685 0.880 0.637 
% 0.942 0.588 1.005 0.536 1.068 0.482 1.131 0.426 1.194 0.368 
% 1.257 0.309 1.319 0.249 1.382 0.187 1.445 0.125 1.508 0.063 
% 1.571 0.000 1.634 -0.063 1.696 -0.125 1.759 -0.187 1.822 -0.249 
% 1.885 -0.309 1.948 -0.368 2.011 -0.426 2.073 -0.482 2.136 -0.536 
% 2.199 -0.588 2.262 -0.637 2.325 -0.685 2.388 -0.729 2.450 -0.771 
% 2.513 -0.809 2.576 -0.844 2.639 -0.876 2.702 -0.905 2.765 -0.930 
% 2.827 -0.951 2.890 -0.969 2.953 -0.982 3.016 -0.992 3.079 -0.998 
% 3.142 -1.000 3.204 -0.998 3.267 -0.992 3.330 -0.982 3.393 -0.969 
% 3.456 -0.951 3.519 -0.930 3.581 -0.905 3.644 -0.876 3.707 -0.844 
% 3.770 -0.809 3.833 -0.771 3.896 -0.729 3.958 -0.685 4.021 -0.637 
% 4.084 -0.588 4.147 -0.536 4.210 -0.482 4.273 -0.426 4.335 -0.368 
% 4.398 -0.309 4.461 -0.249 4.524 -0.187 4.587 -0.125 4.650 -0.063 
% 4.712 0.000 4.775 0.063 4.838 0.125 4.901 0.187 4.964 0.249 
% 5.027 0.309 5.089 0.368 5.152 0.426 5.215 0.482 5.278 0.536 
% 5.341 0.588 5.404 0.637 5.466 0.685 5.529 0.729 5.592 0.771 
% 5.655 0.809 5.718 0.844 5.781 0.876 5.843 0.905 5.906 0.930 
% 5.969 0.951 6.032 0.969 6.095 0.982 6.158 0.992 6.220 0.998 
% 6.283 1.000 /
% \endpicture}
% \figrdef{fig:sin and cos}
% \endfigure{The sine and cosine.}

\begin{exercises}
In 1--13,
find all critical points and identify them as
local maximum points, local minimum points, or neither.

\twocol

\begin{exercise} $\ds y=x^2-x$ 
\begin{answer} min at $x=1/2$
\end{answer}\end{exercise}

\begin{exercise} $\ds y=2+3x-x^3$ 
\begin{answer} min at $x=-1$, max at $x=1$
\end{answer}\end{exercise}

\begin{exercise} $\ds y=x^3-9x^2+24x$
\begin{answer} max at $x=2$, min at $x=4$
\end{answer}\end{exercise}

\begin{exercise} $\ds y=x^4-2x^2+3$ 
\begin{answer} min at $x=\pm 1$, max at $x=0$.
\end{answer}\end{exercise}

\begin{exercise} $\ds y=3x^4-4x^3$
\begin{answer} min at $x=1$
\end{answer}\end{exercise}

\begin{exercise} $\ds y=(x^2-1)/x$
\begin{answer} none
\end{answer}\end{exercise}

\begin{exercise} $\ds y=3x^2-(1/x^2)$ 
\begin{answer} none
\end{answer}\end{exercise}

\begin{exercise} $y=\cos(2x)-x$ 
\begin{answer} min at $x=7\pi/12+k\pi$, max at $x=-\pi/12+k\pi$, for integer $k$.
\end{answer}\end{exercise}

\begin{exercise}
$\ds f(x) = (5-x)/(x+2)$
\begin{answer} none
\end{answer}\end{exercise}

\begin{exercise} $\ds f(x) = |x^2 - 121|$
\begin{answer} max at $x=0$, min at $x=\pm 11$
\end{answer}\end{exercise}

\begin{exercise} $\ds f(x) = x^3/(x+1)$
\begin{answer} min at $x=-3/2$, neither at $x=0$
\end{answer}\end{exercise}

\begin{exercise} $\ds f(x)= \begin{cases}
x^2 \sin(1/x)  & x\neq 0  \\
 0  & x=0 \end{cases}$
\end{exercise}

\begin{exercise} $\ds f(x) = \sin ^2 x$
\begin{answer} min at $n\pi$, max at $\pi/2+n\pi$
\end{answer}\end{exercise}

\endtwocol
\bsk
\begin{exercise} Find the maxima and minima of $f(x)=\sec x$.
\begin{answer} min at $2n\pi$, max at $(2n+1)\pi$
\end{answer}\end{exercise}

\begin{exercise}  Let $\ds f(\theta) = \cos^2(\theta) -
 2\sin(\theta)$.  Find the intervals where $f$ is increasing and the
 intervals where $f$ is decreasing in $[0,2\pi]$.  Use this
 information to classify the critical points of $f$ as either local
 maximums, local minimums, or neither.
\begin{answer} min at $\pi/2+2n\pi$, max at $3\pi/2+2n\pi$
\end{answer}\end{exercise}

\begin{exercise} Let $r>0$. Find the local
maxima and minima of the function $\ds f(x)
=\sqrt{r^2 -x^2 }$ on its domain $[-r,r]$.
\end{exercise}

\begin{exercise} Let $\ds f(x) =a x^2 + bx + c$ with $a\neq 0$. Show that $f$
has exactly one critical point using the first derivative test. Give
conditions on $a$ and $b$ which guarantee that the critical point will
be a maximum. It is possible to see this without using calculus at
all; explain.
\end{exercise}

\end{exercises}

