\chapter{Limits}


\section{The Basic Ideas of Limits}{}{}



Consider the function:
\[
f(x) = \frac{x^2 - 3x + 2}{x-2}
\]
While $f(x)$ is undefined at $x=2$, we can still plot $f(x)$ at other
values, see Figure~\ref{plot:(x^2 - 3x + 2)/(x-2)}. Examining
Table~\ref{table:(x^2 - 3x + 2)/(x-2)}, we see that as $x$ approaches
$2$, $f(x)$ approaches $1$. We write this: 
\[
\text{As $x \to 2$, $f(x) \to 1$}\qquad\text{or}\qquad \lim_{x\to 2} f(x) = 1.
\]
Intuitively, $\lim_{x\to a} f(x) = L$ when the value of $f(x)$ can
be made arbitrarily close to $L$ by making $x$ sufficiently close, but
not equal to, $a$.  This leads us to the formal definition of a
\textit{limit}.
\begin{marginfigure}[-5in]
\begin{tikzpicture}
	\begin{axis}[
            domain=-2:4,
            axis lines =middle, xlabel=$x$, ylabel=$y$,
            every axis y label/.style={at=(current axis.above origin),anchor=south},
            every axis x label/.style={at=(current axis.right of origin),anchor=west},
            grid=both,
            grid style={dashed, gridColor},
            xtick={-2,...,4},
            ytick={-3,...,3},
          ]
	  \addplot [very thick, penColor, smooth] {x-1};
          \addplot[color=penColor,fill=background,only marks,mark=*] coordinates{(2,1)};  %% open hole
        \end{axis}
\end{tikzpicture}
\caption{A plot of $f(x)=\protect\frac{x^2 - 3x + 2}{x-2}$.}
\label{plot:(x^2 - 3x + 2)/(x-2)}
\end{marginfigure}

\begin{margintable}[-1in]
\[
\begin{tchart}{ll}
 x & f(x) \\ \hline
 1.7 &  0.7 \\
 1.9 &  0.9 \\
 1.99 &  0.99 \\
 1.999 &  0.999 \\
  2 &  \text{undefined}
\end{tchart}\qquad
\begin{tchart}{ll}
 x & f(x) \\ \hline
  2 & \text{undefined}\\
 2.001&  1.001\\
 2.01&  1.01\\
 2.1 &  1.1 \\
 2.3 &  1.3 \\
\end{tchart}
\]
\caption{Values of $f(x)=\protect\frac{x^2 - 3x + 2}{x-2}$.}
\label{table:(x^2 - 3x + 2)/(x-2)}
\end{margintable}

\marginnote[1in]{Equivalently, $\lim_{x\to a}f(x)=L$, if for
  every $\epsilon>0$ there is a $\delta > 0$ so that whenever $x\ne a$
  and $a- \delta < x < a+ \delta$, we have $L-\epsilon<
  f(x)<L+\epsilon$.}
\begin{definition}\label{def:limit}\index{limit!definition}
The \textbf{limit} of $f(x)$ as $x$ goes to $a$ is $L$,
\[
\lim_{x\to a}f(x)=L,
\] 
if for every $\epsilon>0$ there is a $\delta > 0$ so that whenever
\[
0 < |x-a| < \delta, \qquad\text{we have} \qquad |f(x)-L|<\epsilon.
\] 
If no such value of $L$ can be
found, then we say that $\lim_{x\to a}f(x)$ \textbf{does not exist}.
\end{definition}

In Figure~\ref{figure:epsilon-delta}, we see a geometric
interpretation of this definition.

\begin{figure}
\begin{tikzpicture}
	\begin{axis}[
            domain=0:2, 
            axis lines =left, xlabel=$x$, ylabel=$y$,
            every axis y label/.style={at=(current axis.above origin),anchor=south},
            every axis x label/.style={at=(current axis.right of origin),anchor=west},
            xtick={0.7,1,1.3}, ytick={3,4,5},
            xticklabels={$a-\delta$,$a$,$a+\delta$}, yticklabels={$L-\epsilon$,$L$,$L+\epsilon$},
            axis on top,
          ]          
          \addplot [color=textColor, fill=fill2, smooth, domain=(0:1.570)] {5} \closedcycle;
          \addplot [color=textColor, dashed, fill=fill1, smooth, domain=(0:1.3)] {4.537} \closedcycle;
          \addplot [color=textColor, dashed, fill=fill2, domain=(0:.7)] {3.283} \closedcycle;       
          \addplot [textColor, very thick, smooth, domain=(0:1)] {4};
          \addplot [color=textColor, fill=background, smooth, domain=(0:0.607)] {3} \closedcycle;
	  \addplot [draw=none, fill=background, smooth] {x*(x-2)^2+3*x} \closedcycle;
          \addplot [fill=fill1, draw=none, domain=.7:1.3] {x*(x-2)^2+3*x} \closedcycle;
          \addplot [textColor, very thick] plot coordinates {(1,0) (1,4)};
          \addplot [textColor] plot coordinates {(.7,0) (.7,3.283)};
          \addplot [textColor] plot coordinates {(1.3,0) (1.3,4.537)};
	  \addplot [very thick,penColor, smooth] {x*(x-2)^2+3*x};
        \end{axis}
\end{tikzpicture}
\caption{A geometric interpretation of the
  $(\epsilon,\delta)$-criterion for limits.  If $0<|x-a|<\delta$, then we have that $a
  -\delta < x < a+\delta$. In our diagram, we see that for all such
  $x$ we are sure to have $L - \epsilon< f(x) < L+\epsilon$, and hence
  $|f(x) - L|<\epsilon$.}
\label{figure:epsilon-delta}
\end{figure}
Limits need not exist, let's examine two cases of this.

\begin{example}
Let $f(x) = \lfloor x\rfloor$. Explain why the limit
\[
\lim_{x\to 2} f(x)
\]
does not exist.
\end{example}

\begin{marginfigure}[0in]
\begin{tikzpicture}
	\begin{axis}[
            domain=-2:4,
            axis lines =middle, xlabel=$x$, ylabel=$y$,
            every axis y label/.style={at=(current axis.above origin),anchor=south},
            every axis x label/.style={at=(current axis.right of origin),anchor=west},
            clip=false,
            %axis on top,
          ]
          \addplot [draw=none,, fill=fill1, domain=(0:2)] {2} \closedcycle; %% fill for epsilon  
          \addplot [color=textColor, dashed, domain=(0:2)] {2};  %% bndry for epsilon  
          \addplot [draw=none, fill=background, domain=(0:1.8)] {1} \closedcycle;
          \addplot [color=textColor, dashed, domain=(0:1)] {1};  %% bndry for epsilon  
          \addplot [draw=none, fill=fill1, domain=(1.8:2)] {1} \closedcycle;
          \addplot [draw=none, fill=fill1, domain=(2:2.2)] {2} \closedcycle;
          \addplot [textColor, very thick] plot coordinates {(2,0) (2,2)};
          \addplot [textColor] plot coordinates {(1.8,0) (1.8,1)};
          \addplot [textColor] plot coordinates {(2.2,0) (2.2,2)};
          \addplot [textColor, very thin, domain=(0:2.3)] {0}; % puts the axis back, axis on top clobbers our open holes
          \addplot [textColor, very thin] plot coordinates {(0,0) (0,2)}; % puts the axis back, axis on top clobbers our open holes
	  \addplot [very thick, penColor, domain=(-2:-1)] {-2};
          \addplot [very thick, penColor, domain=(-1:0)] {-1};
          \addplot [very thick, penColor, domain=(0:1)] {0};
          \addplot [very thick, penColor, domain=(1:2)] {1};
          \addplot [very thick, penColor, domain=(2:3)] {2};
          \addplot [very thick, penColor, domain=(3:4)] {3};
          \addplot[color=penColor,fill=penColor,only marks,mark=*] coordinates{(-2,-2)};  %% closed hole          
          \addplot[color=penColor,fill=penColor,only marks,mark=*] coordinates{(-1,-1)};  %% closed hole          
          \addplot[color=penColor,fill=penColor,only marks,mark=*] coordinates{(0,0)};  %% closed hole          
          \addplot[color=penColor,fill=penColor,only marks,mark=*] coordinates{(1,1)};  %% closed hole          
          \addplot[color=penColor,fill=penColor,only marks,mark=*] coordinates{(2,2)};  %% closed hole  
          \addplot[color=penColor,fill=penColor,only marks,mark=*] coordinates{(3,3)};  %% closed hole                  
          \addplot[color=penColor,fill=background,only marks,mark=*] coordinates{(-1,-2)};  %% open hole
          \addplot[color=penColor,fill=background,only marks,mark=*] coordinates{(0,-1)};  %% open hole
          \addplot[color=penColor,fill=background,only marks,mark=*] coordinates{(1,0)};  %% open hole
          \addplot[color=penColor,fill=background,only marks,mark=*] coordinates{(2,1)};  %% open hole
          \addplot[color=penColor,fill=background,only marks,mark=*] coordinates{(3,2)};  %% open hole
          \addplot[color=penColor,fill=background,only marks,mark=*] coordinates{(4,3)};  %% open hole
        \end{axis}
\end{tikzpicture}
\caption{A plot of $f(x)=\lfloor x\rfloor$. Note, no matter which
  $\delta>0$ is chosen, we can only at best bound $f(x)$ in the
  interval $[1,2]$.}
\label{plot:greatest-integer}
\end{marginfigure}
\begin{solution}
The function $\lfloor x \rfloor$ is the function that returns the
greatest integer less than or equal to $x$. Since $f(x)$ is defined
for all real numbers, one might be tempted to think that the limit
above is simply $f(2) = 2$. However, this is not the case.  If $x<2$,
then $f(x) =1$. Hence if $\epsilon = .5$, we can \textbf{always} find
a value for $x$ (just to the left of $2$) such that
\[
0< |x -2|< \delta, \qquad\text{where} \qquad \epsilon < |f(x)-2|.
\]
On the other hand, $\lim_{x\to 2} f(x)\ne 1$, as in this case if
$\epsilon=.5$, we can \textbf{always} find a value for $x$ (just to
the right of $2$) such that
\[
0<|x- 2|<\delta, \qquad\text{where} \qquad  \epsilon<|f(x)-1|.
\]
We've illustrated this in
Figure~\ref{plot:greatest-integer}. Moreover, no matter what value one
chooses for $\lim_{x\to 2} f(x)$, we will always have a similar
issue.
\end{solution}
\marginnote[-1in]{With the example of $f(x) = \lfloor x \rfloor$, we
  see that taking limits is truly different from evaluating
  functions.}



Limits may not exist even if the formula for the function looks
innocent.

\begin{example}
Let  $f(x) = \sin\left(\frac{1}{x}\right)$. Explain why the limit
\[
\lim_{x\to 0} f(x)
\]
does not exist.
\end{example}
\begin{solution}
In this case $f(x)$ oscillates ``wildly'' as $x$ approaches $0$, see
Figure~\ref{plot:sin 1/x}. In fact, one can show that for any given
  $\delta$, There is a value for $x$ in the interval
\[
0-\delta < x < 0+\delta
\]
such that $f(x)$ is \textbf{any} value in the interval $[-1,1]$. Hence
the limit does not exist.
\end{solution}
\begin{marginfigure}[-1in]
\begin{tikzpicture}
	\begin{axis}[
            domain=-.2:.2,    
            samples=500,
            axis lines =middle, xlabel=$x$, ylabel=$y$,
            yticklabels = {}, 
            every axis y label/.style={at=(current axis.above origin),anchor=south},
            every axis x label/.style={at=(current axis.right of origin),anchor=west},
            clip=false,
          ]
	  \addplot [very thick, penColor, smooth, domain=(-.2:-.02)] {sin(deg(1/x))};
          \addplot [very thick, penColor, smooth, domain=(.02:.2)] {sin(deg(1/x))};
	  \addplot [color=penColor, fill=penColor, very thick, smooth,domain=(-.02:.02)] {1} \closedcycle;
          \addplot [color=penColor, fill=penColor, very thick, smooth,domain=(-.02:.02)] {-1} \closedcycle;
        \end{axis}
\end{tikzpicture}
\caption{A plot of $f(x)=\protect\sin\left(\frac{1}{x}\right)$.}
\label{plot:sin 1/x}
\end{marginfigure}

Sometimes the limit of a function exists from one side or the other
(or both) even though the limit does not exist. Since it is useful to
be able to talk about this situation, we introduce the concept of a
\textit{one-sided limit}\index{one-sided limit}:

\begin{definition} We say that the \textbf{limit} of $f(x)$ as $x$ goes to $a$ from the \textbf{left} is $L$,
\[
\lim_{x\to a-}f(x)=L
\]
if for every $\epsilon>0$ there is a $\delta > 0$ so that whenever $x< a$ and 
\[
a-\delta < x \qquad\text{we have}\qquad |f(x)-L|<\epsilon.
\]

We say that the \textbf{limit} of $f(x)$ as $x$ goes to $a$ from the \textbf{right} is $L$,
\[
\lim_{x\to a+}f(x)=L
\] 
if for every $\epsilon>0$ there is a $\delta > 0$ so that whenever $x > a$ and 
\[
x<a+\delta \qquad\text{we have}\qquad |f(x)-L|<\epsilon.
\]
\end{definition}
\marginnote[-1in]{Limits from the left, or from the right, are collectively called \textbf{one-sided limits}.}


\begin{example}
Let $f(x) = \lfloor x\rfloor$. Discuss
\[
\lim_{x\to 2-} f(x), \qquad \lim_{x\to 2+} f(x), \qquad\text{and}\qquad\lim_{x\to 2} f(x).
\]
\end{example}
\begin{solution}
From the plot of $f(x)$, see Figure~\ref{plot:greatist-integer}, we see that 
\[
\lim_{x\to 2-} f(x)=1, \qquad\text{and}\quad \lim_{x\to 2+} f(x) = 2.
\]
Since these limits are different, $\lim_{x\to 2} f(x)$ does not exist.
\end{solution}



\begin{exercises}
\begin{exercise} Evaluate the expressions by referencing the plot in Figure~\ref{plot:piecewise-exercise}.
\begin{marginfigure}
\begin{tikzpicture}
	\begin{axis}[
            domain=-4:6, xmin=-4, xmax=6, ymin=-3,ymax=10,    
            unit vector ratio*=1 1 1,
            axis lines =middle, xlabel=$x$, ylabel=$y$,
            every axis y label/.style={at=(current axis.above origin),anchor=south},
            every axis x label/.style={at=(current axis.right of origin),anchor=west},
            xtick={-4,...,6}, ytick={-3,...,10},
            xticklabels={-4,,-2,,0,,2,,4,,6}, yticklabels={,-2,,0,,2,,4,,6,,8,,10},
            grid=major,
            grid style={dashed, gridColor},
          ]
	  \addplot [very thick, penColor, smooth, domain=(-4:-2)] {6};
	  \addplot [very thick, penColor, smooth, domain=(-2:0)] {x^2-2};
          \addplot [very thick, penColor, smooth, domain=(0:2)] {(x-1)^3+3*(x-1)+3};
          \addplot [very thick, penColor, smooth, domain=(2:6)] {(x-4)^3+8};
          \addplot[color=penColor,fill=background,only marks,mark=*] coordinates{(-2,6)};  %% open hole
          \addplot[color=penColor,fill=background,only marks,mark=*] coordinates{(-2,2)};  %% open hole
          \addplot[color=penColor,fill=background,only marks,mark=*] coordinates{(0,-2)};  %% open hole
          \addplot[color=penColor,fill=background,only marks,mark=*] coordinates{(0,-1)};  %% open hole
          \addplot[color=penColor,fill=background,only marks,mark=*] coordinates{(2,0)};  %% open hole
          \addplot[color=penColor,fill=penColor,only marks,mark=*] coordinates{(-2,8)};  %% closed hole
          \addplot[color=penColor,fill=penColor,only marks,mark=*] coordinates{(0,-1.5)};  %% closed hole
          \addplot[color=penColor,fill=penColor,only marks,mark=*] coordinates{(2,7)};  %% closed hole
        \end{axis}
\end{tikzpicture}
\caption{A plot of $f(x)$, a piecewise defined function.}
\label{plot:piecewise-exercise}
\end{marginfigure}
\begin{enumerate}
\begin{multicols}{3}
\item $\lim_{x\to 4} f(x)$  
\item $\lim_{x\to -3} f(x)$  
\item $\lim_{x\to 0} f(x)$ 
\item $\lim_{x\to 0-} f(x)$  
\item $\lim_{x\to 0+} f(x)$  
\item $f(-2)$  
\item $\lim_{x\to 2-} f(x)$  
\item $\lim_{x\to -2-} f(x)$  
\item $\lim_{x\to 0} f(x+1)$  
\item $f(0)$ 
\item $\lim_{x\to 1-} f(x-4)$  
\item $\lim_{x\to 0+} f(x-2)$
\end{multicols}  
\end{enumerate}
\begin{answer}
 (a) $8$, (b) $6$, (c) DNE,
 (d) $-2$, (e) $-1$, (f) $8$,
 (g) $7$, (h) $6$, (i) $3$,
 (j) $-3/2$, (k) $6$, (l) $2$
\end{answer}
\end{exercise}


\begin{exercise} 
Use a table and a calculator to estimate $\lim_{x\to 0}
\frac{\sin(x)}{x}$.
\begin{answer}
  $1$
\end{answer}
\end{exercise}

\begin{exercise} 
Use a table and a calculator to estimate $\lim_{x\to 0} \frac{\sin(2x)}{x}$.
\begin{answer}
  $2$
\end{answer}
\end{exercise}

\begin{exercise} 
Use a table and a calculator to estimate $\lim_{x\to 0} \frac{x}{\sin\left(\frac{x}{3}\right)}$.
\begin{answer}
  $3$
\end{answer}
\end{exercise}

\begin{exercise} 
Use a table and a calculator to estimate $\lim_{x\to 0}\frac{\tan(3x)}{\tan(5x)}$.
\begin{answer}
  $3/5$
\end{answer}
\end{exercise}

\begin{exercise} 
Use a table and a calculator to estimate $\lim_{x\to 0}
\frac{2^x-1}{x}$.
\begin{answer}
  $0.6931\approx\ln(2)$
\end{answer}
\end{exercise}

\begin{exercise} 
Use a table and a calculator to estimate $\lim_{x\to 0} (1+x)^{1/x}$. 
\begin{answer}
  $2.718 \approx e$
\end{answer}
\end{exercise}



\begin{exercise} 
Sketch a plot of $f(x) = \dfrac{x}{|x|}$ and explain why $\lim_{x\to
  0} \frac{x}{|x|}$ does not exist.
\begin{answer}
  Consider what happens when $x$ is near zero and positive, as compared to when $x$ is near zero and negative.
\end{answer}
\end{exercise}



\begin{exercise} 
Let $f(x) = \sin\left(\dfrac{\pi}{x}\right)$. Construct three tables
of the following form
\[
\begin{array}{l|l}
 x & f(x) \\ \hline
 0.d &   \\
 0.0d &  \\
 0.00d &   \\
 0.000d &  
\end{array}
\]
where $d = 1,3,7$. What do you notice? How do you reconcile the
entries in your tables with the value of $\lim_{x\to 0} f(x)$?
\begin{answer}
  The limit does not exist, so it is not surprising that the resulting values are so different.
\end{answer}
\end{exercise}


\begin{exercise}
In the theory of special relativity, a moving clock ticks slower than
a stationary observer's clock. If the stationary observer records that
$t_s$ seconds have passed, then the clock moving at velocity $v$ has
recorded that
\[
t_v = t_s \sqrt{1 - v^2/c^2}
\]
seconds have passed, where $c$ is the speed of light. What happens as $v\to c$ from below?
\begin{answer}
  When $v$ approaches $c$ from below, then $t_v$ approaches zero---meaning that one second to the stationary observations seems like very little time at all for our traveler.
\end{answer}
\end{exercise}


\end{exercises}















\section{Limits by the Definition}


Now we are going to get our hands dirty, and really use the definition
of a limit.\marginnote{Recall, $\lim_{x\to a}f(x)=L$, if for every
  $\epsilon>0$ there is a $\delta > 0$ so that whenever $0< |x -a|<
  \delta$, we have $|f(x)- L|<\epsilon$.}

\begin{marginfigure}[.5in]
\begin{tikzpicture}
	\begin{axis}[
            domain=1:3, 
            axis lines =left, xlabel=$x$, ylabel=$y$,
            every axis y label/.style={at=(current axis.above origin),anchor=south},
            every axis x label/.style={at=(current axis.right of origin),anchor=west},
            xtick={1.8,2,2.2}, ytick={3,4,5},
            xticklabels={$2-\delta$,$2$,$2+\delta$}, yticklabels={$4-\epsilon$,$4$,$4+\epsilon$},
            axis on top,
          ]          
          \addplot [color=textColor, fill=fill2, smooth, domain=(1:2.236)] {5} \closedcycle;
          \addplot [color=textColor, dashed, fill=fill1, domain=(1:2.2)] {4.84} \closedcycle;       
          \addplot [color=textColor, dashed, fill=fill2, domain=(1:1.8)] {3.24} \closedcycle;       
          \addplot [textColor, very thick, smooth, domain=(1:2)] {4};
          \addplot [color=textColor, fill=background, smooth, domain=(1:1.8)] {3} \closedcycle;
	  \addplot [draw=none, fill=background, smooth] {x^2} \closedcycle;
          \addplot [fill=fill1, draw=none, domain=1.8:2.2] {x^2} \closedcycle;
          \addplot [textColor, very thick] plot coordinates {(2,0) (2,4)};
          \addplot [textColor] plot coordinates {(1.8,0) (1.8,3.24)};
          \addplot [textColor] plot coordinates {(2.2,0) (2.2,4.84)};
	  \addplot [very thick,penColor, smooth] {x^2};
        \end{axis}
\end{tikzpicture}
\label{plot:x^2 lim dfn}
\caption{The $(\epsilon,\delta)$-criterion for $\lim_{x\to 2}
  x^2=4$. Here $\delta= \min\left(\dfrac{\epsilon}{5},1\right)$.}
\end{marginfigure}
\begin{example} Show that $\lim_{x\to 2} x^2=4$.
\end{example}
\begin{solution}
We want to show that for any given $\epsilon>0$, we can find a
$\delta>0$ such that
\[
|x^2 -4|<\epsilon
\]
whenever $0<|x - 2|<\delta$. Start by factoring the left-hand side of
the inequality above
\[
|x+2||x-2|<\epsilon.
\]
Since we are going to assume that $0<|x - 2|<\delta$, we will focus on
the factor $|x+2|$. Since $x$ is assumed to be close to $2$, suppose that $x\in[1,3]$. In this case
\[
|x+2| \le 3+2 = 5,
\]
and so we want
\begin{align*}
5\cdot |x-2| &< \epsilon\\
|x-2| &< \frac{\epsilon}{5}
\end{align*}
Recall, we assumed that $x\in[1,3]$, which is equivalent to
$|x-2|<1$. Hence we must set $\delta = \min\left(\dfrac{\epsilon}{5},1\right)$.
\end{solution}


When dealing with limits of polynomials, the general strategy is
always the same. Let $p(x)$ be a polynomial. If showing
\[
\lim_{x\to a} p(x) = L,
\]
one must first factor out $|x-a|$ from $|p(x) - L|$. Next bound $x\in
[a-1,a+1]$ and estimate the largest possible value of
\[
\left|\frac{p(x) -L}{x-a}\right|
\]
for $x\in[a-1,a+1]$. Call this estimation $M$. Finally, one must set
$\delta = \min\left(\frac{\epsilon}{M}, 1\right)$.

As you work with limits, you find that you need to do the same
procedures again and again. The next theorems will expedite this
process.
\begin{theorem}[Limit Product Law]\label{theorem:limit-product} 
Suppose $\lim_{x\to a} f(x)=L$ and $\lim_{x\to a}g(x)=M$. Then
\[
\lim_{x\to a} f(x)g(x) = LM.
\]
\end{theorem}

\marginnote[1.5in]{We will use this same trick again of ``adding $0$'' in
  the proof of Theorem~\ref{theorem:product-rule}.}
\marginnote[.5in]{This is all straightforward except perhaps for the
  ``$\le$''. This follows from the \textit{Triangle
    Inequality}\index{triangle inequality}. The \textbf{Triangle
    Inequality} states: If $a$ and $b$ are any real numbers then
  $|a+b|\le |a|+|b|$.}
\begin{proof} 
Given any $\epsilon$ we need to find a $\delta$ such that
\[
0<|x - a|< \delta
\]
implies 
\[
|f(x)g(x)-LM|< \epsilon.  
\]
Here we use an algebraic trick, add $0 = -f(x)M+f(x)M$:
\begin{align*}
|f(x)g(x)-LM| &= |f(x)g(x){\color{penColor2}-f(x)M+f(x)M}-LM| \\
&=|f(x)(g(x)-M)+(f(x)-L)M| \\
&\le |f(x)(g(x)-M)|+|(f(x)-L)M| \\
&=|f(x)||g(x)-M|+|f(x)-L||M|.
\end{align*}
Since $\lim_{x\to a}f(x) =L$, there is a value $\delta_1$ so that
$0<|x-a|<\delta_1$ implies $|f(x)-L|<|\epsilon/(2M)|$. This means that
$0<|x-a|<\delta_1$ implies $|f(x)-L||M|< \epsilon/2$. 
\[
|f(x)g(x)-LM|\le|f(x)||g(x)-M|+\underbrace{|f(x)-L||M|}_{<\dfrac{\epsilon}{2}}.
\]
If we can make $|f(x)||g(x)-M|<\epsilon/2$, then we'll be done. We can
make $|g(x)-M|$ smaller than any fixed number by making $x$ close
enough to $a$. Unfortunately, $\epsilon/(2f(x))$ is not a fixed number
since $x$ is a variable.

Here we need another trick. We can find a $\delta_2$ so that
$|x-a|<\delta_2$ implies that $|f(x)-L|<1$, meaning that $L-1 < f(x) <
L+1$. This means that $|f(x)|<N$, where $N$ is either $|L-1|$ or
$|L+1|$, depending on whether $L$ is negative or positive. The
important point is that $N$ doesn't depend on $x$. Finally, we know
that there is a $\delta_3$ so that $0<|x-a|<\delta_3$ implies
$|g(x)-M|<\epsilon/(2N)$. Now we're ready to put everything
together. Let $\delta$ be the smallest of $\delta_1$, $\delta_2$, and
$\delta_3$. Then $|x-a|<\delta$ implies that
\[
|f(x)g(x)-LM|\le\underbrace{|f(x)|}_{<N}\underbrace{|g(x)-M|}_{<\dfrac{\epsilon}{2N}}+\underbrace{|f(x)-L||M|}_{<\dfrac{\epsilon}{2}}.
\]
so
\begin{align*}
|f(x)g(x)-LM|&\le|f(x)||g(x)-M|+|f(x)-L||M| \\
&<N{\epsilon\over 2N}+\left|{\epsilon\over 2M}\right||M| \\
&={\epsilon\over 2}+{\epsilon\over 2}=\epsilon.
\end{align*}
This is just what we needed, so by the definition of a limit,
$\lim_{x\to a}f(x)g(x)=LM$.
\end{proof}



Another useful way to put functions together is
composition\index{composition of functions}. If $f(x)$ and $g(x)$ are
functions, we can form two functions by composition: $f(g(x))$ and
$g(f(x))$. For example, if $f(x)=\sqrt{x}$ and $g(x)=x^2+5$, then
$f(g(x))=\sqrt{x^2+5}$ and $g(f(x))=(\sqrt{x})^2+5=x+5$.  This brings
us to our next theorem.

\marginnote[.5in]{This is sometimes written as \[
\lim_{x\to a}f(g(x)) = \lim_{g(x)\to M}f(g(x)).
\]}
\begin{theorem}[Limit Composition Law]\label{thm:limit of composition}
Suppose that $\lim_{x\to a}g(x)=M$ and $\lim_{x\to M}f(x) = f(M)$. Then
\[
\lim_{x\to a} f(g(x)) = f(M).
\]
\end{theorem}

Note the special form of the condition on $f(x)$: it is not enough to
know that $\lim_{x\to L}f(x)$ exists, though it is a bit tricky to see
why. Consider
\[
f(x) =\begin{cases}
3 & \text{if $x=2$,}\\
4 & \text{if $x\ne 2$.}
\end{cases}
\]
and $g(x) = 2$. Now the conditions of Theorem~\ref{thm:limit of
  composition} are not satisfied, and
\[
\lim_{x\to 1} f(g(x)) = 3 \qquad{but}\qquad \lim_{x\to 2} f(x) = 4.
\]


Many of the most familiar functions do satisfy the conditions of
Theorem~\ref{thm:limit of composition}. For example:

\begin{theorem}[Limit Root Law]
Suppose that $n$ is a positive integer. Then
$$\lim_{x\to a}\root n\of{x} = \root n\of{a},$$
provided that $a$ is positive if $n$ is even.
\label{thm:continuity of roots}
\end{theorem}

This theorem is not too difficult to prove from the definition of limit.








\begin{exercises}


\begin{exercise} 
For each of the following limits, $\lim_{x\to a} f(x) =L$, use a
graphing device to find $\delta$ such that $0<|x -a|<\delta$ implies
that $|f(x)-L|<\epsilon$ where $\epsilon = .1$.
\begin{enumerate}
\begin{multicols}{3}
\item $\lim_{x\to 2}(3x+1) = 7$  
\item $\lim_{x\to 1} (x^2+2) = 3$  
\item $\lim_{x\to \pi} \sin(x) = 0$  
\item $\lim_{x\to 0} \tan(x) = 0$
\item $\lim_{x\to 1} \sqrt{3x+1} = 2$
\item $\lim_{x\to -2} \sqrt{1-4x} = 3$
\end{multicols}  
\end{enumerate}
\begin{answer}
  For these problems, there are many possible values of $\delta$, so we provide an inequality that $\delta$ must satisfy when $\epsilon = 0.1$.\hspace{1em}
  (a)~$\delta < 1/30$,
  (b)~$\delta < \frac{\sqrt{110}}{10} - 1 \approx 0.0488$,
  (c)~$\delta < \arcsin \left( 1/10 \right) \approx 0.1002$,
  (d)~$\delta < \arctan \left( 1/10 \right) \approx 0.0997$
  (e)~$\delta < 13/100$,
  (f)~$\delta < 59/400$
\end{answer}
\end{exercise}

\noindent The next set of exercises are for advanced students and can be
skipped on first reading.

\begin{exercise} 
Use the definition of limits to explain why $\lim_{x\to 0 } x\sin
\left( {1\over x}\right) = 0$.  Hint: Use the fact that $|\sin(a) |< 1
$ for any real number $a$.
\begin{answer}
 Let $\epsilon >0$. Set $\delta = \epsilon$. If $0<|x-0| <\delta$,
 then $|x\cdot 1|<\epsilon$, since $\sin\left(\frac{1}{x}\right)\le
 1$, $|x\sin \left( {1\over x}\right)-0|< \epsilon$.
\end{answer}
\end{exercise}

\begin{exercise} 
Use the definition of limits to explain why $\lim_{x\to 4} (2x-5)
= 3$.
\begin{answer}
  Let $\epsilon > 0$.  Set $\delta = \epsilon/2$.  If $0 < |x - 4| < \delta$, then $|2x - 8| < 2 \delta = \epsilon$, and then because $|2x - 8| = |(2x - 5) - 3|$, we conclude $|(2x - 5) - 3| < \epsilon$.
\end{answer}
\end{exercise}

\begin{exercise} 
Use the definition of limits to explain why $\lim_{x\to -3} (-4x-11)
= 1$.
\begin{answer}
  Let $\epsilon > 0$.  Set $\delta = \epsilon/4$.  If $0 < |x - (-3)| < \delta$, then $|-4x - 12| < 4 \delta = \epsilon$, and then because $|-4x - 12| = |(-4x - 11) - 1|$, we conclude $|(-4x - 11) - 1| < \epsilon$.
\end{answer}
\end{exercise}

\begin{exercise} 
Use the definition of limits to explain why $\lim_{x\to -2} \pi = \pi$.
\begin{answer}
  Let $\epsilon > 0$.  No matter what I choose for $\delta$, if $x$ is within $\delta$ of $-2$, then $\pi$ is within $\epsilon$ of $\pi$.
\end{answer}
\end{exercise}

\begin{exercise} 
Use the definition of limits to explain why $\lim_{x\to -2} \frac{x^2-4}{x+2} = -4$.
\begin{answer}
  As long as $x \neq -2$, we have $\frac{x^2-4}{x+2} = x-2$, and the limit is not sensitive to the value of the function at the point $-2$; the limit only depends on nearby values, so we really want to compute $\lim_{x \to -2} (x-2)$.  Let $\epsilon > 0$.  Set $\delta = \epsilon$.  Then if $0 < |x - (-2)| < \delta$, we have $|(x - 2) - (-4)| < \epsilon$.
\end{answer}
\end{exercise}

\begin{exercise} 
Use the definition of limits to explain why $\lim_{x\to 4} x^3 = 64$.
\begin{answer}
  Let $\epsilon > 0$.   Pick $\delta$ so that $\delta < 1$ and $\delta < \frac{\epsilon}{61}$.  Suppose $0 < |x-4| < \delta$.  Then $4 - \delta < x < 4 + \delta$.  Cube to get $\left( 4 - \delta \right)^3 < x^3 < \left( 4 + \delta \right)^3$.  Expanding the right-side inequality, we get $x^3 < \delta^3 + 12\cdot\delta^2 + 48\cdot\delta + 64 < \delta + 12 \delta + 48 \delta + 64 = 64 + \epsilon$.  The other inequality is similar.
\end{answer}
\end{exercise}

\begin{exercise} 
Use the definition of limits to explain why $\lim_{x\to 1} (x^2+3x-1) = 3$.
\begin{answer}
  Let $\epsilon > 0$.  Pick $\delta$ small enough so that $\delta < \epsilon / 6$ and $\delta < 1$.  Assume $|x - 1| < \delta$, so $6 \cdot |x - 1| < \epsilon$.  Since $x$ is within $\delta < 1$ of $1$, we know $0 < x < 2$.  So $|x+4| < 6$.  Putting it together, $|x+4| \cdot |x-1| < \epsilon$, so $|x^2 + 3x - 4| < \epsilon$, and therefore $|(x^2 + 3x - 1) - 3| < \epsilon$.
\end{answer}
\end{exercise}


\begin{exercise} 
Use the definition of limits to explain why $\lim_{x\to 9} \frac{x-9}{\sqrt{x}-3}
= 6$.
\begin{answer}
  Let $\epsilon > 0$.  Set $\delta = 3\epsilon$.  Assume $0 < |x-9| < \delta$.  Divide both sides by $3$ to get $\frac{|x-9|}{3} < \epsilon$.  Note that $\sqrt{x}+3 > 3$, so $\frac{|x-9|}{\sqrt{x} + 3} < \epsilon$.  This can be rearranged to conclude $\left|\frac{x-9}{\sqrt{x} - 3} - 6\right| < \epsilon$.
\end{answer}
\end{exercise}


\begin{exercise} 
Use the definition of limits to explain why $\lim_{x\to 2}\frac{1}{x}
= \frac{1}{2}$.
\begin{answer}
  Let $\epsilon > 0$.  Set $\delta$ to be the minimum of $2\epsilon$ and $1$.  Assume $x$ is within $\delta$ of $2$, so $|x - 2| < 2 \epsilon$ and $1 < x < 3$.  So $\left| \frac{x-2}{2} \right| < \epsilon$.  Since $1 < x < 3$, we also have $2x > 2$, so $\left| \frac{x-2}{2x} \right| < \epsilon$.  Simplifying, $\left| \frac{1}{2} - \frac{1}{x} \right| < \epsilon$, which is what we wanted.
\end{answer}
\end{exercise}






\end{exercises}


\section{Limit Laws}

In this section, we present a handful of tools to compute many limits
without explicitly working with the definition of limit. Each of these
could be proved directly as we did in the previous section.

\begin{mainTheorem}[Limit Laws]\index{limit laws}\label{theorem:limit-laws}
Suppose that $\lim_{x\to a}f(x)=L$, $\lim_{x\to a}g(x)=M$, $k$
is some constant, and $n$ is a positive integer.
\begin{itemize}
\item[\textbf{Constant Law}] $\lim_{x\to a} kf(x) = k\lim_{x\to a}f(x)=kL$.
\item[\textbf{Sum Law}] $\lim_{x\to a} (f(x)+g(x)) = \lim_{x\to a}f(x)+\lim_{x\to a}g(x)=L+M$.  
\item[\textbf{Product Law}] $\lim_{x\to a} (f(x)g(x)) = \lim_{x\to a}f(x)\cdot\lim_{x\to a}g(x)=LM$. 
\item[\textbf{Quotient Law}] $\lim_{x\to a} \frac{f(x)}{g(x)} =
  \frac{\lim_{x\to a}f(x)}{\lim_{x\to a}g(x)}=\frac{L}{M}$, if $M\ne0$.
\item[\textbf{Power Law}] $\lim_{x\to a} f(x)^n = \left(\lim_{x\to a}f(x)\right)^n=L^n$.
\item[\textbf{Root Law}] $\lim_{x\to a} \sqrt[n]{f(x)} = \sqrt[n]{\lim_{x\to
    a}f(x)}=\sqrt[n]{L}$ provided if $n$ is even, then $f(x)\ge 0$
  near $a$.
\item[\textbf{Composition Law}] If $\lim_{x\to a}g(x)=M$ and
  $\lim_{x\to M}f(x) = f(M)$, then $\lim_{x\to a} f(g(x)) = f(M)$.
\end{itemize}
\label{thm:limit laws}
\end{mainTheorem}

Roughly speaking, these rules say that to compute the limit of an
algebraic expression, it is enough to compute the limits of the
``innermost bits'' and then combine these limits. This often means
that it is possible to simply plug in a value for the variable, since
$\lim_{x\to a} x =a$.


\begin{example}
Compute $\lim_{x\to 1}{x^2-3x+5\over x-2}$. 
\end{example}
\begin{solution}
Using limit laws, 
\begin{align*}
\lim_{x\to 1}{x^2-3x+5\over x-2}&=
\dfrac{\lim_{x\to 1}x^2-3x+5}{\lim_{x\to1}(x-2)} \\
&=\frac{\lim_{x\to 1}x^2-\lim_{x\to1}3x+\lim_{x\to1}5}{\lim_{x\to1}x-\lim_{x\to1}2} \\
&=\dfrac{\left(\lim_{x\to 1}x\right)^2-3\lim_{x\to1}x+5}{\lim_{x\to1}x-2} \\
&=\dfrac{1^2-3\cdot1+5}{1-2} \\
&=\dfrac{1-3+5}{-1} = -3.
\end{align*}
\end{solution}


It is worth commenting on the trivial limit $\lim_{x\to1}5$. From one
point of view this might seem meaningless, as the number 5 can't
``approach'' any value, since it is simply a fixed number. But 5 can,
and should, be interpreted here as the function that has value 5
everywhere, $f(x)=5$, with graph a horizontal line. From this point of
view it makes sense to ask what happens to the height of the function
as $x$ approaches 1.

We're primarily interested in limits that aren't so easy, namely
limits in which a denominator approaches zero. The basic idea is to
``divide out'' by the offending factor. This is often easier said than
done---here we give two examples of algebraic tricks that work on many
of these limits.


\begin{example}
Compute $\lim_{x\to1}{x^2+2x-3\over x-1}$. 
\end{example}
\begin{solution}
We can't simply plug in $x=1$ because that makes the denominator zero.
However, when taking limits we assume $x\ne 1$:
\begin{align*}
\lim_{x\to1}{x^2+2x-3\over x-1}&=\lim_{x\to1}{(x-1)(x+3)\over x-1} \\
&=\lim_{x\to1}(x+3)=4
\end{align*}
\end{solution}
\marginnote[-1in]{Limits allow us to examine functions where they are not defined.}

\begin{example}
Compute $\lim_{x\to-1} {\sqrt{x+5}-2\over x+1}$.
\end{example}
\begin{solution} 
Using limit laws,
\begin{align*}
\lim_{x\to-1} {\sqrt{x+5}-2\over x+1}&=
\lim_{x\to-1} {\sqrt{x+5}-2\over x+1}{\sqrt{x+5}+2\over \sqrt{x+5}+2} \\
&=\lim_{x\to-1} {x+5-4\over (x+1)(\sqrt{x+5}+2)} \\
&=\lim_{x\to-1} {x+1\over (x+1)(\sqrt{x+5}+2)} \\
&=\lim_{x\to-1} {1\over \sqrt{x+5}+2}={1\over4}.
\end{align*}
\end{solution}
\marginnote[-1.5in]{Here we are rationalizing the numerator by multiplying by the conjugate.}


We'll conclude with one more theorem that will allow us to compute
more difficult limits.

\begin{mainTheorem}[Squeeze Theorem]\label{theorem:squeeze}\index{Squeeze Theorem}
Suppose that $g(x) \le f(x) \le h(x)$ for all $x$
close to $a$ but not necessarily equal to $a$. If 
\[
\lim_{x\to a} g(x) = L = \lim_{x\to a} h(x),
\] 
then $\lim_{x\to a} f(x) = L$.
\end{mainTheorem}

\marginnote[.2in]{For a nice discussion of this limit, see: Richman,
  Fred. \textit{A circular argument}. College Math. J. 24 (1993),
  no. 2, 160--162.}
\begin{example}\label{example:sinx/x}
Compute
\[
\lim_{x\to 0} \frac{\sin(x)}{x}.
\]
\end{example}

\marginnote[.2in]{The limit in this example will be used in
  Theorem~\ref{theorem:deriv sin}, and we will give another derivation
  of this limit in Example~\ref{example:sinx/x-lhopital}.}

\break

\begin{solution}
To compute this limit, use the Squeeze Theorem,
Theorem~\ref{theorem:squeeze}. First note that we only need to examine
$x\in \left(\frac{-\pi}{2}, \frac{\pi}{2}\right)$ and for the present
time, we'll assume that $x$ is positive---consider the diagrams below:

%\vspace{5cm}

\begin{fullwidth}
\begin{tabular}{ccc}
\begin{tikzpicture}
	\begin{axis}[
            xmin=-.1,xmax=1.1,ymin=-.1,ymax=1.1,
            axis lines=center,
            ticks=none,
            unit vector ratio*=1 1 1,
            xlabel=$u$, ylabel=$v$,
            every axis y label/.style={at=(current axis.above origin),anchor=south},
            every axis x label/.style={at=(current axis.right of origin),anchor=west},
          ]        
          \addplot [very thick, penColor2, smooth, domain=(-.2:.2+pi/2)] ({cos(deg(x))},{sin(deg(x))});
          \addplot [textColor] plot coordinates {(0,0) (1,.839)}; %% 40 degrees
          \addplot [very thick, penColor] plot coordinates {(.766,0) (.766,.643)}; %% 40 degrees
          \addplot [textColor] plot coordinates {(1,0) (1,.839)}; %% 40 degrees
          \addplot [very thick,penColor,fill=fill1] plot coordinates {(0,0) (.766,.643)}\closedcycle; %% triangle
          \addplot [textColor,smooth, domain=(0:40)] ({.15*cos(x)},{.15*sin(x)});
          \node at (axis cs:.15,.07) [anchor=west] {$x$};
          \node at (axis cs:.766,.322) [anchor=east] {$\sin(x)$};
          \node at (axis cs:.383,0) [anchor=north] {$\cos(x)$};
        \end{axis}
\end{tikzpicture} & 
\begin{tikzpicture}
	\begin{axis}[
            xmin=-.1,xmax=1.1,ymin=-.1,ymax=1.1,
            axis lines=center,
            ticks=none,
            unit vector ratio*=1 1 1,
            xlabel=$u$, ylabel=$v$,
            every axis y label/.style={at=(current axis.above origin),anchor=south},
            every axis x label/.style={at=(current axis.right of origin),anchor=west},
          ]        
          \addplot [draw=none,fill=fill1] plot coordinates {(0,0) (.766,.643)}\closedcycle; %% sector
          \addplot [draw=none, fill=fill1, samples=100, domain=(0:40)] ({cos(x)},{sin(x)})\closedcycle; %% sector 
          \addplot [very thick, penColor2, smooth, domain=(-.2:.2+pi/2)] ({cos(deg(x))},{sin(deg(x))});
          \addplot [textColor] plot coordinates {(0,0) (1,.839)}; %% 40 degrees
          \addplot [textColor] plot coordinates {(.766,0) (.766,.643)}; %% 40 degrees
          \addplot [textColor] plot coordinates {(1,0) (1,.839)}; %% 40 degrees          
          \addplot [textColor,smooth, domain=(0:40)] ({.15*cos(x)},{.15*sin(x)});
          \addplot [very thick,penColor] plot coordinates {(0,0) (.766,.643)}; %% sector
          \addplot [very thick,penColor] plot coordinates {(0,0) (1,0)}; %% sector
          \addplot [very thick, penColor, smooth, domain=(0:40)] ({cos(x)},{sin(x)}); %% sector
          \node at (axis cs:.15,.07) [anchor=west] {$x$};
          \node at (axis cs:.5,0) [anchor=north] {$1$};
        \end{axis}
\end{tikzpicture} & 
\begin{tikzpicture}
	\begin{axis}[clip=false,
            xmin=-.1,xmax=1.1,ymin=-.1,ymax=1.1,
            axis lines=center,
            ticks=none,
            unit vector ratio*=1 1 1,
            xlabel=$u$, ylabel=$v$,
            every axis y label/.style={at=(current axis.above origin),anchor=south},
            every axis x label/.style={at=(current axis.right of origin),anchor=west},
          ]        
          \addplot [very thick,penColor,fill=fill1] plot coordinates {(0,0) (1,.839)}\closedcycle; %% triangle          
          \addplot [very thick, penColor2, smooth, domain=(-.1:1.671)] ({cos(deg(x))},{sin(deg(x))});
          \addplot [very thick, penColor] plot coordinates {(0,0) (1,.839)}; %% 40 degrees
          \addplot [textColor] plot coordinates {(.766,0) (.766,.643)}; %% 40 degrees
          \addplot [very thick, penColor] plot coordinates {(1,0) (1,.839)}; %% 40 degrees          
          \addplot [textColor,smooth, domain=(0:40)] ({.15*cos(x)},{.15*sin(x)});
          \node at (axis cs:.15,.07) [anchor=west] {$x$};
          \node at (axis cs:.5,0) [anchor=north] {$1$};
          \node at (axis cs:1,.42) [anchor=west] {$\tan(x)$};
        \end{axis}
\end{tikzpicture}\\
Triangle $A$ & Sector & Triangle $B$ \\
\end{tabular}
\end{fullwidth}



From our diagrams above we see that
\[
\text{Area of Triangle $A$} \le \text{Area of Sector} \le \text{Area of Triangle $B$}
\]
and computing these areas we find
\[
\frac{\cos(x)\sin(x)}{2} \le \left(\frac{x}{2\pi}\right)\cdot\pi \le \frac{\tan(x)}{2}.
\]
Multiplying through by $2$, and recalling that $\tan(x) =
\frac{\sin(x)}{\cos(x)}$ we obtain
\[
\cos(x)\sin(x) \le x \le \frac{\sin(x)}{\cos(x)}.
\]
Dividing through by $\sin(x)$ and taking the reciprocals, we find
\[
\cos(x) \le \frac{\sin(x)}{x} \le \frac{1}{\cos(x)}.
\]
Note, $\cos(-x) = \cos(x)$ and $\frac{\sin(-x)}{-x} =
\frac{\sin(x)}{x}$, so these inequalities hold for all $x\in
\left(\frac{-\pi}{2}, \frac{\pi}{2}\right)$.  Additionally, we know
\[
\lim_{x \to 0}\cos(x) = 1 = \lim_{x\to 0}\frac{1}{\cos(x)},
\]
and so we conclude by the Squeeze Theorem,
Theorem~\ref{theorem:squeeze}, $\lim_{x \to 0}\frac{\sin(x)}{x} = 1$.
\end{solution}


\begin{exercises}

\noindent Compute the limits. If a limit does not exist, explain why.

\twocol

\begin{exercise} $\lim_{x\to 3}{x^2+x-12\over x-3}$
\begin{answer} $7$
\end{answer}\end{exercise}

\begin{exercise} $\lim_{x\to 1}{x^2+x-12\over x-3}$
\begin{answer} $5$
\end{answer}\end{exercise}

\begin{exercise} $\lim_{x\to -4}{x^2+x-12\over x-3}$
\begin{answer} $0$
\end{answer}\end{exercise}

\begin{exercise} $\lim_{x\to 2} {x^2+x-12\over x-2}$
\begin{answer} DNE
\end{answer}\end{exercise}

\begin{exercise} $\lim_{x\to 1} {\sqrt{x+8}-3\over x-1}$
\begin{answer} $1/6$
\end{answer}\end{exercise}

\begin{exercise} $\lim_{x\to 0+} \sqrt{{1\over x}+2} - \sqrt{1\over x}$
\begin{answer} $0$
\end{answer}\end{exercise}

\begin{exercise} $\lim _{x\to 2} 3$
\begin{answer} $3$
\end{answer}\end{exercise}

\begin{exercise} $\lim _{x\to 4 } 3x^3 - 5x $
\begin{answer} $172$
\end{answer}\end{exercise}

\begin{exercise} $\lim _{x\to 0 } {4x - 5x^2\over x-1}$
\begin{answer} $0$
\end{answer}\end{exercise}

\begin{exercise} $\lim _{x\to 1 } {x^2 -1 \over x-1 }$
\begin{answer} $2$
\end{answer}\end{exercise}

\begin{exercise} $\lim _{x\to 0 + } {\sqrt{2-x^2 }\over x}$
\begin{answer} DNE
\end{answer}\end{exercise}

\begin{exercise} $\lim _{x\to 0 + } {\sqrt{2-x^2}\over x+1}$
\begin{answer} $\sqrt2$
\end{answer}\end{exercise}

\begin{exercise} $\lim _{x\to a } {x^3 -a^3\over x-a}$
\begin{answer} $3a^2$
\end{answer}\end{exercise}

\begin{exercise} $\lim _{x\to 2 } (x^2 +4)^3$
\begin{answer} $512$
\end{answer}\end{exercise}

\begin{exercise} $\lim _{x\to 1 } \begin{cases}
x-5 & \text{if $x\ne 1$}, \\
7 & \text{if $x=1$}. \end{cases}$
\begin{answer} $-4$
\end{answer}\end{exercise}

\endtwocol

\end{exercises}



