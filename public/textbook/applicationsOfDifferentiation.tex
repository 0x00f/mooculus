\chapter{Applications of Differentiation}

\section{The Derivative as a Rate}

The world is constantly changing around us. To simplify matters we
will only consider change in one dimension. This means that if we
think of a ball being tossed in the air, we will consider its vertical
movement separately from its lateral and forward movement.  To
understand how things change, we need to understand the \textit{rate}
of change. Let's start out rather basic.

\begin{definition}\index{average rate of change}
Given a function $f(x)$, the \textbf{average rate of change} over the
interval $[a, a+\Delta x]$ is given by
\[
\frac{f(a+\Delta x) - f(a)}{\Delta x}.
\]
\end{definition}
\begin{marginfigure}
\begin{tikzpicture}
	\begin{axis}[
            xmin=0,xmax=12,ymin=0,ymax=600,
            axis lines=center,
            ytick={100,200,300,400,500,600},
            xtick={1,...,12},
            grid=both,
            grid style={dashed, gridColor},
            xlabel=$t$, ylabel=$d$,
            every axis y label/.style={at=(current axis.above origin),anchor=south},
            every axis x label/.style={at=(current axis.right of origin),anchor=west},
          ]                  
          
          \addplot[mark=none,penColor,very thick] coordinates { 

            (0,0) (1.5, 50.)  (4.5, 270)  (6.5, 270.) (10.5, 540.)  (12.,600.)        
          };
        \end{axis}
\end{tikzpicture}
\caption{Here we see a plot of the distance traveled on a $600$ mile road trip.}
\label{figure:road-trip}
\end{marginfigure}

\begin{example}
Suppose you drive a car $600$ mile road trip. Your distance from home is
recorded by the plot shown in Figure~\ref{figure:road-trip}. What was
your average velocity during hours $4$--$8$ of your trip?
\end{example}

\begin{solution}
Examining Figure~\ref{figure:road-trip}, we see that we were around
$240$ miles from home at hour $4$, and $360$ miles from home at hour
$8$. Hence our average velocity was
\[
\frac{360-240}{8-4} = \frac{120}{4} = 30~\text{miles per hour.}
\]
\end{solution}

Of course if you look at Figure~\ref{figure:road-trip} closely, you
see that sometime we were driving faster and other times we were
driving slower. To get more information, we need to know the
\textit{instantaneous rate of change}.

\begin{definition}\index{instantaneous rate of change}
Given a function, the \textbf{instantaneous rate of change} at $x=a$ is given by
\[
\left.\ddx f(x) \right|_{x=a}.
\] 
\end{definition}

\begin{example} 
Again suppose, you drive a car 600 mile road trip. Your distance from
home is recorded by the plot shown in
Figure~\ref{figure:road-trip}. What was your instantaneous velocity
$8$ hours into your trip?
\end{example}
\begin{solution}
Since the instantaneous rate of change is measured by the derivative,
we need to find the slope of the tangent line to the curve. At $7$
hours, the curve is growing at an essentially constant rate. In fact,
the growth rate seems to be constant from $(7,300)$ to
$(10,500)$. This gives us an instantaneous growth rate at hour $8$ of
about $200/3 = 67$ miles per hour.
\end{solution}




\subsection*{Physical Applications}

In physical applications, we are often concerned about
\textit{position}, \textit{velocity}, \textit{speed},
\textit{acceleration}.
\begin{align*}
p(t) &= \text{position with respect to time.}\\
v(t) &= p'(t) = \text{velocity with respect to time.}\\
s(t) &= |v(t)| = \text{speed, the absolute value of velocity.}\\
a(t) &=v'(t) = \text{acceleration with respect to time.}
\end{align*}

Let's see an example.

\begin{example}
The Mostar bridge in Bosnia is 25 meters above the river
Neretva. For fun, you decided to dive off the bridge. Your position
$t$ seconds after jumping off is
\[
p(t) = -4.9t^2 + 25.
\]
When do you hit the water? What is your instantaneous velocity as you
enter the water?  What is your average velocity during your dive?
\end{example}

\begin{solution}
To find when you hit the water, you must solve
\[
-4.9t^2 + 25 = 0
\]
Write
\begin{align*}
-4.9t^2 &= -25 \\
t^2 &\approx 5.1 \\ 
t &\approx 2.26
\end{align*}
Hence after approximately 2.26 seconds, you gracefully enter the
river.

Your instantaneous velocity is given by $p'(t)$. Write
\[
p'(t) = -9.8t
\]
so your instantaneous velocity is approximately $-9.8\cdot 2.26\approx
-22$ meters per second.

Finally, your average velocity is given by
\[
\frac{p'(2.26) -p'(0)}{2.26} \approx \frac{-22}{2.26} =
-9.7~\text{meters per second}
\]
\end{solution}


Sometimes we are also interested in \textit{angular velocity}. If an
object is moving past you, you can measure how the angle between you, the object and the FIX

\begin{example}
The Earth make a complete rotation on its axis every 24 hours. If the
Earth is approximately 4000 miles, what is the velocity of someone
standing at $42^\circ$ latitude? What is their angular velocity?
\end{example}

\begin{solution}
Since the Earth spins once every 24 hours, its angular velocity is $\frac{2\pi}{24}$
\end{solution}




\subsection*{Biological Applications}

In biological applications, we are often concerned with how animals
and plants grow, though there are numerous other applications too.

\begin{example}
A certain bacterium divides into two cells every 20 minutes. The
initial population of a culture is 120 cells. Find a formula for the
population.  What is the average growth rate during the first 4 hours?
What is the instanteauosu growth rate of the population at 4 hours?
What rate is the population growing at 20 hours?
\end{example}


\begin{solution}
Since we start with 120 cells, and this population doubles every 20
minutes, the the population doubles three times an hour. So the
formula for the population is
\[
p(t) = 120\cdot 2^{3t}
\]
where $t$ is time measured in hours.

Now, the average growth rate during the first 4 hours is given by
\[
\frac{p(4)-p(0)}{4} =\frac{491520-120}{4} = 122850~\text{cells per hour.}
\]

We comput the instantaneous growth rate of the population with
\[
p'(t) = \ln(2)\cdot 360\cdot 2^{3t}.
\]
So $p'(4) \approx  1022087$ cells per hour.
\end{solution}


\subsection*{Economic Applications}

There are three main players when dealing with very basic economic
questions: The \textit{revenue}, the amount of money acquired from
selling a product; the \textit{cost}, the amount of money it takes to
produce an item; and the \textit{profit}, the revenue minus the cost.
\begin{align*}
r(x) &= \text{revenue from selling $x$ items.}\\
c(x) &= \text{cost of producing $x$ items.}\\
p(x) &= r(x)-c(x) = \text{the profit from selling $x$ items}.
\end{align*}
The rate of change in these functions is denoted by the word \textit{marginal}, hence 
\begin{align*}
\dd[r]{x} &= \text{the marginal revenue, how the revenue is changing.}\\
\dd[c]{x} &= \text{the marginal cost, how the cost is changing.}
\end{align*}

\begin{exercises}

\noindent Exercises related to physical applications:

\begin{exercise}
1
\begin{answer}
\end{answer}
\end{exercise}

\begin{exercise}
2
\begin{answer}
\end{answer}
\end{exercise}

\begin{exercise}
3
\begin{answer}
\end{answer}
\end{exercise}

\begin{exercise}
4
\begin{answer}
\end{answer}
\end{exercise}

\noindent Exercises related to biological applications:

\begin{exercise}
5
\begin{answer}
\end{answer}
\end{exercise}

\begin{exercise}
6
\begin{answer}
\end{answer}
\end{exercise}

\begin{exercise}
7
\begin{answer}
\end{answer}
\end{exercise}


\noindent Exercises related to economic applications:

\begin{exercise}
8
\begin{answer}
\end{answer}
\end{exercise}

\begin{exercise}
9
\begin{answer}
\end{answer}
\end{exercise}

\begin{exercise}
10
\begin{answer}
\end{answer}
\end{exercise}

\end{exercises}







\section{Related Rates Problems}

Suppose we have two variables $x$ and $y$ (in most problems the
letters will be different, but for now let's use $x$ and $y$) which
are both changing with time.  A ``related rates'' problem is a problem
in which we know one of the rates of change at a given instant---say,
$\ds \dot x = dx/dt$---and we want to find the other rate $\ds \dot y = dy/dt$ at that
instant. (The use of $\ds \dot x$ \index{dot notation}
\index{derivative!dot notation}
to mean $dx/dt$ goes back to Newton and
is still used for this purpose, especially by physicists\index{physicists}.)

If $y$ is written in terms of $x$, i.e., $y=f(x)$, then this is easy
to do using the chain rule:
$$
\dot y = {dy\over dt}={dy\over dx}\cdot{dx\over dt}={dy\over dx}\dot x.
$$
That is, find the derivative of $f(x)$, plug in the value of
$x$ at the instant in question, and multiply by the given value of
$\ds \dot{x}=dx/dt$ to get $\ds \dot{y}=dy/dt$.

\begin{example}
Suppose an object is moving along a path described by $\ds y=x^2$, that
is, it is moving on a parabolic path. At a particular time, say $t=5$,
the $x$ coordinate is 6 and 
we measure the speed at which the $x$ coordinate of the object is
changing and find that $dx/dt = 3$. At the same time, how fast is the
$y$ coordinate changing?

Using the chain rule, $\ds dy/dt = 2x\cdot dx/dt$. At $t=5$ we know that
$x=6$ and $dx/dt=3$, so $dy/dt = 2\cdot 6\cdot 3 = 36$.
\end{example}

In many cases, particularly interesting ones,
$x$ and $y$ will be related in some other way, for example
$x=f(y)$, or $F(x,y)=k$, or perhaps $F(x,y)=G(x,y)$, where $F(x,y)$
and $G(x,y)$ are expressions involving both variables.  In all cases, you
can solve the related rates problem by taking the derivative of both sides,
plugging in all the known values (namely, $x$, $y$, and $\ds \dot{x}$), and
then solving for $\ds \dot{y}$.

To summarize, here are the steps in doing a related rates problem:


\begin{verbatim}
http://mathoverflow.net/questions/83097/are-there-any-related-rates-calculus-problems-that-dont-feel-contrived
\end{verbatim}


\begin{itemize} % BADBAD

\item{1.} Decide what the two variables are.
\item{2.}  Find an equation relating them.
\item{3.}  Take $d/dt$ of both sides.
\item{4.}  Plug in all known values at the instant in question.
\item{5.}  Solve for the unknown rate.

\end{itemize}

\begin{example}
\label{exam:receding airplane}
A plane is flying directly away from you at 500 mph at an altitude of
3 miles.  How fast is the plane's distance from you increasing at the
moment when the plane is flying over a point on the ground 4 miles
from you?

To see what's going on, we first draw a schematic representation of
the situation, as in figure~\xrefn{fig:airplane}.

%%BADBAD
%% \figure
%% \vbox{\beginpicture
%% \normalgraphs
%% \ninepoint
%% \setcoordinatesystem units <.75truecm,.75truecm>
%% \setplotarea x from 0 to 6, y from 0 to 3
%% \axis bottom shiftedto y=0 /
%% \setlinear
%% \setdashes
%% \plot 0 0 5 3 /
%% \putrule from 5 0 to 5 3
%% \put {$\longrightarrow$} [l] at 5 3
%% \put {$x$} [t] <0pt,-4pt> at 2.5 0
%% \put {$y$} [br] <-2pt,2pt> at 2.5 1.5
%% \put {$3$} [l] <4pt,0pt> at 5 1.5
%% \endpicture}
%% \figrdef{fig:airplane}
%% \endfigure{Receding airplane.}

Because the plane is in level flight directly away from you, the rate
at which $x$ changes is the speed of the plane, $dx/dt=500$. The
distance between you and the plane is $y$; it is $dy/dt$ that we wish
to know. By the Pythagorean Theorem we know that $\ds x^2+9=y^2$. Taking
the derivative:
$$ 2x \dot x = 2y\dot y.$$
We are interested in the time at which $x=4$; at this time we know
that $\ds 4^2+9=y^2$, so $y=5$. Putting together all the information we
get
$$2(4)(500)=2(5)\dot y.$$
Thus, $\ds \dot y=400$ mph.
\end{example}

\begin{example}
You are inflating a spherical balloon at the rate of 7 cm${}^3$/sec.  How
fast is its radius increasing when the radius is 4 cm?

Here the variables are the radius $r$ and the volume $V$.  We know $dV/dt$,
and we want $dr/dt$.  The two variables are related by means of the
equation $\ds V=4\pi r^3/3$.  Taking the derivative of both sides gives
$\ds dV/dt=4\pi r^2\dot r$.  We now substitute the values we know at the
instant in question: $\ds 7=4\pi 4^2\dot r$, so
$\ds \dot r=7/(64\pi)$ cm/sec.
\end{example}

\begin{example} Water is poured into a conical container at the rate of 10
cm${}^3$/sec.  The cone points directly down, and it has a height of
30 cm and a base radius of 10 cm; see figure~\xrefn{fig:cone tank}.
How fast is the water level rising when the water is 4 cm deep (at its
deepest point)?

The water forms a conical shape within the big cone; its
height and base radius and volume are all increasing
as water is poured into the container.  This means that we actually have
three things varying with time: the water level $h$ (the height of the cone
of water), the radius $r$ of the circular top surface of water (the base
radius of the cone of water), and the volume of water $V$.  
The volume of a cone is given by
$\ds V=\pi r^2h/3$.  We know $dV/dt$, and we want $dh/dt$.  At
first something seems to be wrong: we have a third variable $r$ whose rate
we don't know.  

%%--- figure MB --------------------------------------------------
%\begin{psfigure}{1.85in}{1.9in}{124.figMB.ps}
%\puttext{-3,-20}{\Tiny$h$}
%\puttext{6,-3}{\Tiny$r$}
%\puttext{15, 43}{\Tiny$10$}
%\puttext{40, 0}{\Tiny$30$}
%\end{psfigure}
%%------------------------------
%%BADBAD
%% \figure
%% \vbox{\beginpicture
%% \normalgraphs
%% \sevenpoint
%% \setcoordinatesystem units <1truecm,1truecm>
%% \setplotarea x from -1.5 to 1.5, y from -5 to 1
%% \ellipticalarc  axes ratio 3:1  360 degrees from 1.5 0 center at 0 0
%% \ellipticalarc  axes ratio 3:1  360 degrees from 0.75 -2.5 center at 0 -2.5
%% \betweenarrows {10} from 0 0.75 to 1.5 0.75
%% \betweenarrows {30} from 1.7 -5 to 1.7 0
%% \betweenarrows {$h$} from -1 -5 to -1 -2.5
%% \put {$r$} at 0.385 -2.1
%% \setlinear
%% \plot 1.5 0 0 -5 -1.5 0 /
%% \setplotsymbol ({\teeny.})
%% \plotsymbolspacing=.2pt
%% \arrow <2pt> [0.7, 2] from 0.25 -2.1 to 0 -2.1
%% \arrow <2pt> [0.7, 2] from 0.50 -2.1 to 0.75 -2.1
%% \endpicture}
%% \figrdef{fig:cone tank}
%% \endfigure{Conical water tank.}

But the dimensions of the cone of water must have the same
proportions as those of the container.  
That is, because of similar triangles, 
$r/h=10/30$ so $r=h/3$.  Now we can eliminate $r$ from the
problem entirely: $\ds V=\pi(h/3)^2h/3=\pi h^3/27$.  We take
the derivative of both sides and plug in $h=4$ and $dV/dt=10$, obtaining
$\ds 10=(3\pi\cdot 4^2/27)(dh/dt)$.  Thus, $dh/dt=90/(16\pi)$
cm/sec.
\end{example}

\begin{example}
A swing consists of a board at the end of a 10 ft long rope.  Think of the
board as a point $P$ at the end of the rope, and let $Q$ be the point of
attachment at the other end.  Suppose that the swing is directly below $Q$
at time $t=0$, and is being pushed by someone who walks at 6
ft/sec from left to right.  Find (a) how fast the swing is rising after 1
sec; (b) the angular speed of the rope in deg/sec after 1 sec.

We  start out by asking: What is the geometric
quantity whose rate of change we know, and what is the geometric quantity
whose rate of change we're being asked about?  Note that the person pushing
the swing is moving horizontally at a rate we know.  In other words,
the  horizontal coordinate of $P$ is increasing at 6 ft/sec.  In the
$xy$-plane let us make the convenient choice of putting the origin at the
location of $P$ at time $t=0$, i.e., a distance 10 directly below the point
of attachment.  Then the rate we know is $dx/dt$, and in part 
(a) the rate we want is $dy/dt$ (the rate at which $P$ is rising).  In part
(b) the rate we want is $\ds \dot{\theta}=d\theta/dt$, where $\theta$ stands for
the angle in radians through which the swing has swung from the vertical.
(Actually, since we want our answer in deg/sec, at the end we must convert
$d\theta/dt$ from rad/sec by multiplying by $180/\pi$.)

%%--- figure MC --------------------------------------------------
%\begin{psfigure}{1.95in}{2.0in}{124.figMC.ps}
%%\putcrosshair{0,0}
%\puttext{-6.67, 46.7}{\Small$Q$}
%\puttext{5,16}{\Small$\theta$}
%\puttext{46.7, -40.}{\Small$P$}
%\puttext{20., -26.7}{\Small$x$}
%\puttext{-6.67, -36.7}{\Small$y$}
%\puttext{33.3, 0.}{\Small$10$}
%\end{psfigure}
%%------------------------------
%%BADBAD
%% \figure
%% \vbox{\beginpicture
%% \normalgraphs
%% \sevenpoint
%% \setcoordinatesystem units <0.5truecm,0.5truecm>
%% \setplotarea x from -6 to 6, y from 0 to 10
%% \circulararc  73.74 degrees from -6 2 center at 0 10
%% \circulararc  36.87 degrees from 0 8 center at 0 10
%% %\betweenarrows {10} from 0 0.75 to 1.5 0.75
%% %\betweenarrows {30} from 1.7 0 to 1.7 -5
%% %\betweenarrows {$h$} from -1 -2.5 to -1 -5
%% %\put {$r$} at 0.385 -2.1
%% \setlinear
%% \plot 0 10 6 2 /
%% \setdashes
%% \putrule from 0 0 to 0 10
%% \putrule from 0 2 to 6 2
%% \multiput {$\bullet$} at 0 10 6 2 /
%% \put {$P$} [l] <4pt,0pt> at 6 2
%% \put {$Q$} [b] <0pt,4pt> at 0 10
%% \put {$x$} [b] <0pt,4pt> at 3 2
%% \put {$y$} [r] <-4pt,0pt> at 0 1
%% \put {$\theta$} [tl] at 0.6 7.7
%% \endpicture}
%% \figrdef{fig:swing}
%% \endfigure{Swing.}

\noindent
(a)~From the diagram we see that we have a right triangle whose legs
are $x$ and $10-y$, and whose hypotenuse is 10.  Hence
$\ds x^2+(10-y)^2=100$.  Taking the derivative of both sides we obtain:
$2x\dot{x}+2(10-y)(0-\dot{y})=0$.  We now look at what we know after 1
second, namely $x=6$ (because $x$ started at 0 and has been increasing at
the rate of 6 ft/sec for 1 sec), $y=2$ (because we get $10-y=8$ from
the Pythagorean theorem applied to the triangle with hypotenuse 10 and
leg 6), and $\ds \dot{x}=6$.  Putting in these values gives us
$2\cdot 6\cdot 6-2\cdot 8\dot{y}=0$, from which we can easily solve
for $\ds \dot{y}$: $\ds \dot{y}=4.5$ ft/sec.

\noindent
(b)~Here our two variables are $x$ and $\theta$, so we want to use the
same right triangle as in part (a), but this time relate $\theta$ to
$x$.  Since the hypotenuse is constant (equal to 10), the best way to
do this is to use the sine: $\sin\theta=x/10$.  Taking derivatives we
obtain $\ds (\cos\theta)\dot{\theta}=0.1\dot{x}$.  At the instant in
question ($t=1$ sec), when we have a right triangle with sides
6--8--10, $\ds \cos\theta=8/10$ and $\ds \dot{x}=6$. Thus
$(8/10)\dot{\theta}=6/10$, i.e., $\ds \dot{\theta}=6/8=3/4$ rad/sec, or
approximately $43$ deg/sec.  
\end{example} 

We have seen that sometimes there are apparently more than two
variables that change with time, but in reality there are just two, as
the others can be expressed in terms of just two. But sometimes there
really are several variables that change with time; as long as you
know the rates of change of all but one of them you can find the rate
of change of the remaining one.  As in the case when there are just two
variables, take the derivative of both sides of the equation relating all of
the variables, and then substitute all of the known values and solve for
the unknown rate.

\begin{example}
A road running north to south crosses a road going east to west at the
point $P$.  Car A is driving north along the first road, and car B is
driving east along the second road.  At a particular time car A is $10$
kilometers to the north of $P$ and traveling at 80 km/hr, while car B
is 15 kilometers to the east of $P$ and traveling at 100 km/hr.
How fast is the distance between the two cars
changing?
%%BADBAD
%% \figure
%% \vbox{\beginpicture
%% \normalgraphs
%% \sevenpoint
%% \setcoordinatesystem units <0.5truecm,0.5truecm>
%% \setplotarea x from 0 to 10, y from 0 to 6
%% \axis left shiftedto x=0 /
%% \axis bottom shiftedto y=0 /
%% \multiput {$\bullet$} at 0 0 0 4 7 0 /
%% \setdashes\setlinear
%% \plot 0 4 7 0 /
%% \put {$(0,a(t))$} [r] <-4pt,0pt> at 0 4
%% \put {$(b(t),0)$} [t] <0pt,-6pt> at 7 0
%% \put {$c(t)$} [bl] <2pt,2pt> at 3.5 2
%% \put {$P$} [tr] <-2pt,-2pt> at 0 0
%% \setsolid
%% \setplotsymbol ({\tenrm.}) 
%% \arrow <5pt> [.25, 1] from 0 4 to 0 5
%% \arrow <5pt> [.25, 1] from 7 0 to 8 0
%% \endpicture}
%% \figrdef{fig:departing cars}
%% \endfigure{Cars moving apart.}

Let $a(t)$ be the distance of car A north of $P$ at time $t$, and
$b(t)$ the distance of car B east of $P$ at time $t$, and let $c(t)$
be the distance from car A to car B at time $t$.  By the Pythagorean
Theorem, $\ds c(t)^2=a(t)^2+b(t)^2$. Taking derivatives
we get $\ds 2c(t)c'(t)=2a(t)a'(t)+2b(t)b'(t)$, so
$$
  \dot{c}={a\dot{a}+b\dot{b}\over c}={a\dot{a}+b\dot{b}\over \sqrt{a^2+b^2}}.
$$
Substituting known values we get:
$$
\dot{c}={10\cdot 80+15\cdot100\over
  \sqrt{10^2+15^2}}={460\over\sqrt{13}} \approx 127.6 \hbox{km/hr}
$$
at the time of interest.
\end{example}
\label{exam:departing cars}

%\begin{psfigure}{2.0in}{2.0in}{124.figMD.ps}

Notice how this problem differs from example~\xrefn{exam:receding
airplane}.  In both cases we started with the Pythagorean Theorem and
took derivatives on both sides.  However, in
example~\xrefn{exam:receding airplane} one of the sides was a constant
(the altitude of the plane), and so the derivative of the square of
that side of the triangle was simply zero.  In this example, on the
other hand, all three sides of the right triangle are variables, even
though we are interested in a specific value of each side of the
triangle (namely, when the sides have lengths 10 and 15). Make sure that
you understand at the start of the problem what are the variables and
what are the constants.

\begin{exercises}

\begin{exercise}
A cylindrical tank standing upright (with one circular base on the
ground) has radius 20 cm.  How fast does the water level in the
tank drop when the water is being drained at 25 cm${}^3$/sec?
\begin{answer} $1/(16\pi)$ cm/s
\end{answer}\end{exercise}

\begin{exercise}
A cylindrical tank standing upright (with one circular base on the
ground) has radius 1 meter.  How fast does the water level in the
tank drop when the water is being drained at 3 liters per second?
\begin{answer} $3/(1000\pi)$ meters/second
\end{answer}\end{exercise}

\begin{exercise} A ladder 13 meters long rests on horizontal ground and leans
against a vertical wall.  The foot of the ladder is pulled away from
the wall at the rate of 0.6 m/sec.  How fast is the top sliding down
the wall when the foot of the ladder is 5 m from the wall?
\begin{answer} $1/4$ m/s
\end{answer}\end{exercise}

\begin{exercise} A ladder 13 meters long rests on horizontal ground and leans
against a vertical wall. The top of the ladder is being pulled up the
wall at $0.1$ meters per second.
How fast is the foot of the ladder approaching 
the wall when the foot of the ladder is 5 m from the wall?
\begin{answer} $-6/25$ m/s
\end{answer}\end{exercise}

\begin{exercise}
A rotating beacon is located 2 miles out in the water.  Let $A$ be the
point on the shore that is closest to the beacon.  As the beacon rotates at
10 rev/min, the beam of light sweeps down the shore once each time it revolves.
Assume that the shore is straight.  How fast is the point where the beam
hits the shore moving at an instant when the beam is lighting up a point 2
miles along the shore from the point $A$?
\begin{answer} $80\pi$ mi/min
\end{answer}\end{exercise}

\begin{exercise}
A baseball diamond is a square 90 ft on a side.  A player runs from first
base to second base at 15 ft/sec.  At what rate is the player's distance
from third base decreasing when she is half way from first to second base?
\begin{answer} $\ds 3\sqrt5$ ft/s
\end{answer}\end{exercise}

\begin{exercise} Sand is poured onto a surface at 15 cm${}^3$/sec, forming a
conical pile whose base diameter is always equal to its altitude.  How
fast is the altitude of the pile increasing when the pile is 3 cm
high?
\begin{answer} $20/(3\pi)$ cm/s
\end{answer}\end{exercise}

\begin{exercise}
A boat is pulled in to a dock by a rope with one end attached to the front
of the boat and the other end passing through a ring attached to the dock
at a point 5 ft higher than the front of the boat.  The rope is being
pulled through the ring at the rate of 0.6 ft/sec.  How fast is the boat
approaching the dock when 13 ft of rope are out?
\begin{answer} $13/20$ ft/s
\end{answer}\end{exercise}

\begin{exercise}
A balloon is at a height of 50 meters, and is rising at the constant rate
of 5 m/sec.  A bicyclist passes beneath it, traveling in a
straight line at the constant speed of 10 m/sec.  How fast is the distance
between the bicyclist and the balloon increasing 2 seconds later?
\begin{answer} $\ds 5\sqrt{10}/2$ m/s
\end{answer}\end{exercise}

\begin{exercise} A pyramid-shaped vat has square cross-section and stands on its
tip.  The dimensions at the top are 2 m $\times$ 2 m, and the depth is
5 m.  If water is flowing into the vat at 3 m${}^3$/min, how fast is
the water level rising when the depth of water (at the deepest point)
is 4 m?  Note: the volume of any ``conical'' shape (including
pyramids) is $(1/3)(\hbox{height})(\hbox{area of base})$.
\begin{answer} $75/64$ m/min
\end{answer}\end{exercise}

\begin{exercise}
The sun is rising at the rate of $1/4$ deg/min, and appears to be
climbing into the sky perpendicular to the
horizon, as depicted in figure~\xrefn{fig:sunrise sunset}.
How fast is the shadow of a 200 meter building
shrinking at the moment when the shadow is 500 meters long? 
\begin{answer} $145\pi/72$ m/s
\end{answer}\end{exercise}

\begin{exercise} The sun is setting at the rate of $1/4$ deg/min, and appears
to be dropping perpendicular to the horizon, as depicted in
figure~\xrefn{fig:sunrise sunset}. How fast is the shadow of a 25
meter wall lengthening at the moment when the shadow is 50 meters long?
\begin{answer} $25\pi/144$ m/min
\end{answer}\end{exercise}

%% BADBAD
%% \font\miscsymbols miscsymbols10 scaled 2000
%% %\begin{psfigure}{2.45in}{1.25in}{124.figME.ps}
%% \figure
%% \vbox{\beginpicture
%% \normalgraphs
%% \sevenpoint
%% \setcoordinatesystem units <0.3truecm,0.3truecm>
%% \setplotarea x from -10 to 10, y from 0 to 8
%% \axis bottom shiftedto y=0 /
%% \setdashes\setlinear
%% \plot 5 0 -8 8 /
%% \setsolid
%% \setplotsymbol ({\tenrm.}) 
%% \plot 0 0 0 3 /
%% \put {\miscsymbols k} at -9 8.6
%% \endpicture}
%% \figrdef{fig:sunrise sunset}
%% \endfigure{Sunrise or sunset.}



\begin{exercise}
The trough shown in figure~\xrefn{fig:trough}
is constructed by fastening together three
slabs of wood of dimensions 10 ft $\times$ 1 ft, and then attaching the
construction to a wooden wall at each end.  The angle $\theta$ was
originally $\ds 30^\circ$, but because of poor construction the sides are
collapsing.  The trough is full of water.  At what rate (in ft${}^3$/sec) 
is 
the water spilling out over the top of
the trough if the sides have each fallen to an angle of $\ds 45^\circ$, and are
collapsing at the rate of $\ds 1^\circ$ per second?
\begin{answer} $\ds \pi\sqrt2/36$ ft$^3$/s
\end{answer}\end{exercise}

%%BADBAD
%\begin{psfigure}{2.5in}{1.5in}{124.figMF.ps}
%% \figure
%% \vbox{\beginpicture
%% \normalgraphs
%% \sevenpoint
%% \setcoordinatesystem units <0.9truecm,0.9truecm>
%% \setplotarea x from -1 to 12, y from 0 to 4
%% \setlinear
%% \plot -0.5 0.866 0 0 1 0 1.5 0.866 -0.5 0.866 /
%% \plot 1 0 10.4 3.42 10.9 4.29 8.9 4.29 -0.5 0.866 /
%% \plot 10.9 4.29 1.5 0.866 /
%% \put {$\theta$} [b] <0pt,3pt> at -0.15 0.3
%% \put {$\theta$} [b] <0pt,3pt> at 1.15 0.3
%% \put {$1$} [t] <0pt,-4pt> at 0.5 0
%% \put {$1$} [tr] <-2pt,-2pt> at -0.25 0.433
%% \put {$1$} [tl] <2pt,-2pt> at 10.65 3.853
%% \put {$10$} [tl] <2pt,-2pt> at 5.7 1.71
%% \circulararc 30 degrees from 0 0.3 center at 0 0
%% \circulararc -30 degrees from 1 0.3 center at 1 0
%% \setdashes <2pt>
%% \plot 0 0 9.4 3.42 10.4 3.42 / 
%% \plot 9.4 3.42 8.9 4.29 / 
%% \plot 0 0 0 0.866 /
%% \plot 1 0 1 0.866 /
%% \endpicture}
%% \figrdef{fig:trough}
%% \endfigure{Trough.}

\begin{exercise}
A woman 5 ft tall walks at the rate of 3.5 ft/sec away from a streetlight
that is 12 ft above the ground.  At what rate is the tip of her shadow
moving?  At what rate is her shadow lengthening?
\begin{answer} tip: 6 ft/s, length: $5/2$ ft/s
\end{answer}\end{exercise}

\begin{exercise} A man 1.8 meters tall walks at the rate of 1 meter per
second toward a streetlight that is 4 meters above the ground.  At
what rate is the tip of his shadow moving?  At what rate is his shadow
shortening?
\begin{answer} tip: $20/11$ m/s, length: $9/11$ m/s
\end{answer}\end{exercise}

\begin{exercise}
A police helicopter is flying at 150 mph at a constant altitude of 0.5 mile
above a straight road.  The pilot uses radar to determine that an oncoming
car is at a distance of exactly 1 mile from the helicopter, and that this
distance is decreasing at 190 mph.  Find the speed of the car.
\begin{answer} $\ds 380/\sqrt3-150\approx 69.4$ mph
\end{answer}\end{exercise}

\begin{exercise} A police helicopter is flying at 200 kilometers per hour at
a constant altitude of 1 km above a straight road.  The pilot uses
radar to determine that an oncoming car is at a distance of exactly 2
kilometers from the helicopter, and that this distance is decreasing at 250
kph.  Find the speed of the car.
\begin{answer} $\ds 500/\sqrt3-200\approx 88.7$ km/hr
\end{answer}\end{exercise}

%BADBAD
%\font\miscsymbols miscsymbols10 scaled 2000
\begin{exercise}
A light shines from the top of a pole 20 m high. A ball is falling 10
meters from the pole, casting a shadow on a building 30 meters away,
as shown in figure~\xrefn{fig:falling ball}.
When the ball is 25 meters from the ground it is falling at 6 meters
per second. How fast is its shadow moving?
\begin{answer} 18 m/s
\end{answer}\end{exercise}

%% BADBAD
%% \figure
%% \vbox{\beginpicture
%% \normalgraphs
%% \ninepoint
%% \setcoordinatesystem units <.7truecm,.7truecm>
%% \setplotarea x from 0 to 4.5, y from 0 to 4
%% \axis bottom /
%% \setlinear
%% \linethickness1pt
%% \putrule from 0 0 to 0 2
%% \put {$\bullet$} at 1 2.5
%% \put {\miscsymbols k} at 0.05 2
%% \putrule from 3 0 to 3 4
%% \putrule from 3 4 to 4.5 4
%% \setdashes <2pt>
%% \plot 0 2 3 3.5 /
%% \endpicture}
%% \figrdef{fig:falling ball}
%% \endfigure{Falling ball.}

\begin{exercise} Do example~\xrefn{exam:departing cars} assuming that the angle
between the two roads is 120${}^\circ$ instead of 90${}^\circ$ (that
is, the ``north--south'' road actually goes in a somewhat northwesterly
direction from $P$).  Recall the law of cosines:
$\ds c^2=a^2+b^2-2ab\cos\theta$.
\begin{answer} $\ds 136\sqrt{475}/19\approx 156$ km/hr
\end{answer}\end{exercise}

\begin{exercise}
Do example~\xrefn{exam:departing cars} assuming that
car A is 300 meters north of $P$, car B is 400 meters east of $P$, both
cars are going at constant speed toward $P$, and the two cars will collide in
10 seconds.
\begin{answer} $-50$ m/s
\end{answer}\end{exercise}

\begin{exercise}
Do example~\xrefn{exam:departing cars} assuming that
8 seconds ago car A started from rest at $P$ and has been picking up
speed at the steady rate of 5 m/sec${}^2$, and 6 seconds after car A
started car B passed $P$ moving east at constant speed 60 m/sec.
\begin{answer} $68$ m/s
\end{answer}\end{exercise}

\begin{exercise} Referring again to example~\xrefn{exam:departing cars},
suppose that instead of car B an airplane is flying at speed $200$
km/hr to the east of $P$ at an altitude of 2 km, as depicted in
figure~\xrefn{fig:car and airplane}. How fast is the distance between
car and airplane changing?  
\begin{answer} $\ds 3800/\sqrt{329}\approx 210$ km/hr 
\end{answer}\end{exercise}

%%BADBAD
%% \figure
%% \vbox{\beginpicture
%% \normalgraphs
%% \sevenpoint
%% \setcoordinatesystem units <0.3truecm,0.3truecm>
%% \setplotarea x from 0 to 8, y from -2 to 10
%% \setlinear
%% \plot 6 3 0 0 8 -2 /
%% \setdashes <2pt>
%% \plot 2.5 1.25 5 8.75 /
%% \plot 5 -1.25 5 8.75 /
%% \multiput {$\bullet$} at 2.5 1.25 5 8.75 /
%% \put {$A$} [br] <-2pt,2pt> at 2.5 1.25
%% \put {$B$} [b] <0pt,3pt> at 5 8.75
%% \put {$c(t)$} [br] <-2pt,2pt> at 3.75 5
%% \setplotsymbol ({\tenrm.}) 
%% \setsolid
%% \arrow <5pt> [.25, 1] from 2.5 1.25 to 4 2
%% \arrow <5pt> [.25, 1] from 5 8.75 to 7 8.25
%% \endpicture}
%% \figrdef{fig:car and airplane}
%% \endfigure{Car and airplane.}

\begin{exercise} Referring again to example~\xrefn{exam:departing cars}, suppose
that instead of car B an airplane is flying at speed $200$
km/hr to the east of $P$ at an altitude of 2 km, and that it is
gaining altitude at 10 km/hr.
How fast is
the distance between car and airplane changing?
\begin{answer} \hbox{$\ds 820/\sqrt{329}+150\sqrt{57}/\sqrt{47}\approx 210$ km/hr}
\end{answer}\end{exercise}

\begin{exercise}
A light shines from the top of a pole 20 m high.  An object is dropped from
the same height from a point 10 m away, so that its height at time $\ds t$
seconds is $\ds h(t)=20-9.8t^2/2$.  How fast is the object's shadow
moving on the ground one second later?
\begin{answer} $4000/49$ m/s
\end{answer}\end{exercise}

\begin{exercise}
The two blades of a pair of scissors are fastened at the point $A$ as
shown in figure~\xrefn{fig:scissors}.  Let
$a$ denote the distance from $A$ to the tip of the blade (the point $B$).
Let $\beta$ denote the angle at the tip of the blade that is formed by the
line $\ds \overline{AB}$ and the bottom edge of the blade, line
$\ds \overline{BC}$, and let $\theta$ denote the angle between
$\ds \overline{AB}$ and the horizontal.
Suppose that a piece of paper is cut in such a way that the center
of the scissors at $A$ is fixed, and the paper is also fixed.  As the
blades are closed (i.e., the angle $\theta$ in the diagram is decreased),
the distance $x$ between $A$ and $C$ increases, cutting the paper.
\begin{enumerate}
\item{\bf a.} Express $x$ in terms of $a$, $\theta$, and $\beta$.

\item{\bf b.} Express $dx/dt$ in terms of $a$,
$\theta$, $\beta$, and $d\theta/dt$.

\item{\bf c.} Suppose that the distance $a$ is 20 cm, and the
angle $\beta$ is $\ds 5^\circ$.  Further suppose that $\theta$ is
decreasing at 50
deg/sec.  At the instant when $\ds \theta=30^\circ$, find the rate (in
cm/sec) at which the paper is being cut.
\end{enumerate}
\begin{answer} (a) $x=a\cos\theta-a\sin\theta\cot(\theta+\beta)=
\hbox{$a\sin\beta/\sin(\theta+\beta$), (c) $\ds \dot x\approx 3.79$ cm/s}$
\end{answer}\end{exercise}


%%BADBAD
%% \figure
%% \vbox{\beginpicture
%% \normalgraphs
%% \sevenpoint
%% \setcoordinatesystem units <0.7truecm,0.7truecm>
%% \setplotarea x from -3 to 6, y from -3.5 to 3.5
%% \setlinear
%% \plot 0.3 0 6 3.5 2 0 0.3 0 /
%% \setdashes <2pt>
%% \put {\beginpicture
%% \setplotarea x from -1 to 1, y from -0.33 to 0.33
%% \startrotation by 0.866 -0.5 about 0 0
%% \ellipticalarc  axes ratio 3:1  360 degrees from 1 0 center at 0 0
%% \stoprotation\endpicture} at -1 1
%% \put {\beginpicture
%% \setplotarea x from -1 to 1, y from -0.33 to 0.33
%% \startrotation by 0.866 0.5 about 0 0
%% \ellipticalarc  axes ratio 3:1  360 degrees from 1 0 center at 0 0
%% \stoprotation\endpicture} at -1 -1
%% \put {\beginpicture
%% \setplotarea x from -1 to 1, y from -0.33 to 0.33
%% \startrotation by 0.866 0.5 about 0 0
%% \ellipticalarc  axes ratio 3:1  245 degrees from 1 0.58 center at 0 0
%% \stoprotation\endpicture} at -1 -1
%% \put {\beginpicture
%% \setplotarea x from -1 to 1, y from -0.33 to 0.33
%% \startrotation by 0.866 -0.5 about 0 0
%% \ellipticalarc  axes ratio 3:1  -245 degrees from 1 -0.58 center at 0 0
%% \stoprotation\endpicture} at -1 1
%% %\multiput {$+$} at 0 0 0.27 0.9 -1 -1 /
%% \setquadratic
%% \plot 0.27 0.9 3 2.5 6 3.5 /
%% \plot 0.27 -0.9 3 -2.5 6 -3.5 /
%% \setlinear
%% \plot 0.3 0 6 -3.5 2 0 /
%% \plot 6 -3.5 6 3.5 /
%% \plot 2 0 6 0 /
%% \put {$B$} [l] <4pt,0pt> at 6 3.5
%% \put {$A$} [r] <-3pt,0pt> at 0.3 0
%% \put {$C$} [br] <0pt,3pt> at 2 0
%% \put {$\theta$} at 1 0.2
%% \endpicture}
%% \figrdef{fig:scissors}
%% \endfigure{Scissors.}

%\epsfbox{124.figMG.ps}

\end{exercises}











\section{l'H\^{o}pital's Rule}


Now consider this limit:
$$\lim_{x\to \pi}{x^2-\pi^2\over \sin x}.$$
As $x$ approaches $\pi$, both the numerator and denominator approach
zero, so it is not obvious what, if anything, the quotient
approaches. We can often compute such limits by application of the
following theorem.

\begin{theorem} (L'H\^opital's Rule) For ``sufficiently nice'' functions $f(x)$
and $g(x)$, if $\ds\lim_{x\to a} f(x)= 0 = \lim_{x\to a}
g(x)$ or $\ds\lim_{x\to a} f(x)= \pm\infty = \lim_{x\to a}
g(x)$, and if $\ds\lim_{x\to a}{f'(x)\over g'(x)}$ exists,
then $\ds\lim_{x\to a}{f(x)\over g(x)}=\lim_{x\to
a}{f'(x)\over g'(x)}$.  This remains true if ``$x\to a$'' is replaced
by ``$x\to \infty$'' or ``$x\to -\infty$''.
\end{theorem}
\index{L'H\^opital's Rule}

This theorem is somewhat difficult to prove, in part because it
incorporates so many different possibilities, so we will not prove it
here. We also will not need to worry about the precise definition of 
``sufficiently nice'', as the functions we encounter will be
suitable. 

%% fixme: do plausibility argument with tan lines at x=a?

\begin{example}
Compute $\ds\lim_{x\to \pi}{x^2-\pi^2\over \sin x}$ in two ways.
\msk
First we use L'H\^opital's Rule: Since the numerator and denominator
both approach zero,
$$\lim_{x\to \pi}{x^2-\pi^2\over \sin x}=
\lim_{x\to \pi}{2x \over \cos x},$$
provided the latter exists. But in fact this is an easy limit, since
the denominator now approaches $-1$, so 
$$\lim_{x\to \pi}{x^2-\pi^2\over \sin x}={2\pi\over -1} = -2\pi.$$

We don't really need L'H\^opital's Rule to do this limit. Rewrite it
as 
$$\lim_{x\to \pi}(x+\pi){x-\pi\over \sin x}$$
and note that 
$$\lim_{x\to \pi}{x-\pi\over \sin x}=
\lim_{x\to \pi}{x-\pi\over -\sin (x-\pi)}=
\lim_{x\to 0}-{x\over \sin x}$$
since $x-\pi$ approaches zero as $x$ approaches $\pi$.
Now
$$\lim_{x\to \pi}(x+\pi){x-\pi\over \sin x}=
\lim_{x\to \pi}(x+\pi)\lim_{x\to 0}-{x\over \sin x}=
2\pi(-1)=-2\pi$$
as before.
\end{example}

\begin{example} Compute $\ds\lim_{x\to \infty}{2x^2-3x+7\over
x^2+47x+1}$ in two ways.
\msk
As $x$ goes to infinity both the numerator and denominator go to
infinity, so we may apply L'H\^opital's Rule:
$$\lim_{x\to \infty}{2x^2-3x+7\over x^2+47x+1}=
\lim_{x\to \infty}{4x-3\over 2x+47}.$$
In the second quotient, it is still the case that the numerator and
denominator both go to infinity, so we are allowed to use
L'H\^opital's Rule again:
$$\lim_{x\to \infty}{4x-3\over 2x+47}=\lim_{x\to \infty}{4\over 2}=2.$$
So the original limit is 2 as well.

Again, we don't really need L'H\^opital's Rule, and in fact a more
elementary approach is easier---we divide the numerator and
denominator by $\ds x^2$:
$$\lim_{x\to \infty}{2x^2-3x+7\over x^2+47x+1}=
\lim_{x\to \infty}{2x^2-3x+7\over x^2+47x+1}{{1\over x^2}\over {1\over
    x^2}}=
\lim_{x\to \infty}{2-{3\over x}+{7\over x^2}\over
1+{47\over x}+{1\over x^2}}.$$
Now as $x$ approaches infinity, all the quotients with some power of
$x$ in the denominator approach zero, leaving 2 in the numerator and 1
in the denominator, so the limit again is 2.
\end{example}

\begin{example} Compute $\ds\lim_{x\to 0}{\sec x - 1\over \sin x}$.
\msk
Both the numerator and denominator approach zero, so applying 
L'H\^opital's Rule:
$$\lim_{x\to 0}{\sec x - 1\over \sin x}=
\lim_{x\to 0}{\sec x\tan x\over \cos x}={1\cdot 0\over 1}=0.$$
\vskip -16pt
\end{example}

\begin{example} Compute $\ds\lim_{x\to 0^+} x\ln x$.

This doesn't appear to be suitable for L'H\^opital's Rule, but it also
is not ``obvious''. As $x$ approaches zero, $\ln x$ goes to $-\infty$,
so the product looks like $(\hbox{something very small})\cdot 
(\hbox{something very large and negative})$. But this could be
anything: it depends on {\it how small\/} and {\it how large}. 
For example, consider $\ds (x^2)(1/x)$, $(x)(1/x)$, and $\ds (x)(1/x^2)$. As
$x$ approaches zero, each of these is $(\hbox{something very small})\cdot 
(\hbox{something very large})$, yet the limits are respectively 
zero, $1$, and $\infty$.

We can
in fact turn this into a L'H\^opital's Rule problem:
$$x\ln x = {\ln x\over 1/x}={\ln x\over x^{-1}}.$$
Now as $x$ approaches zero, both the numerator and denominator
approach infinity (one $-\infty$ and one $+\infty$, but only the size
is important). Using  L'H\^opital's Rule:
$$\lim_{x\to 0^+} {\ln x\over x^{-1}}=
\lim_{x\to 0^+} {1/x\over -x^{-2}} =\lim_{x\to 0^+} {1\over x}(-x^2)=
\lim_{x\to 0^+} -x = 0.$$
One way to interpret this is that since $\ds\lim_{x\to
  0^+}x\ln x = 0$, the $x$ approaches zero much faster than the $\ln x$
approaches $-\infty$.
\end{example}

% Most from Keisler
\begin{exercises}

Compute the limits.

\twocol

\begin{exercise} $\ds\lim_{x\to 0} {\cos x -1\over \sin x}$
\begin{answer} $0$
\end{answer}\end{exercise}

\begin{exercise} $\ds\lim_{x\to \infty} {e^x\over x^3}$
\begin{answer} $\infty$
\end{answer}\end{exercise}

\begin{exercise} $\ds\lim_{x\to \infty} \sqrt{x^2+x}-\sqrt{x^2-x}$
\begin{answer} $1$
\end{answer}\end{exercise}

\begin{exercise} $\ds\lim_{x\to \infty} {\ln x\over x}$
\begin{answer} $0$
\end{answer}\end{exercise}

\begin{exercise} $\ds\lim_{x\to \infty} {\ln x\over \sqrt{x}}$
\begin{answer} $0$
\end{answer}\end{exercise}

\begin{exercise} $\ds\lim_{x\to\infty} {e^x + e^{-x}\over e^x -e^{-x}}$
\begin{answer} 1
\end{answer}\end{exercise}

\begin{exercise} $\ds\lim_{x\to0}{\sqrt{9+x}-3\over x}$
\begin{answer} $1/6$
\end{answer}\end{exercise}

\begin{exercise} $\ds\lim_{t\to1^+}{(1/t)-1\over t^2-2t+1}$
\begin{answer} $-\infty$
\end{answer}\end{exercise}

\begin{exercise} $\ds\lim_{x\to2}{2-\sqrt{x+2}\over 4-x^2}$
\begin{answer} $1/16$
\end{answer}\end{exercise}

\begin{exercise} $\ds\lim_{t\to\infty}{t+5-2/t-1/t^3\over 3t+12-1/t^2}$
\begin{answer} $1/3$
\end{answer}\end{exercise}

\begin{exercise} $\ds\lim_{y\to\infty}{\sqrt{y+1}+\sqrt{y-1}\over y}$
\begin{answer} $0$
\end{answer}\end{exercise}

\begin{exercise} $\ds\lim_{x\to1}{\sqrt{x}-1\over \root 1/3\of{x}-1}$
\begin{answer} $3/2$
\end{answer}\end{exercise}

\begin{exercise} $\ds\lim_{x\to0}{(1-x)^{1/4}-1\over x}$
\begin{answer} $-1/4$
\end{answer}\end{exercise}

\begin{exercise} $\ds\lim_{t\to 0}{\left(t+{1\over t}\right)((4-t)^{3/2}-8)}$
\begin{answer} $-3$
\end{answer}\end{exercise}

\begin{exercise} $\ds\lim_{t\to 0^+}\left({1\over t}+{1\over\sqrt{t}}\right)
(\sqrt{t+1}-1)$
\begin{answer} $1/2$
\end{answer}\end{exercise}

\begin{exercise} $\ds\lim_{x\to 0}{x^2\over\sqrt{2x+1}-1}$
\begin{answer} $0$
\end{answer}\end{exercise}

\begin{exercise} $\ds\lim_{u\to 1}{(u-1)^3\over (1/u)-u^2+3/u-3}$
\begin{answer} $-1$
\end{answer}\end{exercise}

\begin{exercise} $\ds\lim_{x\to 0}{2+(1/x)\over 3-(2/x)}$
\begin{answer} $-1/2$
\end{answer}\end{exercise}

\begin{exercise} $\ds\lim_{x\to 0^+}{1+5/\sqrt{x}\over 2+1/\sqrt{x}}$
\begin{answer} $5$
\end{answer}\end{exercise}

\begin{exercise} $\ds\lim_{x\to 0^+}{3+x^{-1/2}+x^{-1}\over 2+4x^{-1/2}}$
\begin{answer} $\infty$
\end{answer}\end{exercise}

\begin{exercise} $\ds\lim_{x\to\infty}{x+x^{1/2}+x^{1/3}\over x^{2/3}+x^{1/4}}$
\begin{answer} $\infty$
\end{answer}\end{exercise}

\begin{exercise} $\ds\lim_{t\to\infty}
{1-\sqrt{t\over t+1}\over 2-\sqrt{4t+1\over t+2}}$
\begin{answer} $2/7$
\end{answer}\end{exercise}

\begin{exercise} $\ds\lim_{t\to\infty}{1-{t\over t-1}\over 1-\sqrt{t\over t-1}}$
\begin{answer} $2$
\end{answer}\end{exercise}

\begin{exercise} $\ds\lim_{x\to-\infty}{x+x^{-1}\over 1+\sqrt{1-x}}$
\begin{answer} $-\infty$
\end{answer}\end{exercise}

\begin{exercise} $\ds\lim_{x\to1}{\int_1^x 1/t\,dt\over\int_1^x 1/(2t+1)\,dt}$
\begin{answer} $3$
\end{answer}\end{exercise}

\begin{exercise} $\ds\lim_{x\to\infty}{\int_1^x\sqrt{t+(1/t)}\,dt\over x\sqrt{x}}$
\begin{answer} $2/3$
\end{answer}\end{exercise}

\begin{exercise} $\ds\lim_{x\to\pi/2}{\cos x\over (\pi/2)-x}$
\begin{answer} $1$
\end{answer}\end{exercise}

\begin{exercise} $\ds\lim_{x\to0}{e^x-1\over x}$
\begin{answer} $1$
\end{answer}\end{exercise}

\begin{exercise} $\ds\lim_{x\to0}{x^2\over e^x-x-1}$
\begin{answer} $2$
\end{answer}\end{exercise}

\begin{exercise} $\ds\lim_{x\to1}{\ln x\over x-1}$
\begin{answer} $1$
\end{answer}\end{exercise}

\begin{exercise} $\ds\lim_{x\to0}{\ln(x^2+1)\over x}$
\begin{answer} $0$
\end{answer}\end{exercise}

\begin{exercise} $\ds\lim_{x\to1}{x\ln x\over x^2-1}$
\begin{answer} $1/2$
\end{answer}\end{exercise}

\begin{exercise} $\ds\lim_{x\to0}{\sin(2x)\over\ln(x+1)}$
\begin{answer} $2$
\end{answer}\end{exercise}

\begin{exercise} $\ds\lim_{x\to1}{x^{1/4}-1\over x}$
\begin{answer} $0$
\end{answer}\end{exercise}

\begin{exercise} $\ds\lim_{x\to1^+}{\sqrt{x}\over x-1}$
\begin{answer} $\infty$
\end{answer}\end{exercise}

\begin{exercise} $\ds\lim_{x\to1}{\sqrt{x}-1\over x-1}$
\begin{answer} $1/2$
\end{answer}\end{exercise}

\begin{exercise} $\ds\lim_{x\to\infty}{x^{-1}+x^{-1/2}\over x+x^{-1/2}}$
\begin{answer} $0$
\end{answer}\end{exercise}

\begin{exercise} $\ds\lim_{x\to\infty}{x+x^{-2}\over 2x+x^{-2}}$
\begin{answer} $1/2$
\end{answer}\end{exercise}

\begin{exercise} $\ds\lim_{x\to\infty}{5+x^{-1}\over 1+2x^{-1}}$
\begin{answer} $5$
\end{answer}\end{exercise}

\begin{exercise} $\ds\lim_{x\to\infty}{4x\over\sqrt{2x^2+1}}$
\begin{answer} $\ds 2\sqrt2$
\end{answer}\end{exercise}

\begin{exercise} $\ds\lim_{x\to0}{3x^2+x+2\over x-4}$
\begin{answer} $-1/2$
\end{answer}\end{exercise}

\begin{exercise} $\ds\lim_{x\to0}{\sqrt{x+1}-1\over \sqrt{x+4}-2}$
\begin{answer} $2$
\end{answer}\end{exercise}

\begin{exercise} $\ds\lim_{x\to0}{\sqrt{x+1}-1\over \sqrt{x+2}-2}$
\begin{answer} $0$
\end{answer}\end{exercise}

\begin{exercise} $\ds\lim_{x\to0^+}{\sqrt{x+1}+1\over\sqrt{x+1}-1}$
\begin{answer} $\infty$
\end{answer}\end{exercise}

\begin{exercise} $\ds\lim_{x\to0}{\sqrt{x^2+1}-1\over\sqrt{x+1}-1}$
\begin{answer} $0$
\end{answer}\end{exercise}

\begin{exercise} $\ds\lim_{x\to\infty}{(x+5)\left({1\over 2x}+{1\over x+2}\right)}$
\begin{answer} $3/2$
\end{answer}\end{exercise}

\begin{exercise} $\ds\lim_{x\to0^+}{(x+5)\left({1\over 2x}+{1\over x+2}\right)}$
\begin{answer} $\infty$
\end{answer}\end{exercise}

\begin{exercise} $\ds\lim_{x\to1}{(x+5)\left({1\over 2x}+{1\over x+2}\right)}$
\begin{answer} $5$
\end{answer}\end{exercise}

\begin{exercise} $\ds\lim_{x\to2}{x^3-6x-2\over x^3+4}$
\begin{answer} $-1/2$
\end{answer}\end{exercise}

\begin{exercise} $\ds\lim_{x\to2}{x^3-6x-2\over x^3-4x}$
\begin{answer} does not exist
\end{answer}\end{exercise}

\begin{exercise} $\ds\lim_{x\to1+}{x^3+4x+8\over 2x^3-2}$
\begin{answer} $\infty$
\end{answer}\end{exercise}

\endtwocol

\msk
\begin{exercise} The function $\ds f(x) = {x\over\sqrt{x^2+1}}$ has two horizontal
 asymptotes.  Find them and give a rough sketch of $f$ with its horizontal
 asymptotes. 
\begin{answer} $y=1$ and $y=-1$
\end{answer}\end{exercise}

\end{exercises}

