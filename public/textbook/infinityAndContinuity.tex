\chapter{Infinity and Continuity}


\section{Infinite Limits}


Consider the function
\[
f(x) = \frac{1}{(x+1)^2}
\]
While the $\lim_{x\to -1} f(x)$ does not exist, see
Figure~\ref{plot:1/(x+1)^2}, something can still be said.
\begin{marginfigure}
\begin{tikzpicture}
	\begin{axis}[
            domain=-2:1,
            ymax=100,
            samples=100,
            axis lines =middle, xlabel=$x$, ylabel=$y$,
            every axis y label/.style={at=(current axis.above origin),anchor=south},
            every axis x label/.style={at=(current axis.right of origin),anchor=west}
          ]
	  \addplot [very thick, penColor, smooth, domain=(-2:-1.1)] {1/(x+1)^2};
          \addplot [very thick, penColor, smooth, domain=(-.9:1)] {1/(x+1)^2};
          \addplot [textColor, dashed] plot coordinates {(-1,0) (-1,100)};
        \end{axis}
\end{tikzpicture}
\caption{A plot of $f(x)=\protect\frac{1}{(x+1)^2}$.}
\label{plot:1/(x+1)^2}
\end{marginfigure}


\begin{definition}\label{def:inflimit}\index{limit!infinite}\index{infinite limit}
If $f(x)$ grows arbitrarily large as $x$ approaches $a$, we write
\[
\lim_{x\to a} f(x) = \infty
\]
and say that the limit of $f(x)$ \textbf{approaches infinity} as $x$
goes to $a$.

If $|f(x)|$ grows arbitrarily large as $x$ approaches $a$ and $f(x)$ is
negative, we write
\[
\lim_{x\to a} f(x) = -\infty
\]
and say that the limit of $f(x)$ \textbf{approaches negative infinity}
as $x$ goes to $a$.
\end{definition}

On the other hand, if we consider the function 
\[
f(x) = \frac{1}{(x-1)}
\]
While we have $\lim_{x\to 1} f(x) \ne \pm\infty$, we do have one-sided
limits, $\lim_{x\to 1+} f(x) = \infty$ and $\lim_{x\to 1-} f(x) =
-\infty$, see Figure~\ref{plot:1/(x-1)}.


\begin{definition}\label{def:vert asymptote}\index{asymptote!vertical}\index{vertical asymptote}
If 
\[
\lim_{x\to a} f(x) = \pm\infty, \qquad \lim_{x\to a+} f(x) = \pm\infty, \qquad\text{or}\qquad \lim_{x\to a-} f(x) = \pm\infty,
\]
then the line $x=a$ is a \textbf{vertical asymptote} of $f(x)$.
\end{definition}


\begin{example}
Find the vertical asymptotes of 
\[
f(x) = \frac{x^2-9x+14}{x^2-5x+6}.
\]
\end{example}

\begin{marginfigure}
\begin{tikzpicture}
	\begin{axis}[
            domain=-1:2,
            ymax=50,
            ymin=-50,
            samples=100,
            axis lines =middle, xlabel=$x$, ylabel=$y$,
            every axis y label/.style={at=(current axis.above origin),anchor=south},
            every axis x label/.style={at=(current axis.right of origin),anchor=west}
          ]
	  \addplot [very thick, penColor, smooth, domain=(1.02:2)] {1/(x-1)};
          \addplot [very thick, penColor, smooth, domain=(-1:.98)] {1/(x-1)};
          \addplot [textColor, dashed] plot coordinates {(1,-50) (1,50)};
        \end{axis}
\end{tikzpicture}
\caption{A plot of $f(x)=\protect\frac{1}{x-1}$.}
\label{plot:1/(x-1)}
\end{marginfigure}




\begin{solution}
Start by factoring both the numerator and the denominator:
\[
\frac{x^2-9x+14}{x^2-5x+6} = \frac{(x-2)(x-7)}{(x-2)(x-3)}
\]
Using limits, we must investigate when $x\to 2$ and $x\to 3$. Write
\begin{align*}
\lim_{x\to 2} \frac{(x-2)(x-7)}{(x-2)(x-3)} &= \lim_{x\to 2} \frac{(x-7)}{(x-3)}\\
&= \frac{-5}{-1}\\
&=5.
\end{align*}
Now write
\begin{align*}
\lim_{x\to 3} \frac{(x-2)(x-7)}{(x-2)(x-3)} &= \lim_{x\to 3} \frac{(x-7)}{(x-3)}\\
&= \lim_{x\to 3}\frac{-4}{x-3}.
\end{align*}
Since $\lim_{x\to 3+} x-3$ approaches $0$ from the right and the
numerator is negative, $\lim_{x\to 3+} f(x) = -\infty$. Since
$\lim_{x\to 3-} x-3$ approaches $0$ from the left and the numerator is
negative, $\lim_{x\to 3-} f(x) = \infty$. Hence we have a vertical
asymptote at $x=3$, see Figure~\ref{plot:(x^2-9x+14)/(x^2-5x+6)}.
\end{solution}
\begin{marginfigure}[0in]
\begin{tikzpicture}
	\begin{axis}[
            domain=1:4,
            ymax=50,
            ymin=-50,
            samples=100,
            axis lines =middle, xlabel=$x$, ylabel=$y$,
            every axis y label/.style={at=(current axis.above origin),anchor=south},
            every axis x label/.style={at=(current axis.right of origin),anchor=west}
          ]
	  \addplot [very thick, penColor, smooth, domain=(3.02:4)] {(x-7)/(x-3)};
          \addplot [very thick, penColor, smooth, domain=(1:2.98)] {(x-7)/(x-3)};
          \addplot [textColor, dashed] plot coordinates {(3,-50) (3,50)};
          \addplot[color=penColor,fill=background,only marks,mark=*] coordinates{(2,5)};  %% open hole
        \end{axis}
\end{tikzpicture}
\caption{A plot of $f(x)=\protect\frac{x^2-9x+14}{x^2-5+6}$.}
\label{plot:(x^2-9x+14)/(x^2-5x+6)}
\end{marginfigure}


\begin{exercises}

\noindent Compute the limits. If a limit does not exist, explain why.
\twocol
\begin{exercise}
$\lim_{x\to 1-} \frac{1}{x^2-1}$
\begin{answer}
  $-\infty$
\end{answer}
\end{exercise}

\begin{exercise}
$\lim_{x\to 4-} \frac{3}{x^2-2}$
\begin{answer}
  $3/14$
\end{answer}
\end{exercise}

\begin{exercise}
$\lim_{x\to -1+} \frac{1+2x}{x^3-1}$
\begin{answer}
  $1/2$
\end{answer}
\end{exercise}

\begin{exercise}
$\lim_{x\to 3+} \frac{x-9}{x^2-6x+9}$
\begin{answer}
  $-\infty$
\end{answer}
\end{exercise}

\begin{exercise}
$\lim_{x\to 5} \frac{1}{(x-5)^4}$
\begin{answer}
  $\infty$
\end{answer}
\end{exercise}


\begin{exercise}
$\lim_{x\to -2} \frac{1}{(x^2+3x+2)^2}$
\begin{answer}
  $\infty$
\end{answer}
\end{exercise}


\begin{exercise}
$\lim_{x\to 0} \frac{1}{\frac{x}{x^5}-\cos(x)}$
\begin{answer}
  0
\end{answer}
\end{exercise}

\begin{exercise}
$\lim_{x\to 0+} \frac{x-11}{\sin(x)}$
\begin{answer}
  $-\infty$
\end{answer}
\end{exercise}


\endtwocol


\begin{exercise}
Find the vertical asymptotes of
\[
f(x) = \frac{x-3}{x^2+2x-3}.
\]
\begin{answer}
  $x = 1$ and $x = -3$
\end{answer}
\end{exercise}


\begin{exercise}
Find the vertical asymptotes of
\[
f(x) = \frac{x^2-x-6}{x+4}.
\]
\begin{answer}
  $x = -4$
\end{answer}
\end{exercise}
\end{exercises}






\section{Limits at Infinity}


Consider the function:
\[
f(x) = \frac{6x-9}{x-1}
\]
\begin{marginfigure}[0in]
\begin{tikzpicture}
	\begin{axis}[
            domain=1:4,
            ymax=20,
            ymin=-10,
            samples=100,
            axis lines =middle, xlabel=$x$, ylabel=$y$,
            every axis y label/.style={at=(current axis.above origin),anchor=south},
            every axis x label/.style={at=(current axis.right of origin),anchor=west}
          ]
	  \addplot [very thick, penColor, smooth, domain=(0:.9)] {(6*x-9)/(x-1)};
          \addplot [very thick, penColor, smooth, domain=(1.1:3)] {(6*x-9)/(x-1)};
          \addplot [textColor, dashed] plot coordinates {(1,-10) (1,20)};
        \end{axis}
\end{tikzpicture}
\caption{A plot of $f(x)=\protect\frac{6x-9}{x-1}$.}
\label{plot:(6x-9)/(x-1)}
\end{marginfigure}
As $x$ approaches infinity, it seems like $f(x)$ approaches a specific
value. This is a \textit{limit at infinity}.

\begin{definition}\label{def:limitAtInfty}\index{limit!at infinity}
If $f(x)$ becomes arbitrarily close to a specific value $L$ by making
$x$ sufficiently large, we write
\[
\lim_{x\to \infty} f(x) = L
\]
and we say, the \textbf{limit at infinity} of $f(x)$ is $L$.  

If $f(x)$ becomes
arbitrarily close to a specific value $L$ by making $x$ sufficiently
large and negative, we write
\[
\lim_{x\to -\infty} f(x) = L
\]
and we say, the \textbf{limit at negative infinity} of $f(x)$ is $L$.  
\end{definition}

\begin{example} Compute
\[
\lim_{x\to\infty} \frac{6x-9}{x-1}.
\]
\end{example}


\begin{solution}
Write
\begin{align*}
\lim_{x\to\infty}\frac{6x-9}{x-1} &= \lim_{x\to\infty}\frac{6x-9}{x-1} \frac{1/x}{1/x}\\
&=\lim_{x\to\infty}\frac{\frac{6x}{x} - \frac{9}{x}}{\frac{x}{x} - \frac{1}{x}}\\
&= \lim_{x\to\infty} \frac{6}{1}\\
&= 6.
\end{align*}
\end{solution}

Sometimes one must be careful, consider this example.

\begin{example}
Compute
\[
\lim_{x\to -\infty} \frac{x+1}{\sqrt{x^2}}
\]
\end{example}

\begin{solution}
In this case, the denominator is positive when $x$ is large, and the numerator is negative, hence
\begin{align*}
\lim_{x\to -\infty} \frac{x+1}{\sqrt{x^2}} &= -\lim_{x\to \infty} \frac{x+1}{\sqrt{x^2}}\\
&= -\lim_{x\to \infty} \frac{x+1}{\sqrt{x^2}} \cdot \frac{1/x}{1/x}\\
&= -\lim_{x\to \infty} \frac{1+1/x}{\sqrt{x^2/x^2}}\\
&= -1.
\end{align*}
\end{solution}


Here is a somewhat different example of a limit at infinity.

\begin{example}
Compute
\[
\lim_{x\to \infty} \frac{\sin(7x)}{x}+4.
\]
\end{example}

\begin{marginfigure}[0in]
\begin{tikzpicture}
	\begin{axis}[
            domain=2:20,
            ymax=5,
            ymin=3,
            samples=100,
            axis lines =middle, xlabel=$x$, ylabel=$y$,
            every axis y label/.style={at=(current axis.above origin),anchor=south},
            every axis x label/.style={at=(current axis.right of origin),anchor=west}
          ]
	  \addplot [very thick, penColor, smooth] {(1/x) * sin(deg(7*x))+4};
        \end{axis}
\end{tikzpicture}
\caption{A plot of $f(x)=\frac{\sin(7x)}{x}+4$.}
\label{plot:sin7x/x+4}
\end{marginfigure}

\begin{solution}
We can bound our function
\[
-1/x + 4 \le \frac{\sin(7x)}{x}+4 \le 1/x + 4.
\]
Since 
\[
\lim_{x\to \infty} -1/x + 4 = 4 = \lim_{x\to \infty}1/x + 4
\] 
we conclude by the Squeeze Theorem, Theorem~\ref{theorem:squeeze},
$\lim_{x\to\infty}\frac{\sin(7x)}{x}+4 = 4$.
\end{solution}






\begin{definition}\label{def:horiz asymptote}\index{asymptote!horizontal}\index{horizontal asymptote}
If  
\[
\lim_{x\to \infty} f(x) = L \qquad\text{or}\qquad \lim_{x\to -\infty} f(x) = L,
\]
then the line $y=L$ is a \textbf{horizontal asymptote} of $f(x)$.
\end{definition}

\begin{example} 
Give the horizontal asymptotes of
\[
f(x) = \frac{6x-9}{x-1}
\]
\end{example}

\begin{solution}
From our previous work, we see that $\lim_{x\to \infty} f(x) = 6$, and
upon further inspection, we see that $\lim_{x\to -\infty} f(x) =
6$. Hence the horizontal asymptote of $f(x)$ is the line $y=6$.
\end{solution}


It is a common misconception that a function cannot cross an
asymptote. As the next example shows, a function can cross an
asymptote, and in this case this occurs an infinite number of times!

\begin{example}
Give a horizontal asymptote of
\[
f(x) = \frac{\sin(7x)}{x}+4.
\]
\end{example}

\begin{solution}
Again from previous work, we see that $\lim_{x\to \infty} f(x) =
4$. Hence $y=4$ is a horizontal asymptote of $f(x)$.
\end{solution}


We conclude with an infinite limit at infinity.

\begin{example}
Compute
\[
\lim_{x\to \infty} \ln(x)
\]
\end{example}
\begin{marginfigure}[0in]
\begin{tikzpicture}
	\begin{axis}[
            domain=0:20,
            ymax=4,
            ymin=-1,
            samples=100,
            axis lines =middle, xlabel=$x$, ylabel=$y$,
            every axis y label/.style={at=(current axis.above origin),anchor=south},
            every axis x label/.style={at=(current axis.right of origin),anchor=west}
          ]
	  \addplot [very thick, penColor, smooth] {ln(x)};
        \end{axis}
\end{tikzpicture}
\caption{A plot of $f(x)=\ln(x)$.}
\label{plot:lnx}
\end{marginfigure}

\begin{solution}
The function $\ln(x)$ grows very slowly, and seems like it may have a
horizontal asymptote, see Figure~\ref{plot:lnx}. However, if we
consider the definition of the natural log
\[
\ln(x) = y \qquad \Leftrightarrow\qquad e^y =x
\]
Since we need to raise $e$ to higher and higher values to obtain
larger numbers, we see that $\ln(x)$ is unbounded, and hence
$\lim_{x\to\infty}\ln(x)=\infty$.
\end{solution}


\begin{exercises}

\noindent Compute the limits.
\twocol
\begin{exercise}
$\lim_{x\to \infty} \frac{1}{x}$
\begin{answer}
$0$
\end{answer}
\end{exercise}

\begin{exercise}
$\lim_{x\to \infty} \frac{-x}{\sqrt{4+x^2}}$
\begin{answer}
$-1$
\end{answer}
\end{exercise}

\begin{exercise}
$\lim_{x\to \infty} \frac{2x^2-x+1}{4x^2-3x-1}$
\begin{answer}
$\frac{1}{2}$
\end{answer}
\end{exercise}

\begin{exercise}
$\lim_{x\to -\infty} \frac{3x+7}{\sqrt{x^2}}$
\begin{answer}
$-3$
\end{answer}
\end{exercise}

\begin{exercise}
$\lim_{x\to -\infty} \frac{2x+7}{\sqrt{x^2+2x-1}}$
\begin{answer}
$-2$
\end{answer}
\end{exercise}

\begin{exercise}
$\lim_{x\to -\infty} \frac{x^3-4}{3x^2+4x-1}$
\begin{answer}
$-\infty$
\end{answer}
\end{exercise}


\begin{exercise}
$\lim_{x\to \infty} \left(\frac{4}{x}+\pi\right)$
\begin{answer}
$\pi$
\end{answer}
\end{exercise}

\begin{exercise}
$\lim_{x\to \infty} \frac{\cos(x)}{\ln(x)}$
\begin{answer}
$0$
\end{answer}
\end{exercise}

\begin{exercise}
$\lim_{x\to \infty} \frac{\sin\left(x^7\right)}{\sqrt{x}}$
\begin{answer}
$0$
\end{answer}
\end{exercise}

\begin{exercise}
$\lim_{x\to \infty} \left(17 + \frac{32}{x} - \frac{\left(\sin(x/2)\right)^2}{x^3}\right)$
\begin{answer}
$17$
\end{answer}
\end{exercise}

\endtwocol

\begin{exercise}
Suppose a population of feral cats on a certain college campus $t$
years from now is approximated by
\[
p(t) = \frac{1000}{5+ 2e^{-0.1 t}}.
\]
Approximately how many feral cats are on campus 10 years from now? 50
years from now? 100 years from now? 1000 years from now? What do you
notice about the prediction---is this realistic?
\begin{answer}
After 10 years, $\approx 174$ cats; after 50 years, $\approx 199$
cats; after 100 years, $\approx 200$ cats; after 1000 years, $\approx
200$ cats; in the sense that the population of cats cannot grow
indefinitely this is somewhat realistic.
\end{answer}
\end{exercise}

\begin{exercise}
The amplitude of an oscillating spring is given by
\[
a(t) = \frac{\sin(t)}{t}.
\]
What happens to the amplitude of the oscillation over a long period of
time?
\begin{answer}
The amplitude goes to zero. 
\end{answer}
\end{exercise}
\end{exercises}




\section{Continuity}


Informally, a function is continuous if you can ``draw it'' without
``lifting your pencil.'' We need a formal definition.

\begin{definition} A function $f$ is \textbf{continuous
at a point} $a$ if $\lim_{x\to a} f(x) = f(a)$.  
\end{definition}
\index{continuous}
\begin{marginfigure}[0in]
\begin{tikzpicture}
	\begin{axis}[
            domain=0:10,
            ymax=5,
            ymin=0,
            samples=100,
            axis lines =middle, xlabel=$x$, ylabel=$y$,
            every axis y label/.style={at=(current axis.above origin),anchor=south},
            every axis x label/.style={at=(current axis.right of origin),anchor=west}
          ]
	  \addplot [very thick, penColor, smooth, domain=(4:10)] {3 + sin(deg(x*2))/(x-1)};
          \addplot [very thick, penColor, smooth, domain=(0:4)] {1};
          \addplot[color=penColor,fill=background,only marks,mark=*] coordinates{(4,3.30)};  %% open hole
          \addplot[color=penColor,fill=background,only marks,mark=*] coordinates{(6,2.893)};  %% open hole
          \addplot[color=penColor,fill=penColor,only marks,mark=*] coordinates{(4,1)};  %% closed hole
          \addplot[color=penColor,fill=penColor,only marks,mark=*] coordinates{(6,2)};  %% closed hole
        \end{axis}
\end{tikzpicture}
\caption{A plot of a function with discontinuities at $x=4$ and $x=6$.}
\label{plot:discontinuous-function}
\end{marginfigure}

\begin{example}
Find the discontinuities (the values for $x$ where a function is not
continuous) for the function given in Figure~\ref{plot:discontinuous-function}.
\end{example}
\begin{solution}
From Figure~\ref{plot:discontinuous-function} we see that $\lim_{x\to 4} f(x)$ does not exist as
\[
\lim_{x\to 4-}f(x) = 1\qquad\text{and}\qquad \lim_{x\to 4+}f(x) \approx 3.5
\]
Hence $\lim_{x\to 4} f(x) \ne f(4)$, and so $f(x)$ is not
continuous at $x=4$.

We also see that $\lim_{x\to 6} f(x) \approx 3$ while $f(6) =
2$. Hence $\lim_{x\to 6} f(x) \ne f(6)$, and so $f(x)$ is not
continuous at $x=6$.
\end{solution}

Building from the definition of \textit{continuous at a point}, we can
now define what it means for a function to be \textit{continuous} on
an interval.

\begin{definition} A function $f$ is \textbf{continuous on an interval} if it is
continuous at every point in the interval.
\end{definition}

In particular, we should note that if a function is not defined on an
interval, then it \textbf{cannot} be continuous on that interval.
\begin{marginfigure}[0in]
\begin{tikzpicture}
	\begin{axis}[
            domain=-.2:.2,    
            samples=500,
            axis lines =middle, xlabel=$x$, ylabel=$y$,
            yticklabels = {}, 
            every axis y label/.style={at=(current axis.above origin),anchor=south},
            every axis x label/.style={at=(current axis.right of origin),anchor=west},
            clip=false,
          ]
	  \addplot [very thick, penColor, smooth, domain=(-.2:-.02)] {abs(x)^(1/5)*sin(deg(1/x))};
          \addplot [very thick, penColor, smooth, domain=(.02:.2)] {x^(1/5)*sin(deg(1/x))};
	  \addplot [color=penColor, fill=penColor, very thick, smooth,domain=(-.02:.02)] {abs(x)^(1/5)} \closedcycle;
          \addplot [color=penColor, fill=penColor, very thick, smooth,domain=(-.02:.02)] {-abs(x)^(1/5)} \closedcycle;
        \end{axis}
\end{tikzpicture}
\caption[A continuous function.]{A plot of
\[
f(x)=
\begin{cases}
\sqrt[5]{x}\sin\left(\frac{1}{x}\right) & \text{if $x \ne 0$,}\\
 0 & \text{if $x = 0$.}
\end{cases}
\]
}
\label{plot:sqrt[5]{x}sin 1/x}
\end{marginfigure}

\begin{example}
Consider the function
\[
f(x) = 
\begin{cases}
\sqrt[5]{x}\sin\left(\frac{1}{x}\right) & \text{if $x \ne 0$,}\\
0 & \text{if $x = 0$,}
\end{cases}
\]
see Figure~\ref{plot:sqrt[5]{x}sin 1/x}. Is this function continuous?
\end{example}

\begin{solution}
Considering $f(x)$, the only issue is when $x=0$. We must show that
$\lim_{x\to 0} f(x) = 0$. Note
\[
-|\sqrt[5]{x}|\le f(x) \le |\sqrt[5]{x}|.
\]
Since
\[
\lim_{x\to 0} -|\sqrt[5]{x}| = 0 = \lim_{x\to 0}|\sqrt[5]{x}|,
\]
we see by the Squeeze Theorem, Theorem~\ref{theorem:squeeze}, that
$\lim_{x\to 0} f(x) = 0$. Hence $f(x)$ is continuous.

Here we see how the informal definition of continuity being that you
can ``draw it'' without ``lifting your pencil'' differs from the
formal definition.
\end{solution}

We close with a useful theorem about continuous functions:

\begin{mainTheorem}[Intermediate Value Theorem]\label{theorem:IVT}
If $f(x)$ is a continuous function for all $x$ in the closed interval
$[a,b]$ and $d$ is between $f(a)$ and $f(b)$, then there is a number
$c$ in $[a, b]$ such that $f(c) = d$.
\end{mainTheorem}
\marginnote[-1.2in]{The Intermediate Value Theorem is most frequently
  used when $d=0$.}  
\marginnote[-.7in]{For a nice proof of this theorem, see: Walk, Stephen
  M. \textit{The intermediate value theorem is NOT obvious---and I am
    going to prove it to you}. College Math. J. 42 (2011), no. 4,
  254--259.}
In Figure~\ref{figure:intermediate-value}, we see a geometric
interpretation of this theorem.

\begin{marginfigure}
\begin{tikzpicture}
	\begin{axis}[
            domain=0:6, ymin=0, ymax=2.2,xmax=6,
            axis lines =left, xlabel=$x$, ylabel=$y$,
            every axis y label/.style={at=(current axis.above origin),anchor=south},
            every axis x label/.style={at=(current axis.right of origin),anchor=west},
            xtick={1,3.597,5}, ytick={.203,1,1.679},
            xticklabels={$a$,$c$,$b$}, yticklabels={$f(a)$,$f(c)=d$,$f(b)$},
            axis on top,
          ]
          \addplot [draw=none, fill=fill2, domain=(0:7)] {1.679} \closedcycle;
          \addplot [draw=none, fill=background, domain=(0:7)] {.203} \closedcycle;
          \addplot [textColor,dashed] plot coordinates {(0,1.679) (6,1.679)};
          \addplot [textColor,dashed] plot coordinates {(0,.203) (6,.203)};
          \addplot [textColor,dashed] plot coordinates {(5,0) (5,1.679)};
          \addplot [textColor,dashed] plot coordinates {(1,0) (1,.203)};
          \addplot [textColor,dashed] plot coordinates {(3.587,0) (3.597,1)};
          \addplot [penColor2,domain=(0:6)] {1};
          \addplot [very thick,penColor, smooth,domain=(0:2.5)] {sin(deg((x - 4)/2)) + 1.2};
          \addplot [very thick,penColor, smooth,domain=(4:6)] {sin(deg((x - 4)/2)) + 1.2};
          \addplot [very thick,dashed,penColor!50!background, smooth,domain=(2.5:4)] {sin(deg((x - 4)/2)) + 1.2}; 
          \addplot [color=penColor!50!background,fill=penColor!50!background,only marks,mark=*] coordinates{(3.587,1)};  %% closed hole          
          \addplot [color=penColor,fill=penColor,only marks,mark=*] coordinates{(1,.203)};  %% closed hole          
          \addplot [color=penColor,fill=penColor,only marks,mark=*] coordinates{(5,1.679)};  %% closed hole          
        \end{axis}
\end{tikzpicture}
\caption{A geometric interpretation of the Intermediate Value
  Theorem. The function $f(x)$ is continuous on the interval
  $[a,b]$. Since $d$ is in the interval $[f(a),f(b)]$, there exists a
  value $c$ in $[a,b]$ such that $f(c) = d$.}
\label{figure:intermediate-value}
\end{marginfigure}





\begin{example} 
Explain why the function $f(x) =x^3 + 3x^2+x-2$ has a root between 0
and 1.
\end{example}

\begin{solution}
By Theorem~\ref{theorem:limit-laws}, $\lim_{x\to a} f(x) = f(a)$, for
all real values of $a$, and hence $f$ is continuous.  Since $f(0)=-2$
and $f(1)=3$, and $0$ is between $-2$ and $3$, by the Intermediate
Value Theorem, Theorem~\ref{theorem:IVT}, there is a $c\in[0,1]$ such
that $f(c)=0$.
\end{solution}

This example also points the way to a simple method for approximating
roots. 

\begin{example} 
Approximate a root of $f(x) =x^3 + 3x^2+x-2$ to one decimal place.
\end{example}
\begin{solution}
If we compute $f(0.1)$, $f(0.2)$, and so on, we find that $f(0.6)<0$
and $f(0.7)>0$, so by the Intermediate Value Theorem, $f$ has a root
between $0.6$ and $0.7$. Repeating the process with $f(0.61)$,
$f(0.62)$, and so on, we find that $f(0.61)<0$ and $f(0.62)>0$, so by
the Intermediate Value Theorem, Theorem~\ref{theorem:IVT}, $f(x)$ has
a root between $0.61$ and $0.62$, and the root is $0.6$ rounded to one
decimal place.
\end{solution}





\begin{exercises}

\begin{exercise} 
Consider the function
\[
f(x) = \sqrt{x-4} 
\]
Is $f(x)$ continuous at the point $x=4$?  Is $f(x)$ a continuous
function on $\R$?
\begin{answer}
  $f(x)$ is continuous at $x=4$ but it is not continuous on $\R$.
\end{answer}
\end{exercise}


\begin{exercise} 
Consider the function
\[
f(x) = \frac{1}{x+3}
\]
Is $f(x)$ continuous at the point $x=3$?  Is $f(x)$ a continuous
function on $\R$?
\begin{answer}
  $f(x)$ is continuous at $x=3$ but it is not continuous on $\R$.
\end{answer}
\end{exercise}

\begin{exercise} 
Consider the function
\[
f(x) = 
\begin{cases} 
2x - 3 & \text{if $x<1$,} \\ 
0      & \text{if $x\geq 1$.}
\end{cases}
\]
Is $f(x)$ continuous at the point $x=1$?  Is $f(x)$ a continuous
function on $\R$?
\begin{answer}
  $f(x)$ is not continuous at $x=1$ and it is not continuous on $\R$.
\end{answer}
\end{exercise}



\begin{exercise} 
Consider the function
\[
f(x) = 
\begin{cases} 
\frac{x^2 + 10x + 25}{x-5} & \text{if $x\ne 5$,} \\ 
10      & \text{if $x= 5$.}
\end{cases}
\]
Is $f(x)$ continuous at the point $x=5$?  Is $f(x)$ a continuous
function on $\R$?
\begin{answer}
  $f(x)$ is not continuous at $x=5$ and it is not continuous on $\R$.
\end{answer}
\end{exercise}


\begin{exercise} 
Consider the function
\[
f(x) = 
\begin{cases} 
\frac{x^2 + 10x + 25}{x+5} & \text{if $x\ne -5$,} \\ 
0      & \text{if $x= -5$.}
\end{cases}
\]
Is $f(x)$ continuous at the point $x=-5$?  Is $f(x)$ a continuous
function on $\R$?
\begin{answer}
  $f(x)$ is continuous at $x=-5$ and it is also continuous on $\R$.
\end{answer}
\end{exercise}



\begin{exercise} 
Determine the interval(s) on which the function $f(x) = x^7+3x^5-2x +
4$ is continuous.
\begin{answer}
  $\R$
\end{answer}
\end{exercise}



\begin{exercise}
Determine the interval(s) on which the function $f(x) = \frac{x^2-2x+1}{x+4}$ is continuous. 
\begin{answer}
  $(-\infty,-4)\cup(-4,\infty)$
\end{answer}
\end{exercise}



\begin{exercise}
Determine the interval(s) on which the function $f(x) = \frac{1}{x^2
  -9}$ is continuous.
\begin{answer}
  $(-\infty,-3)\cup(-3,3)\cup(3,\infty)$
\end{answer}
\end{exercise}


\begin{exercise}
Approximate a root of $f(x)=x^3-4x^2+2x+2$ to two decimal places.
\begin{answer}
$x=-0.48$, $x=1.31$, or $x=3.17$
\end{answer}
\end{exercise}

\begin{exercise}
Approximate a root of $f(x)=x^4+x^3-5x+1$ to two decimal places.
\begin{answer}
$x=0.20$, or $x=1.35$
\end{answer}
\end{exercise}
\end{exercises}
