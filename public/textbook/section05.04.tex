\section{Concavity and inflection points} {}{}
\nobreak
We know that the sign of the derivative tells us whether a function is
increasing or decreasing; for example, when $f'(x)>0$,
$f(x)$ is increasing. The sign of the second derivative
$f''(x)$ tells us whether $f'$ is increasing or decreasing; we have
seen that if $f'$ is zero and increasing at a point then there is a
local minimum at the point, and if $f'$ is zero and decreasing at a
point then there is a local maximum at the point. Thus, we extracted
information about $f$ from information about $f''$. 

We can get information from the sign of $f''$ even when $f'$ is not
zero. Suppose that $f''(a)>0$. This means that near $x=a$, $f'$ is
increasing. If $f'(a)>0$, this means that $f$ slopes up and is getting
steeper; if $f'(a)<0$, this means that $f$ slopes down and is getting
{\it less\/} steep. The two situations are shown in
figure~\xrefn{fig:concave up}. A curve that is shaped like this is
called {\dfont concave\index{concave up} up.}

% BADBAD
% \figure
% \vbox{\beginpicture
% \normalgraphs
% \ninepoint
% \setcoordinatesystem units <2truecm,2truecm>
% \setplotarea x from 0 to 1, y from 0 to 1
% \axis left shiftedto x=0 /
% \axis bottom shiftedto y=0 ticks withvalues {$a$} / at 0.5 / /
% \setquadratic
% \plot 0.1 0.2 0.5 0.4 0.9 0.9 /
% \setcoordinatesystem units <2truecm,2truecm> point at -2 0
% \setplotarea x from 0 to 1, y from 0 to 1
% \axis left shiftedto x=0 /
% \axis bottom shiftedto y=0 ticks withvalues {$a$} / at 0.5 / /
% \setquadratic
% \plot 0.1 0.9 0.5 0.3 0.9 0.1 /
% \endpicture}
% \figrdef{fig:concave up}
% \endfigure{$f''(a)>0$: $f'(a)$ positive and increasing, $f'(a)$ negative and
%   increasing.}

Now suppose that $f''(a)<0$. This means that near $x=a$, $f'$ is
decreasing. If $f'(a)>0$, this means that $f$ slopes up and is getting
less steep; if $f'(a)<0$, this means that $f$ slopes down and is getting
steeper. The two situations are shown in
figure~\xrefn{fig:concave down}. A curve that is shaped like this is
called {\dfont concave\index{concave down} down.}

% BADBAD
% \figure
% \vbox{\beginpicture
% \normalgraphs
% \ninepoint
% \setcoordinatesystem units <2truecm,2truecm>
% \setplotarea x from 0 to 1, y from 0 to 1
% \axis left shiftedto x=0 /
% \axis bottom shiftedto y=0 ticks withvalues {$a$} / at 0.5 / /
% \setquadratic
% \plot 0.1 0.2 0.5 0.7 0.9 0.9 /
% \setcoordinatesystem units <2truecm,2truecm> point at -2 0
% \setplotarea x from 0 to 1, y from 0 to 1
% \axis left shiftedto x=0 /
% \axis bottom shiftedto y=0 ticks withvalues {$a$} / at 0.5 / /
% \setquadratic
% \plot 0.1 0.9 0.5 0.6 0.9 0.1 /
% \endpicture}
% \figrdef{fig:concave down}
% \endfigure{$f''(a)<0$: $f'(a)$ positive and decreasing, $f'(a)$ negative and
%   decreasing.}

If we are trying to understand the shape of the graph of a function,
knowing where it is concave up and concave down helps us to get a more
accurate picture. Of particular interest are points at which the
concavity changes from up to down or down to up; such points are
called {\dfont inflection\index{inflection point} points.} If the
concavity changes from up to down at $x=a$, $f''$ changes from
positive to the left of $a$ to negative to the right of $a$, and
usually $f''(a)=0$. We can identify such points by first finding where
$f''(x)$ is zero and then checking to see whether $f''(x)$ does in
fact go from positive to negative or negative to positive at these
points. Note that it is possible that $f''(a)=0$ but the concavity is
the same on both sides; $\ds f(x)=x^4$ at $x=0$ is an example.

\begin{example}
Describe the concavity of $\ds f(x)=x^3-x$. $\ds f'(x)=3x^2-1$, $f''(x)=6x$.
Since $f''(0)=0$, there is potentially an inflection point at
zero. Since $f''(x)>0$ when $x>0$ and $f''(x)<0$ when $x<0$ the
concavity does change from down to up at zero, and the curve is
concave down for all $x<0$ and concave up for all $x>0$.
\end{example}

Note that we need to compute and analyze the second derivative to
understand concavity, so we may as well try to use the second
derivative test for maxima and minima. If for some reason this fails
we can then try one of the other tests.

\begin{exercises}
Describe the concavity of the functions in 1--18.

\twocol

\begin{exercise} $\ds y=x^2-x$ 
\begin{answer} concave up everywhere
\end{answer}\end{exercise}

\begin{exercise} $\ds y=2+3x-x^3$ 
\begin{answer} concave up when $x<0$, concave down when $x>0$
\end{answer}\end{exercise}

\begin{exercise} $\ds y=x^3-9x^2+24x$
\begin{answer} concave down when $x<3$, concave up when $x>3$
\end{answer}\end{exercise}

\begin{exercise} $\ds y=x^4-2x^2+3$ 
\begin{answer} concave up when $\ds x<-1/\sqrt3$ or $\ds x>1/\sqrt3$,
concave down when $\ds -1/\sqrt3<x<1/\sqrt3$
\end{answer}\end{exercise}

\begin{exercise} $\ds y=3x^4-4x^3$
\begin{answer} concave up when $x<0$ or $x>2/3$,
concave down when $0<x<2/3$
\end{answer}\end{exercise}

\begin{exercise} $\ds y=(x^2-1)/x$
\begin{answer} concave up when $x<0$, concave down when $x>0$
\end{answer}\end{exercise}

\begin{exercise} $\ds y=3x^2-(1/x^2)$ 
\begin{answer} concave up when $x<-1$ or $x>1$, concave down when
$-1<x<0$ or $0<x<1$
\end{answer}\end{exercise}

\begin{exercise} $y=\sin x + \cos x$ 
\begin{answer} concave down on $((8n-1)\pi/4,(8n+3)\pi/4)$,
concave up on $((8n+3)\pi/4,(8n+7)\pi/4)$, for integer $n$
\end{answer}\end{exercise}

\begin{exercise} $\ds y = 4x+\sqrt{1-x}$
\begin{answer} concave down everywhere
\end{answer}\end{exercise}

\begin{exercise} $\ds y = (x+1)/\sqrt{5x^2 + 35}$
\begin{answer} concave up on $\ds (-\infty,(21-\sqrt{497})/4)$ and 
$\ds (21+\sqrt{497})/4,\infty)$
\end{answer}\end{exercise}

\begin{exercise} $\ds y= x^5 - x$
\begin{answer} concave up on $(0,\infty)$
\end{answer}\end{exercise}

\begin{exercise} $\ds y = 6x + \sin 3x$
\begin{answer} concave down on $(2n\pi/3,(2n+1)\pi/3)$
\end{answer}\end{exercise}

\begin{exercise} $\ds y = x+ 1/x$
\begin{answer} concave up on $(0,\infty)$
\end{answer}\end{exercise}

\begin{exercise} $\ds y = x^2+ 1/x$
\begin{answer} concave up on $(-\infty,-1)$ and $(0,\infty)$
\end{answer}\end{exercise}

\begin{exercise} $\ds y = (x+5)^{1/4}$
\begin{answer} concave down everywhere
\end{answer}\end{exercise}

\begin{exercise} $\ds y = \tan^2 x$
\begin{answer} concave up everywhere
\end{answer}\end{exercise}

\begin{exercise} $\ds y =\cos^2 x - \sin^2 x$
\begin{answer} concave up on $(\pi/4+n\pi,3\pi/4+n\pi)$
\end{answer}\end{exercise}

\begin{exercise} $\ds y = \sin^3 x$
\begin{answer} inflection points at $n\pi$, $\ds \pm\arcsin(\sqrt{2/3})+n\pi$
\end{answer}\end{exercise}

\endtwocol

\msk \begin{exercise} Identify the intervals on which the graph of the function
$\ds f(x) = x^4-4x^3 +10$ is of one of these four
shapes: concave up and increasing; concave up and decreasing; concave
down and increasing; concave down and decreasing.
\begin{answer} up/incr: $(3,\infty)$, up/decr: $(-\infty,0)$, $(2,3)$,
down/decr: $(0,2)$
\end{answer}\end{exercise}

\begin{exercise} Describe the concavity of $\ds y =  x^3 + bx^2 + cx + d$.
You will need to consider different cases, depending on the values of
the coefficients.
\end{exercise}

\begin{exercise} Let $n$ be an integer greater than or equal to
two, and suppose $f$ is a polynomial of degree $n$. How many inflection points
can $f$ have?  Hint: Use the second derivative test and the
fundamental theorem of algebra.
\end{exercise}

% Mike Wills stuff
% \iflatetranscendentals
% 
% \begin{remark}{Definition} Let $f: I \to \R$ be a
% function. Let $a$ and $b$ be distinct points in $I$.  The
% {\dfont secant line\/} of $f$ from $a$ to $b$ is the (unique) line that
% passes through the points $(a,f(a))$ and $(b, f(b))$.
% 
% We now give a formal definition of concavity.
% 
% \end{remark}
% 
% \begin{remark}{Definition} 
% Let $I$ be an (open or closed) interval. Let $f:I \to \R$ be a given function.
% Suppose that for every $t$ in $[0,1]$ and every $a, b$ in $I$ the following inequality holds:
% $$ f(ta + (1-t)b ) \leq t f(a) + (1-t) f(b) .$$
% Then $f$ is said to  be {\dfont concave up\/} on $I$.
% If $g:I\to \R$ is a function such that $-g$ is concave up, then $g$
% is {\dfont concave down}.
% 
%  In more advanced texts concave up functions are called
%  {\dfont convex\/}, while concave down functions are called
%  {\dfont concave\/}. Additionally, several texts define concave
%  up/concave down in terms of tangent lines which presupposes that the
%  function is differentiable; taking that point of view, requiring a
%  function to be concave up is a stronger requirement than requiring
%  the function to be convex.
% 
% \end{remark}
% 
% \begin{exercise} Illustrate the definition of concave up with a picture.
% % (Hint: what goes the graph of
% %$h(t) = tf(a) + (1-t)f(b) $ look like?)
% Conclude that if a function is concave up on an interval then the
% graph of each secant line lies on or above the graph of the function.
% 
% \begin{exercise} It can be shown that if $f$ is concave up on an open
% interval then $f$ is continuous.  Give an example of a function that
% is concave up on the closed interval $[-1, 1]$ but fails to be
% continuous.  (Hint: note that the function must be continuous on $(-1,
% 1)$ so the function will be discontinuous only at one or both of the
% endpoints.
% 
% 
% \begin{exercise} Let $f(x) = mx+b$. Show that $f$ is concave up
% and concave down.
% \label{exer:linear concavity} 
% 
% \begin{exercise} Let $\ds f(x) = ax^2 + bx +c$ with $a \neq 0$. Use the second
% derivative test to show that $f$ is concave up on all of $\R$
% if $a>0$ and that $f$ is concave down on all of $\R$ if $a<0$.
% \label{exer:quadratic concavity}
% 
% \begin{exercise} Give an example of a function $f$ on the interval $(-1,1)$ which is
% concave up but is not differentiable at $0$.
% 
% \begin{exercise} Let $f,g :I \to \R$ be concave up functions. Let $c\geq
% 0$. Show that $f+g$ and $cf$ are both concave up.
%  
% 
%  \begin{exercise} Let $f,g :I \to \R$ be concave up
%  functions. Show by means of an example that $fg$ need not be concave
%  up. Hint: Use exercises \xrefn{exer:linear concavity} and 
% \xrefn{exer:quadratic concavity}.
% 
% \fi

\end{exercises}

