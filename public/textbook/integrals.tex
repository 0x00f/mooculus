\chapter{Integrals}

\section{Definite Integrals Compute Signed Area}

Definite integrals, often simply called integrals, compute signed area. 

\begin{definition}\index{integral}\index{definite integral}
The \textbf{definite integral}
\[
\int_a^b f(x) \d x
\]
computes the signed area in the region $[a,b]$ between $f(x)$ and the
$x$-axis. If the region is above the $x$-axis, then the area has
positive sign. If the region is below the $x$-axis, then the area has
negative sign.
\end{definition}

\begin{example}
Compute
\[
\int_0^3 x \d x.
\]
\end{example}
\begin{marginfigure}
\begin{tikzpicture}
  \begin{axis}[
      xmin=0, xmax=3,ymin=0,ymax=3,domain=0:3,
      axis lines =center, xlabel=$x$, ylabel=$y$,
      every axis y label/.style={at=(current axis.above origin),anchor=south},
      every axis x label/.style={at=(current axis.right of origin),anchor=west},
      axis on top,
    ] 
    \addplot [draw=none, fill=fillp] {x} \closedcycle;
    \addplot [penColor,very thick] {x};
  \end{axis}
\end{tikzpicture}
\caption{The integral $\int_0^3 x \d x$ measures the shaded area.}
\label{figure:intfirst}
\end{marginfigure}

\begin{solution}
The definite integral $\int_0^3 x \d x$ measures signed area of the
shaded region shown in figure~\ref{figure:intfirst}. Since this region
is a triangle, we can use the formula for the area of the triangle to
compute
\[
\int_0^2 x \d x = \frac{1}{2} 3\cdot 3 = 9/2. 
\]
\end{solution}

When working with signed area, positive and negative area cancel each
other out.

\begin{example} Compute
\[
\int_{-1}^3 \lfloor x \rfloor \d x.
\]
\end{example}

\begin{marginfigure}[0in]
\begin{tikzpicture}
  \begin{axis}[
      domain=-1:3,
      axis lines =middle, xlabel=$x$, ylabel=$y$,
      every axis y label/.style={at=(current axis.above origin),anchor=south},
      every axis x label/.style={at=(current axis.right of origin),anchor=west},
      clip=false,
      axis on top,
    ]
    \addplot [draw=none, fill=filln, domain=(-1:0)] {-1} \closedcycle;
    \addplot [draw=none, fill=fillp, domain=(0:1)] {0} \closedcycle;
    \addplot [draw=none, fill=fillp, domain=(1:2)] {1} \closedcycle;
    \addplot [draw=none, fill=fillp, domain=(2:3)] {2} \closedcycle;
    \addplot [very thick, penColor, domain=(-1:0)] {-1};
    \addplot [very thick, penColor, domain=(0:1)] {0};
    \addplot [very thick, penColor, domain=(1:2)] {1};
    \addplot [very thick, penColor, domain=(2:3)] {2};
    \addplot[color=penColor,fill=penColor,only marks,mark=*] coordinates{(-1,-1)};  %% closed hole          
    \addplot[color=penColor,fill=penColor,only marks,mark=*] coordinates{(0,0)};  %% closed hole          
    \addplot[color=penColor,fill=penColor,only marks,mark=*] coordinates{(1,1)};  %% closed hole          
    \addplot[color=penColor,fill=penColor,only marks,mark=*] coordinates{(2,2)};  %% closed hole  
    \addplot[color=penColor,fill=penColor,only marks,mark=*] coordinates{(3,3)};  %% closed hole                  
    \addplot[color=penColor,fill=background,only marks,mark=*] coordinates{(0,-1)};  %% open hole
    \addplot[color=penColor,fill=background,only marks,mark=*] coordinates{(1,0)};  %% open hole
    \addplot[color=penColor,fill=background,only marks,mark=*] coordinates{(2,1)};  %% open hole
    \addplot[color=penColor,fill=background,only marks,mark=*] coordinates{(3,2)};  %% open hole

    \node at (axis cs:-.5,-.5) [textColor] {\scalebox{2}{$\boldsymbol-$}};
    \node at (axis cs:2,.5) [textColor] {\scalebox{2}{$\boldsymbol+$}};
  \end{axis}
\end{tikzpicture}
\caption{The integral $\int_{-1}^3 \lfloor x\rfloor \d x$ measures the
  shaded area. Area above the $x$-axis has positive sign and the area below the $x$-axis has negative sign.}
\label{plot:int-greatist-integer}
\end{marginfigure}

\begin{solution}
The definite integral $\int_{-1}^3 \lfloor x\rfloor \d x$ measures
signed area of the shaded region shown in
figure~\ref{plot:int-greatist-integer}. We see that 
\[
\int_{-1}^3 \lfloor x\rfloor \d x = 
\int_{-1}^0 \lfloor x\rfloor \d x + \int_{0}^1 \lfloor x\rfloor \d x + \int_{1}^2 \lfloor x\rfloor \d x + \int_{2}^3 \lfloor x\rfloor \d x.
\]
So computing each of these areas separately
\begin{align*}
\int_{-1}^3 \lfloor x\rfloor \d x &= -1 + 0 + 1+2 \\
&= 2.
\end{align*}
\end{solution}

Our previous examples hopefully give us enough insight that this next
theorem is unsurprising.

\begin{mainTheorem}[Properties of Definite Integrals]
\begin{enumerate}
\item $\int_a^b k \d x= kb-ka$, where $k$ is a constant.
\item $\int_a^b \left( f(x) + g(x) \right) \d x = \int_a^b f(x) \d x + \int_a^b
  g(x) \d x$.
\item $\int_a^b k \cdot f(x) \d x = k \int_a^b f(x) \d x$.
\end{enumerate}
\end{mainTheorem}
Each of these properties follows from the notion that definite
integrals compute signed area.

\subsection*{Accumulation Functions}

While the definite integral computes a signed area, which is a fixed
number, there is a way to turn it into a function.
\begin{definition}
Given a function $f(x)$, an \textbf{accumulation function} for
$f(x)$ is given by
\[
F(x) = \int_a^x f(t) \d t.
\]
\end{definition}

One thing that you might note is that an accumulation function seems
to have two variables $x$ and $t$. Let's see if we can explain
this. Consider the following plot:

\begin{tikzpicture}
	\begin{axis}[
            domain=0:6, ymax=2.2,xmax=6,
            axis lines =left, xlabel=$t$, ylabel=$y$,
            every axis y label/.style={at=(current axis.above origin),anchor=south},
            every axis x label/.style={at=(current axis.right of origin),anchor=west},
            xtick={1,5}, ytick={.203,1.679},
            xticklabels={$a$,$x$}, yticklabels={$f(a)$,$f(x)$},
            axis on top,
          ]
          \addplot [draw=none,fill=fillp,domain=(1:5)] {sin(deg((x - 4)/2)) + 1.2} \closedcycle;
          \addplot [very thick,penColor, smooth,domain=(0:6)] {sin(deg((x - 4)/2)) + 1.2};

          \addplot [textColor,dashed] plot coordinates {(0,1.679) (5,1.679)};
          \addplot [textColor,dashed] plot coordinates {(0,.203) (1,.203)};
          %\addplot [textColor,dashed] plot coordinates {(5,0) (5,1.679)};
          \addplot [textColor] plot coordinates {(1,0) (1,.203)};

          \addplot [color=penColor,fill=penColor,only marks,mark=*] coordinates{(1,.203)};  %% closed hole         
          \addplot [color=penColor,fill=penColor,only marks,mark=*] coordinates{(5,1.679)};  %% closed hole       
          \node at (axis cs:3.4,.3) [textColor] {$F(x) = \int_a^x f(t) \d t$};
          \node at (axis cs:3.4,1.1) [penColor] {$f(t)$};
        \end{axis}
\end{tikzpicture}


An accumulation function $F(x)$ is measuring the signed area in the
region $[a,x]$ between $f(t)$ and the $t$-axis. Hence $t$ is playing
the role of a ``place-holder'' and represents numbers where we are
evaluating $f(t)$. On the other hand, $x$ is the specific number that
we are using to bound the region that will determine the area between
$f(t)$ and the $t$-axis.


\begin{example} 
Consider the following accumulation function for $f(x) = x^3$.
\[
F(x) = \int_{-1}^x t^3 \d t.
\]
Considering the interval $[-1,1]$, where is $F(x)$ increasing? Where
is $F(x)$ decreasing? When does $F(x)$ have a local extrema?
\end{example}

\begin{marginfigure}
\begin{tikzpicture}
  \begin{axis}[
      xmin=-1.2, xmax=1,ymin=-1,ymax=1,domain=-1:1,
      axis lines =center, xlabel=$t$, ylabel=$y$,
      every axis y label/.style={at=(current axis.above origin),anchor=south},
      every axis x label/.style={at=(current axis.right of origin),anchor=west},
      xtick={-1,.8}, 
      xticklabels={$-1$,$x$}, 
      axis on top,
    ] 
    \addplot [draw=none, fill=fillp,domain=0:.8] {x^3} \closedcycle;
    \addplot [draw=none, fill=filln,domain=-1:0] {x^3} \closedcycle;
    \addplot [penColor,very thick,domain=-1.2:1,] {x^3};
    
    \addplot [textColor] plot coordinates {(-1,0) (-1,-1)};

    \node at (axis cs:.67,.15) [textColor] {\scalebox{2}{$\boldsymbol+$}};
    \node at (axis cs:-.85,-.3) [textColor] {\scalebox{2}{$\boldsymbol-$}};
  \end{axis}
\end{tikzpicture}
\caption{The integral $\int_{-1}^x t^3 \d t$ measures the shaded area.}
\label{figure:accumulationeg}
\end{marginfigure}

\begin{solution}
We can see a plot of $f(t)$ along with the signed area measured by the
accumulation function in Figure~\ref{figure:accumulationeg}. The
accumulation function starts off at zero, and then is decreasing as it
accumulates negatively signed area. However when $x>0$, $F(x)$ starts
to accumulate positively signed area, and hence is increasing. Thus
$F(x)$ is increasing on $(0,1)$, decreasing on $(-1,0)$ and hence has
a local minimum at $(0,0)$.
\end{solution}

Working with the accumulation function leads us to a question, what is  
\[
\int_a^x f(x) \d x
\]
when $x< a$? The general convention is that 
\[
\int_a^b f(x) \d x = -\int_b^a f(x) \d x. 
\]
With this in mind, let's consider one more example.


\begin{example} 
Consider the following accumulation function for $f(x) = x^3$.
\[
F(x) = \int_{-1}^x t^3 \d t.
\]
Where is $F(x)$ increasing? Where is $F(x)$ decreasing? When does
$F(x)$ have a extrema?
\end{example}

\begin{marginfigure}
\begin{tikzpicture}
  \begin{axis}[
      xmin=-3, xmax=1,ymin=-10,ymax=1,domain=-3:1,
      axis lines =center, xlabel=$t$, ylabel=$y$,
      every axis y label/.style={at=(current axis.above origin),anchor=south},
      every axis x label/.style={at=(current axis.right of origin),anchor=west},
      xtick={-2,-1}, 
      xticklabels={$x$,$-1$}, 
      axis on top,
    ] 
    \addplot [draw=none, fill=fillp,domain=-2:-1] {x^3} \closedcycle;
    \addplot [penColor,very thick] {x^3};

    \addplot [textColor] plot coordinates {(-1,0) (-1,-1)};
    
    \node at (axis cs:-1.5,-1.5) [textColor] {\scalebox{2}{$\boldsymbol+$}};
  \end{axis}
\end{tikzpicture}
\caption{The integral $\int_{-1}^x t^3 \d t$ measures the shaded
  area. Note, since $x<-1$, the area has positive sign.}
\label{figure:accumulationegreal}
\end{marginfigure}

\begin{solution}
From our previous example, we know that $F(x)$ is increasing on
$(0,1)$. Since $f(t)$ continues to be positive at $t=1$ and beyond,
  $F(x)$ is increasing on $(0,\infty)$. On the other hand, we know
  from our previous example that $F(x)$ is decreasing on $(-1,0)$. For
  values to the left of $t=-1$, $F(x)$ is still decreasing, as less
  and less positively signed area is accumulated. Hence $F(x)$ is
  increasing on $(0,\infty)$, decreasing on $(-\infty,0)$ and hence
  has an absolute minimum at $(0,0)$.
\end{solution}

The key point to take from these examples is that an accumulation function
\[
\int_a^x f(t) \d x 
\]
is increasing precisely when $f(t)$ is positive and is decreasing
precisely when $f(t)$ is negative. In short, it seems that $f(x)$ is
behaving in a similar fashion to $F'(x)$.




\begin{exercises}

\noindent For the following exercises consider the plot of $f(x)$
shown in Figure~\ref{figure:intExerPlot1}.
\begin{marginfigure}
\begin{tikzpicture}
  \begin{axis}[
      xmin=-2.5, xmax=2.5,ymin=-1,ymax=1,domain=-2.2:2.2,
      axis lines =center, xlabel=$x$, ylabel=$y$,
      every axis y label/.style={at=(current axis.above origin),anchor=south},
      every axis x label/.style={at=(current axis.right of origin),anchor=west},
      axis on top,
    ] 
    \addplot [penColor,very thick,smooth] {sin(deg(x))*sin(deg(x^2/1.3))};
  \end{axis}
\end{tikzpicture}
\caption{A plot of $f(x)$.}
\label{figure:intExerPlot1}
\end{marginfigure}
\twocol
\begin{exercise}
Is $\int_1^2 f(x) \d x$ \\
positive, negative, or zero?
\begin{answer}
positive
\end{answer}
\end{exercise}

\begin{exercise}
Is $\int_{-1}^0 f(x) \d x$ \\
positive, negative, or zero?
\begin{answer}
negative
\end{answer}
\end{exercise}

\begin{exercise}
Is $\int_{-1}^1 f(x) \d x$ \\
positive, negative, or zero?
\begin{answer}
zero
\end{answer}
\end{exercise}

\begin{exercise}
Is $\int_{-1}^2 f(x) \d x$ \\
positive, negative, or zero?
\begin{answer}
positive
\end{answer}
\end{exercise}

\endtwocol

\noindent For the following exercises use the plot of $g(x)$ shown in
Figure~\ref{plot:intExerPlot2} to compute the integrals below.
\begin{marginfigure}[0in]
\begin{tikzpicture}
  \begin{axis}[
      domain=-2:3,
      axis lines =middle, xlabel=$x$, ylabel=$y$,
      every axis y label/.style={at=(current axis.above origin),anchor=south},
      every axis x label/.style={at=(current axis.right of origin),anchor=west},
      clip=false,
      grid=both,
      grid style={dashed, gridColor},
    ]
    \addplot [penColor, very thick] plot coordinates {
      (-2,1) (-1,1) (0,-2) (2,2) (3,0)};
  \end{axis}
\end{tikzpicture}
\caption{A plot of $g(x)$.}
\label{plot:intExerPlot2}
\end{marginfigure}

\twocol

\begin{exercise}
$\int_{-2}^{-1} g(x) \d x$
\begin{answer}
$1$
\end{answer}
\end{exercise}

\begin{exercise}
$\int_1^3 g(x) \d x$
\begin{answer}
$2$
\end{answer}
\end{exercise}

\begin{exercise}
$\int_0^3 g(x) \d x$
\begin{answer}
$1$
\end{answer}
\end{exercise}

\begin{exercise}
$\int_{-1}^3 g(x) \d x$
\begin{answer}
$1/2$
\end{answer}
\end{exercise}

\endtwocol

\begin{exercise}
Suppose you know that $\int_{-1}^1 x^2 \d x = \frac{2}{3}$ and that
$\int_{-1}^1 e^x \d x= e-\frac{1}{e}$. Use properties of definite
integrals to compute
\[
\int_{-1}^1 \left(4 e^x - 3 x^2\right) \d x.
\]
\begin{answer}
$4e - \frac{4}{e} - 2$
\end{answer}
\end{exercise}

\begin{exercise}
Suppose you know that $\int_{1}^2 x^2 \d x = \frac{7}{3}$, $\int_1^2
\ln(x)\d x = \ln(4) - 1$, and that $\int_1^2 \sin(\pi x) \d x =
\frac{-2}{\pi}$. Use properties of definite integrals to compute
\[
\int_1^2 \left(6 x^2  - 2 \ln(x) +\pi \sin(\pi x) \right) \d x.
\]
\begin{answer}
$14 -2 \ln(4)$
\end{answer}
\end{exercise}


\noindent For the following exercises consider the accumulation
function $F(x) = \int_{-2}^x \frac{\sin(t)}{t} \d t$ on the interval
$[-2\pi,2\pi]$.

\begin{exercise} On what subinterval(s) is $F(x)$ increasing?
\begin{answer}
$(-\pi,\pi)$
\end{answer}
\end{exercise}

\begin{exercise} On what subinterval(s) is $F(x)$ decreasing?
\begin{answer}
$(-2\pi,-\pi)\cup(\pi,2\pi)$
\end{answer}
\end{exercise}

\end{exercises}




\section{Riemann Sums}

In the first section we learned that integrals compute signed
area. However, we gave no indication as to how this area is
computed. Suppose you want to integrate $f(x)$ from $a$ to $b$, see
Figure~\ref{figure:areacompute1}. Start by partitioning the interval
$[a,b]$ by making a list

\begin{marginfigure}
\begin{tikzpicture}
	\begin{axis}[
            domain=.5:5.5, ymax=2.2,xmax=5.5,xmin=.5,ymin=0,
            axis lines =left, xlabel=$x$, ylabel=$y$,
            every axis y label/.style={at=(current axis.above origin),anchor=south},
            every axis x label/.style={at=(current axis.right of origin),anchor=west},
            xtick={1,5}, ytickmin=4, ytickmax=1,
            xticklabels={$a$,$b$}, 
            axis on top,
          ]
          \addplot [draw=none,fill=fillp,domain=(1:5)] {sin(deg((x - 4)/2)) + 1.2} \closedcycle;
          \addplot [very thick,penColor, smooth,domain=(0:6)] {sin(deg((x - 4)/2)) + 1.2};
          %\addplot [color=penColor,fill=penColor,only marks,mark=*] coordinates{(1,.203)};  %% closed hole         
          %\addplot [color=penColor,fill=penColor,only marks,mark=*] coordinates{(5,1.679)};  %% closed hole       
        \end{axis}
\end{tikzpicture}
\caption{A plot of $f(x)$ along with the area computed by a definite
  integral.}
\label{figure:areacompute1}
\end{marginfigure}

\[
a = x_0 < x_1 <x_2 < \cdots x_{n-1}< x_n = b
\]
and considering the subintervals where
\[
[x_0,x_1]\cup [x_1, x_2]\cup \dots \cup [x_{n-1},x_n] = [a,b].
\]

\begin{marginfigure}
\begin{tikzpicture}
	\begin{axis}[
            domain=.5:5.5, ymax=2.2,xmax=5.5, xmin=.5,ymin=0,
            axis lines =left, xlabel=$x$, ylabel=$y$,
            every axis y label/.style={at=(current axis.above origin),anchor=south},
            every axis x label/.style={at=(current axis.right of origin),anchor=west},
            xtick={1,2.1,2.7,4.2,5}, ytickmin=4, ytickmax=1,
            xticklabels={$a=x_0$, $x_1$, $x_2$, $x_3$, $x_4=b$}, 
            axis on top,
          ]

%          \addplot [draw=penColor,fill=fillp] plot coordinates {(1,.24) (2.1,.24)} \closedcycle;
%          \addplot [draw=penColor,fill=fillp] plot coordinates {(2.1,.45) (2.7,.45)} \closedcycle;
%          \addplot [draw=penColor,fill=fillp] plot coordinates {(2.7,.72) (4.2,.72)} \closedcycle;
%          \addplot [draw=penColor,fill=fillp] plot coordinates {(4.2,1.54) (5,1.54)} \closedcycle;

          \addplot [very thick,penColor, smooth,domain=(0:6)] {sin(deg((x - 4)/2)) + 1.2};
          
          \addplot [color=penColor,fill=penColor,only marks,mark=*] coordinates{(1.4,.24)};  %% closed hole         
          \addplot [color=penColor,fill=penColor,only marks,mark=*] coordinates{(2.3,.45)};  %% closed hole       
          \addplot [color=penColor,fill=penColor,only marks,mark=*] coordinates{(3,.72)};  %% closed hole  
          \addplot [color=penColor,fill=penColor,only marks,mark=*] coordinates{(4.7,1.54)};  %% closed hole       
          
          \addplot [dashed,textColor] plot coordinates {(1.4,0) (1.4,.24)};
          \addplot [dashed,textColor] plot coordinates {(2.3,0) (2.3,.45)};
          \addplot [dashed,textColor] plot coordinates {(3,0) (3,.72)};
          \addplot [dashed,textColor] plot coordinates {(4.7,0) (4.7,1.54)};

          \node at (axis cs:1.4,.24) [textColor,above] {$f(x_0^*)$};
          \node at (axis cs:2.3,.5) [textColor,above] {$f(x_1^*)$};
          \node at (axis cs:3,.8) [textColor,above] {$f(x_2^*)$};
          \node at (axis cs:4.7,1.6) [textColor,above] {$f(x_3^*)$};
        \end{axis}
\end{tikzpicture}
\caption{A plot of $f(x)$ along with the partition 
\[
[x_0,x_1]\cup [x_1, x_2]\cup [x_2,x_3]\cup [x_3,x_4]\cup[x_4,x_5] = [a,b]
\]
and the $y$-values $f(x_0^*)$, $f(x_1^*)$, $f(x_2^*)$, $f(x_3^*)$, $f(x_4^*)$.}
\label{figure:areacompute2}
\end{marginfigure}
For each subinterval pick a point $x_i^*\in [x_i,x_{i+1}]$ and
evaluate your function $f(x)$ at each of these points, see
Figure~\ref{figure:areacompute2}. We can now compute the area of the
rectangles defined by the width of the subinterval $[x_i,x_{i+1}]$ and
the height $f(x_i^*)$. Adding the areas of these rectangles together
we find
\[
\sum_{i=0}^{n-1} f(x_i^*) \cdot (x_{i+1}-x_i) \approx \int_a^b f(x) \d x.
\]

\begin{tikzpicture}
	\begin{axis}[
            domain=.5:5.5, ymax=2.2,xmax=5.5, xmin=.5,ymin=0,
            axis lines =left, xlabel=$x$, ylabel=$y$,
            every axis y label/.style={at=(current axis.above origin),anchor=south},
            every axis x label/.style={at=(current axis.right of origin),anchor=west},
            xtick={1,2.1,2.7,4.2,5}, ytickmin=4, ytickmax=1,
            xticklabels={$a=x_0$, $x_1$, $x_2$, $x_3$, $x_4=b$}, 
            axis on top,
          ]

          \addplot [draw=penColor,fill=fillp] plot coordinates {(1,.24) (2.1,.24)} \closedcycle;
          \addplot [draw=penColor,fill=fillp] plot coordinates {(2.1,.45) (2.7,.45)} \closedcycle;
          \addplot [draw=penColor,fill=fillp] plot coordinates {(2.7,.72) (4.2,.72)} \closedcycle;
          \addplot [draw=penColor,fill=fillp] plot coordinates {(4.2,1.54) (5,1.54)} \closedcycle;

          \addplot [very thick,penColor, smooth,domain=(0:6)] {sin(deg((x - 4)/2)) + 1.2};
          
          \addplot [color=penColor,fill=penColor,only marks,mark=*] coordinates{(1.4,.24)};  %% closed hole         
          \addplot [color=penColor,fill=penColor,only marks,mark=*] coordinates{(2.3,.45)};  %% closed hole       
          \addplot [color=penColor,fill=penColor,only marks,mark=*] coordinates{(3,.72)};  %% closed hole  
          \addplot [color=penColor,fill=penColor,only marks,mark=*] coordinates{(4.7,1.54)};  %% closed hole       
          
          \addplot [dashed,textColor] plot coordinates {(1.4,0) (1.4,.24)};
          \addplot [dashed,textColor] plot coordinates {(2.3,0) (2.3,.45)};
          \addplot [dashed,textColor] plot coordinates {(3,0) (3,.72)};
          \addplot [dashed,textColor] plot coordinates {(4.7,0) (4.7,1.54)};

          \node at (axis cs:1.4,.24) [textColor,above] {$f(x_0^*)$};
          \node at (axis cs:2.3,.5) [textColor,above] {$f(x_1^*)$};
          \node at (axis cs:3,.8) [textColor,above] {$f(x_2^*)$};
          \node at (axis cs:4.7,1.6) [textColor,above] {$f(x_3^*)$};
        \end{axis}
\end{tikzpicture}

If we take the limit of all such sums as the partitions get finer and
finer, we obtain closer and closer approximations, see
Figure~\ref{figure:partitionsfiner}. Sums of the form we are
describing are called \textit{Riemann sums}.

\break

\begin{marginfigure}
\begin{tikzpicture}
	\begin{axis}[
            domain=.5:5.5, ymax=2.2,xmax=5.5, xmin=.5,ymin=0,
            axis lines =left, xlabel=$x$, ylabel=$y$,
            every axis y label/.style={at=(current axis.above origin),anchor=south},
            every axis x label/.style={at=(current axis.right of origin),anchor=west},
            xtick={1,5}, ytickmin=4, ytickmax=1,
            xticklabels={$a$, $b$}, 
            axis on top,
          ]

          \addplot [draw=penColor,fill=fillp] plot coordinates {(1,.21) (1.13,.21)} \closedcycle;
          \addplot [draw=penColor,fill=fillp] plot coordinates {(1.13,.21) (1.19,.21)} \closedcycle;
          \addplot [draw=penColor,fill=fillp] plot coordinates {(1.19,.225) (1.5,.225)} \closedcycle;
          \addplot [draw=penColor,fill=fillp] plot coordinates {(1.5,.29) (1.74,.29)} \closedcycle;
          \addplot [draw=penColor,fill=fillp] plot coordinates {(1.74,.33) (1.93,.33)} \closedcycle;
          \addplot [draw=penColor,fill=fillp] plot coordinates {(1.93,.36) (2.16,.36)} \closedcycle;
          \addplot [draw=penColor,fill=fillp] plot coordinates {(2.16,.42) (2.24,.42)} \closedcycle;
          \addplot [draw=penColor,fill=fillp] plot coordinates {(2.24,.45) (2.35,.45)} \closedcycle;
          \addplot [draw=penColor,fill=fillp] plot coordinates {(2.35,.56) (2.75,.56)} \closedcycle;
          \addplot [draw=penColor,fill=fillp] plot coordinates {(2.75,.62) (2.8,.62)} \closedcycle;
          \addplot [draw=penColor,fill=fillp] plot coordinates {(2.8,.65) (2.85,.65)} \closedcycle;
          \addplot [draw=penColor,fill=fillp] plot coordinates {(2.85,.77) (3.2,.77)} \closedcycle;
          \addplot [draw=penColor,fill=fillp] plot coordinates {(3.2,.86) (3.47,.86)} \closedcycle;
          \addplot [draw=penColor,fill=fillp] plot coordinates {(3.47,.95) (3.65,.95)} \closedcycle;
          \addplot [draw=penColor,fill=fillp] plot coordinates {(3.65,1.05) (3.72,1.05)} \closedcycle;
          \addplot [draw=penColor,fill=fillp] plot coordinates {(3.72,1.1) (4.04,1.1)} \closedcycle;
          \addplot [draw=penColor,fill=fillp] plot coordinates {(4.04,1.24) (4.15,1.24)} \closedcycle;
          \addplot [draw=penColor,fill=fillp] plot coordinates {(4.15,1.3) (4.23,1.3)} \closedcycle;
          \addplot [draw=penColor,fill=fillp] plot coordinates {(4.23,1.35) (4.58,1.35)} \closedcycle;
          \addplot [draw=penColor,fill=fillp] plot coordinates {(4.58,1.5) (4.74,1.5)} \closedcycle;
          \addplot [draw=penColor,fill=fillp] plot coordinates {(4.74,1.57) (4.8,1.57)} \closedcycle;
          \addplot [draw=penColor,fill=fillp] plot coordinates {(4.8,1.63) (5,1.63)} \closedcycle;
          
          \addplot [very thick,penColor, smooth,domain=(0:6)] {sin(deg((x - 4)/2)) + 1.2};
        \end{axis}
\end{tikzpicture}
\caption{Using finer and finer partitions, the closer the approximation
\[
\sum_{i=0}^{n-1} f(x_i^*) \cdot (x_{i+1}-x_i) \approx \int_a^b f(x) \d x.
\]}
\label{figure:partitionsfiner}
\end{marginfigure}


\begin{definition}\index{Riemann sum}
Given an interval $[a,b]$ and a partition defined by
\[
a = x_0 < x_1 <x_2 < \cdots x_{n-1}< x_n = b,
\]
a \textbf{Riemann sum} for $f(x)$ is a sum of the form
\[
\sum_{i=0}^{n-1} f(x_i^*) \cdot (x_{i+1}-x_i)
\]
where $x_i^*\in [x_i,x_{i+1}]$.
\end{definition}

There are actually at least five special Riemann sums, \textit{left},
\textit{right}, \textit{midpoint}, \textit{upper}, and \textit{lower}.

\begin{definition}
Consider the following Riemann sum:
\[
\sum_{i=0}^{n-1} f(x_i^*) \cdot (x_{i+1}-x_i)
\]
\begin{itemize}
\item This is called a \textbf{left} Riemann sum if each $x_i^* =
  x_i$.
\item This is called a \textbf{right} Riemann sum if each $x_i^* =
  x_{i+1}$.
\item This is called a \textbf{midpoint} Riemann sum if each $x_i^*
  = \frac{x_i+x_{i+1}}{2}$.
\item This is called a \textbf{upper} Riemann sum if each $x_i^*$ is
  a point that gives a maximum value $f(x)$ on the interval
  $[x_i,x_{i+1}]$.
\item This is called a \textbf{lower} Riemann sum if each $x_i^*$ is a
  point that gives a minimum value $f(x)$ on the interval
  $[x_i,x_{i+1}]$.
\end{itemize}
\end{definition}
Riemann sums give a mechanism though which integrals could be
computed. Let's give it a try.

\begin{example}
Compute the left Riemann sum that approximates
\[
\int_1^2 \left(x^2-2x+2\right)\d x
\]
using four equally spaced partitions of the interval $[1,2]$.
\end{example}

\begin{marginfigure}

\begin{tikzpicture}
	\begin{axis}[
            domain=.75:2.25, ymax=3,xmax=2.25, xmin=.75,ymin=0,
            axis lines =left, xlabel=$x$, ylabel=$y$,
            every axis y label/.style={at=(current axis.above origin),anchor=south},
            every axis x label/.style={at=(current axis.right of origin),anchor=west},
            xtick={1,1.25,1.5,1.75,2}, ytickmin=4, ytickmax=1,
            axis on top,
          ]

          \addplot [draw=penColor,fill=fillp] plot coordinates {(1,1) (1.25,1)} \closedcycle;
          \addplot [draw=penColor,fill=fillp] plot coordinates {(1.25,1.06) (1.5,1.06)} \closedcycle;
          \addplot [draw=penColor,fill=fillp] plot coordinates {(1.5,1.25) (1.75,1.25)} \closedcycle;
          \addplot [draw=penColor,fill=fillp] plot coordinates {(1.75,1.56) (2,1.56)} \closedcycle;

          \addplot [very thick,penColor, smooth,domain=(0:6)] {x^2-2*x+2};
          
          \addplot [color=penColor,fill=penColor,only marks,mark=*] coordinates{(1,1)};  %% closed hole         
          \addplot [color=penColor,fill=penColor,only marks,mark=*] coordinates{(1.25,1.06)};  %% closed hole       
          \addplot [color=penColor,fill=penColor,only marks,mark=*] coordinates{(1.5,1.25)};  %% closed hole  
          \addplot [color=penColor,fill=penColor,only marks,mark=*] coordinates{(1.75,1.56)};  %% closed hole       
          
          \node at (axis cs:1,1) [textColor,above] {$f(x_0^*)$};
          \node at (axis cs:1.25,1.06) [textColor,above] {$f(x_1^*)$};
          \node at (axis cs:1.5,1.3) [textColor,above] {$f(x_2^*)$};
          \node at (axis cs:1.75,1.6) [textColor,above] {$f(x_3^*)$};
        \end{axis}
\end{tikzpicture}
\caption{Here we see the interval $[1,2]$ partitioned into four
  subintervals.}
\label{figure:drawRiemann1}
\end{marginfigure}

\begin{solution}
Start by setting $f(x) = x^2-2x+2$ and examining
Figure~\ref{figure:drawRiemann1}.  Our partition of $[1,2]$ is
\[
[1,1.25]\cup [1.25,1.5]\cup [1.5,1.75]\cup [1.75,2]. 
\]
Hence our left Riemann sum is given by 
\[
f(1)(1.25-1) + f(1.25)(1.5-1.25) + f(1.5)(1.75-1.5) + f(1.75)(2-1.75).
\]
This is equal to 
\[
\frac{1}{4} + \frac{17}{64} + \frac{5}{16} + \frac{25}{64} = \frac{39}{32} \approx 1.22.
\]
\end{solution}

To guarantee that a Riemann sum is to equal the value of the related
integral, we need the number of subintervals to go to infinity as the
width of our partitions goes to zero. We'll work through an example of
this.

\begin{example}
Compute
\[
\int_3^7 \left(2x-1\right)\d x
\]
via a left Riemann sum.
\end{example}

\begin{solution}
Start by setting $f(x) = 2x-1$ and examining
Figure~\ref{figure:drawRiemann2}.  The interval $[3,7]$ is divided
into $n$ subintervals each of width $(7-3)/n$. Our left Riemann sum is
now
\[
\sum_{i=0}^{n-1} f(3+(7-3)i/n) \left(\frac{7-3}{n}\right).
\]
Simplifying a bit we find
\begin{align*}
\sum_{i=0}^{n-1} f(3+4i/n) \frac{4}{n} &= \sum_{i=0}^{n-1} \left((2(3+4i/n) -1 )\frac{4}{n}\right)\\
&= \sum_{i=0}^{n-1} \left((5+8i/n )\frac{4}{n}\right)\\
&= \sum_{i=0}^{n-1} \left(\frac{20}{n} + \frac{32i}{n^2}\right)\\
&= \sum_{i=0}^{n-1} \frac{20}{n} + \sum_{i=0}^{n-1}\frac{32i}{n^2}\\
&= \frac{20}{n}\sum_{i=0}^{n-1} 1 +\frac{32}{n^2} \sum_{i=0}^{n-1} i
\end{align*}
At this point we need two formulas
\[
\sum_{i=0}^{n-1}1 = n-1 \qquad\text{and}\qquad \sum_{i=0}^{n-1} i = \frac{n^2-n}{2}.
\]
Substituting these formulas for the sums above, we find 
\begin{align*}
\frac{20}{n}\sum_{i=0}^{n-1} 1 +\frac{32}{n^2} \sum_{i=0}^{n-1} i &=\frac{20}{n}(n-1) +\frac{32}{n^2}\frac{n^2-n}{2}\\
&=20 -\frac{20}{n}+16 -\frac{16}{n}\\
&=36 - \frac{36}{n}.
\end{align*}
By construction
\[
\int_3^7 \left(2x-1\right) \d x = \lim_{n\to \infty}\sum_{i=0}^{n-1} f(3+(7-3)i/n) \left(\frac{7-3}{n}\right)
\]
hence
\[
\int_3^7 \left(2x-1 \right)\d x = \lim_{n\to \infty} \left(36 - \frac{36}{n}\right) = 36.
\]
\end{solution}

\begin{marginfigure}[-8in]
\begin{tikzpicture}
	\begin{axis}[
            domain=2:8, ymax=14,xmax=8,ymin=0,
            axis lines =left, xlabel=$x$, ylabel=$y$,
            every axis y label/.style={at=(current axis.above origin),anchor=south},
            every axis x label/.style={at=(current axis.right of origin),anchor=west},
            xtick={3,7}, ytickmin=2,ytickmax=1,
            axis on top,
          ]
          \addplot [draw=penColor,fill=fillp] plot coordinates {(3,5) (3.3,5)} \closedcycle;
          \addplot [draw=penColor,fill=fillp] plot coordinates {(3.3,5.6) (3.6,5.6)} \closedcycle;
          \addplot [draw=penColor,fill=fillp] plot coordinates {(3.6,6.2) (3.9,6.2)} \closedcycle;
          %% It might be cool to have rectangles fade out... but I'll put dots in for now.
          \addplot [draw=penColor,fill=fillp] plot coordinates {(6.7,12.4) (7,12.4)} \closedcycle;
          \node at (axis cs:5.4,2) [penColor] {\scalebox{3}{$\cdots$}};
          \addplot [very thick,penColor, smooth,domain=(2:8)] {2*x-1};
        \end{axis}
\end{tikzpicture}
\caption{We'll use a sum to compute 
\[
\int_3^7 2x -1 \d x.
\]
 Note if there are $n$ rectangles, then each rectangle is of width
 $4/n$.}
\label{figure:drawRiemann2}
\end{marginfigure}


Computing Riemann sums can be difficult. In particular, simply
integrating polynomials with Riemann sums requires one to evaluate
sums of the form
\[
\sum_{i=0}^{n-1} i^a
\]
for whole number values of $a$. Is there an easier way to compute
integrals? Read on to find out.


\begin{exercises}

\begin{margintable}[1in]
\[
\begin{tchart}{ll}
x & f(x) \\
1.0 & 2.3 \\
1.2 & 3.9 \\ 
1.4 & 7.0 \\
1.6 & 12.9 \\
1.8 & 24.9 \\ 
2 & 49.6 
\end{tchart}
\]
\caption{Values for $f(x)$.}
\label{table:intEx1}
\end{margintable}

\begin{exercise}
Use the Table~\ref{table:intEx1} to compute a left Riemann sum
estimating $\int_1^2 f(x) \d x$.
\begin{answer}
$10.2$
\end{answer}
\end{exercise}

\begin{exercise}
Use the Table~\ref{table:intEx1} to compute a right Riemann sum
estimating $\int_1^2 f(x) \d x$.
\begin{answer}
$19.66$
\end{answer}
\end{exercise}

\begin{margintable}[1in]
\[
\begin{tchart}{ll}
x & g(x) \\
-1.0 & 0.8\\
-0.8 & 0.5 \\ 
-0.6 & 0.1 \\
-0.4 & -0.1 \\
-0.2 & -0.1 \\ 
0.0 &  0.0 
\end{tchart}
\]
\caption{Values for $g(x)$.}
\label{table:intEx2}
\end{margintable}


\begin{exercise}
Use the Table~\ref{table:intEx2} to compute a left Riemann sum
estimating $\int_{-1}^0 g(x) \d x$.
\begin{answer}
$0.24$
\end{answer}
\end{exercise}

\begin{exercise}
Use the Table~\ref{table:intEx2} to compute a right Riemann sum
estimating $\int_{-1}^0 g(x) \d x$.
\begin{answer}
$0.08$
\end{answer}
\end{exercise}

\begin{exercise}
Write an expression in summation notation for the left Riemann sum
with $n$ equally spaced partitions that approximates $\int_1^3 \left(4-x^2\right) \d x$.
\begin{answer}
$\sum_{i=0}^{n-1} \left( 4 - (1+2i/n)^2\right)\cdot \frac{2}{n}$
\end{answer}
\end{exercise}

\begin{exercise}
Write an expression in summation notation for the right Riemann sum
with $n$ equally spaced partitions that approximates $\int_{-\pi}^\pi
\frac{\sin(x)}{x} \d x$.
\begin{answer}
$\sum_{i=1}^{n} \left(\frac{\sin\left(-\pi + \frac{2\pi i}{n}\right)}{-\pi + \frac{2\pi i}{n}}\right)\cdot \frac{2\pi}{n}$
\end{answer}
\end{exercise}

\begin{exercise}
Write an expression in summation notation for the midpoint Riemann sum
with $n$ equally spaced partitions that approximates $\int_{0}^1 e^{(x^2)} \d x$.
\begin{answer}
$\sum_{i=0}^{n-1} e^{((1+2i)/2n)^2} \cdot \frac{1}{n}$
\end{answer}
\end{exercise}

\begin{exercise} 
Use a Riemann sum to compute $\int_1^2 x \d x$.
\begin{answer}
$3/2$
\end{answer}
\end{exercise}

\begin{exercise}
Use a Riemann sum to compute $\int_{-1}^3 \left(4-x\right) \d x$.
\begin{answer}
$12$
\end{answer}
\end{exercise}


\begin{exercise}
Use a Riemann sum to compute $\int_{2}^4 3x^2 \d x$. Hint,
$\sum_{i=0}^{n-1} i^2 = \frac{(n-1)n(2n-1)}{6}$.
\begin{answer}
$56$
\end{answer}
\end{exercise}



\end{exercises}

