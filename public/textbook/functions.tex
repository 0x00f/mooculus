\chapter{Functions}

\section{For Each Input, Exactly One Output}

Life is complex. Part of this complexity stems from the fact that
there are many relations between seemingly independent events. Armed
with mathematics we seek to understand the world, and hence we need
tools for talking about these relations.

\marginnote{Something as simple as a dictionary could be thought of as
  a relation, as it connects \textit{words} to
  \textit{definitions}. However, a dictionary is not a function, as
  there are words with multiple definitions. On the other hand, if
  each word only had a single definition, then it would be a
  function.}

A \textit{function} is a relation between sets of objects that can be
thought of as a ``mathematical machine.'' This means for each input,
there is exactly one output. Moreover, whenever we talk about
functions, we should try to explicitly state what type of things the
inputs are and what type of things the outputs are.

In calculus, functions often define a relation from (a subset of) the
real numbers to (a subset of) the real numbers.


\marginnote[.5in]{While the name of the function is technically ``$f$,'' we
  will abuse notation and call the function ``$f(x)$'' to remind the
  reader that it is a function.}
\begin{example}
Consider the function $f$ that maps from the real numbers to the real
numbers by taking a number and mapping it to its cube:
\begin{align*}
1 &\mapsto 1\\
-2 &\mapsto -8\\
1.5 &\mapsto 3.375
\end{align*}
and so on. This function can be described by the formula $f(x)=x^3$ or
by the plot shown in Figure~\ref{plot:fxn x^3}.
\end{example}

\begin{warning}
A function is a mapping (such that for each input, there is exactly one
output) between sets and should not be confused with either its
formula or its plot.
\begin{itemize}
\item A formula merely describes the mapping using algebra.
\item A plot merely describes the mapping using pictures. 
\end{itemize}
\end{warning}


\begin{marginfigure}[0in]
\begin{tikzpicture}
	\begin{axis}[
            domain=-2:2,
            axis lines =middle, xlabel=$x$, ylabel=$y$,
            every axis y label/.style={at=(current axis.above origin),anchor=south},
            every axis x label/.style={at=(current axis.right of origin),anchor=west},
          ]
	  \addplot [very thick, penColor, smooth] {x^3};
        \end{axis}
\end{tikzpicture}
\caption{A plot of $f(x)=x^3$. Here we can see that for each input---a
  value on the $x$-axis, there is exactly one output---a value on the
  $y$-axis.}
\label{plot:fxn x^3}
\end{marginfigure}



\begin{example}
Consider the \textit{greatest integer function}, denoted by
\[
f(x) = \lfloor x \rfloor.
\]
This is the function that maps any real number $x$ to the the greatest
integer less than or equal to $x$. See Figure~\ref{plot:greatest-integer fxn} for a plot of
this function. Some might be confused because here we have multiple
inputs that give the same output. However, this is not a problem. To
be a function, we merely need to check that for each input, there is exactly
one output, and this is satisfied.
\end{example}
\begin{marginfigure}[0in]
\begin{tikzpicture}
	\begin{axis}[
            domain=-2:4,
            axis lines =middle, xlabel=$x$, ylabel=$y$,
            every axis y label/.style={at=(current axis.above origin),anchor=south},
            every axis x label/.style={at=(current axis.right of origin),anchor=west},
            clip=false,
            %axis on top,
          ]
          \addplot [textColor, very thin, domain=(0:2.3)] {0}; % puts the axis back, axis on top clobbers our open holes
          \addplot [textColor, very thin] plot coordinates {(0,0) (0,2)}; % puts the axis back, axis on top clobbers our open holes
	  \addplot [very thick, penColor, domain=(-2:-1)] {-2};
          \addplot [very thick, penColor, domain=(-1:0)] {-1};
          \addplot [very thick, penColor, domain=(0:1)] {0};
          \addplot [very thick, penColor, domain=(1:2)] {1};
          \addplot [very thick, penColor, domain=(2:3)] {2};
          \addplot [very thick, penColor, domain=(3:4)] {3};
          \addplot[color=penColor,fill=penColor,only marks,mark=*] coordinates{(-2,-2)};  %% closed hole          
          \addplot[color=penColor,fill=penColor,only marks,mark=*] coordinates{(-1,-1)};  %% closed hole          
          \addplot[color=penColor,fill=penColor,only marks,mark=*] coordinates{(0,0)};  %% closed hole          
          \addplot[color=penColor,fill=penColor,only marks,mark=*] coordinates{(1,1)};  %% closed hole          
          \addplot[color=penColor,fill=penColor,only marks,mark=*] coordinates{(2,2)};  %% closed hole  
          \addplot[color=penColor,fill=penColor,only marks,mark=*] coordinates{(3,3)};  %% closed hole                  
          \addplot[color=penColor,fill=background,only marks,mark=*] coordinates{(-1,-2)};  %% open hole
          \addplot[color=penColor,fill=background,only marks,mark=*] coordinates{(0,-1)};  %% open hole
          \addplot[color=penColor,fill=background,only marks,mark=*] coordinates{(1,0)};  %% open hole
          \addplot[color=penColor,fill=background,only marks,mark=*] coordinates{(2,1)};  %% open hole
          \addplot[color=penColor,fill=background,only marks,mark=*] coordinates{(3,2)};  %% open hole
          \addplot[color=penColor,fill=background,only marks,mark=*] coordinates{(4,3)};  %% open hole
        \end{axis}
\end{tikzpicture}
\caption{A plot of $f(x)=\lfloor x\rfloor$. Here we can see that for each input---a
  value on the $x$-axis, there is exactly one output---a value on the
  $y$-axis.}
\label{plot:greatest-integer fxn}
\end{marginfigure}


Just to remind you, a function maps from one set to another. We call
the set a function is mapping from the \textbf{domain}\index{domain}
or \textit{source} and we call the set a function is mapping to the
\textbf{range}\index{range} or \textit{target}.  In our previous
examples the domain and range have both been the real numbers, denoted
by $\R$. In our next examples we show that this is not always the
case.


\begin{example}
Consider the function that maps non-negative real numbers to their positive square root. This function is denoted by 
\[
f(x) = \sqrt{x}.
\]
Note, since this is a function, and its range consists of the positive real numbers, we have that 
\[
\sqrt{x^2} = |x|.
\]
See Figure~\ref{plot:sqrt fxn} for a plot of this
function.
\end{example}

Finally, we will consider a function whose domain is all real numbers
except for a single point.

\begin{example}
Consider the function defined by 
\[
f(x) = \frac{x^2 - 3x + 2}{x-2}
\]
This function may seem innocent enough; however, it is undefined at
$x=2$. See Figure~\ref{plot:point undfed fxn} for a plot of this function.
\end{example}

\begin{marginfigure}[0in]
\begin{tikzpicture}
	\begin{axis}[
            xmin=-8,xmax=8,
            ymin=-5,ymax=5,
            domain=0:8,
            axis lines =middle, xlabel=$x$, ylabel=$y$,
            every axis y label/.style={at=(current axis.above origin),anchor=south},
            every axis x label/.style={at=(current axis.right of origin),anchor=west},
          ]
	  \addplot [very thick, penColor, smooth,samples=100] {sqrt(x)};
        \end{axis}
\end{tikzpicture}
\caption{A plot of $f(x)=\sqrt{x}$. Here we can see that for each
  input---a non-negative value on the $x$-axis, there is exactly one
  output---a positive value on the $y$-axis.}
\label{plot:sqrt fxn}
\end{marginfigure}


\begin{marginfigure}[0in]
\begin{tikzpicture}
	\begin{axis}[
            domain=-2:4,
            axis lines =middle, xlabel=$x$, ylabel=$y$,
            every axis y label/.style={at=(current axis.above origin),anchor=south},
            every axis x label/.style={at=(current axis.right of origin),anchor=west},
            xtick={-2,...,4},
            ytick={-3,...,3},
          ]
	  \addplot [very thick, penColor, smooth] {x-1};
          \addplot[color=penColor,fill=background,only marks,mark=*] coordinates{(2,1)};  %% open hole
        \end{axis}
\end{tikzpicture}
\caption{A plot of $f(x)=\protect\frac{x^2 - 3x + 2}{x-2}$. Here we
  can see that for each input---any value on the $x$-axis except for
  $x=2$, there is exactly one output---a value on the $y$-axis.}
\label{plot:point undfed fxn}
\end{marginfigure}


\begin{exercises}

\begin{exercise} 
In Figure~\ref{} we see a plot of $y=f(x)$. What is $f(4)$?    %% insert picture of f(x) = abs(x-2) on [-5, 5].
\begin{answer}
$2$
\end{answer}
\end{exercise}


\begin{exercise} 
In Figure~\ref{} we see a plot of $y=f(x)$. What is $f(-2)$?  %% insert picture of f(x) = (x-1)/(x+3) on [-5, 5].
\begin{answer}
$-3$
\end{answer}
\end{exercise}


\begin{exercise} Consider the following points:
\[
(5,8),\qquad (3,6), \qquad(-6,-9), \qquad(-1,-4), \qquad(-10,7)
\]
Could these points all be on the graph of a function $y =f(x)$?
\begin{answer}
Yes.  Every input has exactly one output.
\end{answer}
\end{exercise}


\begin{exercise} Consider the following points:
\[
(7,-4),\qquad (0,3), \qquad(-2,-2), \qquad(-1,-8), \qquad(10,4)
\]
Could these points all be on the graph of a function $y =f(x)$?
\begin{answer}
Yes.  Every input has exactly one output.
\end{answer}
\end{exercise}


\begin{exercise} Consider the following points that lie on the graph of $y =f(x)$.
\[
(-5,8),\qquad (5,-1), \qquad(-4,0), \qquad(2,-9), \qquad(4,10)
\]
If $f(x)=-9$ find the value of $x$.
\begin{answer}
$x=2$
\end{answer}
\end{exercise}

\begin{exercise} A student thinks the set of points does not define a function.
\[
(-7,-4),\qquad (10,-4), \qquad(0,-4), \qquad(3,-4)
\]
They argue that the output -4 has four different inputs.  Are they correct?
\begin{answer}
No.  These points define a function as every input has a unique output.
\end{answer}
\end{exercise}

\begin{exercise} Consider the following points:
\[
(-1,5),\qquad (-3,4), \qquad($x$,3), \qquad(5,-3), \qquad(8,5)
\]
Name a value of $x$ so that these points do not define a function. 
\begin{answer}
If $x$ were one of -1, -3, 5, or 8.
\end{answer}
\end{exercise}

\begin{exercise} Let $f(x) = 18x^5-27x^4-32x^3+11x^2 -7x +4$, evaluate $f(0)$.
\begin{answer}
$4$
\end{answer}
\end{exercise}

\begin{exercise} Let $f(x) = x^5+2x^4+3x^3+4x^2+5x+6$, evaluate $f(1)$.
\begin{answer}
$21$
\end{answer}
\end{exercise}

\begin{exercise} Let $f(x) = x^5+2x^4+3x^3+4x^2+5x+6$, evaluate $f(-1)$.
\begin{answer}
$3$
\end{answer}
\end{exercise}

\begin{exercise} Let $f(x) =\sqrt{x^2+x+1}$, evaluate $f(w)$.
\begin{answer}
$\sqrt{w^2+w+1}$
\end{answer}
\end{exercise}

\begin{exercise} Let $f(x) =\sqrt{x^2+x+1}$, evaluate $f(x+h)$.
\begin{answer}
$\sqrt{(x+h)^2+(x+h)+1}$
\end{answer}
\end{exercise}

\begin{exercise} Let $f(x) = \sqrt{x^2+x+1}$, evaluate $f(x+h) - f(x)$.
\begin{answer}
$\sqrt{(x+h)^2+(x+h)+1} - \sqrt{x^2+x+1}$
\end{answer}
\end{exercise}

\begin{exercise} Let $f(x) = x+1$. What is $f(f(f(f(1))))$?
\begin{answer}
$5$
\end{answer}
\end{exercise}

\begin{exercise} Let $f(x) = x+1$. What is $f(f(f(f(x+h))))$?
\begin{answer}
$4+x+h$
\end{answer}
\end{exercise}

\begin{exercise} 
If $f(8) = 8$ and $g(x)=3\cdot f(x)$, what point must satisfy $y=g(x)$?
\begin{answer}
$x=8$, $y=24$
\end{answer}
\end{exercise}

\begin{exercise} 
If $f(7) = 6$ and $g(x)=f(8\cdot x)$, what point must satisfy $y=g(x)$?
\begin{answer}
$x=7/8$, $y=6$
\end{answer}
\end{exercise}

\begin{exercise} 
If $f(-1) = -7$ and $f(x)=g(-6\cdot x)$, what point must satisfy
$y=g(x)$?
\begin{answer}
$x=6$, $y=-7$
\end{answer}
\end{exercise}


\end{exercises}








\section{Inverses of Functions}


If a function maps every input to exactly one output, an inverse of that
function maps every ``output'' to exactly one ``input.'' While this
might sound somewhat esoteric, let's see if we can ground this in
some real-life contexts.

\begin{example}
Suppose that you are filling a swimming pool using a garden
hose---though because it rained last night, the pool starts with some
water in it. The volume of water in gallons after $t$ hours of
filling the pool is given by:
\[
v(t) = 700t + 200
\]
What does the inverse of this function tell you? What is the inverse
of this function?
\end{example}


\marginnote[.5in]{Here we abuse notation slightly, allowing $v$ and $t$
  to simultaneously be names of variables and functions.  This is
  standard practice in calculus classes.}
\begin{solution}
While $v(t)$ tells you how many gallons of water are in the pool after
a period of time, the inverse of $v(t)$ tells you how must time must
be spent to obtain a given volume. To compute the inverse function,
first set $v=v(t)$ and write
\[
v = 700t + 200.
\]
Now solve for $t$:
\[
t = v/700 - 2/7
\]
This is a function that maps volumes to times, and 
$t(v) = v/700-2/7$.
\end{solution}


Now let's consider a different example.

\begin{example}\label{E:example-ball-bridge}
Suppose you are standing on a bridge that is 60 meters above
sea-level. You toss a ball up into the air with an initial velocity of
30 meters per second.  If $t$ is the time (in seconds) after we toss
the ball, then the height at time $t$ is approximately $h(t) = -5 t^2
+30t+60$. What does the inverse of this function tell you? What is the inverse
of this function?
\end{example}


\begin{solution}
While $h(t)$ tells you how height the ball is above sea-level at an
instant of time, the inverse of $h(t)$ tells you what time it is when
the ball is at a given height. There is only one problem: There is no
function that is the inverse of $h(t)$. Consider Figure~\ref{plot:fxn
  ball}, we can see that for some heights---namely 60 meters, there
are two times. 

While there is no inverse function for $h(t)$, we can find one if we
\textit{restrict the domain} of $h(t)$. Take it as given that the
maximum of $h(t)$ is at $105$ meters and $t=3$ seconds, later on in
this course you'll know how to find points like this with ease. In
this case, we may find an inverse of $h(t)$ on the interval
$[3,\infty)$. Write
\begin{align*}
h &=  -5 t^2 +30t+60\\
0 &= -5 t^2 +30t+(60 - h)
\end{align*}
and solve for $t$ using the quadratic formula
\begin{align*}
t &= \frac{-30\pm \sqrt{30^2 -4(-5)(60-h)}}{2(-5)}\\
&= \frac{-30\pm \sqrt{30^2 +20(60-h)}}{-10}\\
&=3\mp \sqrt{3^2+ .2(60-h)}\\
&=3\mp \sqrt{9+ .2(60-h)}\\
&=3\mp \sqrt{21-.2h}
\end{align*}
Now we must think about what it means to restrict the domain of $h(t)$
to values of $t$ in $[3,\infty)$. Since $h(t)$ has its maximum value
  of $105$ when $t=3$, the largest $h$ could be is $105$. This means
  that $21-.2h >0$ and so $\sqrt{21-.2h}$ is a real number. We know
  something else too, $t>3$. This means that the ``$\mp$'' that we see
  above must be a ``$+$.''  So the inverse of $h(t)$ on the interval
  $[3,\infty)$ is $t(h) = 3+ \sqrt{21-.2h}$. A similar argument will
    show that the inverse of $h(t)$ on the interval $(-\infty, 3]$ is
  $t(h) = 3- \sqrt{21-.2h}$.
\end{solution}

\begin{marginfigure}[-7in]
\begin{tikzpicture}
	\begin{axis}[
            clip=false, domain=0:7.58, axis lines =middle, xlabel=$t$,
            ylabel=$h$, every axis y label/.style={at=(current
              axis.above origin),anchor=south}, every axis x
            label/.style={at=(current axis.right of
              origin),anchor=west}, ] \addplot [very thick, penColor,
            smooth] {-5*x^2 +30*x+60};
        \end{axis}
\end{tikzpicture}
\caption{A plot of $h(t)=-5t^2+30t+60$. Here we can see that for each
  input---a value on the $t$-axis, there is exactly one output---a
  value on the $h$-axis. However, for each value on the $h$ axis,
  sometimes there are two values on the $t$-axis. Hence there is no
  function that is the inverse of $h(t)$.}
\label{plot:fxn ball}
\end{marginfigure}



\begin{marginfigure}[-2in]
\begin{tikzpicture}
	\begin{axis}[
            clip=false, domain=0:7.58, axis lines =middle, xlabel=$t$,
            ylabel=$h$, every axis y label/.style={at=(current
              axis.above origin),anchor=south}, every axis x
            label/.style={at=(current axis.right of
              origin),anchor=west}, ] 
          \addplot [very thick, penColor, smooth] {-5*x^2 +30*x+60};
          \addplot [very thick, penColor2] {80};
          \addplot [very thick, penColor4] plot coordinates {(5,0) (5,110)};
        \end{axis}
\end{tikzpicture}
\caption{A plot of $h(t)=-5t^2+30t+60$. While this plot passes the
  vertical line test, and hence represents $y$ as a function of $x$,
  it does not pass the horizontal line test, so the function is not one-to-one.}
\label{plot:fxn vert/horiz}
\end{marginfigure}



We see two different cases with our examples above. To clearly
describe the difference we need a definition.

\begin{definition}\index{one-to-one} 
A function is \textbf{one-to-one} if for every value in the range,
there is exactly one value in the domain.
\end{definition}

You may recall that a plot gives $y$ as a function of $x$ if every
vertical line cross the plot at most once, this is commonly known as
the vertical line test. A function is one-to-one if every horizontal
line cross the plot at most once, this is commonly known as the
horizontal line test, see Figure~\ref{plot:fxn vert/horiz}.  We can
only find an inverse to a function when it is one-to-one, otherwise we
must restrict the domain as we did in
Example~\ref{E:example-ball-bridge}.


Let's look at several examples.



\begin{example}
Consider the function
\[
f(x) = x^3.
\]
Does $f(x)$ have an inverse? If so what is it? If not, attempt to
restrict the domain of $f(x)$ and find an inverse on the restricted
domain.
\end{example}


\begin{solution}
In this case $f(x)$ is one-to-one and $f^{-1}(x) = \sqrt[3]{x}$. See Figure~\ref{plot:fxn and inverse x^3}.
\end{solution}

\begin{marginfigure}[-2in]
\begin{tikzpicture}
	\begin{axis}[
            domain=-2:2,
            xmin=-2, xmax=2,
            ymin=-2, ymax=2,
            axis lines =middle, xlabel=$x$, ylabel=$y$,
            every axis y label/.style={at=(current axis.above origin),anchor=south},
            every axis x label/.style={at=(current axis.right of origin),anchor=west},
          ]
	  \addplot [very thick, penColor, smooth] {x^3};
          \addplot [very thick, penColor2, smooth, samples=100,domain=.01:2] {x^(1/3)};
          \addplot [very thick, penColor2, smooth, samples=100,domain=-2:-.01] {-abs(x)^(1/3)};
          \addplot [very thick, penColor2] plot coordinates {(.01,.215) (-.01,-.215)};
          \addplot [dashed, textColor] {x};
          \node at (axis cs:-1.2,-.42) [penColor,anchor=west] {$f(x)$};
          \node at (axis cs:1.2,.9) [penColor2, anchor=west] {$f^{-1}(x)$};
        \end{axis}
\end{tikzpicture}
\caption{A plot of $f(x)=x^3$ and $f^{-1}(x) = \sqrt[3]{x}$. Note
  $f^{-1}(x)$ is the image of $f(x)$ after being flipped over the line
  $y=x$.}
\label{plot:fxn and inverse x^3}
\end{marginfigure}


\begin{example}
Consider the function
\[
f(x) = x^2.
\]
Does $f(x)$ have an inverse? If so what is it? If not, attempt to
restrict the domain of $f(x)$ and find an inverse on the restricted
domain.
\end{example}


\begin{solution}
In this case $f(x)$ is not one-to-one. However, it is one-to-one on
the interval $[0,\infty)$. Hence we can find an inverse of $f(x)=x^2$
  on this interval, and it is our familiar function $\sqrt{x}$.  See
  Figure~\ref{plot:fxn and inverse x^2}.
\end{solution}

\begin{marginfigure}[0in]
\begin{tikzpicture}
	\begin{axis}[
            domain=-2:2,
            xmin=-2, xmax=2,
            ymin=-2, ymax=2,
            axis lines =middle, xlabel=$x$, ylabel=$y$,
            every axis y label/.style={at=(current axis.above origin),anchor=south},
            every axis x label/.style={at=(current axis.right of origin),anchor=west},
          ]
	  \addplot [very thick, penColor, smooth] {x^2};
          \addplot [very thick, penColor2, smooth, samples=100,domain=0:2] {sqrt(x)};
          \addplot [dashed, textColor] {x};
          \node at (axis cs:-1.2,.55) [penColor,anchor=west] {$f(x)$};
          \node at (axis cs:1.4,1) [penColor2, anchor=west] {$f^{-1}(x)$};
        \end{axis}
\end{tikzpicture}
\caption{A plot of $f(x)=x^3$ and $f^{-1}(x) = \sqrt{x}$. While
  $f(x)=x^2$ is not one-to-one on $\R$, it is one-to-one on
  $[0,\infty$.}
\label{plot:fxn and inverse x^2}
\end{marginfigure}


\subsection{A Word on Notation}

Given a function $f(x)$, we have a way of writing an inverse of $f(x)$, assuming it exists
\[
f^{-1}(x) = \text{the inverse of $f(x)$, if it exists.}
\]
On the other hand,
\[
f(x)^{-1} = \frac{1}{f(x)}.
\]

\begin{warning}
It is not usually the case that 
\[
f^{-1}(x) = f(x)^{-1}.
\]
\end{warning}

This confusing notation is often exasperated by the fact that 
\[
\sin^2(x) = (\sin(x))^2\qquad \text{but} \qquad \sin^{-1}(x)
\ne(\sin(x))^{-1}.
\]

In the case of trigonometric functions, this confusion can be avoided
by using the notation $\arcsin$ and so on for other trionometric
functions.




\begin{exercises}

\begin{exercise}
The length in centimeters of Rapunzel's hair after $t$ months is given
by
\[
\l(t) = \frac{8t}{3}+8.
\]
Give the inverse of $\l(t)$.  What does the inverse of $\l(t)$
 represent?
\begin{answer}
$t(\l) = \frac{\l}{8} - 3$, this function gives the number of months
  required to grow hair to a given length. 
\end{answer}
\end{exercise}

\begin{exercise}
The value of someone's savings account in dollars is given by
\[
m(t) = 900t + 300
\]
where $t$ is time in months. Give the inverse of $m(t)$.  What does 
the inverse of $m(t)$ represent?
\begin{answer}
$t(m) = \frac{m}{900} - 1/3$, this function gives the number of months
  required to aquire a given amount of money. 
\end{answer}
\end{exercise}

\begin{exercise}
At graduation the students all grabbed their caps and threw them into
the air.  The height of their caps can be described by 
\[
h(t) = -5t^2+10t+2
\]
where $h(t)$ is the height in meters and $t$ is in seconds after
letting go. Give two different inverses on two different restricted
domains. What do these inverses represent?
\end{exercise}

\begin{exercise}
The number $n$ of bacteria in refrigerated food can be modeled by
\[
n(t) =17t^2 - 20t + 700
\]
where $t$ is the temperature of the food in degrees Celsius.  Give two
different inverses on two different restricted domains. What do these
inverses represent?
\end{exercise}


\begin{exercise}
sin word problem
\end{exercise}


\begin{exercise}
The value $v$ of a car in dollars after $t$ years of ownership can be
modeled by
\[
v(t) = 10000\cdot 0.8^{t}.
\]
Find an equation for the inverse $v^{-1}(t)$ and explain in words
what it represents.
\end{exercise}



\begin{exercise}
ln e word problem
\end{exercise}



\begin{exercise}
What is the difference in meaning between $f^{-1}(x)$ and $f(x)^{-1}$?
\begin{answer}
$f^{-1}(x)$ is the inverse function (if it exists) of $f(x)$;
  $f(x)^{-1}$ is $1/f(x)$, the multiplicative inverse.
\end{answer}
\end{exercise}



\begin{exercise}
$\sin^2 x$, $\sin(x)^2$  $(\sin x)^2$ $\sin(x^2)$ $\sin x^2$ $\sin(x) \sin(x)$ $(\sin x)(\sin x)$
\end{exercise}

\begin{exercise}
$\arcsin(x)$ $\sin^{-1}(x)$ $\sin(x)^{-1}$ $\frac{1}{\sin(x)}$ $\sin(x^{-1})$  $(\sin x)^{-1}$
\end{exercise}

\begin{exercise} 
Is $\sqrt{x^2} = \sqrt[3]{x^3}$? Explain your reasoning.
\begin{answer}
No. Consider $x = -1$. $\sqrt{(-1)^2} = \sqrt{1} = 1$. However, $\sqrt[3]{(-1)^3} =  \sqrt[3]{-1} = -1$.  
\end{answer}
\end{exercise}



\end{exercises}
