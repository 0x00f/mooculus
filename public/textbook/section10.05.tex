\section{Calculus with Parametric Equations}{}{}
\nobreak
We have already seen how to compute slopes of curves given by
parametric equations---it is how we computed slopes in polar
coordinates.

\begin{example} Find the slope of the cycloid $x=t-\sin t$, $y=1-\cos t$.
We compute $x'=1-\cos t$, $y'=\sin t$, so 
$${dy\over dx} ={\sin t\over  1-\cos t}.$$
Note that when $t$ is an odd multiple of $\pi$, like $\pi$ or $3\pi$,
this is $(0/2)=0$, so there is a horizontal tangent line, in agreement
with figure~\xrefn{fig:cycloid}. At even multiples of $\pi$, the
fraction is $0/0$, which is undefined. The figure shows that
there is no tangent line at such points.
\end{example}

Areas can be a bit trickier with parametric equations, depending on
the curve and the area desired. We can potentially compute areas
between the curve and the $x$-axis quite easily.

\begin{example} Find the area under one arch of the cycloid
$x=t-\sin t$, $y=1-\cos t$. We would like to compute
$$\int_0^{2\pi} y\;dx,$$
but we do not know $y$ in terms of $x$. However, the parametric
equations allow us to make a substitution: use $y=1-\cos t$
to replace $y$, and compute $dx=(1-\cos t)\;dt$. Then the integral
becomes 
$$\int_0^{2\pi} (1-\cos t)(1-\cos t)\;dt=3\pi.$$
Note that we need to convert the original $x$ limits to $t$ limits
using $x=t-\sin t$. When $x=0$, $t=\sin t$, which happens only when
$t=0$. Likewise, when $x=2\pi$, $t-2\pi=\sin t$ and
$t=2\pi$. Alternately, because we understand how the cycloid is
produced, we can see directly that one arch is generated by 
$0\le t\le 2\pi$. In general, of course, the $t$ limits will be
different than the $x$ limits.
\end{example}

This technique will allow us to compute some quite interesting areas,
as illustrated by the exercises.

As a final example, we see how to compute the length of a curve given
by parametric equations. Section~\xrefn{sec:arc length} investigates
arc length for functions given as $y$ in terms of $x$, and develops
the formula for length:
$$\int_a^b \sqrt{1+\left({dy\over dx}\right)^2}\;dx.$$
Using some properties of derivatives, including the chain rule, we can
convert this to use parametric equations $x=f(t)$, $y=g(t)$:
$$\eqalign{
  \int_a^b \sqrt{1+\left({dy\over dx}\right)^2}\;dx&=
  \int_a^b \sqrt{\left({dx\over dt}\right)^2
  +\left({dx\over dt}\right)^2\left({dy\over dx}\right)^2}\;{dt\over dx}\;dx \\
  &=\int_u^v \sqrt{\left({dx\over dt}\right)^2+
    \left({dy\over dt}\right)^2}\;dt \\
  &=\int_u^v \sqrt{(f'(t))^2+(g'(t))^2}\;dt. \\}
$$
Here $u$ and $v$ are the $t$ limits corresponding to the $x$ limits
$a$ and $b$.

\begin{example} Find the length of one arch of the cycloid.
From $x=t-\sin t$, $y=1-\cos t$, we get the derivatives
$f'=1-\cos t$ and $g'=\sin t$, so the length is 
$$
  \int_0^{2\pi} \sqrt{(1-\cos t)^2+\sin^2 t}\;dt=
  \int_0^{2\pi} \sqrt{2-2\cos t}\;dt.
$$
Now we use the formula $\ds \sin^2(t/2)=(1-\cos(t))/2$ or
$\ds 4\sin^2(t/2)=2-2\cos t$ to get
$$\int_0^{2\pi} \sqrt{4\sin^2(t/2)}\;dt.$$
Since $0\le t\le2\pi$, $\sin(t/2)\ge 0$, so we can rewrite this as
$$\int_0^{2\pi} 2\sin(t/2)\;dt = 8.$$
\end{example}

\begin{exercises}

\begin{exercise} Consider the curve of 
exercise~\xrefn{exer:pseudo cycloid} in 
section~\xrefn{sec:parametric equations}. Find all values of
$t$ for which the curve has a horizontal tangent line.
\begin{answer} There is a horizontal tangent at all multiples of $\pi$.
\end{answer}\end{exercise}

\begin{exercise} Consider the curve of 
exercise~\xrefn{exer:pseudo cycloid} in 
section~\xrefn{sec:parametric equations}. Find the area under
one arch of the curve.
\begin{answer} $9\pi/4$
\end{answer}\end{exercise}

\begin{exercise} Consider the curve of 
exercise~\xrefn{exer:pseudo cycloid} in 
section~\xrefn{sec:parametric equations}. Set up an integral
for the length of one arch of the curve.
\begin{answer} $\ds \int_0^{2\pi}{1\over2} \sqrt{5-4\cos t}\;dt$
\end{answer}\end{exercise}

\begin{exercise} Consider the hypercycloid of 
exercise~\xrefn{exer:hypercycloid} in 
section~\xrefn{sec:parametric equations}. Find all points at
which the curve has a horizontal tangent line.
\begin{answer} Four points:\hfill\break
$\ds \left({-3-3\sqrt5\over4},
\pm\sqrt{5-\sqrt5\over8}\right)$,\hfill\break
$\ds \left({-3+3\sqrt5\over4},
\pm\sqrt{5+\sqrt5\over8}\right)$ 
\end{answer}\end{exercise}

\begin{exercise} Consider the hypercycloid of 
exercise~\xrefn{exer:hypercycloid} in 
section~\xrefn{sec:parametric equations}. Find the area between the
large circle and
one arch of the curve.
\begin{answer} $11\pi/3$
\end{answer}\end{exercise}

\begin{exercise} Consider the hypercycloid of
exercise~\xrefn{exer:hypercycloid} in section~\xrefn{sec:parametric
  equations}. Find the length of one arch of the curve.  
\begin{answer} $32/3$
\end{answer}\end{exercise}

\begin{exercise} Consider the hypocycloid of 
exercise~\xrefn{exer:hypocycloid} in 
section~\xrefn{sec:parametric equations}. Find the area inside the curve.
\begin{answer} $2\pi$
\end{answer}\end{exercise}

\begin{exercise} Consider the hypocycloid of
exercise~\xrefn{exer:hypocycloid} in section~\xrefn{sec:parametric
  equations}. Find the length of one arch of the curve.  
\begin{answer} $16/3$
\end{answer}\end{exercise}

\begin{exercise} Recall the involute of a circle from
exercise~\xrefn{exer:involute of a circle} in
section~\xrefn{sec:parametric equations}. Find the point in the first
quadrant in figure~\xrefn{fig:involute} 
at which the tangent line is vertical.
\begin{answer} $(\pi/2,1)$
\end{answer}\end{exercise}

\begin{exercise} Recall the involute of a circle from
exercise~\xrefn{exer:involute of a circle} in
section~\xrefn{sec:parametric equations}. Instead of an infinite
string, suppose we have a string of length $\pi$ attached to the unit
circle at $(-1,0)$, and initially laid around the top of the circle
with its end at $(1,0)$. If we grasp the end of the string and begin
to unwind it, we get a piece of the involute, until the string is
vertical. If we then keep the string taut and continue to rotate it
counter-clockwise, the end traces out a semi-circle with center at
$(-1,0)$, until the string is vertical again. Continuing, the end of
the string traces out the mirror image of the initial portion of the
curve; see figure~\xrefn{fig:involute plus semicircle}. Find the area
of the region inside this curve and outside the unit circle.
\begin{answer} $\ds 5\pi^3/6$
\end{answer}\end{exercise}

\begin{exercise} Find the length of the curve from the previous exercise,
shown in figure~\xrefn{fig:involute plus semicircle}.
\begin{answer} $\ds 2\pi^2$
\end{answer}\end{exercise}

\begin{exercise} Find the length of the spiral of Archimedes
(figure~\xrefn{fig:area inside spiral}) for $0\le\theta\le2\pi$.
\begin{answer} $\ds(2\pi\sqrt{4\pi^2+1}+\ln(2\pi+\sqrt{4\pi^2+1}))/2$
\end{answer}\end{exercise}

\figure
\vbox{\beginpicture
\normalgraphs
\sevenpoint
\setcoordinatesystem units <7truemm,7truemm>
\setplotarea x from -5 to 2, y from -4  to 3.5
\axis left shiftedto x=0 /
\axis bottom shiftedto y=0 /
\circulararc 360 degrees from 0 1 center at 0 0
\setquadratic
\textRed
\plot 1.000 0.000 1.005 0.000 1.022 0.003 1.048 0.010 1.084 0.024
1.128 0.047 1.178 0.079 1.234 0.124 1.292 0.183 1.350 0.255
1.407 0.342 1.459 0.445 1.504 0.563 1.540 0.695 1.563 0.841
1.571 1.000 1.562 1.170 1.533 1.348 1.484 1.534 1.411 1.723
1.314 1.913 1.191 2.102 1.043 2.285 0.868 2.459 0.668 2.621
0.443 2.767 0.194 2.894 -0.077 2.998 -0.369 3.076 -0.677 3.125
-1.000 3.142 /
\plot -1.000 -3.142 -0.677 -3.125 -0.369 -3.076 -0.077 -2.998 0.194 -2.894
0.443 -2.767 0.668 -2.621 0.868 -2.459 1.043 -2.285 1.191 -2.102
1.314 -1.913 1.411 -1.723 1.484 -1.534 1.533 -1.348 1.562 -1.170
1.571 -1.000 1.563 -0.841 1.540 -0.695 1.504 -0.563 1.459 -0.445
1.407 -0.342 1.350 -0.255 1.292 -0.183 1.234 -0.124 1.178 -0.079
1.128 -0.047 1.084 -0.024 1.048 -0.010 1.022 -0.003 1.005 -0.000
1.000 0.000 /
\circulararc 180 degrees from -1.000 3.142 center at -1 0
\textBlack
\setlinear\setdashes <2pt>
\plot -1 -3.142 -1 3.142 /
\plot -1 0 -3.221 2.221 /
\plot -0.42 0.91 1.4 1.74 /
\endpicture}
\figrdef{fig:involute plus semicircle}
\endfigure{A region formed by the end of a string.}

\end{exercises}
