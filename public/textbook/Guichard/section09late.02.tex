\section{The natural logarithm}{}{}
\nobreak
The function $f(t)=1/t$ is continuous on $(0, \infty ) $.
By the fundamental theorem of calculus, $f$ has an antiderivative on
on the interval with end points $x$ and $1$ whenever $x>0 $. This
observation allows us to make the following definition.

\begin{definition} The {\dfont natural logarithm} $\ln(x)$ is an antiderivative of
$1/x$, given by 
$$\ln x = \int _1 ^x {1\over t}\,dt.$$
\end{definition}

Figure~\xrefn{fig:log as area} gives a geometric interpretation of $\ln$.
Note that when $x<1$, $\ln x$ is negative.

\figure
\vbox{\beginpicture
\normalgraphs
\ninepoint
\setcoordinatesystem units <1.2truecm,1.2truecm>
\setplotarea x from 0 to 3.1, y from 0 to 3.4
\axis left /
\axis bottom ticks withvalues {$1$} {$x$} / at 1 2 / /
\put {\hbox{area is $\ln x$}} [bl] <2pt,2pt> at 1.75 1
\arrow <4pt> [0.35, 1] from 1.75 1 to 1.5 0.4
\putrule from 2 0 to 2 0.5
\putrule from 1 0 to 1 1
\plot 0.300 3.333 0.368 2.721 0.435 2.299 0.502 1.990 0.570 1.754 
0.638 1.569 0.705 1.418 0.772 1.294 0.840 1.190 0.908 1.102 
0.975 1.026 1.042 0.959 1.110 0.901 1.178 0.849 1.245 0.803 
1.312 0.762 1.380 0.725 1.448 0.691 1.515 0.660 1.582 0.632 
1.650 0.606 1.718 0.582 1.785 0.560 1.852 0.540 1.920 0.521 
1.988 0.503 2.055 0.487 2.122 0.471 2.190 0.457 2.258 0.443 
2.325 0.430 2.392 0.418 2.460 0.407 2.528 0.396 2.595 0.385 
2.662 0.376 2.730 0.366 2.798 0.357 2.865 0.349 2.932 0.341 
3.000 0.333 /
\setcoordinatesystem units <1.2truecm,1.2truecm> point at -5 0
\setplotarea x from 0 to 3.1, y from 0 to 3.4
\axis left /
\axis bottom ticks withvalues {$x$} {$1$} / at 0.5 1 / /
\put {\hbox{area is $-\ln x$}} [bl] <2pt,2pt> at 1 1.5
\arrow <4pt> [0.35, 1] from 1 1.5 to 0.75 1
\putrule from 0.5 0 to 0.5 2
\putrule from 1 0 to 1 1
\plot 0.300 3.333 0.368 2.721 0.435 2.299 0.502 1.990 0.570 1.754 
0.638 1.569 0.705 1.418 0.772 1.294 0.840 1.190 0.908 1.102 
0.975 1.026 1.042 0.959 1.110 0.901 1.178 0.849 1.245 0.803 
1.312 0.762 1.380 0.725 1.448 0.691 1.515 0.660 1.582 0.632 
1.650 0.606 1.718 0.582 1.785 0.560 1.852 0.540 1.920 0.521 
1.988 0.503 2.055 0.487 2.122 0.471 2.190 0.457 2.258 0.443 
2.325 0.430 2.392 0.418 2.460 0.407 2.528 0.396 2.595 0.385 
2.662 0.376 2.730 0.366 2.798 0.357 2.865 0.349 2.932 0.341 
3.000 0.333 /
\endpicture}
\figrdef{fig:log as area}
\endfigure{$\ln(x)$ is an area.}

Some properties of this function $\ln x$ are now easy to see.

\begin{theorem} Suppose that $x,y >0 $ and $q\in \Q$.
\label{thm:log rules}
\begin{itemize} % BADBAD
\item{a.} $\ds{d\over dx} \ln x = {1\over x}.$
\item{b.} $\ln(1) = 0$.
\item{c.} $\ln (xy) = \ln x+ \ln y $
\item{d.} $\ln(x/y) = \ln x - \ln y $
\item{e.} $\ln x^q  = q\ln x $.
\end{itemize}

\begin{proof} Part (a) is simply the Fundamental Theorem of Calculus
(\xrefn{thm:fundamental_theorem_II}). Part (b) follows directly from
the definition, since 
$$\ln(1)=\int_1^1 {1\over t}\,dt.$$

Part (c) is a bit more involved; start with:
$$\ln(xy)=\int_1^{xy} {1\over t}\,dt=
\int_1^{x} {1\over t}\,dt+\int_x^{xy} {1\over t}\,dt
=\ln(x)+\int_x^{xy} {1\over t}\,dt.$$

In the remaining integral, use the substitution $u=t/x$ to get
$$\int_x^{xy} {1\over t}\,dt=\int_1^{y} {1\over xu}x\,du
=\int_1^{y} {1\over u}\,du=\ln(y).$$

% Fix $y>0$ and set $g(x) =\ln (xy)$.
% By the chain rule,
% $\ds g'(x) = {1\over xy} y = {1\over x} = {d\over dx} \ln x$.
% Since $g(x)$ and $\ln x$ have the same derivative, 
% $g(x) =\ln x + C$ for some constant $C$.
% To ascertain $C$ we substitute an appropriate value for $x$.
% When $x=1$,
% $g(1) = \ln (1\cdot y)= \ln(y)$, and also 
% $g(1) = \ln(1)+C= C$, so $\ln(y) = C$ and thus
% $\ln(xy)=\ln(x)+\ln(y)$.



Parts (d) and (e) are left as exercises.
\end{proof}

Part (e) is in fact true for any real number $q$
(not just rationals) but one of the points of our approach here is
to give a rigorous definition of real powers which so far we have
not done.

 We now turn to the task of sketching the graph of $\ln x$.

 \begin{theorem} $\ln x $ is increasing and 
its graph is concave down everywhere.

\begin{proof} Since ${d\over dx}\ln x =1/x$ is positive 
for $x>0$, the Mean Value Theorem (\xrefn{thm:mvt}) implies that $\ln x$
is increasing.
The second derivative of $\ln x$ is then $-1/x^2$ which is negative,
so the graph is concave down.
\end{proof}

Notice that this theorem implies that $\ln x$ is
injective.

\begin{theorem} $\ds\lim_{x\to\infty} \ln x =\infty$

\begin{proof} Note that $\ln 2 >0 $ and for $n\in\N$,
$\ln 2^n =n\ln 2$. Since $\ln x$ is increasing, 
when $x> 2^n$, $\ln(x)>n\ln2$. Since
$\ds\lim_{n\to\infty}n\ln2=\infty$,
also $\ds\lim_{x\to\infty} \ln x =\infty$.
\end{proof}


\cor $\lim_{x\to 0^+} \ln x =-\infty$

\begin{proof} If $0< x< 1 $, then $(1/x)>1$ and
$\lim_{x\to 0^+} (1/x) = \infty $. Let $y=1/x$; then
$\lim _{x\to 0^+} \ln x= \lim_{y\to \infty}\ln(1/y)
=\lim_{y\to \infty}\ln(1)-\ln(y)=\lim_{y\to \infty}-\ln(y)
=-\infty$.
\end{proof}


 Thus, the domain of $\ln $ is $(0, \infty ) $ and the range is
 $\R$; $\ln(x)$ is shown in figure~\xrefn{fig:graph of natural logarithm}.

\figure
\vbox{\beginpicture
\normalgraphs
\ninepoint
\setcoordinatesystem units <1.2truecm,1.2truecm>
\setplotarea x from 0 to 6.1, y from -2.5 to 2
\axis left ticks numbered from -2 to 2 by 1 /
\axis bottom shiftedto y=0 ticks numbered from 1 to 6 by 1 /
\axis bottom shiftedto y=0 ticks numbered withvalues {$e$} / at 2.718 / /
\put {$\bullet$} at 2.718 1
\plot 0.100 -2.303 0.248 -1.396 0.395 -0.929 0.542 -0.612 0.690 -0.371 
0.838 -0.177 0.985 -0.015 1.132 0.124 1.280 0.247 1.428 0.356 
1.575 0.454 1.722 0.544 1.870 0.626 2.018 0.702 2.165 0.772 
2.312 0.838 2.460 0.900 2.608 0.958 2.755 1.013 2.902 1.066 
3.050 1.115 3.198 1.162 3.345 1.207 3.492 1.251 3.640 1.292 
3.788 1.332 3.935 1.370 4.082 1.407 4.230 1.442 4.378 1.476 
4.525 1.510 4.672 1.542 4.820 1.573 4.968 1.603 5.115 1.632 
5.262 1.661 5.410 1.688 5.558 1.715 5.705 1.741 5.852 1.767 
6.000 1.792 /
\endpicture}
\figrdef{fig:graph of natural logarithm}
\endfigure{The graph of $\ln(x)$.}

By the intermediate value theorem (\xrefn{thm:intermediate value theorem})
there is a number $e$ such
that $\ln e = 1 $. The number $e$ is also known as
{\dfont Napier's constant\index{Napier's constant}}.

It turns out that $e$ is not rational. In fact, $e$ is not the root of
a polynomial with rational coefficients which means that $e$ is a
{\dfont transcendental number\index{transcendental number}}. We will
not prove these assertions here.
The value of $e$ is 
approximately $2.718$.

% We now consider several examples of calculus computations involving
% the natural logarithm.
% \begin{lem} Let $g(x) >0 $ for every $x$ in the domain of $g$. If
% $g$ is differentiable, then
% \begin{equation} [\ln g(x)]' = \frac{g'(x) }{g(x) } . \end{equation} \end{lem}
% The result follows by an immediate application of the chain rule.

\begin{example} Let $f(x) =\ln (x^5 + 7x +12) $. Compute $f'(x) $.

Using the chain rule:
$\ds f'(x) = {1\over x^5 +7x+ 12}(5x+7)$.
\end{example}

\begin{example} Let $f(x) =\ln (-x) $ for $x<0 $. Compute $f'(x) $.
$$f'(x) ={1\over -x}(-1) ={1\over x}.$$
So the derivatives of $\ln(x)$ and $\ln(-x)$ are the same. Thus, you
will often see $\ds\int{1\over x}\,dx=\ln|x|+C$ as the general
antiderivative of $1/x$.
\end{example}

\begin{example} Compute $\ds\int \tan x \,dx $.

Use $u=\cos x$:
$$
\int\tan x\,dx=\int{\sin x\over\cos x}\,dx=\int -{1\over u}\,du
=-\ln |u|+C = -\ln|\cos x| +C.$$
Using one of the properties of the logarithm, we could go further:
$$-\ln|\cos x| +C=\ln|(\cos x)^{-1}|+C=\ln|\sec x|+C.$$
\vglue-20pt\end{example}

\begin{example} Let $\ds f(x) = {x^{18} (x+2)^{6/7}\over (x^{10} + 4x^2 +1 )^6 }$.
 Compute $f'(x) $

 Computing the derivative directly is straightforward
 but irritating. We therefore take an indirect approach. Note that
 $f(x) >0 $ for every $x$.
 Let $g(x) = \ln f(x) $. Then $g'(x) = f'(x)/f(x)$ and so 
$f'(x) =f(x)g'(x)$. Now
$$\eqalign{
g(x) &=\ln\left({x^{18} (x+2)^{6/7}\over (x^{10} + 4x^2 +1 )^6 }\right) \\
&= 18\ln x +{6\over 7} \ln (x+2) - 6\ln (x^{10} +4x^2 +1) \\}$$
Hence, 
$$g'(x) = {18\over x} +{6\over 7(x+2)}
-{6(10x^9 +8x )\over x^{10} +4x^2 +1 }.$$
Therefore,
$$f'(x) = {x^{18} (x+2)^{6/7}\over (x^{10} + 4x^2 +1 )^6 }
\left({18\over x} +{6\over 7(x+2)}
-{6(10x^9 +8x )\over x^{10} +4x^2 +1 }\right).$$
\vglue-20pt\end{example}

\begin{exercises}

\exer Prove parts (d) and (e) of theorem~\xrefn{thm:log rules}.

\noindent
In subsequent exercises, it is understood that the arguments in any
logarithms are positive unless otherwise stated.

\exer Expand $\ln ((x+45)^7 (x-2))$.

\exer Expand $\ds\ln {x^3\over 3x-5 +(7/x)} $.

\exer Sketch the graph of $y= \ln (x-7)^3  + 14 $.

\exer Sketch the graph of $y=\ln |x| $ for $x\neq 0$.

\exer Write $\ln 3x + 17 \ln (x-2) -
2\ln (x^2 + 4x + 1) $ as a single logarithm.

\exer Differentiate $f(x) =x\ln x $.

\exer Differentiate $f(x) =\ln (\ln (3x) )$.

\exer Sketch the graph of $\ln (x^2 - 2x) $. 

\exer Differentiate $\ds f(x) ={1+\ln (3x^2 )\over 1+ \ln
(4x)} $. 

\exer Differentiate $f(x) =\ln |\sec x +\tan x |
$.

\exer Find the second derivative of $f(x) =\sqrt{\ln (x^4
-2) } $. 

\exer Find the equation of the tangent line to $f(x) =\ln x
$ at $x=a $. 

\exer Differentiate $\ds f(x) ={x^8 (x-23)^{1/2}\over 
27 x^6(4x-6)^8 }$.
  

\exer If $f(x) = \ln (x^3 + 2 ) $ compute $f'(e^{1/3} )
$.

\exer Compute
$\ds\int _1 ^e {1\over x}\,dx$.


\exer Compute the derivative with respect to $x$ of $\ds\int _1 ^{\ln
  x} \ln t\, dt $. (Assume that $x>1$.)

\exer Compute
$\ds\int _0 ^{\pi/6} \tan (2x)\, dx$.


\exer Compute $\ds\int {\ln x\over x}\,dx$.


\exer Compute $\ds\int {\sin (2x)\over 1+\cos ^2 x }\,dx$.


\exer Find the volume of the solid obtained by rotating the
region under $y=1/\sqrt{x}$ from $1$ to $e$ about the
$x$-axis.

\end{exercises}
