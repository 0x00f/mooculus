\section{Hyperbolic Functions}{}{}
\label{sec:hyperbolic functions}
\nobreak
The hyperbolic functions appear with some frequency in applications,
and are quite similar in many respects to the trigonometric
functions. This is a bit surprising given our initial definitions.

\begin{definition}
The {\dfont hyperbolic cosine\index{hyperbolic cosine}} is the function
$$\cosh x ={e^x +e^{-x }\over2},$$
and the {\dfont hyperbolic sine\index{hyperbolic sine}} is the function
$$\sinh x ={e^x -e^{-x}\over 2}.$$
\end{definition}

Notice that $\cosh$ is even (that is, $\cosh(-x)=\cosh(x)$)
while $\sinh$ is odd ($\sinh(-x)=-\sinh(x)$), and
$\ds\cosh x + \sinh x = e^x$. Also, for all $x$,
$\cosh x >0$, while $\sinh x=0$ if and only if $\ds e^x -e^{-x }=0$,
which is true precisely when $x=0$.

\begin{lemma} 
The range of $\cosh x$ is $[1,\infty)$.
\end{lemma}
\begin{proof}
Let $y= \cosh x$. We solve for $x$:
\begin{align*}
y&={e^x +e^{-x }\over 2} \\
2y &= e^x + e^{-x } \\
2ye^x &= e^{2x} + 1 \\
 0 &= e^{2x}-2ye^x +1 \\
 e^{x} &= {2y \pm \sqrt{4y^2 -4}\over 2} \\
 e^{x} &= y\pm \sqrt{y^2 -1} \\
\end{align*}
From the last equation, we see $\ds y^2 \geq 1$, and since
$y\geq 0$, it follows that $y\geq 1$.

Now suppose $y\geq 1$, so $\ds y\pm \sqrt{y^2 -1}>0$. Then
$\ds x = \ln(y\pm \sqrt{y^2 -1})$ is a real number, and 
$y =\cosh x$, so $y$ is in the range of $\cosh(x)$.
\end{proof}

\begin{definition} The other hyperbolic functions are
\begin{align*}
\tanh x &= {\sinh x\over\cosh x} \\
\coth x &= {\cosh x\over\sinh x} \\
\sech x &= {1\over\cosh x} \\
\csch x &= {1\over\sinh x} \\
\end{align*}
The domain  of $\coth$ and $\csch$  is $x\neq 0$ while the domain of
the other hyperbolic functions is all real numbers. Graphs are shown
in figure~\xrefn{fig:hyperbolic functions}
\end{definition}

% BADBAD
% \figure
% \vbox{\beginpicture
% \normalgraphs
% \eightpoint
% \setcoordinatesystem units <0.6truecm,0.6truecm>
% \setplotarea x from -2 to 2, y from 0 to 4
% \axis left shiftedto x=0 ticks length <2pt> numbered from 1 to 4 by 1 /
% \axis bottom ticks length <2pt> numbered from -2 to 2 by 1 /
% \plot -2.000 3.762 -1.900 3.418 -1.800 3.107 -1.700 2.828 -1.600 2.577 
% -1.500 2.352 -1.400 2.151 -1.300 1.971 -1.200 1.811 -1.100 1.669 
% -1.000 1.543 -0.900 1.433 -0.800 1.337 -0.700 1.255 -0.600 1.185 
% -0.500 1.128 -0.400 1.081 -0.300 1.045 -0.200 1.020 -0.100 1.005 
% 0.000 1.000 0.100 1.005 0.200 1.020 0.300 1.045 0.400 1.081 
% 0.500 1.128 0.600 1.185 0.700 1.255 0.800 1.337 0.900 1.433 
% 1.000 1.543 1.100 1.669 1.200 1.811 1.300 1.971 1.400 2.151 
% 1.500 2.352 1.600 2.577 1.700 2.828 1.800 3.107 1.900 3.418 
% 2.000 3.762 /
% \setcoordinatesystem units <0.6truecm,0.6truecm> point at -5 0
% \setplotarea x from -2 to 2, y from -4 to 4
% \axis left shiftedto x=0 ticks length <2pt> withvalues {$-4$} {$-2$}
% {$4$} {$2$} / at -4  -2  4 2 / /
% \axis left shiftedto x=0 ticks length <2pt> from -4 to 4 by 1 /
% \axis bottom shiftedto y=0 ticks length <2pt> from -2 to 2 by 1 /
% \axis bottom shiftedto y=0 ticks length <2pt> withvalues {$-2$} {$2$} / at -2 2 / /
% \plot -2.000 -3.627 -1.900 -3.268 -1.800 -2.942 -1.700 -2.646 -1.600 -2.376 
% -1.500 -2.129 -1.400 -1.904 -1.300 -1.698 -1.200 -1.509 -1.100 -1.336 
% -1.000 -1.175 -0.900 -1.027 -0.800 -0.888 -0.700 -0.759 -0.600 -0.637 
% -0.500 -0.521 -0.400 -0.411 -0.300 -0.305 -0.200 -0.201 -0.100 -0.100 
% 0.000 0.000 0.100 0.100 0.200 0.201 0.300 0.305 0.400 0.411 
% 0.500 0.521 0.600 0.637 0.700 0.759 0.800 0.888 0.900 1.027 
% 1.000 1.175 1.100 1.336 1.200 1.509 1.300 1.698 1.400 1.904 
% 1.500 2.129 1.600 2.376 1.700 2.646 1.800 2.942 1.900 3.268 
% 2.000 3.627 /
% \setcoordinatesystem units <0.6truecm,0.6truecm> point at -10 0
% \setplotarea x from -2 to 2, y from -1 to 1
% \axis left shiftedto x=0 ticks length <2pt> withvalues {$-1$} {$1$} / at -1 1 / /
% \axis bottom shiftedto y=0 ticks length <2pt> from -2 to 2 by 1 /
% \axis bottom shiftedto y=0 ticks length <2pt> withvalues {$2$} / at 2 / /
% \plot -2.000 -0.964 -1.900 -0.956 -1.800 -0.947 -1.700 -0.935 -1.600 -0.922 
% -1.500 -0.905 -1.400 -0.885 -1.300 -0.862 -1.200 -0.834 -1.100 -0.800 
% -1.000 -0.762 -0.900 -0.716 -0.800 -0.664 -0.700 -0.604 -0.600 -0.537 
% -0.500 -0.462 -0.400 -0.380 -0.300 -0.291 -0.200 -0.197 -0.100 -0.100 
% 0.000 0.000 0.100 0.100 0.200 0.197 0.300 0.291 0.400 0.380 
% 0.500 0.462 0.600 0.537 0.700 0.604 0.800 0.664 0.900 0.716 
% 1.000 0.762 1.100 0.800 1.200 0.834 1.300 0.862 1.400 0.885 
% 1.500 0.905 1.600 0.922 1.700 0.935 1.800 0.947 1.900 0.956 
% 2.000 0.964 /
% \setcoordinatesystem units <0.6truecm,0.6truecm> point at -15 0
% \setplotarea x from -2 to 2, y from 0 to 1
% \axis left shiftedto x=0 ticks length <2pt> withvalues {$1$} / at 1 / /
% \axis bottom shiftedto y=0 ticks length <2pt> from -2 to 2 by 1 /
% \axis bottom shiftedto y=0 ticks length <2pt> withvalues {$-2$} {$2$} / at -2 2 / /
% \plot -2.000 0.266 -1.900 0.293 -1.800 0.322 -1.700 0.354 -1.600 0.388 
% -1.500 0.425 -1.400 0.465 -1.300 0.507 -1.200 0.552 -1.100 0.599 
% -1.000 0.648 -0.900 0.698 -0.800 0.748 -0.700 0.797 -0.600 0.844 
% -0.500 0.887 -0.400 0.925 -0.300 0.957 -0.200 0.980 -0.100 0.995 
% 0.000 1.000 0.100 0.995 0.200 0.980 0.300 0.957 0.400 0.925 
% 0.500 0.887 0.600 0.844 0.700 0.797 0.800 0.748 0.900 0.698 
% 1.000 0.648 1.100 0.599 1.200 0.552 1.300 0.507 1.400 0.465 
% 1.500 0.425 1.600 0.388 1.700 0.354 1.800 0.322 1.900 0.293 
% 2.000 0.266 /
% \setcoordinatesystem units <0.6truecm,0.6truecm> point at -20 0
% \setplotarea x from -2 to 2, y from -4 to 4
% \axis left shiftedto x=0 ticks length <2pt> withvalues {$-4$}
% {$4$} {$2$} / at -4 4 2 / /
% \axis left shiftedto x=0 ticks length <2pt> from -4 to 4 by 1 /
% \axis bottom shiftedto y=0 ticks length <2pt> from -2 to 2 by 1 /
% \axis bottom shiftedto y=0 ticks length <2pt> withvalues {$-2$} {$2$} / at -2 2 / /
% \plot -2.000 -0.276 -1.912 -0.302 -1.825 -0.331 -1.738 -0.363 -1.650 -0.399 
% -1.562 -0.438 -1.475 -0.483 -1.388 -0.533 -1.300 -0.589 -1.212 -0.653 
% -1.125 -0.726 -1.038 -0.810 -0.950 -0.910 -0.862 -1.027 -0.775 -1.170 
% -0.688 -1.346 -0.600 -1.571 -0.512 -1.868 -0.425 -2.284 -0.338 -2.907 
% -0.250 -3.959 /
% \plot 0.250 3.959 0.338 2.907 0.425 2.284 0.512 1.868 0.600 1.571 
% 0.688 1.346 0.775 1.170 0.862 1.027 0.950 0.910 1.038 0.810 
% 1.125 0.726 1.212 0.653 1.300 0.589 1.388 0.533 1.475 0.483 
% 1.562 0.438 1.650 0.399 1.738 0.363 1.825 0.331 1.912 0.302 
% 2.000 0.276 /
% \setcoordinatesystem units <0.6truecm,0.6truecm> point at -25 0
% \setplotarea x from -2 to 2, y from -4 to 4
% \axis left shiftedto x=0 ticks length <2pt> withvalues {$-4$}
% {$4$} {$2$} / at -4 4 2 / /
% \axis left shiftedto x=0 ticks length <2pt> from -4 to 4 by 1 /
% \axis bottom shiftedto y=0 ticks length <2pt> from -2 to 2 by 1 /
% \axis bottom shiftedto y=0 ticks length <2pt> withvalues {$-2$} {$2$} / at -2 2 / /
% \plot -2.000 -1.037 -1.912 -1.045 -1.825 -1.053 -1.738 -1.064 -1.650 -1.077 
% -1.562 -1.092 -1.475 -1.110 -1.388 -1.133 -1.300 -1.160 -1.212 -1.194 
% -1.125 -1.236 -1.038 -1.287 -0.950 -1.352 -0.862 -1.434 -0.775 -1.539 
% -0.688 -1.677 -0.600 -1.862 -0.512 -2.119 -0.425 -2.493 -0.338 -3.075 
% -0.250 -4.083 /
% \plot 0.250 4.083 0.338 3.075 0.425 2.493 0.512 2.119 0.600 1.862 
% 0.688 1.677 0.775 1.539 0.862 1.434 0.950 1.352 1.038 1.287 
% 1.125 1.236 1.212 1.194 1.300 1.160 1.388 1.133 1.475 1.110 
% 1.562 1.092 1.650 1.077 1.738 1.064 1.825 1.053 1.912 1.045 
% 2.000 1.037 /
% \endpicture}
% \figrdef{fig:hyperbolic functions}
% \endfigure{The hyperbolic functions: cosh, sinh, tanh, sech, csch, coth.}

% \footnote{The notation here may seem a bit strange. At first
% glance, it is not at all obvious that the hyperbolic functions have
% anything to do with the trigonometric functions. The connection
% becomes clearest in the context of the theory of complex variables
% which is a bit beyond the scope of our main discussion. It turns out
% that
% \begin{equation} \cos x +i\sin x = e^{ix } \end{equation}
% where $i^2 =-1 $ and $x$ is any complex number. (Of course, the
% expressions $\cos x $, $\sin x$, and $e^{ix} $ have to be defined
% properly when $x$ is a nonreal number.)
% 
% From there it can be shown that $\cos x =\frac{e^{ix } + e^{-ix}
% }{2} $ and hence $\cos x = \cosh (ix) $. Similarly, $i\sin x = \sinh
% (ix) $.} 

Certainly the hyperbolic functions do not closely resemble the
trigonometric functions graphically. But they do have analogous
properties, beginning with the following identity.


\begin{theorem} For all $x$ in $\R$, $\ds \cosh ^2 x -\sinh ^2 x = 1$.
\end{theorem}
\begin{proof} The proof is a straightforward computation:
$$\cosh ^2 x -\sinh ^2 x =
 {(e^x +e^{-x} )^2\over 4} -{(e^x -e^{-x} )^2\over 4}=
 {e^{2x} + 2 + e^{-2x } - e^{2x } + 2 - e^{-2x}\over 4}=
 {4\over 4} = 1.
$$
\end{proof}

This immediately gives two additional identities:
$$1-\tanh^2 x =\sech^2 x\qquad\hbox{and}\qquad
\coth^2 x - 1  =\csch^2 x.$$

The identity of the theorem also helps to provide a geometric
motivation. Recall that the graph of $\ds x^2 -y^2 =1$ is a hyperbola
with asymptotes $x=\pm y$ whose $x$-intercepts are $\pm 1$. If
$(x,y)$ is a point on the right half of the hyperbola, and if
we let $x=\cosh t$, then
$\ds y=\pm\sqrt{x^2-1}=\pm\sqrt{\cosh^2x-1}=\pm\sinh t$. So for some
suitable $t$, $\cosh t$ and $\sinh t$ are the coordinates of a typical
point on the hyperbola. In fact, it turns out that $t$ is twice the
area shown in the first graph of 
figure~\xrefn{fig:geometric def trigh}.  Even
this is analogous to trigonometry; $\cos t$ and $\sin t$ are the
coordinates of a typical point on the unit circle, and $t$ is twice
the area shown in the second graph of figure~\xrefn{fig:geometric def
trigh}. 

% BADBAD
% \figure
% \hbox to \hsize{\hfill
% \tikzpicture[domain=0:3,x=6mm,y=6mm,baseline=0]
% \draw[->] (0,0) -- (3,0) ;
% \draw[->] (0,-3) -- (0,3) ;
% \foreach \x in {1,2,3} \draw (\x,0) -- (\x,-2pt) node[anchor=north] {\eightpoint $\x$};
% \foreach \y in {-3,-2,-1,0,1,2,3} \draw (0,\y) -- (-2pt,\y) node[anchor=east]
% {\eightpoint $\y$};
% \node[anchor=west] at (2,1.3) {$(\cosh t,\sinh t)$};
% \global\advance\gpnum by 1
% \draw[color=black] plot[parametric,id=\the\gpnum,domain=-1:1] 
% function{1+2*t**2,2*t*sqrt(t**2+1)};
% \draw (0,0) -- (2,1.732); 
% \gpad
% \fill[opacity=0.5,fill=red!20] 
% (0,0) -- (1,0) 
% plot[parametric,id=\the\gpnum,domain=0:.707]
% function{1+2*t**2,2*t*sqrt(t**2+1);} -- (0,0);
% \endtikzpicture
% \hskip2cm
% \tikzpicture[domain=-1.2:1.2,baseline=0]
% \draw[->] (-1.2,0) -- (1.2,0) ;
% \draw[->] (0,-1.2) -- (0,1.2) ;
% \foreach \x in {1} \draw (\x,0) -- (\x,-2pt) node[anchor=north west] {\ninepoint $\x$};
% \draw[color=black] (0,0) circle (1);
% \draw (0,0) -- (0.5,0.886); 
% \fill[opacity=0.5,fill=red!20] 
% (0,0) -- (1,0) arc (0:60:1) -- (0,0);
% \endtikzpicture\hfill}
% \figrdef{fig:geometric def trigh}
% \endfigure{Geometric definitions of sin, cos, sinh, cosh.}

Given the definitions of the hyperbolic functions, finding their
derivatives is straightforward. Here again we see similarities to the
trigonometric functions.

\begin{theorem} $\ds{d\over dx}\cosh x=\sinh x$ and 
\label{thm:hyperbolic derivatives}
$\ds{d\over dx}\sinh x = \cosh x$.
\end{theorem}

\begin{proof}
$\ds{d\over dx}\cosh x= {d\over dx}{e^x +e^{-x}\over 2} = 
{e^x- e^{-x}\over 2} =\sinh x$, and 
$\ds\ds{d\over dx}\sinh x = {d\over dx}{e^x -e^{-x}\over 2} = 
{e^x +e^{-x }\over 2} =\cosh x$.
\end{proof}

Since $\cosh x > 0$, $\sinh x$ is increasing and hence injective, so
$\sinh x$ has an inverse, $\arcsinh x$. Also, $\sinh x > 0$ when
$x>0$, so $\cosh x$ is injective on $[0,\infty)$ and has a (partial)
inverse, $\arccosh x$. The other hyperbolic functions have inverses
as well, though $\arcsech x$ is only a partial inverse. 
We may compute the derivatives of these functions as we have other
inverse functions.

\begin{theorem} $\ds{d\over dx}\arcsinh x = {1\over\sqrt{1+x^2}}$.
\end{theorem}
\begin{proof}
Let $y=\arcsinh x$, so $\sinh y=x$. Then 
$\ds{d\over dx}\sinh y = \cosh(y)\cdot y'  = 1$, and
so $\ds y' ={1\over\cosh y} ={1\over\sqrt{1 +\sinh^2 y}} =
{1\over\sqrt{1+x^2}}$.
\end{proof}

The other derivatives are left to the exercises.


\begin{exercises}

\begin{exercise} Show that the range of $\sinh x$ is all real
numbers. (Hint: show that if $y=\sinh x$ then 
$\ds x =\ln (y+\sqrt{y^2+1})$.) 
\end{exercise}

\begin{exercise} Compute the following limits:
\begin{itemize} % BADBAD

\item{a.} $\ds  \lim_{x\to \infty } \cosh x$
\item{b.} $\ds  \lim_{x\to \infty } \sinh x$
\item{c.} $\ds  \lim_{x\to \infty } \tanh x$
\item{d.} $\ds  \lim_{x\to \infty } (\cosh x -\sinh x)$

\end{itemize}\end{exercise}

\begin{exercise} Show that the range of $\tanh x$ is $(-1,1)$. What
are the ranges of $\coth$, $\sech$, and $\csch$? 
(Use the fact that they are reciprocal functions.) 
\end{exercise}

\begin{exercise}
\label{exer:sinhid} 
Prove that for every $x,y\in\R$, $\sinh (x+y)
=\sinh x \cosh y + \cosh x \sinh y$. Obtain a similar identity for
$\sinh(x-y)$.
\end{exercise}

\begin{exercise}
\label{exer:coshid} 
Prove that for every $x,y\in\R$, $\cosh (x+y) =\cosh x \cosh y + \sinh x
  \sinh y$. Obtain a similar identity for $\cosh(x-y)$.
\end{exercise}

\begin{exercise} 
\label{exer:hyperbolic double angle formulas}
Use exercises~\xrefn{exer:sinhid} and~\xrefn{exer:coshid} 
to
show that $\sinh(2x)=2\sinh x \cosh x$ and $\ds \cosh(2x)=\cosh^2 x
+\sinh^2 x$ for every $x$.  Conclude also that $\ds (\cosh (2x) -1)/2 = \sinh
^2 x$.
\end{exercise}

\begin{exercise} Show that $\ds {d\over dx} (\tanh x) =\sech^2 x$. Compute
  the derivatives of the remaining hyperbolic functions as well.
\end{exercise}

\begin{exercise} What are the domains of the six inverse hyperbolic
functions?
\end{exercise}

\begin{exercise} Sketch the graphs of all six inverse hyperbolic
  functions. 
\end{exercise}

\end{exercises}
