\section{Inverse Trigonometric Functions}{}{}
\nobreak
The trigonometric functions frequently arise in problems, and often it
is necessary to invert the functions, for example, to find an angle with
a specified sine. Of course, there are many angles with the same sine,
so the sine function doesn't actually have an inverse that reliably
``undoes'' the sine function. If you know that $\sin x=0.5$, you can't
reverse this to discover $x$, that is, you can't solve for
$x$, as there are infinitely many angles with sine
$0.5$. Nevertheless, it is useful to have something like an inverse to
the sine, however imperfect. The usual approach is to pick out some
collection of angles that produce all possible values of the sine
exactly once. If we ``discard'' all other angles, the resulting
function does have a proper inverse.

The sine takes on all values between $-1$ and $1$ exactly once on the
interval $[-\pi/2,\pi/2]$. If we truncate the sine, keeping only the
interval $[-\pi/2,\pi/2]$, as shown in figure~\xrefn{fig:truncated
sine}, then this truncated sine has an inverse function. We call this
the inverse\index{inverse sine} sine or the arcsine\index{arcsine}, and
write $y=\arcsin(x)$.

% BADBAD
% \figure
% \vbox{\beginpicture
% \normalgraphs
% \sevenpoint
% \setcoordinatesystem units <1truecm,1truecm> point at 0 0
% \setplotarea x from -6.3 to 6.3, y from -1.1 to 1.1
% \axis left shiftedto x=0 ticks length <2pt> withvalues {$-1$} {$1$} / at -1 1 / /
% \axis bottom shiftedto y=0 ticks length <2pt> withvalues 
%   {$\pi/2$} {$\pi$} {$3\pi/2$} {$2\pi$} {$-\pi/2$} {$-\pi$} {$-3\pi/2$} {$-2\pi$} /
%   at 1.57 3.14 4.71 6.28 -1.57 -3.14 -4.71 -6.28 / /
% \setquadratic
% \plot -6.283 0.000 -6.158 0.125 -6.032 0.249 -5.906 0.368 -5.781 0.482 
% -5.655 0.588 -5.529 0.685 -5.404 0.771 -5.278 0.844 -5.152 0.905 
% -5.027 0.951 -4.901 0.982 -4.775 0.998 -4.650 0.998 -4.524 0.982 
% -4.398 0.951 -4.273 0.905 -4.147 0.844 -4.021 0.771 -3.896 0.685 
% -3.770 0.588 -3.644 0.482 -3.519 0.368 -3.393 0.249 -3.267 0.125 
% -3.142 0.000 -3.016 -0.125 -2.890 -0.249 -2.765 -0.368 -2.639 -0.482 
% -2.513 -0.588 -2.388 -0.685 -2.262 -0.771 -2.136 -0.844 -2.011 -0.905 
% -1.885 -0.951 -1.759 -0.982 -1.634 -0.998 -1.508 -0.998 -1.382 -0.982 
% -1.257 -0.951 -1.131 -0.905 -1.005 -0.844 -0.880 -0.771 -0.754 -0.685 
% -0.628 -0.588 -0.503 -0.482 -0.377 -0.368 -0.251 -0.249 -0.126 -0.125 
% 0.000 0.000 0.126 0.125 0.251 0.249 0.377 0.368 0.503 0.482 
% 0.628 0.588 0.754 0.685 0.880 0.771 1.005 0.844 1.131 0.905 
% 1.257 0.951 1.382 0.982 1.508 0.998 1.634 0.998 1.759 0.982 
% 1.885 0.951 2.011 0.905 2.136 0.844 2.262 0.771 2.388 0.685 
% 2.513 0.588 2.639 0.482 2.765 0.368 2.890 0.249 3.016 0.125 
% 3.142 0.000 3.267 -0.125 3.393 -0.249 3.519 -0.368 3.644 -0.482 
% 3.770 -0.588 3.896 -0.685 4.021 -0.771 4.147 -0.844 4.273 -0.905 
% 4.398 -0.951 4.524 -0.982 4.650 -0.998 4.775 -0.998 4.901 -0.982 
% 5.027 -0.951 5.152 -0.905 5.278 -0.844 5.404 -0.771 5.529 -0.685 
% 5.655 -0.588 5.781 -0.482 5.906 -0.368 6.032 -0.249 6.158 -0.125 
% 6.283 0.000 /
% \setcoordinatesystem units <1truecm,1truecm> point at 3 3.5
% \setplotarea x from -1.57 to 1.57, y from -1.1 to 1.1
% \axis left shiftedto x=0 ticks length <2pt> withvalues {$-1$} {$1$} / at -1 1 / /
% \axis bottom shiftedto y=0 ticks length <2pt> withvalues 
%   {$\pi/2$} {$-\pi/2$} /
%   at 1.57 -1.57 / /
% \setquadratic
% \plot -1.571 -1.000 -1.466 -0.995 -1.361 -0.978 -1.257 -0.951 -1.152 -0.914 
% -1.047 -0.866 -0.942 -0.809 -0.838 -0.743 -0.733 -0.669 -0.628 -0.588 
% -0.524 -0.500 -0.419 -0.407 -0.314 -0.309 -0.209 -0.208 -0.105 -0.105 
% 0.000 0.000 0.105 0.105 0.209 0.208 0.314 0.309 0.419 0.407 
% 0.524 0.500 0.628 0.588 0.733 0.669 0.838 0.743 0.942 0.809 
% 1.047 0.866 1.152 0.914 1.257 0.951 1.361 0.978 1.466 0.995 
% 1.571 1.000 /
% \setcoordinatesystem units <1truecm,1truecm> point at -3 3.5
% \setplotarea x from -1 to 1, y from -1.6 to 1.6
% \axis left shiftedto x=0 ticks length <2pt> withvalues {$-\pi/2$} {$\pi/2$} / 
%   at -1.57 1.57 / /
% \axis bottom shiftedto y=0 ticks length <2pt> withvalues 
%   {$-1$} {$1$} /
%   at -1 1 / /
% \setquadratic
% \plot -1.000 -1.571 -0.980 -1.370 -0.960 -1.287 -0.940 -1.223 -0.920 -1.168 
% -0.900 -1.120 -0.880 -1.076 -0.860 -1.035 -0.840 -0.997 -0.820 -0.961 
% -0.800 -0.927 -0.780 -0.895 -0.760 -0.863 -0.740 -0.833 -0.720 -0.804 
% -0.700 -0.775 -0.680 -0.748 -0.660 -0.721 -0.640 -0.694 -0.620 -0.669 
% -0.600 -0.644 -0.580 -0.619 -0.560 -0.594 -0.540 -0.570 -0.520 -0.547 
% -0.500 -0.524 -0.480 -0.501 -0.460 -0.478 -0.440 -0.456 -0.420 -0.433 
% -0.400 -0.412 -0.380 -0.390 -0.360 -0.368 -0.340 -0.347 -0.320 -0.326 
% -0.300 -0.305 -0.280 -0.284 -0.260 -0.263 -0.240 -0.242 -0.220 -0.222 
% -0.200 -0.201 -0.180 -0.181 -0.160 -0.161 -0.140 -0.140 -0.120 -0.120 
% -0.100 -0.100 -0.080 -0.080 -0.060 -0.060 -0.040 -0.040 -0.020 -0.020 
% 0.000 0.000 0.020 0.020 0.040 0.040 0.060 0.060 0.080 0.080 
% 0.100 0.100 0.120 0.120 0.140 0.140 0.160 0.161 0.180 0.181 
% 0.200 0.201 0.220 0.222 0.240 0.242 0.260 0.263 0.280 0.284 
% 0.300 0.305 0.320 0.326 0.340 0.347 0.360 0.368 0.380 0.390 
% 0.400 0.412 0.420 0.433 0.440 0.456 0.460 0.478 0.480 0.501 
% 0.500 0.524 0.520 0.547 0.540 0.570 0.560 0.594 0.580 0.619 
% 0.600 0.644 0.620 0.669 0.640 0.694 0.660 0.721 0.680 0.748 
% 0.700 0.775 0.720 0.804 0.740 0.833 0.760 0.863 0.780 0.895 
% 0.800 0.927 0.820 0.961 0.840 0.997 0.860 1.035 0.880 1.076 
% 0.900 1.120 0.920 1.168 0.940 1.223 0.960 1.287 0.980 1.370 
% 1.000 1.571 /
% \endpicture}
% \figrdef{fig:truncated sine}
% \endfigure{The sine, the truncated sine, the inverse sine.}

Recall that a function and its inverse undo each other in either
order, for example, $\ds (\root3\of x)^3=x$ and $\ds \root3\of{x^3}=x$. This
does not work with the sine and the ``inverse sine'' because the
inverse sine is the inverse of the truncated sine function, not the
real sine function. It is true that $\sin(\arcsin(x))=x$, that is, the
sine undoes the arcsine. It is not true that the arcsine undoes the
sine, for example, $\sin(5\pi/6)=1/2$ and $\arcsin(1/2)=\pi/6$, so
doing first the sine then the arcsine does not get us back where we
started. This is because $5\pi/6$ is not in the domain of the
truncated sine. If we start with an angle between $-\pi/2$ and $\pi/2$
then the arcsine does reverse the sine: $\sin(\pi/6)=1/2$ and
$\arcsin(1/2)=\pi/6$.

What is the derivative of the arcsine? Since this is an inverse
function, we can discover the derivative by using implicit
differentiation. Suppose $y=\arcsin(x)$. Then 
$$\sin(y)=\sin(\arcsin(x))=x.$$
Now taking the derivative of both sides, we get
\begin{align*}
y'\cos y &= 1 \\
y'={1\over \cos y} \\
\end{align*}
As we expect when using implicit differentiation, $y$ appears on the
right hand side here. We would certainly prefer to have $y'$ written
in terms of $x$, and as in the case of $\ln x$ we can actually do that
here. Since $\ds \sin^2y+\cos^2 y=1$, $\ds \cos^2y=1-\sin^2y=1-x^2$. So 
$\ds \cos y=\pm\sqrt{1-x^2\ }$, but which is it---plus or minus? It could
in general be either, but this isn't ``in general'': since
$y=\arcsin(x)$ we know that $-\pi/2\le y\le \pi/2$, and the cosine of
an angle in this interval is always positive. Thus
$\ds \cos y=\sqrt{1-x^2\ }$ and
$${d\over dx}\arcsin(x)={1\over \sqrt{1-x^2\ }}.$$
Note that this agrees with figure~\xrefn{fig:truncated sine}: the
graph of the arcsine has positive slope everywhere.

We can do something similar for the cosine. As with the sine, we must
first truncate the cosine so that it can be inverted, as shown in
figure~\xrefn{fig:truncated cosine}. Then we use
implicit differentiation to find that
$${d\over dx}\arccos(x)={-1\over \sqrt{1-x^2\ }}.$$ Note that the
truncated cosine uses a different interval than the truncated sine, so
that if $y=\arccos(x)$ we know that $0\le y\le \pi$. The computation
of the derivative of the arccosine\index{arccosine} is left as an exercise.

% BADBAD
% \figure
% \vbox{\beginpicture
% \normalgraphs
% \sevenpoint
% \setcoordinatesystem units <1truecm,1truecm> point at 0 0
% \setplotarea x from 0 to 3.14, y from -1.1 to 1.1
% \axis left shiftedto x=0 ticks length <2pt> withvalues {$-1$} {$1$} / at -1 1 / /
% \axis bottom shiftedto y=0 ticks length <2pt> withvalues 
%   {$\pi/2$} {$\pi$} /
%   at 1.57 3.14 / /
% \setquadratic
% \plot 0.000 1.000 0.105 0.995 0.209 0.978 0.314 0.951 0.419 0.914 
% 0.524 0.866 0.628 0.809 0.733 0.743 0.838 0.669 0.942 0.588 
% 1.047 0.500 1.152 0.407 1.257 0.309 1.361 0.208 1.466 0.105 
% 1.571 0.000 1.676 -0.105 1.780 -0.208 1.885 -0.309 1.990 -0.407 
% 2.094 -0.500 2.199 -0.588 2.304 -0.669 2.409 -0.743 2.513 -0.809 
% 2.618 -0.866 2.723 -0.914 2.827 -0.951 2.932 -0.978 3.037 -0.995 
% 3.142 -1.000  /
% \setcoordinatesystem units <1truecm,1truecm> point at -6 1
% \setplotarea x from -1 to 1, y from 0 to 3.2
% \axis left shiftedto x=0 ticks length <2pt> withvalues {$\pi/2$} {$\pi$} / 
%   at 1.57 3.14 / /
% \axis bottom shiftedto y=0 ticks length <2pt> withvalues 
%   {$-1$} {$1$} /
%   at -1 1 / /
% \setquadratic
% \plot -1.000 3.142 -0.950 2.824 -0.900 2.691 -0.850 2.587 -0.800 2.498 
% -0.750 2.419 -0.700 2.346 -0.650 2.278 -0.600 2.214 -0.550 2.153 
% -0.500 2.094 -0.450 2.038 -0.400 1.982 -0.350 1.928 -0.300 1.875 
% -0.250 1.823 -0.200 1.772 -0.150 1.721 -0.100 1.671 -0.050 1.621 
% 0.000 1.571 0.050 1.521 0.100 1.471 0.150 1.420 0.200 1.369 
% 0.250 1.318 0.300 1.266 0.350 1.213 0.400 1.159 0.450 1.104 
% 0.500 1.047 0.550 0.988 0.600 0.927 0.650 0.863 0.700 0.795 
% 0.750 0.723 0.800 0.644 0.850 0.555 0.900 0.451 0.950 0.318 
% 1.000 0.000  /
% \endpicture}
% \figrdef{fig:truncated cosine}
% \endfigure{The truncated cosine, the inverse cosine.}

Finally we look at the tangent; the other trigonometric functions also
have ``partial inverses'' but the sine, cosine and tangent are enough
for most purposes. The tangent, truncated tangent and inverse tangent
are shown in figure~\xrefn{fig:truncated tangent}; the derivative of
the arctangent is left as an exercise.

% BADBAD
% \figure
% \vbox{\beginpicture
% \normalgraphs
% \sevenpoint
% \setcoordinatesystem units <0.5truecm,0.2truecm> point at 8 0
% \setplotarea x from -4.7 to 4.7, y from -14.1 to 14.1
% \axis left shiftedto x=0 /
% \axis bottom shiftedto y=0 ticks length <2pt> withvalues 
%   {$-\pi/2$} {$\pi/2$} /
%   at -1.57 1.57 / /
% \setquadratic
% \plot -4.640 -13.790 -4.565 -6.737 -4.490 -4.424 -4.415 -3.264 -4.340 -2.561 
% -4.265 -2.085 -4.190 -1.738 -4.115 -1.471 -4.040 -1.257 -3.965 -1.080 
% -3.890 -0.929 -3.815 -0.799 -3.740 -0.682 -3.666 -0.578 -3.591 -0.482 
% -3.516 -0.392 -3.441 -0.308 -3.366 -0.228 -3.291 -0.150 -3.216 -0.074 
% -3.141 0.001 -3.066 0.076 -2.991 0.152 -2.916 0.230 -2.841 0.310 
% -2.766 0.394 -2.691 0.484 -2.616 0.580 -2.541 0.685 -2.466 0.801 
% -2.391 0.932 -2.316 1.083 -2.241 1.261 -2.166 1.476 -2.091 1.744 
% -2.016 2.094 -1.941 2.573 -1.866 3.283 -1.792 4.457 -1.717 6.812 
% -1.642 14.102 /
% \plot -1.500 -14.101 -1.425 -6.810 -1.350 -4.455 -1.275 -3.282 -1.200 -2.572 
% -1.125 -2.093 -1.050 -1.743 -0.975 -1.475 -0.900 -1.260 -0.825 -1.083 
% -0.750 -0.932 -0.675 -0.800 -0.600 -0.684 -0.525 -0.579 -0.450 -0.483 
% -0.375 -0.394 -0.300 -0.309 -0.225 -0.229 -0.150 -0.151 -0.075 -0.075 
% 0.000 0.000 0.075 0.075 0.150 0.151 0.225 0.229 0.300 0.309 
% 0.375 0.394 0.450 0.483 0.525 0.579 0.600 0.684 0.675 0.800 
% 0.750 0.932 0.825 1.083 0.900 1.260 0.975 1.475 1.050 1.743 
% 1.125 2.093 1.200 2.572 1.275 3.282 1.350 4.455 1.425 6.810 
% 1.500 14.101  /
% \plot 1.642 -14.102 1.717 -6.812 1.792 -4.457 1.866 -3.283 1.941 -2.573 
% 2.016 -2.094 2.091 -1.744 2.166 -1.476 2.241 -1.261 2.316 -1.083 
% 2.391 -0.932 2.466 -0.801 2.541 -0.685 2.616 -0.580 2.691 -0.484 
% 2.766 -0.394 2.841 -0.310 2.916 -0.230 2.991 -0.152 3.066 -0.076 
% 3.141 -0.001 3.216 0.074 3.291 0.150 3.366 0.228 3.441 0.308 
% 3.516 0.392 3.591 0.482 3.666 0.578 3.740 0.682 3.815 0.799 
% 3.890 0.929 3.965 1.080 4.040 1.257 4.115 1.471 4.190 1.738 
% 4.265 2.085 4.340 2.561 4.415 3.264 4.490 4.424 4.565 6.737 
% 4.640 13.790 /
% \setcoordinatesystem units <0.5truecm,0.2truecm> point at 0 0
% \setplotarea x from -1.57 to 1.57, y from -14.1 to 14.1
% \axis left shiftedto x=0 /
% \axis bottom shiftedto y=0 ticks length <2pt> withvalues 
%   {$-\pi/2$} {$\pi/2$} /
%   at -1.57 1.57 / /
% \setquadratic
% \plot -1.500 -14.101 -1.425 -6.810 -1.350 -4.455 -1.275 -3.282 -1.200 -2.572 
% -1.125 -2.093 -1.050 -1.743 -0.975 -1.475 -0.900 -1.260 -0.825 -1.083 
% -0.750 -0.932 -0.675 -0.800 -0.600 -0.684 -0.525 -0.579 -0.450 -0.483 
% -0.375 -0.394 -0.300 -0.309 -0.225 -0.229 -0.150 -0.151 -0.075 -0.075 
% 0.000 0.000 0.075 0.075 0.150 0.151 0.225 0.229 0.300 0.309 
% 0.375 0.394 0.450 0.483 0.525 0.579 0.600 0.684 0.675 0.800 
% 0.750 0.932 0.825 1.083 0.900 1.260 0.975 1.475 1.050 1.743 
% 1.125 2.093 1.200 2.572 1.275 3.282 1.350 4.455 1.425 6.810 
% 1.500 14.101  /
% \setcoordinatesystem units <0.3truecm,0.5truecm> point at -18 0
% \setplotarea x from -10 to 10, y from -1.57 to 1.57
% \axis left shiftedto x=0 ticks length <2pt> withvalues {$-\pi/2$} {$\pi/2$} /
%   at -1.57 1.57 / /
% \axis bottom shiftedto y=0 /
% \setquadratic
% \plot  -10.000 -1.471 -9.500 -1.466 -9.000 -1.460 -8.500 -1.454 -8.000 -1.446 
% -7.500 -1.438 -7.000 -1.429 -6.500 -1.418 -6.000 -1.406 -5.500 -1.391 
% -5.000 -1.373 -4.500 -1.352 -4.000 -1.326 -3.500 -1.292 -3.000 -1.249 
% -2.500 -1.190 -2.000 -1.107 -1.500 -0.983 -1.000 -0.785 -0.500 -0.464 
% 0.000 0.000 0.500 0.464 1.000 0.785 1.500 0.983 2.000 1.107 
% 2.500 1.190 3.000 1.249 3.500 1.292 4.000 1.326 4.500 1.352 
% 5.000 1.373 5.500 1.391 6.000 1.406 6.500 1.418 7.000 1.429 
% 7.500 1.438 8.000 1.446 8.500 1.454 9.000 1.460 9.500 1.466 
% 10.000 1.471 /
% \endpicture}
% \figrdef{fig:truncated tangent}
% \endfigure{The tangent, the truncated tangent, the inverse tangent.}

\begin{exercises}

\begin{exercise} Show that the derivative of $\arccos x$ is $\ds -{1\over
  \sqrt{1-x^2}}$. 
\end{exercise}

\begin{exercise} Show that the derivative of 
$\arctan x$ is $\ds {1\over 1+x^2}$. 
\end{exercise}

\begin{exercise} The inverse of $\cot$ is usually defined so that the
 range of arccot is $(0, \pi )$.  Sketch the graph of $y=\arccot
 x$. In the process you will make it clear what the domain of arccot
 is. Find the derivative of the arccotangent.
\begin{answer} $\ds {-1\over 1+x^2}$
\end{answer}\end{exercise}

\begin{exercise} Show that $\arccot x + \arctan x =\pi/2$.
\end{exercise}

\begin{exercise} Find the derivative of $\ds \arcsin(x^2)$.
\begin{answer} $\ds {2x\over\sqrt{1-x^4}}$
\end{answer}\end{exercise}

\begin{exercise} Find the derivative of $\ds \arctan(e^x)$.
\begin{answer} $\ds {e^x\over1+e^{2x}}$
\end{answer}\end{exercise}

\begin{exercise} Find the derivative of $\ds \arccos (\sin x^3 )$
\begin{answer} $\ds -3x^2\cos(x^3)/\sqrt{1-\sin^2(x^3)}$
\end{answer}\end{exercise}

\begin{exercise} Find the derivative of $\ds \ln( (\arcsin x )^2)$
\begin{answer} $\ds {2\over(\arcsin x)\sqrt{1-x^2}}$
\end{answer}\end{exercise}

\begin{exercise} Find the derivative of $\ds \arccos e^x$
\begin{answer} $\ds-e^x/\sqrt{1-e^{2x}}$
\end{answer}\end{exercise}

\begin{exercise} Find the derivative of $\arcsin x + \arccos x$
\begin{answer} $0$
\end{answer}\end{exercise}

\begin{exercise} Find the derivative of $\ds \log _5 (\arctan (x^x ) )$
\begin{answer} $\ds {(1+\ln x)x^x\over\ln5(1+x^{2x})\arctan(x^x)}$
\end{answer}\end{exercise}

\end{exercises}
