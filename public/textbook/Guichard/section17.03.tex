\section{First Order Linear Equations}{}{}
%\label{sec:first order homogeneous linear}
\nobreak
As you might guess, a first order linear differential equation has the form 
$\ds \dot y + p(t)y = f(t)$. Not only is this closely related in form
to the first order homogeneous linear equation, we can use what we
know about solving homogeneous equations to solve the general linear
equation. 

Suppose that $y_1(t)$ and $y_2(t)$ are solutions to 
$\ds \dot y + p(t)y = f(t)$. Let $\ds g(t)=y_1-y_2$. Then
$$\eqalign{
 g'(t)+p(t)g(t)&=y_1'-y_2'+p(t)(y_1-y_2) \\
&=(y_1'+p(t)y_1)-(y_2'+p(t)y_2) \\
&=f(t)-f(t)=0. \\}
$$
In other words, $\ds g(t)=y_1-y_2$ is a solution to the homogeneous
equation $\ds \dot y + p(t)y = 0$. Turning this around, any solution
to the linear equation $\ds \dot y + p(t)y = f(t)$, call it $y_1$, can
be written as $y_2+g(t)$, for some particular $y_2$ and some solution
$g(t)$ of the homogeneous equation $\ds \dot y + p(t)y = 0$. Since we
already know how to find all solutions of the homogeneous equation,
finding just one solution to the equation $\ds \dot y + p(t)y = f(t)$
will give us all of them.

How might we find that one particular solution to $\ds \dot y + p(t)y
= f(t)$? Again, it turns out that what we already know helps. We know
that the general solution to the homogeneous equation
$\ds \dot y + p(t)y = 0$ looks like $\ds Ae^{P(t)}$. We now make an
inspired guess: consider the function $\ds v(t)e^{P(t)}$, in which we
have replaced the constant parameter $A$ with the function
$v(t)$. This technique is called 
{\dfont variation of parameters\index{variation of parameters}}.
For
convenience write this as $s(t)=v(t)h(t)$ where $\ds h(t)=e^{P(t)}$ 
is a solution to the
homogeneous equation. Now let's compute a bit with $s(t)$:
$$\eqalign{
s'(t)+p(t)s(t)&=v(t)h'(t)+v'(t)h(t)+p(t)v(t)h(t) \\
&=v(t)(h'(t)+p(t)h(t)) + v'(t)h(t) \\
&=v'(t)h(t). \\}
$$
The last equality is true because $\ds h'(t)+p(t)h(t)=0$, since $h(t)$
is a solution to the homogeneous equation. We are hoping to find a
function $s(t)$ so that $\ds s'(t)+p(t)s(t)=f(t)$; we will have such a
function if we can arrange to have $\ds v'(t)h(t)=f(t)$, that is,
$\ds v'(t)=f(t)/h(t)$. But this is as easy (or hard) as finding an
anti-derivative of $\ds f(t)/h(t)$. Putting this all together, the
general solution to $\ds \dot y + p(t)y = f(t)$ is
$$v(t)h(t)+Ae^{P(t)} = v(t)e^{P(t)}+Ae^{P(t)}.$$

\begin{example} Find the solution of the initial value problem
$\ds \dot y+3y/t=t^2$, $y(1)=1/2$. First we find the general solution;
since we are interested in a solution with a given condition at $t=1$,
we may assume $t>0$.
We start by solving the homogeneous equation as usual; call the
solution $g$:
$$g=Ae^{-\int (3/t)\,dt}=Ae^{-3\ln t}=At^{-3}.$$
Then as in the discussion, $\ds h(t)=t^{-3}$ and
$\ds v'(t)=t^2/t^{-3}=t^5$, so $\ds v(t)=t^6/6$. Finally, we know that
every solution to the equation looks like
$$v(t)t^{-3}+At^{-3}={t^6\over6}t^{-3}+At^{-3}={t^3\over6}+At^{-3}.$$
Finally we substitute to find $A$:
$$\eqalign{
{1\over 2}&={(1)^3\over6}+A(1)^{-3}={1\over6}+A \\
A&={1\over 2}-{1\over6}={1\over3}. \\}
$$
The solution is then
$$y={t^3\over6}+{1\over3}t^{-3}.$$
\end{example}

\begin{exercises}

In problems 1--10, find the general solution of the equation.

\begin{exercise} $\ds\dot y +4y=8$
\begin{answer} $\ds y=Ae^{-4t}+2$
\end{answer}\end{exercise}

\begin{exercise} $\ds\dot y-2y=6$
\begin{answer} $\ds y=Ae^{2t}-3$
\end{answer}\end{exercise}

\begin{exercise} $\ds\dot y +ty=5t$
\begin{answer} $\ds y=Ae^{-(1/2)t^2}+5$
\end{answer}\end{exercise}

\begin{exercise} $\ds\dot y+e^ty=-2e^t$
\begin{answer} $\ds y=Ae^{-e^t}-2$
\end{answer}\end{exercise}

\begin{exercise} $\ds\dot y-y=t^2$
\begin{answer} $\ds y=Ae^{t}-t^2-2t-2$
\end{answer}\end{exercise}

\begin{exercise} $\ds 2\dot y +y=t$
\begin{answer} $\ds y=Ae^{-t/2}+t-2$
\end{answer}\end{exercise}

\begin{exercise} $\ds t\dot y -2y=1/t$, $t>0$
\begin{answer} $\ds y=At^2-{1\over3t}$
\end{answer}\end{exercise}

\begin{exercise} $\ds t\dot y+y=\sqrt{t}$, $t>0$
\begin{answer} $\ds y={c\over t}+{2\over3}\sqrt t$
\end{answer}\end{exercise}

\begin{exercise} $\ds\dot y\cos t+y\sin t=1$, $-\pi/2<t<\pi/2$
\begin{answer} $\ds y= A\cos t+\sin t$
\end{answer}\end{exercise}

\begin{exercise} $\ds\dot y + y\sec t=\tan t$, $-\pi/2<t<\pi/2$
\begin{answer} $\ds y= {A\over\sec t+\tan t}+1-{t\over\sec t+\tan t}$
\end{answer}\end{exercise}

\end{exercises}
