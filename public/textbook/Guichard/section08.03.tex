\section{Trigonometric Substitutions}{}{}
\nobreak
So far we have seen that it sometimes helps to replace a subexpression
of a function by a single variable. Occasionally it can help to
replace the original variable by something more complicated. This
seems like a ``reverse'' substitution, but it is really no different
in principle than ordinary substitution.

\begin{example}\relax
\label{exam:cos squared}
Evaluate $\ds\int \sqrt{1-x^2}\,dx$. Let $x=\sin u$ so 
$dx=\cos u\,du$. Then
$$
  \int \sqrt{1-x^2}\,dx=\int\sqrt{1-\sin^2 u}\cos u\,du=
  \int\sqrt{\cos^2 u}\cos u\,du.
$$
We would like to replace $\ds \sqrt{\cos^2 u}$ by $\cos u$, but this is
valid only if $\cos u $ is positive, since $\ds \sqrt{\cos^2 u}$ is
positive. Consider again the substitution $x=\sin u$. We could just as
well think of this as $u=\arcsin x$. If we do, then by the definition
of the arcsine, $-\pi/2\le u\le\pi/2$, so $\cos u\ge0$. Then we
continue:
$$\eqalign{
  \int\sqrt{\cos^2 u}\cos u\,du&=\int\cos^2u\,du=\int {1+\cos
    2u\over2}\,du = {u\over 2}+{\sin 2u\over4}+C \\
  &={\arcsin x\over2}+{\sin(2\arcsin x)\over4}+C. \\
}$$
This is a perfectly good answer, though the term
$\sin(2\arcsin x)$ is a bit unpleasant. It is possible to simplify
this. Using the identity $\sin 2x=2\sin x\cos x$, we can write
$\ds \sin 2u=2\sin u\cos u=2\sin(\arcsin x)\sqrt{1-\sin^2 u}=
2x\sqrt{1-\sin^2(\arcsin x)}=2x\sqrt{1-x^2}.$ Then the full
antiderivative is 
$$
  {\arcsin x\over2}+{2x\sqrt{1-x^2}\over4}=
  {\arcsin x\over2}+{x\sqrt{1-x^2}\over2}+C.
$$
\vskip-10pt\end{example}

This type of substitution is usually indicated when the function you
wish to integrate contains a polynomial expression that might allow
you to use the fundamental identity $\ds \sin^2x+\cos^2x=1$ in
one of three forms:
$$
  \cos^2 x=1-\sin^2x
  \qquad
  \sec^2x=1+\tan^2x
  \qquad
  \tan^2x=\sec^2x-1.
$$
If your function contains $\ds 1-x^2$, as in the example above, try
$x=\sin u$; if it contains $\ds 1+x^2$ try $x=\tan u$; and if it contains
$\ds x^2-1$, try $x=\sec u$. Sometimes you will need to try something a
bit different to handle constants other than one.

\begin{example}
Evaluate $\ds\int\sqrt{4-9x^2}\,dx$. We start by rewriting this so
that it looks more like the previous example:
$$
  \int\sqrt{4-9x^2}\,dx=\int\sqrt{4(1-(3x/2)^2)}\,dx
  =\int 2\sqrt{1-(3x/2)^2}\,dx.
$$
Now let $3x/2=\sin u$ so $(3/2)\,dx=\cos u \,du$ or
$dx=(2/3)\cos u\,du$. Then
$$\eqalign{
\int 2\sqrt{1-(3x/2)^2}\,dx&=\int 2\sqrt{1-\sin^2u}\,(2/3)\cos u\,du
={4\over3}\int \cos^2u\,du \\
&={4u\over 6}+{4\sin 2u\over12}+C \\
&={2\arcsin(3x/2)\over3}+{2\sin u \cos u\over3}+C \\
&={2\arcsin(3x/2)\over3}+{2\sin(\arcsin(3x/2))\cos(\arcsin(3x/2))\over3}+C \\
&={2\arcsin(3x/2)\over3}+{2(3x/2)\sqrt{1-(3x/2)^2}\over3}+C \\
&={2\arcsin(3x/2)\over3}+{x\sqrt{4-9x^2}\over2}+C, \\
}$$
using some of the work from example~\xrefn{exam:cos squared}.
\end{example}

\begin{example}
Evaluate $\ds\int\sqrt{1+x^2}\,dx$. Let $x=\tan u$, 
$\ds dx=\sec^2 u\,du$, so
$$
  \int\sqrt{1+x^2}\,dx=\int \sqrt{1+\tan^2 u}\sec^2u\,du=
  \int\sqrt{\sec^2u}\sec^2u\,du.
$$
Since $u=\arctan(x)$, $-\pi/2\le u\le\pi/2$ and $\sec u\ge0$, so 
$\ds \sqrt{\sec^2u}=\sec u$. Then
$$\int\sqrt{\sec^2u}\sec^2u\,du=\int \sec^3 u \,du.$$
In problems of this type, two integrals come up frequently:
$\ds \int\sec^3u\,du$ and $\int\sec u\,du$. Both have relatively nice
expressions but they are a bit tricky to discover. 

First we do $\int\sec u\,du$\index{integral!of $\sec x$}, which we
will need to compute $\ds \int\sec^3u\,du$:
$$\eqalign{
  \int\sec u\,du&=\int\sec u\,{\sec u +\tan u\over \sec u +\tan u}\,du \\
  &=\int{\sec^2 u +\sec u\tan u\over \sec u +\tan u}\,du. \\
}$$
Now let $w=\sec u +\tan u$, $\ds dw=\sec u \tan u + \sec^2u\,du$, exactly
the numerator of the function we are integrating. Thus
$$\eqalign{
  \int\sec u\,du=\int{\sec^2 u +\sec u\tan u\over \sec u +\tan u}\,du&=
  \int{1\over w}\,dw=\ln |w|+C \\
  &=\ln|\sec u +\tan u|+C. \\
}$$

Now for $\ds \int\sec^3 u\,du$\index{integral!of $\sec^3x$}:
$$\eqalign{
  \sec^3u&={\sec^3u\over2}+{\sec^3u\over2}={\sec^3u\over2}+{(\tan^2u+1)\sec
    u\over 2} \\
  &={\sec^3u\over2}+{\sec u \tan^2 u\over2}+{\sec u\over 2}=
  {\sec^3u+\sec u \tan^2u\over 2}+{\sec u\over 2}. \\
}$$
We already know how to integrate $\sec u$, so we just need the first
quotient. This is ``simply'' a matter of recognizing the product rule
in action:
$$\int \sec^3u+\sec u \tan^2u\,du=\sec u \tan u.$$

So putting these together we get
$$
  \int\sec^3u\,du={\sec u \tan u\over2}+{\ln|\sec u +\tan u|
  \over2}+C,
$$
and reverting to the original variable $x$:
$$\eqalign{
  \int\sqrt{1+x^2}\,dx&={\sec u \tan u\over2}+{\ln|\sec u +\tan
    u|\over2}+C \\
  &={\sec(\arctan x) \tan(\arctan x)\over2}
    +{\ln|\sec(\arctan x) +\tan(\arctan x)|\over2}+C \\
  &={ x\sqrt{1+x^2}\over2}
    +{\ln|\sqrt{1+x^2} +x|\over2}+C, \\
}$$
using $\tan(\arctan x)=x$ and 
$\ds \sec(\arctan x)=\sqrt{1+\tan^2(\arctan x)}=\sqrt{1+x^2}$.
\end{example}

\begin{exercises}

Find the antiderivatives.

\twocol

\begin{exercise} $\ds\int\csc x\,dx$
\begin{answer} $-\ln|\csc x+\cot x|+C$
\end{answer}\end{exercise}

\begin{exercise} $\ds\int\csc^3 x\,dx$
\begin{answer} $-\csc x\cot x/2-(1/2)\ln|\csc x+\cot x|+C$
\end{answer}\end{exercise}

\begin{exercise} $\ds\int\sqrt{x^2-1}\,dx$
\begin{answer} $\ds x\sqrt{x^2-1}/2-\ln|x+\sqrt{x^2-1}|/2+C$
\end{answer}\end{exercise}

\begin{exercise} $\ds\int\sqrt{9+4x^2}\,dx$
\begin{answer} $\ds x\sqrt{9+4x^2}/2+\hbox{$\ds(9/4)\ln|2x+\sqrt{9+4x^2}|+C$}$
\end{answer}\end{exercise}

\begin{exercise} $\ds\int x\sqrt{1-x^2}\,dx$
\begin{answer} $\ds -(1-x^2)^{3/2}/3+C$
\end{answer}\end{exercise}

\begin{exercise} $\ds\int x^2\sqrt{1-x^2}\,dx$
\begin{answer} $\arcsin(x)/8-\sin(4\arcsin x)/32+C$
\end{answer}\end{exercise}

\begin{exercise} $\ds\int{1\over\sqrt{1+x^2}}\,dx$
\begin{answer} $\ds \ln|x+\sqrt{1+x^2}|+C$
\end{answer}\end{exercise}

\begin{exercise} $\ds\int\sqrt{x^2+2x}\,dx$
\begin{answer} $\ds (x+1)\sqrt{x^2+2x}/2-\hbox{$\ds\ln|x+1+\sqrt{x^2+2x}|/2+C$}$
\end{answer}\end{exercise}

\begin{exercise} $\ds\int{1\over x^2(1+x^2)}\,dx$
\begin{answer} $-\arctan x - 1/x+C$
\end{answer}\end{exercise}

\begin{exercise} $\ds\int{x^2\over\sqrt{4-x^2}}\,dx$
\begin{answer} $\ds 2\arcsin(x/2)-x\sqrt{4-x^2}/2+C$
\end{answer}\end{exercise}

\begin{exercise} $\ds\int{\sqrt{x}\over\sqrt{1-x}}\,dx$
\begin{answer} $\ds \arcsin(\sqrt{x})-\sqrt{x}\sqrt{1-x}+C$
\end{answer}\end{exercise}

\begin{exercise} $\ds\int{x^3\over\sqrt{4x^2-1}}\,dx$
\begin{answer} $\ds (2x^2+1)\sqrt{4x^2-1}/24+C$
\end{answer}\end{exercise}

\endtwocol

\end{exercises}

