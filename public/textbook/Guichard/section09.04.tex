\section{Average value of a function}{}{}
\nobreak
The average of some finite set of values is a familiar concept. If,
for example, the class scores on a quiz are 10, 9, 10, 8, 7, 5, 7, 6,
3, 2, 7, 8, then the average score is the sum of these numbers divided
by the size of the class:
$$
  \hbox{average score} = {10+ 9+ 10+ 8+ 7+ 5+ 7+ 6+
  3+ 2+ 7+ 8\over 12}={82\over 12}\approx 6.83.
$$
Suppose that between $t=0$ and $t=1$ the speed of an object is
$\sin(\pi t)$. What is the average speed of the object over that time?
The question sounds as if it must make sense, yet we can't merely add
up some number of speeds and divide, since the speed is changing
continuously over the time interval.

To make sense of ``average'' in this context, we fall back on the idea
of approximation. Consider the speed of the object at tenth of a
second intervals: $\sin 0$, $\sin(0.1\pi)$, $\sin(0.2\pi)$,
$\sin(0.3\pi)$,\dots, $\sin(0.9\pi)$. The average speed ``should'' be
fairly close to the average of these ten speeds:
$$
  {1\over 10}\sum_{i=0}^9 \sin(\pi i/10)\approx {1\over 10}6.3=0.63.
$$
Of course, if we compute more speeds at more times, the average of
these speeds should be closer to the ``real'' average. If we take the
average of $n$ speeds at evenly spaced times, we get:
$${1\over n}\sum_{i=0}^{n-1} \sin(\pi i/n).$$
Here the individual times are $\ds t_i=i/n$, so rewriting slightly we have
$${1\over n}\sum_{i=0}^{n-1} \sin(\pi t_i).$$
This is almost the sort of sum that we know turns into an integral;
what's apparently missing is $\Delta t$---but in fact, $\Delta t=1/n$,
the length of each subinterval. So rewriting again:
$$
  \sum_{i=0}^{n-1} \sin(\pi t_i){1\over n}=
  \sum_{i=0}^{n-1} \sin(\pi t_i)\Delta t.
$$
Now this has exactly the right form, so that in the limit we get
$$
  \hbox{average speed} = \int_0^1 \sin(\pi t)\,dt=
  \left.-{\cos(\pi t)\over\pi}\right|_0^1=
  -{\cos(\pi)\over \pi}+{\cos(0)\over\pi}={2\over\pi}\approx
  0.6366\approx 0.64.
$$

It's not entirely obvious from this one simple example how to compute
such an average in general. Let's look at a somewhat more complicated
case. Suppose that the velocity of an object is $\ds 16
t^2+5$ feet per second. 
What is the average velocity between $t=1$ and $t=3$? Again we
set up an approximation to the average:
$${1\over n}\sum_{i=0}^{n-1} 16t_i^2+5,$$
where the values $\ds t_i$ are evenly spaced 
times between 1 and 3. Once again we are ``missing'' $\Delta t$, and
this time $1/n$ is not the correct value. What is $\Delta t$ in
general? It is the length of a subinterval; in this case we take the
interval $[1,3]$ and divide it into $n$ subintervals, so each
has length $(3-1)/n=2/n=\Delta t$. Now with the usual ``multiply and
divide by the same thing'' trick we can rewrite the sum:
$$
  {1\over n}\sum_{i=0}^{n-1} 16t_i^2+5=
  {1\over 3-1}\sum_{i=0}^{n-1} (16t_i^2+5){3-1\over n}=
  {1\over 2}\sum_{i=0}^{n-1} (16t_i^2+5){2\over n}=
  {1\over 2}\sum_{i=0}^{n-1} (16t_i^2+5)\Delta t.
$$
In the limit this becomes
$${1\over 2}\int_1^3 16t^2+5\,dt={1\over 2}{446\over 3}={223\over 3}.$$
Does this seem reasonable? Let's picture it: in
figure~\xrefn{fig:average speed} is the velocity function together
with the horizontal line $y=223/3\approx 74.3$. Certainly 
the height of the
horizontal line looks at least plausible for the average height of the
curve.

\figure
\vbox{\beginpicture
\normalgraphs
\ninepoint
\setcoordinatesystem units <3truecm,0.4truemm>
\setplotarea x from 0 to 3.1, y from 0 to 150
\axis bottom ticks numbered from 0 to 3 by 1 /
\axis left ticks numbered from 0 to 150 by 25 /
\putrule from 1 74.3 to 3 74.3
\setquadratic
\plot 0.000 5.000 0.150 5.360 0.300 6.440 0.450 8.240 0.600 10.760 
0.750 14.000 0.900 17.960 1.050 22.640 1.200 28.040 1.350 34.160 
1.500 41.000 1.650 48.560 1.800 56.840 1.950 65.840 2.100 75.560 
2.250 86.000 2.400 97.160 2.550 109.040 2.700 121.640 2.850 134.960 
3.000 149.000 /
\setdashes
\putrule from 1 0 to 1 74.3
\putrule from 3 0 to 3 74.3
\endpicture}
\figrdef{fig:average speed}
\endfigure{Average velocity.}

Here's another way to interpret ``average'' that may make our
computation appear even more reasonable. The object of our example
goes a certain distance between $t=1$ and $t=3$. If instead the object
were to travel at the average speed over the same time, it should go
the same distance. At an average speed of $223/3$ feet per second for
two seconds the object would go $446/3$ feet. How far does it actually
go? We know how to compute this:
$$\int_1^3 v(t)\,dt = \int_1^3 16t^2+5\,dt={446\over 3}.$$
So now we see that another interpretation of the calculation
$${1\over 2}\int_1^3 16t^2+5\,dt={1\over 2}{446\over 3}={223\over 3}$$
is: total distance traveled divided by the time in transit, namely,
the usual interpretation of average speed.

In the case of speed, or more properly velocity, we can always
interpret ``average'' as total (net) distance divided by time. But in
the case of a different sort of quantity this interpretation does not
obviously apply, while the approximation approach always does. We might
interpret the same problem geometrically: what is the average height
of $16x^2+5$ on the interval $[1,3]$? We approximate this in exactly
the same way, by adding up many sample heights and dividing by the
number of samples. In the limit we get the same result:
$$
  \lim_{n\to\infty}{1\over n}\sum_{i=0}^{n-1} 16x_i^2+5=
  {1\over 2}\int_1^3 16x^2+5\,dx={1\over 2}{446\over 3}={223\over 3}.
$$
We can interpret this result in a slightly different way. The area
under $y=16x^2+5$ above $[1,3]$ is
$$\int_1^3 16t^2+5\,dt={446\over 3}.$$ 
The area under $y=223/3$ over the same interval $[1,3]$ is simply the
area of a rectangle that is 2 by $223/3$ with area $446/3$. So the
average height of a function is the height of the horizontal line that
produces the same area over the given interval.

\begin{exercises}

\begin{exercise} Find the average height of $\cos x$ over the intervals
$[0,\pi/2]$, $[-\pi/2,\pi/2]$, and $[0,2\pi]$.
\begin{answer} $2/\pi$; $2/\pi$; $0$
\end{answer}\end{exercise}

\begin{exercise} Find the average height of $\ds x^2$ over the interval
$[-2,2]$.
\begin{answer} $4/3$
\end{answer}\end{exercise}

\begin{exercise} Find the average height of $\ds 1/x^2$ over the interval
$[1,A]$.
\begin{answer} $1/A$
\end{answer}\end{exercise}

\begin{exercise} Find the average height of $\ds \sqrt{1-x^2}$ over the interval
$[-1,1]$.
\begin{answer} $\pi/4$
\end{answer}\end{exercise}

\begin{exercise} An object moves with velocity $\ds v(t)=-t^2+1$ feet per second
between $t=0$ and $t=2$. Find the average velocity and the average
speed of the object between $t=0$ and $t=2$.
\begin{answer} $-1/3$, $1$
\end{answer}\end{exercise}

\begin{exercise} The observation deck on the 102nd floor of the Empire State Building
is 1,224 feet above the ground. If a steel ball is dropped from the
observation deck its velocity at time $t$ is approximately $v(t)=-32t$
feet per second. Find the average speed between the time it is dropped
and the time it hits the ground, and find its speed when it hits the
ground.
\begin{answer} $\ds -4\sqrt{1224}$ ft/s; $\ds -8\sqrt{1224}$ ft/s
\end{answer}\end{exercise}

\end{exercises}

