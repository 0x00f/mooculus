\section{Green's Theorem}{}{}
\nobreak
We now come to the first of three important theorems that extend the
Fundamental Theorem of Calculus to higher dimensions. (The Fundamental
Theorem of Line Integrals has already done this in one way, but in
that case we were still dealing with an essentially one-dimensional
integral.) They all share with the Fundamental Theorem the following
rather vague description: {\it To compute a certain sort of integral over a
region, we may do a computation on the boundary of the region
that involves one fewer integrations.}

Note that this does indeed describe the Fundamental Theorem of
Calculus and the Fundamental Theorem of Line Integrals: to
compute a single integral over an interval, we do a computation on the
boundary (the endpoints) that involves one fewer integrations, namely,
no integrations at all.

\begin{theorem} (Green's Theorem) If the vector field ${\bf F}=\langle
P,Q\rangle$ and the region $D$ are sufficiently nice, and if $C$ is
the boundary of $D$ ($C$ is a closed curve), then
$$\dint{D} {\partial Q\over\partial x}
-{\partial P\over\partial y} \,dA = \int_C P\,dx +Q\,dy ,$$
provided the integration on the right is done counter-clockwise around
$C$.\end{theorem} 
\label{thm:greens theorem}

To indicate that an integral $\ds\int_C$ is being done over a closed
curve in the counter-clockwise direction, we usually write
$\ds\oint_C$. We also use the notation $\partial D$ to mean the
boundary of $D$ {\dfont oriented\index{oriented curve}\/} in the
counterclockwise direction. With this notation,
$\ds\oint_C=\int_{\partial D}$.

We already know one case, not particularly interesting, in which this
theorem is true: If $\bf F$ is conservative, we know that the integral
$\ds\oint_C {\bf F}\cdot d{\bf r}=0$, because any integral of a
conservative vector field around a closed curve is zero. We also know
in this case that $\partial P/\partial y=\partial Q/\partial x$, so
the double integral  in the theorem is simply the integral of the zero
function, namely, 0. So in the case that $\bf F$ is conservative, the
theorem says simply that $0=0$. 

\begin{example} We illustrate the theorem by computing both sides of
$$\int_{\partial D} x^4\,dx + xy\,dy=\dint{D} y-0\,dA,$$
where $D$ is the triangular region with corners $(0,0)$, $(1,0)$,
$(0,1)$. 

Starting with the double integral:
$$\dint{D} y-0\,dA=\int_0^1\int_0^{1-x} y\,dy\,dx=
\int_0^1
{(1-x)^2\over2}\,dx=\left.-{(1-x)^3\over6}\right|_0^1={1\over6}.$$

There is no single formula to describe the boundary of $D$, so to
compute the left side directly we need to compute three separate
integrals corresponding to the three sides of the triangle, and each
of these integrals we break into two integrals, the ``$dx$'' part and
the ``$dy$'' part.
The three sides are described by $y=0$, $y=1-x$, and $x=0$. The
integrals are then
$$\eqalign{
\int_{\partial D}\!\!\! x^4\,dx + xy\,dy&=
\int_0^1 x^4\,dx+\int_0^0 0\,dy+\int_1^0 x^4\,dx+\int_0^1 (1-y)y\,dy+
\int_0^0 0\,dx+\int_1^0 0\,dy \\
&={1\over5}+0-{1\over5}+{1\over6}+0+0={1\over6}. \\}
$$

Alternately, we could describe the three sides in vector form as
$\langle t,0\rangle$, $\langle 1-t,t\rangle$, and $\langle
0,1-t\rangle$. Note that in each case, as $t$ ranges from 0 to 1, we
follow the corresponding side in the correct direction. Now
$$\eqalign{
\int_{\partial D} x^4\,dx + xy\,dy&=
\int_0^1 t^4 + t\cdot 0\,dt + \int_0^1 -(1-t)^4 + (1-t)t\,dt
+\int_0^1 0 + 0\,dt \\
&=\int_0^1 t^4\,dt + \int_0^1 -(1-t)^4 + (1-t)t\,dt
={1\over6}. \\
}$$ 
% The three integrals on the right in the first line deserve a
% bit of explanation. The first arises from
% $$\int_C {\bf F}\cdot d{\bf r},$$
% where $C$ is given by ${\bf r}=\langle t,0\rangle$, that is, by
% $x=t$ and $y=0$. This implies that $dx=dt$ and $dy=0\cdot dt$. 
% Now substituting into $x^4\,dx + xy\,dy$ gives
% $t^4\,dt+t\cdot0\cdot0\,dt$. In the second integral we have 
% ${\bf r}=\langle 1-t,t\rangle$, so $x=1-t$, $y=t$,
% $dx=-dt$, and $dy=dt$. Substitution gives the integral shown. The
% third integral arises in the same way.
%
%Now computing the easy integrals gives
%$(1/5)+(-1/5+1/2-1/3)+(0)=1/6$ as before.
\end{example}

In this case, none of the integrations are difficult, but the second
approach is somewhat tedious because of the necessity to set up three
different integrals. In different circumstances, either of the
integrals, the single or the double, might be easier to
compute. Sometimes it is worthwhile to turn a single integral into the
corresponding double integral, sometimes exactly the opposite approach
is best.

Here is a clever use of Green's Theorem: We know that areas can be
computed using double integrals, namely,
$$\dint{D} 1\,dA$$
computes the area of region $D$. If we can find $P$ and $Q$ so that
$\partial Q/\partial x-\partial P/\partial y=1$, then the area is also
$$\int_{\partial D} P\,dx+Q\,dy.$$
It is quite easy to do this: $P=0,Q=x$ works, as do
$P=-y, Q=0$ and $P=-y/2,Q=x/2$. 

\begin{example} An ellipse centered at the origin, with its two principal axes
aligned with the $x$ and $y$ axes, is given by
$${x^2\over a^2}+{y^2\over b^2}=1.$$ We find the area of the interior
of the ellipse
via Green's theorem. To do this we need a vector equation for the
boundary; one such equation is $\langle a\cos t,b\sin t\rangle$, as
$t$ ranges from 0 to $2\pi$. We
can easily verify this by substitution:
$${x^2\over a^2}+{y^2\over b^2}=
{a^2\cos^2 t\over a^2}+{b^2\sin^2t\over b^2}=
\cos^2t+\sin^2t=1.$$
Let's consider the three possibilities for $P$ and $Q$ above:
Using 0 and $x$ gives
$$\oint_C 0\,dx+x\,dy=\int_0^{2\pi} a\cos(t)b\cos(t)\,dt=
\int_0^{2\pi} ab\cos^2(t)\,dt.$$
Using $-y$ and 0 gives
$$\oint_C -y\,dx+0\,dy=\int_0^{2\pi} -b\sin(t)(-a\sin(t))\,dt=
\int_0^{2\pi} ab\sin^2(t)\,dt.$$
Finally, using $-y/2$ and $x/2$ gives
$$\eqalign{
\oint_C -{y\over2}\,dx+{x\over2}\,dy&=
\int_0^{2\pi} -{b\sin(t)\over2}(-a\sin(t))\,dt
+{a\cos(t)\over2}(b\cos(t))\,dt \\
&=\int_0^{2\pi} {ab\sin^2t\over2}+{ab\cos^2t\over2}\,dt=
\int_0^{2\pi} {ab\over2}\,dt=\pi ab. \\}$$
The first two integrals are not particularly difficult, but the third
is very easy, though the choice of $P$ and $Q$ seems more complicated.
\end{example}

\figure
\vbox{\beginpicture
\normalgraphs
\ninepoint
\setcoordinatesystem units <1truecm,1truecm>
\setplotarea x from -3.1 to 3.1, y from -2.1 to 2.1
\axis left shiftedto x=0 /
\axis bottom shiftedto y=0 /
\put {$(0,b)$} [bl] <2pt,2pt> at 0 2
\put {$(a,0)$} [tl] <2pt,-2pt> at 3 0
\multiput {\eightpoint$\bullet$} at 0 2 3 0 /
\ellipticalarc axes ratio 3:2 360 degrees from 3 0 center at 0 0
\endpicture}
\figrdef{fig:standard ellipse}
\endfigure{A ``standard'' ellipse, ${x^2\over a^2}+{y^2\over b^2}=1$.}

\goodbreak
\noindent{\begin{proof}font Proof of 
Green's Theorem.}\kern1pc\bgroup

We cannot here prove Green's Theorem in general, but we can do a
special case. We seek to prove that 
$$\oint_C P\,dx +Q\,dy = \dint{D} {\partial Q\over\partial x}
-{\partial P\over\partial y} \,dA.$$
It is sufficient to show that
$$\oint_C P\,dx=\dint{D}-{\partial P\over\partial y} \,dA\qquad\hbox{and}
\qquad\oint_C Q\,dy=\dint{D} {\partial Q\over\partial x}\,dA,$$
which we can do if we can compute the double integral in both possible
ways, that is, using $dA=dy\,dx$ and $dA=dx\,dy$.

For the first equation, we start with
$$\dint{D}{\partial P\over\partial y}\,dA=
\int_a^b\int_{g_1(x)}^{g_2(x)} {\partial P\over \partial y}\,dy\,dx=
\int_a^b P(x,g_2(x))-P(x,g_1(x))\,dx.$$
Here we have simply used the ordinary Fundamental Theorem of Calculus,
since for the inner integral we are integrating a derivative with
respect to $y$: an antiderivative of $\partial P/\partial y$ with
respect to $y$ is simply $P(x,y)$, and then we substitute $g_1$ and
$g_2$ for $y$ and subtract.

Now we need to manipulate $\oint_C P\,dx$. The boundary of region $D$
consists of 4 parts, given by the equations $y=g_1(x)$, $x=b$,
$y=g_2(x)$, and $x=a$. On the portions $x=b$ and $x=a$, $dx=0\,dt$, so
the corresponding integrals are zero. For the other two portions, we
use the parametric forms $x=t$, $y=g_1(t)$, $a\le t\le b$, and
$x=t$, $y=g_2(t)$, letting $t$ range from $b$ to $a$, since we are
integrating counter-clockwise around the boundary.
The resulting integrals give us
$$\eqalign{
\oint_C P\,dx = \int_a^b P(t,g_1(t))\,dt+\int_b^a P(t,g_2(t))\,dt
&=\int_a^b P(t,g_1(t))\,dt-\int_a^b P(t,g_2(t))\,dt \\
&=\int_a^b P(t,g_1(t))-P(t,g_2(t))\,dt \\
}$$
which is the result of the double integral times $-1$, as desired.

The equation involving $Q$ is essentially the same, and left as an
exercise.
\end{proof}

\begin{exercises}

\begin{exercise} Compute $\ds\int_{\partial D} 2y\,dx + 3x\,dy$, 
where $D$ is described by $0\le x\le1$, $0\le y\le 1$.
\begin{answer} $1$
\end{answer}\end{exercise}

\begin{exercise} Compute $\ds\int_{\partial D} xy\,dx + xy\,dy$, 
where $D$ is described by $0\le x\le1$, $0\le y\le 1$.
\begin{answer} $0$
\end{answer}\end{exercise}

\begin{exercise} Compute $\ds\int_{\partial D} e^{2x+3y}\,dx + e^{xy}\,dy$, 
where $D$ is described by $-2\le x\le 2$, $-1\le y\le 1$.
\begin{answer} $1/(2e)-1/(2e^7)+e/2-e^7/2$
\end{answer}\end{exercise}

\begin{exercise} Compute $\ds\int_{\partial D} y\cos x\,dx + y\sin x\,dy$, 
where $D$ is described by $0\le x\le \pi/2$, $1\le y\le 2$.
\begin{answer} $1/2$
\end{answer}\end{exercise}

\begin{exercise} Compute $\ds\int_{\partial D} x^2y\,dx + xy^2\,dy$, 
where $D$ is described by $0\le x\le 1$, $0\le y\le x$.
\begin{answer} $-1/6$
\end{answer}\end{exercise}

\begin{exercise} Compute $\ds\int_{\partial D} x\sqrt{y}\,dx + \sqrt{x+y}\,dy$, 
where $D$ is described by $1\le x\le 2$, $2x\le y\le 4$.
\begin{answer} $(2\sqrt3-10\sqrt5+8\sqrt6)/3-2\sqrt2/5+1/5$
\end{answer}\end{exercise}

\begin{exercise} Compute $\ds\int_{\partial D} (x/y)\,dx + (2+3x)\,dy$, 
where $D$ is described by $1\le x\le 2$, $1\le y\le x^2$.
\begin{answer} $11/2-\ln(2)$
\end{answer}\end{exercise}

\begin{exercise} Compute $\ds\int_{\partial D} \sin y\,dx + \sin x\,dy$, 
where $D$ is described by $0\le x\le \pi/2$, $x\le y\le \pi/2$.
\begin{answer} $2-\pi/2$
\end{answer}\end{exercise}

\begin{exercise} Compute $\ds\int_{\partial D} x\ln y\,dx$,
where $D$ is described by $1\le x\le 2$, $\ds e^x\le y\le e^{x^2}$.
\begin{answer} $-17/12$
\end{answer}\end{exercise}

\begin{exercise} Compute $\ds\int_{\partial D} \sqrt{1+x^2}\,dy$, 
where $D$ is described by $-1\le x\le 1$, $x^2\le y\le 1$.
\begin{answer} $0$
\end{answer}\end{exercise}

\begin{exercise} Compute $\ds\int_{\partial D} x^2y\,dx - xy^2\,dy$, 
where $D$ is described by $x^2+y^2\le 1$.
\begin{answer} $-\pi/2$
\end{answer}\end{exercise}

\begin{exercise} Compute $\ds\int_{\partial D} y^3\,dx + 2x^3\,dy$, 
where $D$ is described by $x^2+y^2\le 4$.
\begin{answer} $12\pi$
\end{answer}\end{exercise}

% Albert

\begin{exercise} Evaluate $\ds\oint_C (y-\sin(x))\,dx + \cos(x) \, dy$,
where $C$ is the boundary of the triangle with vertices $(0,0)$,
$(1,0)$, and $(1,2)$ oriented counter-clockwise.
\begin{answer} $2\cos(1)-2\sin(1)-1$
\end{answer}\end{exercise}
%/Albert
\begin{exercise} Finish our proof of Green's Theorem by showing that
$\ds\oint_C Q\,dy=\dint{D} {\partial Q\over\partial x}\,dA$.

\end{exercises}
