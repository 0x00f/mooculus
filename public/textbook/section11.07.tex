\section{The Ratio and Root Tests}{}{}
\nobreak
Does the series $\ds\sum_{n=0}^\infty {n^5\over 5^n}$ converge? It is
possible, but a bit unpleasant, to approach this with the integral
test or the comparison test, but there is an easier way. Consider what
happens as we move from one term to the next in this series:
$$\cdots+{n^5\over5^n}+{(n+1)^5\over 5^{n+1}}+\cdots$$
The denominator goes up by a factor of 5, $\ds 5^{n+1}=5\cdot5^n$, but the
numerator goes up by much less: $\ds (n+1)^5=n^5+5n^4+10n^3+10n^2+5n+1$,
which is much less than $\ds 5n^5$ when $n$ is large, because $\ds 5n^4$ is
much less than $\ds n^5$. So we might guess that in the long run it begins
to look as if each term is $1/5$ of the previous term. We have seen
series that behave like this:
$$\sum_{n=0}^\infty {1\over 5^n} = {5\over4},$$
a geometric series. So we might try comparing the given series to some
variation of this geometric series. This is possible, but a bit
messy. We can in effect do the same thing, but bypass most of the
unpleasant work.

The key is to notice that
$$
  \lim_{n\to\infty} {a_{n+1}\over a_n}=
  \lim_{n\to\infty} {(n+1)^5\over 5^{n+1}}{5^n\over n^5}=
  \lim_{n\to\infty} {(n+1)^5\over n^5}{1\over 5}=1\cdot {1\over5}
    ={1\over 5}.
$$ 
This is really just what we noticed above, done a bit more officially:
in the long run, each term is one fifth of the previous term. Now pick
some number between $1/5$ and $1$, say $1/2$. Because
$$\lim_{n\to\infty} {a_{n+1}\over a_n}={1\over5},$$
then when $n$ is big enough, say $n\ge N$ for some $N$, 
$$
  {a_{n+1}\over a_n}<{1\over2} \hbox{\quad and\quad} a_{n+1}<{a_n\over2}.
$$
So $\ds a_{N+1}< a_N/2$, $\ds a_{N+2}<a_{N+1}/2<a_N/4$,
$\ds a_{N+3}<a_{N+2}/2< a_{N+1}/4<a_N/8$, and so on. The general form is
$\ds a_{N+k}< a_N/2^k$. So if we look at the series
$$
  \sum_{k=0}^\infty a_{N+k}=
  a_N+a_{N+1}+a_{N+2}+a_{N+3}+\cdots+a_{N+k}+\cdots,
$$
its terms are less than or equal to the terms of the sequence
$$
  a_N+{a_N\over2}+{a_N\over4}+{a_N\over8}+\cdots+{a_N\over2^k}+\cdots=
  \sum_{k=0}^\infty {a_N\over 2^k} = 2a_N.
$$
So by the comparison test, $\ds\sum_{k=0}^\infty a_{N+k}$ converges,
and this means that $\ds\sum_{n=0}^\infty a_{n}$ converges, since
we've just added the fixed number $\ds a_0+a_1+\cdots+a_{N-1}$.

Under what circumstances could we do this? What was crucial was that
the limit of $\ds a_{n+1}/a_n$, say $L$, was less than 1 so that we could pick a
value $r$ so that $L<r<1$. The fact that $L<r$ ($1/5<1/2$ in our
example) means that we can compare the series $\sum a_n$ to $\sum
r^n$, and the fact that $r<1$ guarantees that $\sum r^n$
converges. That's really all that is required to make the argument
work. We also made use of the fact that the terms of the series were
positive; in general we simply consider the absolute values of the
terms and we end up testing for absolute convergence.

\begin{theorem} (The Ratio Test)
Suppose that $\ds\lim_{n\to \infty} |a_{n+1}/a_n|=L$. If $L<1$
the series $\sum a_n$ converges absolutely, 
if $L>1$ the series diverges, and if
$L=1$ this test gives no information.
\begin{proof}
The example above essentially proves the first part of this, if we
simply replace $1/5$ by $L$ and $1/2$ by $r$. 
Suppose that $L>1$, and pick $r$ so that $1<r<L$.
Then for $n\ge N$, for some $N$,
$${|a_{n+1}|\over |a_n|} > r \hbox{\quad and\quad} |a_{n+1}| > r|a_n|.$$
This implies that $\ds |a_{N+k}|>r^k|a_N|$, but since $r>1$ this means
that $\ds\lim_{k\to\infty}|a_{N+k}|\not=0$, which means also that
$\ds\lim_{n\to\infty}a_n\not=0$. By the divergence test, the series
diverges. 

To see that we get no information when $L=1$, we need to exhibit two
series with $L=1$, one that converges and one that diverges. It is
easy to see that $\sum 1/n^2$ and $\sum 1/n$ do the job.
\end{proof}

\begin{example} The ratio test is particularly useful for series involving
the factorial function. Consider $\ds\sum_{n=0}^\infty  5^n/n!$. 
$$
  \lim_{n\to\infty} {5^{n+1}\over (n+1)!}{n!\over 5^n}=
  \lim_{n\to\infty} {5^{n+1}\over 5^n}{n!\over (n+1)!}=
  \lim_{n\to\infty} {5}{1\over (n+1)}=0.
$$
Since $0<1$, the series converges.
\end{example}

A similar argument, which we will not do, justifies a similar test
that is occasionally easier to apply. 

\begin{theorem} (The Root Test) 
\label{thm:root test}
Suppose that $\ds\lim_{n\to \infty} |a_n|^{1/n}=L$. If $L<1$
the series $\sum a_n$ converges absolutely, 
if $L>1$ the series diverges, and if
$L=1$ this test gives no information.
\end{proof}

The proof of the root test is actually easier than that of the ratio
test, and is a good exercise.

\begin{example} Analyze $\ds\sum_{n=0}^\infty {5^n\over n^n}$.
\ssk
The ratio test turns out to be a bit difficult on this series (try
it). Using the root test:
$$
  \lim_{n\to\infty} \left({5^n\over n^n}\right)^{1/n}=
  \lim_{n\to\infty} {(5^n)^{1/n}\over (n^n)^{1/n}}=
  \lim_{n\to\infty} {5\over n}=0.
$$
Since $0<1$, the series converges.
\end{example}

The root test is frequently useful when $n$ appears as an exponent in
the general term of the series.

\begin{exercises}

\begin{exercise} Compute $\ds\lim_{n\to\infty} |a_{n+1}/a_n|$ for the series
$\sum 1/n^2$.

\begin{exercise} Compute $\ds\lim_{n\to\infty} |a_{n+1}/a_n|$ for the series
$\sum 1/n$.

\begin{exercise} Compute $\ds\lim_{n\to\infty} |a_n|^{1/n}$ for the series
$\sum 1/n^2$.

\begin{exercise} Compute $\ds\lim_{n\to\infty} |a_n|^{1/n}$ for the series
$\sum 1/n$.

\msk\noindent Determine whether the series converge.
\twocol

\begin{exercise} $\ds\sum_{n=0}^\infty (-1)^{n}{3^n\over 5^n}$
\begin{answer} converges
\end{answer}\end{exercise}

\begin{exercise} $\ds\sum_{n=1}^\infty {n!\over n^n}$
\begin{answer} converges
\end{answer}\end{exercise}

\begin{exercise} $\ds\sum_{n=1}^\infty {n^5\over n^n}$
\begin{answer} converges
\end{answer}\end{exercise}

\begin{exercise} $\ds\sum_{n=1}^\infty {(n!)^2\over n^n}$
\begin{answer} diverges
\end{answer}\end{exercise}

\endtwocol

\ssk
\begin{exercise} Prove theorem \xrefn{thm:root test}, the root test.

\end{exercises}

