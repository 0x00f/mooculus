\chapter{Curve Sketching}



Whether we are interested in a function as a purely mathematical
object or in connection with some application to the real world, it is
often useful to know what the graph of the function looks like. We can
obtain a good picture of the graph using certain crucial information
provided by derivatives of the function and certain limits.

\section{Extrema}

Local \textit{extrema} on a function are points on the graph where the
$y$ coordinate is larger (or smaller) than all other $y$ coordinates
on the graph at points ``close to'' $(x,y)$. 

\begin{definition}\hfil
\begin{enumerate}
\item A point $(x,f(x))$ is a \textbf{local maximum}\index{local
  maximum} if there is an interval $a<x<b$ with $f(x)\ge f(z)$ for
  every $z$ in $(a,b)$.
\item A point $(x,f(x))$ is a \textbf{local minimum}\index{local
  minimum} if there is an interval $a<x<b$ with $f(x)\le f(z)$ for
  every $z$ in $(a,b)$.
\end{enumerate}
A \textbf{local extremum}\index{local extremum} is either a local
maximum or a local minimum.
\end{definition}

Local maximum and minimum points are quite distinctive on the graph of
a function, and are therefore useful in understanding the shape of the
graph. In many applied problems we want to find the largest or
smallest value that a function achieves (for example, we might want
to find the minimum cost at which some task can be performed) and so
identifying maximum and minimum points will be useful for applied
problems as well.

If $(x,f(x))$ is a point where $f(x)$ reaches a local maximum or minimum,
and if the derivative of $f$ exists at $x$, then the graph has a
tangent line and the tangent line must be horizontal. This is
important enough to state as a theorem, though we will not prove it.

\begin{mainTheorem}[Fermat's Theorem]\index{Fermat's Theorem}
If $f(x)$ has a local extremum at $x=a$ and $f(x)$ is differentiable at
$a$, then $f'(a)=0$.
\end{mainTheorem}
\begin{marginfigure}[0in]
\begin{tikzpicture}
	\begin{axis}[
            domain=-3:3,
            ymax=3,
            ymin=-3,
            %samples=100,
            axis lines =middle, xlabel=$x$, ylabel=$y$,
            every axis y label/.style={at=(current axis.above origin),anchor=south},
            every axis x label/.style={at=(current axis.right of origin),anchor=west}
          ]
          \addplot [dashed, textColor, smooth] plot coordinates {(.451,0) (.451,.631)}; %% {.451};
          \addplot [dashed, textColor, smooth] plot coordinates {(2.215,-2.113) (2.215,0)}; %% axis{2.215};
          \addplot [very thick, penColor2, smooth] {3*x^2-8*x+3};
          \addplot [very thick, penColor, smooth] {x^3-4*x^2+3*x};
          \node at (axis cs:2.5,-2) [anchor=west] {\color{penColor}$f(x)$};  
          \node at (axis cs:.2,2) [anchor=west] {\color{penColor2}$f'(x)$};
          \addplot[color=penColor2,fill=penColor2,only marks,mark=*] coordinates{(.451,0)};  %% closed hole
          \addplot[color=penColor2,fill=penColor2,only marks,mark=*] coordinates{(2.215,0)};  %% closed hole
          \addplot[color=penColor,fill=penColor,only marks,mark=*] coordinates{(.451,.631)};  %% closed hole
          \addplot[color=penColor,fill=penColor,only marks,mark=*] coordinates{(2.215,-2.113)};  %% closed hole
        \end{axis}
\end{tikzpicture}
\caption{A plot of $f(x) = x^3-4x^2+3x$ and $f'(x) = 3x^2-8x+3$.}
\label{figure:x^3-4x^2+3x}
\end{marginfigure}
Thus, the only points at which a function can have a local maximum or
minimum are points at which the derivative is zero, see
Figure~\ref{figure:x^3-4x^2+3x}, or the derivative is undefined, as in
Figure~\ref{figure:x^{2/3}}. This brings us to our next definition.
\begin{marginfigure}[0in]
\begin{tikzpicture}
	\begin{axis}[
            domain=-3:3,
            ymax=2,
            ymin=-2,
            axis lines =middle, xlabel=$x$, ylabel=$y$,
            every axis y label/.style={at=(current axis.above origin),anchor=south},
            every axis x label/.style={at=(current axis.right of origin),anchor=west}
          ]
          \addplot [very thick, penColor2, samples=100, smooth,domain=(-3:-.01)] {-(2/3)*abs(x)^(-1/3)};
          \addplot [very thick, penColor2, samples=100, smooth,domain=(.01:3)] {(2/3)*abs(x)^(-1/3)};
          \addplot [very thick, penColor, smooth,domain=(-3:-.01)] {abs(x)^(2/3)};
          \addplot [very thick, penColor, smooth,domain=(.01:3)] {x^(2/3)};         
          \node at (axis cs:-2,1.7) [anchor=west] {\color{penColor}$f(x)$};  
          \node at (axis cs:2,.7) [anchor=west] {\color{penColor2}$f'(x)$};
        \end{axis}
\end{tikzpicture}
\caption{A plot of $f(x) = x^{2/3}$ and $f'(x) = \frac{2}{3x^{1/3}}$.}
\label{figure:x^{2/3}}
\end{marginfigure}

\begin{definition}\index{critical point}
Any value of $x$ for which $f'(x)$ is zero or undefined is called a
\textbf{critical point} for $f(x)$.
\end{definition}

\begin{warning} 
When looking for local maximum and minimum points, you are likely to
make two sorts of mistakes: 
\begin{itemize}
\item You may forget that a maximum or minimum can occur where the
  derivative does not exist, and so forget to check whether the
  derivative exists everywhere. 
\item You might assume that any place that the derivative is zero is a
  local maximum or minimum point, but this is not true, see Figure~\ref{figure:x^3}.
\end{itemize}
\end{warning}



Since the derivative is zero or undefined at both local maximum and
local minimum points, we need a way to determine which, if either,
actually occurs. The most elementary approach is to test directly
whether the $y$ coordinates near the potential maximum or minimum are
above or below the $y$ coordinate at the point of interest. 
\begin{marginfigure}[0in]
\begin{tikzpicture}
	\begin{axis}[
            domain=-3:3,
            ymax=3,
            ymin=-3,
            axis lines =middle, xlabel=$x$, ylabel=$y$,
            every axis y label/.style={at=(current axis.above origin),anchor=south},
            every axis x label/.style={at=(current axis.right of origin),anchor=west}
          ]
          \addplot [very thick, penColor2, smooth] {3*x^2};
          \addplot [very thick, penColor, smooth] {x^3};         
          \node at (axis cs:1,.9) [anchor=west] {\color{penColor}$f(x)$};  
          \node at (axis cs:-.5,1) [anchor=west] {\color{penColor2}$f'(x)$};
        \end{axis}
\end{tikzpicture}
\caption{A plot of $f(x) = x^3$ and $f'(x) = 3x^2$. While $f'(0)=0$,
  there is neither a maximum nor minimum at $(0,f(0))$.}
\label{figure:x^3}
\end{marginfigure}

It is not always easy to compute the value of a function at a
particular point. The task is made easier by the availability of
calculators and computers, but they have their own drawbacks---they do
not always allow us to distinguish between values that are very close
together. Nevertheless, because this method is conceptually simple and
sometimes easy to perform, you should always consider it.

\begin{example}
Find all local maximum and minimum points for the function 
$f(x)=x^3-x$. 
\end{example}

\begin{solution} 
Write
\[
\ddx f(x)=3x^2-1.
\] 
This is defined everywhere and is zero at $x=\pm \sqrt{3}/3$. Looking
first at $x=\sqrt{3}/3$, we see that 
\[
f(\sqrt{3}/3)=-2\sqrt{3}/9.
\] 
Now we test two points on either side of $x=\sqrt{3}/3$, making sure
that neither is farther away than the nearest critical point; since
$\sqrt{3}<3$, $\sqrt{3}/3<1$ and we can use $x=0$ and $x=1$. Since
\[
f(0)=0>-2\sqrt{3}/9\qquad\text{and}\qquad f(1)=0>-2\sqrt{3}/9,
\] 
there must be a local minimum at $x=\sqrt{3}/3$.

For $x=-\sqrt{3}/3$, we see that $f(-\sqrt{3}/3)=2\sqrt{3}/9$. This
time we can use $x=0$ and $x=-1$, and we find that $f(-1)=f(0)=0<
2\sqrt{3}/9$, so there must be a local maximum at $x=-\sqrt{3}/3$, see
Figure~\ref{figure:x^3-x}.
\end{solution}
\begin{marginfigure}[0in]
\begin{tikzpicture}
	\begin{axis}[
            domain=-2:2,
            ymax=2,
            ymin=-2,
            %samples=100,
            axis lines =middle, xlabel=$x$, ylabel=$y$,
            every axis y label/.style={at=(current axis.above origin),anchor=south},
            every axis x label/.style={at=(current axis.right of origin),anchor=west}
          ]
          \addplot [dashed, textColor, smooth] plot coordinates {(-.577,0) (-.577,.385)}; %% {.451};
          \addplot [dashed, textColor, smooth] plot coordinates {(.577,-.385) (.577,0)}; %% axis{2.215};

          \addplot [very thick, penColor2, smooth] {3*x^2-1};
          \addplot [very thick, penColor, smooth] {x^3-x};

          \node at (axis cs:1,1) [anchor=west] {\color{penColor}$f(x)$};  
          \node at (axis cs:-.75,1) [anchor=west] {\color{penColor2}$f'(x)$};

          \addplot[color=penColor2,fill=penColor2,only marks,mark=*] coordinates{(-.577,0)};  %% closed hole
          \addplot[color=penColor2,fill=penColor2,only marks,mark=*] coordinates{(.577,0)};  %% closed hole
          \addplot[color=penColor,fill=penColor,only marks,mark=*] coordinates{(-.577,.385)};  %% closed hole
          \addplot[color=penColor,fill=penColor,only marks,mark=*] coordinates{(.577,-.385)};  %% closed hole
        \end{axis}
\end{tikzpicture}
\caption{A plot of $f(x) = x^3-x$ and $f'(x) = 3x^2-1$.}
\label{figure:x^3-x}
\end{marginfigure}


\begin{exercises} 
\noindent In the following problems, find the $x$ values for local
maximum and minimum points by the method of this section.

\twocol
\begin{exercise} $y=x^2-x$ 
\begin{answer} min at $x=1/2$
\end{answer}\end{exercise}

\begin{exercise} $y=2+3x-x^3$ 
\begin{answer} min at $x=-1$, max at $x=1$
\end{answer}\end{exercise}

\begin{exercise} $y=x^3-9x^2+24x$
\begin{answer} max at $x=2$, min at $x=4$
\end{answer}\end{exercise}

\begin{exercise} $y=x^4-2x^2+3$ 
\begin{answer} min at $x=\pm 1$, max at $x=0$.
\end{answer}\end{exercise}

\begin{exercise} $y=3x^4-4x^3$
\begin{answer} min at $x=1$
\end{answer}\end{exercise}

\begin{exercise} $y=(x^2-1)/x$
\begin{answer} none
\end{answer}\end{exercise}

\begin{exercise} $y=-\frac{x^4}{4}+x^3+x^2$ 
\begin{answer} min at $x=0$, max at $x=\frac{3\pm \sqrt{17}}{2}$
\end{answer}\end{exercise}

\begin{exercise} $f(x) = \begin{cases} x-1 & x < 2  \\
x^2 & x\geq 2 \end{cases}$
\begin{answer} none
\end{answer}\end{exercise}

 \begin{exercise} $f(x) = \begin{cases} x-3 & x < 3  \\
x^3  & 3\leq x \leq 5 \\
1/x  & x>5 \end{cases}$
\begin{answer} local max at $x=5$
\end{answer}\end{exercise}

\begin{exercise} $f(x) = x^2 - 98x + 4$
%(Hint: Complete the square.)
\begin{answer} local min at $x=49$
\end{answer}\end{exercise}

\begin{exercise} $f(x) =\begin{cases} -2 & x = 0  \\
1/x^2 & x \neq 0 \end{cases}$
\begin{answer} local min at $x=0$
\end{answer}\end{exercise}

\endtwocol

\begin{exercise} How many critical points can a quadratic polynomial function have?
\begin{answer} one
\end{answer}\end{exercise}

\begin{exercise} Explore the family of functions $f(x) = x^3 + cx +1$ where $c$
 is a constant.  How many and what types of local extrema are there?
 Your answer should depend on the value of $c$, that is, different
 values of $c$ will give different answers.
\begin{answer} if $c\ge 0$, then there are no local extrema; 
if $c<0$ then there is a local max at $x=-\sqrt{\frac{|c|}{3}}$ and a
local min at $x=\sqrt{\frac{|c|}{3}}$
\end{answer}\end{exercise}


\end{exercises}









\section{The First Derivative Test}

The method of the previous section for deciding whether there is a
local maximum or minimum at a critical point by testing ``near-by''
points is not always convenient. Instead, since we have already had to
compute the derivative to find the critical points, we can use
information about the derivative to decide. Recall that
\begin{itemize}
\item If $f'(x) >0$ on an interval, then $f(x)$ is increasing on that interval.
\item If $f'(x) <0$ on an interval, then $f(x)$ is decreasing on that interval.
\end{itemize}

So how exactly does the derivative tell us whether there is a maximum,
minimum, or neither at a point? Use the \textit{first derivative test}.
\begin{mainTheorem}[First Derivative Test]\index{first derivative test}\label{T:fdt}\hfil
Suppose that $f(x)$ is continuous on an interval, and that $f'(a)=0$
for some value of $a$ in that interval.
\begin{itemize}
\item If $f'(x)>0$ to the left of $a$ and $f'(x)<0$ to the right of
  $a$, then $f(a)$ is a local maximum.
\item If $f'(x)<0$ to the left of $a$ and $f'(x)>0$ to the right of
  $a$, then $f(a)$ is a local minimum.
\item If $f'(x)$ has the same sign to the left and right of $f'(a)$,
  then $f'(a)$ is not a local extremum.
\end{itemize}
\end{mainTheorem}

\begin{example}\label{E:localextrema}
Consider the function 
\[
f(x) = \frac{x^4}{4}+\frac{x^3}{3}-x^2
\]
Find the intervals on which $f(x)$ is increasing and decreasing and
identify the local extrema of $f(x)$.
\end{example}

\begin{solution}
Start by computing
\[
\ddx f(x) = x^3+x^2-2x.
\]
Now we need to find when this function is positive and when it is
negative. To do this, solve 
\[
f'(x) = x^3+x^2-2x =0.
\]
Factor $f'(x)$
\begin{align*}
f'(x) &= x^3+x^2-2x \\
&=x(x^2+x-2)\\
&=x(x+2)(x-1).
\end{align*}
So the critical points (when $f'(x)=0$) are when $x=-2$, $x=0$, and
$x=1$. Now we can check points \textbf{between} the critical points to find
when $f'(x)$ is increasing and decreasing:
\[
f'(-3)=-12 \qquad f'(.5)=-0.625 \qquad f'(-1)=2 \qquad f'(2)=8
\]
From this we can make a sign table:

\flushleft
\begin{tikzpicture}
	\begin{axis}[
            trim axis left,
            scale only axis,
            domain=-3:3,
            ymax=2,
            ymin=-2,
            axis lines=none,
            height=3cm, %% Hard coded height! 
            width=\textwidth, %% width
          ]
          \addplot [draw=none, fill=fill1, domain=(-3:-2)] {2} \closedcycle;
          \addplot [draw=none, fill=fill2, domain=(-2:0)] {2} \closedcycle;
          \addplot [draw=none, fill=fill1, domain=(0:1)] {2} \closedcycle;
          \addplot [draw=none, fill=fill2, domain=(1:3)] {2} \closedcycle;
          
          \addplot [->,textColor] plot coordinates {(-3,0) (3,0)}; %% axis{0};
          
          \addplot [dashed, textColor] plot coordinates {(-2,0) (-2,2)};
          \addplot [dashed, textColor] plot coordinates {(0,0) (0,2)};
          \addplot [dashed, textColor] plot coordinates {(1,0) (1,2)};
          
          \node at (axis cs:-2,0) [anchor=north,textColor] {\footnotesize$-2$};
          \node at (axis cs:0,0) [anchor=north,textColor] {\footnotesize$0$};
          \node at (axis cs:1,0) [anchor=north,textColor] {\footnotesize$1$};

          \node at (axis cs:-2.5,1) [textColor] {\footnotesize$f'(x)<0$};
          \node at (axis cs:.5,1) [textColor] {\footnotesize$f'(x)<0$};
          \node at (axis cs:-1,1) [textColor] {\footnotesize$f'(x)>0$};
          \node at (axis cs:2,1) [textColor] {\footnotesize$f'(x)>0$};

          \node at (axis cs:-2.5,-.5) [anchor=north,textColor] {\footnotesize Decreasing};
          \node at (axis cs:.5,-.5) [anchor=north,textColor] {\footnotesize Decreasing};
          \node at (axis cs:-1,-.5) [anchor=north,textColor] {\footnotesize Increasing};
          \node at (axis cs:2,-.5) [anchor=north,textColor] {\footnotesize Increasing};

        \end{axis}
\end{tikzpicture}


Hence $f(x)$ is increasing on $(-2,0)\cup(1,\infty)$ and $f(x)$ is
decreasing on $(-\infty,-2)\cup(0,1)$. Moreover, from the first
derivative test, Theorem~\ref{T:fdt}, the local maximum is at $x=0$
while the local minima are at $x=-2$ and $x=1$, see
Figure~\ref{figure:(x^4)/4 + (x^3)/3 -x^2}.
\end{solution}
\begin{marginfigure}[-3in]
\begin{tikzpicture}
	\begin{axis}[
            domain=-4:4,
            ymax=5,
            ymin=-5,
            %samples=100,
            axis lines =middle, xlabel=$x$, ylabel=$y$,
            every axis y label/.style={at=(current axis.above origin),anchor=south},
            every axis x label/.style={at=(current axis.right of origin),anchor=west}
          ]
          \addplot [dashed, textColor, smooth] plot coordinates {(-2,0) (-2,-2.667)}; %% {.451};
          \addplot [dashed, textColor, smooth] plot coordinates {(1,0) (1,-.4167)}; %% axis{2.215};

          \addplot [very thick, penColor, smooth] {(x^4)/4 + (x^3)/3 -x^2};
          \addplot [very thick, penColor2, smooth] {x^3 + x^2 -2*x};

          \node at (axis cs:-1.3,-2) [anchor=west] {\color{penColor}$f(x)$};  
          \node at (axis cs:-2.1,2) [anchor=west] {\color{penColor2}$f'(x)$};

          \addplot[color=penColor2,fill=penColor2,only marks,mark=*] coordinates{(-2,0)};  %% closed hole
          \addplot[color=penColor2,fill=penColor2,only marks,mark=*] coordinates{(1,0)};  %% closed hole
          \addplot[color=penColor2,fill=penColor3,only marks,mark=*] coordinates{(0,0)};  %% closed hole
          \addplot[color=penColor,fill=penColor,only marks,mark=*] coordinates{(-2,.-2.667)};  %% closed hole
          \addplot[color=penColor,fill=penColor,only marks,mark=*] coordinates{(1,-.4167)};  %% closed hole
        \end{axis}
\end{tikzpicture}
\caption{A plot of $f(x) =x^4/4 + x^3/3 -x^2$ and $f'(x) = x^3 + x^2 -2x$.}
\label{figure:(x^4)/4 + (x^3)/3 -x^2}
\end{marginfigure}

Hence we have seen that if $f'(x)$ is zero and increasing at a point,
then $f(x)$ has a local minimum at the point. If $f'(x)$ is zero and
decreasing at a point then $f(x)$ has a local maximum at the
point. Thus, we see that we can gain information about $f(x)$ by
studying how $f'(x)$ changes. This leads us to our next section.


\begin{exercises}
\noindent In the following exercises, find all critical points and
identify them as local maximum points, local minimum points, or
neither.

\twocol

\begin{exercise} $y=x^2-x$ 
\begin{answer} min at $x=1/2$
\end{answer}\end{exercise}

\begin{exercise} $y=2+3x-x^3$ 
\begin{answer} min at $x=-1$, max at $x=1$
\end{answer}\end{exercise}

\begin{exercise} $y=x^3-9x^2+24x$
\begin{answer} max at $x=2$, min at $x=4$
\end{answer}\end{exercise}

\begin{exercise} $y=x^4-2x^2+3$ 
\begin{answer} min at $x=\pm 1$, max at $x=0$.
\end{answer}\end{exercise}

\begin{exercise} $y=3x^4-4x^3$
\begin{answer} min at $x=1$
\end{answer}\end{exercise}

\begin{exercise} $y=(x^2-1)/x$
\begin{answer} none
\end{answer}\end{exercise}

\begin{exercise} $f(x) = |x^2 - 121|$
\begin{answer} max at $x=0$, min at $x=\pm 11$
\end{answer}\end{exercise}

\endtwocol

\begin{exercise} Let $f(x) =a x^2 + bx + c$ with $a\neq 0$. Show that $f(x)$
has exactly one critical point using the first derivative test. Give
conditions on $a$ and $b$ which guarantee that the critical point will
be a maximum. 
\begin{answer} $f'(x) = 2ax + b$, this has only one root and hence one critical point; $a<0$ to guarantee a maximum.
\end{answer}
\end{exercise}
\end{exercises}











\section{Concavity and Inflection Points}

We know that the sign of the derivative tells us whether a function is
increasing or decreasing. Likewise, the sign of the second derivative
$f''(x)$ tells us whether $f'(x)$ is increasing or decreasing.


Suppose that $f''(a)>0$. This means that near $x=a$, $f'(x)$ is
increasing. If $f'(a)>0$, this means that $f(x)$ slopes up and is
getting \textit{steeper}. If $f'(a)<0$, this means that $f(x)$ slopes
down and is getting \textit{less steep}. These two situations are
shown in Figure~\ref{figure:concave up}. A curve that is shaped like
this is called \index{concave up} \textbf{concave up}.

\begin{figure}
\begin{tabular}{cc}
\begin{tikzpicture}
	\begin{axis}[
            domain=0:1,
            ymax=1,
            height=4.5cm,
            ymin=0,
            axis lines=none,
          ]
          \addplot [very thick, penColor, smooth] {x^2};
          \node at (axis cs:.3,.4) [textColor] {\footnotesize Concave Up};
        \end{axis}
\end{tikzpicture}

&

\begin{tikzpicture}
	\begin{axis}[
            height=4.5cm,
            domain=0:1,
            ymax=1,
            ymin=0,
            axis lines=none,
          ]
          \addplot [very thick, penColor, smooth] {(x-1)^2};
          \node at (axis cs:.7,.4) [textColor] {\footnotesize Concave Up};
        \end{axis}
\end{tikzpicture}

\\

\begin{minipage}{2in}\footnotesize
Here $f'(x)>0$ and $f''(x)>0$. This means that $f(x)$ slopes up and is
getting \textit{steeper}.
\end{minipage}

& 

\begin{minipage}{2in}\footnotesize
Here $f'(x)<0$ and $f''(x)>0$. This means
that $f(x)$ slopes down and is getting \textit{less steep}.
\end{minipage}
\end{tabular}
\label{figure:concave up}
\caption{Examples of when a curve is concave up.}
\end{figure}



Now suppose that $f''(a)<0$. This means that near $x=a$, $f'(x)$ is
decreasing. If $f'(a)>0$, this means that $f(x)$ slopes up and is
getting less steep; if $f'(a)<0$, this means that $f(x)$ slopes down
and is getting steeper. These two situations are shown in
Figure~\ref{figure:concave down}. A curve that is shaped like this is
called \index{concave down}\textbf{concave down.}


\begin{figure}
\begin{tabular}{cc}
\begin{tikzpicture}
	\begin{axis}[
            height=4.5cm,
            domain=0:1,
            ymax=1,
            ymin=0,
            axis lines=none,
          ]
          \addplot [very thick, penColor, smooth] {-x^2+1};
          \node at (axis cs:.4,.4) [textColor] {\footnotesize Concave Down};
        \end{axis}
\end{tikzpicture}

&

\begin{tikzpicture}
	\begin{axis}[
            height=4.5cm,
            domain=0:1,
            ymax=1,
            ymin=0,
            axis lines=none,
          ]
          \addplot [very thick, penColor, smooth] {-(x-1)^2+1};
          \node at (axis cs:.6,.4) [textColor] {\footnotesize Concave Down};
        \end{axis}
\end{tikzpicture} \\

\begin{minipage}{2in}\footnotesize
Here $f'(x)<0$ and $f''(x)<0$. This means
that $f(x)$ slopes down and is getting \textit{steeper}.
\end{minipage}

&

\begin{minipage}{2in}\footnotesize
Here $f'(x)>0$ and $f''(x)<0$. This means
that $f(x)$ slopes up and is getting less \textit{steep}.
\end{minipage}
\end{tabular}
\label{figure:concave down}
\caption{Examples of when a curve is concave down.}
\end{figure}

If we are trying to understand the shape of the graph of a function,
knowing where it is concave up and concave down helps us to get a more
accurate picture. It is worth summarizing what we have seen already in
to a single theorem.

\begin{mainTheorem}[Test for Concavity]\index{concavity test}
Suppose that $f''(x)$ exists on an interval.
\begin{enumerate}
\item If $f''(x)>0$ on an interval, then $f(x)$ is concave up on that interval.
\item If $f''(x)<0$ on an interval, then $f(x)$ is concave down on that interval.
\end{enumerate}
\end{mainTheorem}


Of particular interest are points at which the concavity changes from
up to down or down to up. 

\begin{definition}\index{inflection point}
If $f(x)$ is continuous and its concavity changes either from up to
down or down to up at $x=a$, then $f(x)$ has an \textbf{inflection
  point} at $x=a$.
\end{definition}

It is instructive to see some examples and nonexamples of inflection
points.

\begin{fullwidth}
\begin{tabular}{cccc}
\begin{tikzpicture}
	\begin{axis}[
            domain=0:2,
            ymax=2,
            height=4.5cm,
            ymin=0,
            axis lines=none,
          ]
          \addplot [very thick, penColor, smooth, domain=(0:1)] {(x-1)^2+1};
          \addplot [very thick, penColor, smooth, domain=(1:2)] {-(x-1)^2+1};
          \addplot[color=penColor,fill=penColor,only marks,mark=*] coordinates{(1,1)};
        \end{axis}
\end{tikzpicture}

&

\begin{tikzpicture}
	\begin{axis}[
            height=4.5cm,
            domain=0:2,
            ymax=1,
            ymin=0,
            axis lines=none,
          ]
          \addplot [very thick, penColor2, smooth] {-(x-1)^2+.75};
          \addplot[color=penColor2,fill=penColor2,only marks,mark=*] coordinates{(1,.75)};
        \end{axis}
\end{tikzpicture} 

&

\begin{tikzpicture}
	\begin{axis}[
            height=4.5cm,
            domain=0:2,
            ymax=2,
            ymin=0,
            samples=100,
            axis lines=none,
          ]
          \addplot [very thick, penColor, smooth,domain=(1:2)] {sqrt(x-1)+1};
          \addplot [very thick, penColor, smooth,domain=(0:1)] {-sqrt(abs(1-x))+1};
          \addplot[color=penColor,fill=penColor,only marks,mark=*] coordinates{(1,1)};
        \end{axis}
\end{tikzpicture}

&

\begin{tikzpicture}
	\begin{axis}[
            height=4.5cm,
            domain=0:2,
            ymax=2,
            ymin=0,
            axis lines=none,
          ]
          \addplot [very thick, penColor2, smooth,domain=(1:2)] {sqrt(x-1)+.5};
          \addplot [very thick, penColor2, smooth,domain=(0:1)] {sqrt(abs(1-x))+.5};
          \addplot[color=penColor2,fill=penColor2,only marks,mark=*] coordinates{(1,.5)};
        \end{axis}
\end{tikzpicture} \\

\begin{minipage}{2in}\footnotesize
This is an inflection point. The concavity changes from concave up to
concave down.
\end{minipage}

& 

\begin{minipage}{2in}\footnotesize
This is \textbf{not} an inflection point. The curve is concave down on either side of the point.
\end{minipage}

& 

\begin{minipage}{2in}\footnotesize
This is an inflection point. The concavity changes from concave up to concave down.
\end{minipage}

&

\begin{minipage}{2in}\footnotesize
This is \textbf{not} an inflection point. The curve is concave down on either side of the point.
\end{minipage}

\end{tabular}
\end{fullwidth}

We identify inflection points by first finding where $f''(x)$ is zero
and then checking to see whether $f''(x)$ does in fact go from
positive to negative or negative to positive at these points.

\begin{warning}
Even if $f''(a) = 0$, the point determined by $x=a$ might \textbf{not}
be an inflection point.
\end{warning}




\begin{example}
Describe the concavity of $f(x)=x^3-x$. 
\end{example}

\begin{solution}
To start, compute the first and second derivative of $f(x)$ with
respect to $x$,
\[
f'(x)=3x^2-1\qquad\text{and}\qquad f''(x)=6x.
\]
Since $f''(0)=0$, there is potentially an inflection point at
zero. Since $f''(x)>0$ when $x>0$ and $f''(x)<0$ when $x<0$ the
concavity does change from down to up at zero---there is an inflection
point at $x=0$. The curve is concave down for all $x<0$ and
concave up for all $x>0$, see Figure~\ref{figure:3x^2-1}.
\end{solution}
\begin{marginfigure}[0in]
\begin{tikzpicture}
	\begin{axis}[
            domain=-3:3,
            ymax=3,
            ymin=-3,
            axis lines =middle, xlabel=$x$, ylabel=$y$,
            every axis y label/.style={at=(current axis.above origin),anchor=south},
            every axis x label/.style={at=(current axis.right of origin),anchor=west}
          ]
          \addplot [very thick, penColor, smooth] {x^3-x};
          \addplot [very thick, penColor4, smooth] {6*x};         
          \node at (axis cs:-.75,.6) [anchor=west] {\color{penColor}$f(x)$};  
          \node at (axis cs:.2,1) [anchor=west] {\color{penColor4}$f''(x)$};
          \addplot[color=penColor4!50!penColor,fill=penColor4!50!penColor,only marks,mark=*] coordinates{(0,0)};  %% closed hole
        \end{axis}
\end{tikzpicture}
\caption{A plot of $f(x) = x^3-x$ and $f''(x) = 6x$. We can see that
  the concavity change at $x=0$.}
\label{figure:3x^2-1}
\end{marginfigure}

Note that we need to compute and analyze the second derivative to
understand concavity, so we may as well try to use the second
derivative test for maxima and minima. If for some reason this fails
we can then try one of the other tests.

\begin{exercises}
\noindent In the following exercises, describe the concavity of the functions.

\twocol

\begin{exercise} $y=x^2-x$ 
\begin{answer} concave up everywhere
\end{answer}\end{exercise}

\begin{exercise} $y=2+3x-x^3$ 
\begin{answer} concave up when $x<0$, concave down when $x>0$
\end{answer}\end{exercise}

\begin{exercise} $y=x^3-9x^2+24x$
\begin{answer} concave down when $x<3$, concave up when $x>3$
\end{answer}\end{exercise}

\begin{exercise} $y=x^4-2x^2+3$ 
\begin{answer} concave up when $x<-1/\sqrt3$ or $x>1/\sqrt3$,
concave down when $-1/\sqrt3<x<1/\sqrt3$
\end{answer}\end{exercise}

\begin{exercise} $y=3x^4-4x^3$
\begin{answer} concave up when $x<0$ or $x>2/3$,
concave down when $0<x<2/3$
\end{answer}\end{exercise}

\begin{exercise} $y=(x^2-1)/x$
\begin{answer} concave up when $x<0$, concave down when $x>0$
\end{answer}\end{exercise}

\begin{exercise} $y=3x^2-\frac{1}{x^2}$ 
\begin{answer} concave up when $x<-1$ or $x>1$, concave down when
$-1<x<0$ or $0<x<1$
\end{answer}\end{exercise}

\begin{exercise} $y= x^5 - x$
\begin{answer} concave up on $(0,\infty)$
\end{answer}\end{exercise}

\begin{exercise} $y = x+ 1/x$
\begin{answer} concave up on $(0,\infty)$
\end{answer}\end{exercise}

\begin{exercise} $y = x^2+ 1/x$
\begin{answer} concave up on $(-\infty,-1)$ and $(0,\infty)$
\end{answer}\end{exercise}

\endtwocol

\begin{exercise} Identify the intervals on which the graph of the
  function $f(x) = x^4-4x^3 +10$ is of one of these four shapes:
  concave up and increasing; concave up and decreasing; concave down
  and increasing; concave down and decreasing.
\begin{answer} up/incr: $(3,\infty)$, up/decr: $(-\infty,0)$, $(2,3)$,
down/decr: $(0,2)$
\end{answer}\end{exercise}


\end{exercises}








\section{The Second Derivative Test}


Recall the first derivative test, Theorem~\ref{T:fdt}:
\begin{itemize}
\item If $f'(x)>0$ to the left of $a$ and $f'(x)<0$ to the right of
  $a$, then $f(a)$ is a local maximum.
\item If $f'(x)<0$ to the left of $a$ and $f'(x)>0$ to the right of
  $a$, then $f(a)$ is a local minimum.
\end{itemize}

If $f'(x)$ changes from positive to negative it is decreasing. In this
case, $f''(x)$ might be negative, and if in fact $f''(x)$ is negative
then $f'(x)$ is definitely decreasing, so there is a local maximum at
the point in question. On the other hand, if $f'(x)$ changes from
negative to positive it is increasing. Again, this means that
$f''(x)$ might be positive, and if in fact $f''(x)$ is positive then
$f'(x)$ is definitely increasing, so there is a local minimum at the
point in question. We summarize this as the \textit{second derivative
  test}.

\begin{mainTheorem}[Second Derivative Test]\index{second derivative test}\label{T:sdt}
Suppose that $f''(x)$ is continuous on an open interval and that
$f'(a)=0$ for some value of $a$ in that interval.
\begin{itemize}
\item If $f''(a) <0$, then $f(x)$ has a local maximum at $a$.
\item If $f''(a) >0$, then $f(x)$ has a local minimum at $a$.
\item If $f''(a) =0$, then the test is inconclusive. In this case,
  $f(x)$ may or may not have a local extremum at $x=a$.
\end{itemize}
\end{mainTheorem}


The second derivative test is often the easiest way to identify local
maximum and minimum points. Sometimes the test fails and sometimes
the second derivative is quite difficult to evaluate. In such cases we
must fall back on one of the previous tests.

\begin{example}
Once again, consider the function 
\[
f(x) = \frac{x^4}{4}+\frac{x^3}{3}-x^2
\]
Use the second derivative test, Theorem~\ref{T:sdt}, to locate the
local extrema of $f(x)$.
\end{example}

\begin{solution}
Start by computing
\[
f'(x) = x^3 + x^2 -2x \qquad\text{and}\qquad f''(x) = 3x^2 + 2x-2.
\] 
Using the same technique as used in the solution of
Example~\ref{E:localextrema}, we find that 
\[
f'(-2) = 0,\qquad f'(0) = 0, \qquad f'(1) = 0. 
\]
Now we'll attempt to use the second derivative test, Theorem~\ref{T:sdt},
\[
f''(-2) = 6, \qquad f''(0) = -2, \qquad f''(1) = 3.
\]
Hence we see that $f(x)$ has a local minimum at $x=-2$, a local
maximum at $x=0$, and a local minimum at $x=1$, see Figure~\ref{figure:SDT(x^4)/4 + (x^3)/3 -x^2}.
\end{solution}
\begin{marginfigure}[-3in]
\begin{tikzpicture}
	\begin{axis}[
            domain=-4:4,
            ymax=7,
            ymin=-4,
            %samples=100,
            axis lines =middle, xlabel=$x$, ylabel=$y$,
            every axis y label/.style={at=(current axis.above origin),anchor=south},
            every axis x label/.style={at=(current axis.right of origin),anchor=west}
          ]
          \addplot [dashed, textColor, smooth] plot coordinates {(-2,-2.667) (-2,6)}; %% {.451};
          \addplot [dashed, textColor, smooth] plot coordinates {(1,0) (1,3)}; %% axis{2.215};

          \addplot [very thick, penColor, smooth] {(x^4)/4 + (x^3)/3 -x^2};
          \addplot [very thick, penColor4, smooth] {3*x^2 + 2*x -2};

          \node at (axis cs:-1.7,-2.7) [anchor=west] {\color{penColor}$f(x)$};  
          \node at (axis cs:-1.5,2) [anchor=west] {\color{penColor4}$f''(x)$};

          \addplot[color=penColor4,fill=penColor4,only marks,mark=*] coordinates{(-2,6)};  %% closed hole
          \addplot[color=penColor4,fill=penColor4,only marks,mark=*] coordinates{(1,3)};  %% closed hole
          \addplot[color=penColor4,fill=penColor4,only marks,mark=*] coordinates{(0,-2)};  %% closed hole
          \addplot[color=penColor,fill=penColor,only marks,mark=*] coordinates{(0,0)};  %% closed hole
          \addplot[color=penColor,fill=penColor,only marks,mark=*] coordinates{(-2,.-2.667)};  %% closed hole
          \addplot[color=penColor,fill=penColor,only marks,mark=*] coordinates{(1,-.4167)};  %% closed hole
        \end{axis}
\end{tikzpicture}
\caption{A plot of $f(x) =x^4/4 + x^3/3 -x^2$ and $f''(x) = 3x^2 + 2x -2$.}
\label{figure:SDT(x^4)/4 + (x^3)/3 -x^2}
\end{marginfigure}




\begin{warning}
If $f''(a)=0$, then the second derivative test gives no information on
whether $x=a$ is a local extremum. 
\end{warning}






\begin{exercises}
Find all local maximum and minimum points by the second derivative
test. 

\twocol
\begin{exercise} $y=x^2-x$ 
\begin{answer} min at $x=1/2$
\end{answer}\end{exercise}

\begin{exercise} $y=2+3x-x^3$ 
\begin{answer} min at $x=-1$, max at $x=1$
\end{answer}\end{exercise}

\begin{exercise} $y=x^3-9x^2+24x$
\begin{answer} max at $x=2$, min at $x=4$
\end{answer}\end{exercise}

\begin{exercise} $y=x^4-2x^2+3$ 
\begin{answer} min at $x=\pm 1$, max at $x=0$.
\end{answer}\end{exercise}

\begin{exercise} $y=3x^4-4x^3$
\begin{answer} min at $x=1$
\end{answer}\end{exercise}

\begin{exercise} $y=(x^2-1)/x$
\begin{answer} none
\end{answer}\end{exercise}

\begin{exercise} $y=3x^2-\frac{1}{x^2}$ 
\begin{answer} none
\end{answer}\end{exercise}

\begin{exercise} $y= x^5 - x$
\begin{answer} max at $-5^{-1/4}$, min at $5^{-1/4}$
\end{answer}\end{exercise}

\begin{exercise} $y = x+ 1/x$
\begin{answer} max at $-1$, min at $1$
\end{answer}\end{exercise}

\begin{exercise} $y = x^2+ 1/x$
\begin{answer} min at $2^{-1/3}$
\end{answer}\end{exercise}


\endtwocol
\end{exercises}












\section{Sketching the Plot of a Function} 


In this section, we will give some general guidelines for sketching
the plot of a function.

\begin{procedureForPlotting}\hfil
\begin{itemize}
\item Find the $y$-intercept, this is the point $(0,f(0))$. Place this
  point on your graph.
\item Find candidates for vertical asymptotes, these are points where
  $f(x)$ is undefined.
\item Compute $f'(x)$ and $f''(x)$.
\item Find the critical points, the points where $f'(x) = 0$. 
\item Use the second derivative test to identify local extrema and/or
  find the intervals where your function is increasing/decreasing.
\item Find the candidates for inflection points, the points where $f''(x) = 0$.
\item Identify inflection points and concavity.
\item If possible find the $x$-intercepts, the points where $f(x) =
  0$. Place these points on your graph.
\item Find horizontal asymptotes.
\item Determine an interval that shows all relevant behavior.
\end{itemize}
At this point you should be able to sketch the plot of your function.
\end{procedureForPlotting}

Let's see this procedure in action. We'll sketch the plot of
$2x^3-3x^2-12x$.  Following our guidelines above, we start by
computing $f(0) = 0$.  Hence we see that the $y$-intercept is
$(0,0)$. Place this point on your plot, see Figure~\ref{figure:CS1}.
\begin{marginfigure}[-2.5in]
\begin{tikzpicture}
	\begin{axis}[
            domain=-2:4,
            xmin=-2,
            xmax=4,
            ymax=25,
            ymin=-25,
            axis lines =middle, xlabel=$x$, ylabel=$y$,
            every axis y label/.style={at=(current axis.above origin),anchor=south},
            every axis x label/.style={at=(current axis.right of origin),anchor=west}
          ]
         \addplot[color=penColor,fill=penColor,only marks,mark=*] coordinates{(0,0)};  %% closed hole
        \end{axis}
\end{tikzpicture}
\caption{We start by placing the point $(0,0)$.}
\label{figure:CS1}
\end{marginfigure}

Note that there are no vertical asymptotes as our function is defined
for all real numbers.  Now compute $f'(x)$ and $f''(x)$,
\[
f'(x) = 6x^2 -6x -12\qquad\text{and}\qquad f''(x) = 12x-6.
\]

The critical points are where $f'(x) = 0$, thus we need to solve $6x^2
-6x -12 = 0$ for x. Write
\begin{align*}
6x^2 -6x -12 &= 0 \\
x^2 - x -2 &=0\\
(x-2)(x+1) &=0.
\end{align*}
Thus
\[
f'(2) = 0\qquad\text{and}\qquad f'(-1) = 0.
\]
Mark the critical points $x=2$ and $x=-1$ on your plot, see
Figure~\ref{figure:CS2}.
\begin{marginfigure}[-3in]
\begin{tikzpicture}
	\begin{axis}[
            domain=-2:4,
            xmin=-2,
            xmax=4,
            ymax=25,
            ymin=-25,
            axis lines =middle, xlabel=$x$, ylabel=$y$,
            every axis y label/.style={at=(current axis.above origin),anchor=south},
            every axis x label/.style={at=(current axis.right of origin),anchor=west}
          ]
         \addplot [dashed, penColor2] plot coordinates {(-1,-25) (-1,25)}; 
         \addplot [dashed, penColor2] plot coordinates {(2,-25) (2,25)}; 
         \addplot[color=penColor,fill=penColor,only marks,mark=*] coordinates{(0,0)};  %% closed hole
        \end{axis}
\end{tikzpicture}
\caption{Now we add the critical points $x=-1$ and $x=2$.}
\label{figure:CS2}
\end{marginfigure}

Check the second derivative evaluated at the critical points. In this
case,
\[
f''(-1) = -18 \qquad\text{and}\qquad f''(2) = 18,
\]
hence $x=-1$, corresponding to the point $(-1,7)$ is a local maximum
and $x=2$, corresponding to the point $(2,-20)$ is local minimum of
$f(x)$. Moreover, this tells us that our function is increasing on
$[-2,-1)$, decreasing on $(-1,2)$, and increasing on $(2,4]$. Identify
this on your plot, see Figure~\ref{figure:CS3}.
\begin{marginfigure}[0in]
\begin{tikzpicture}
	\begin{axis}[
            axis on top=true,
            domain=-2:4,
            xmin=-2,
            xmax=4,
            ymax=25,
            ymin=-25,
            axis lines =middle, xlabel=$x$, ylabel=$y$,
            every axis y label/.style={at=(current axis.above origin),anchor=south},
            every axis x label/.style={at=(current axis.right of origin),anchor=west}
          ]
          \addplot [->, line width=10, penColor!10!background] plot coordinates {(-2,-25) (-1,7)}; 
          \addplot [->, line width=10, penColor!10!background] plot coordinates {(-1,7) (2,-20)}; 
          \addplot [->, line width=10, penColor!10!background] plot coordinates {(2,-20) (4,25)}; 
          \addplot [dashed, penColor2] plot coordinates {(-1,-25) (-1,25)}; 
          \addplot [dashed, penColor2] plot coordinates {(2,-25) (2,25)}; 
          \addplot [color=penColor,fill=penColor,only marks,mark=*] coordinates{(0,0)};  %% closed hole
          \addplot [color=penColor,fill=penColor,only marks,mark=*] coordinates{(-1,7)};  %% closed hole
          \addplot [color=penColor,fill=penColor,only marks,mark=*] coordinates{(2,-20)};  %% closed hole
          %\addplot [very thick, penColor, samples=100, smooth,domain=(-1.2:-.8)] {2*x^3-3*x^2-12*x};
          %\addplot [very thick, penColor, samples=100, smooth,domain=(1.8:2.2)] {2*x^3-3*x^2-12*x};
        \end{axis}
\end{tikzpicture}
\caption{We have identified the local extrema of $f(x)$ and where this
  function is increasing and decreasing.}
\label{figure:CS3}
\end{marginfigure}


The candidates for the inflection points are where $f''(x) = 0$, thus
we need to solve $12x-6=0$ for $x$.  Write
\begin{align*}
12x-6 &=0\\
x - 1/2 &=0\\
x &=1/2.
\end{align*}
Thus $f''(1/2) = 0$. Checking points, $f''(0) = -6$ and $f''(1) = 6$.
Hence $x=1/2$ is an inflection point, with $f(x)$ concave down to the
left of $x=1/2$ and $f(x)$ concave up to the right of $x=1/2$. We can
add this information to our plot, see Figure~\ref{figure:CS4}.

Finally, in this case, $f(x) =2x^3-3x^2-12x$, we can find the
$x$-intercepts. Write
\begin{align*}
2x^3-3x^2-12x &=0\\
x(2x^2 -3x -12) &=0.\\
\end{align*}
Using the quadratic formula, we see that the $x$-intercepts of $f(x)$ are
\[
x = 0, \qquad x= \frac{3-\sqrt{105}}{4}, \qquad x= \frac{3+\sqrt{105}}{4}.
\]
Since all of this behavior as described above occurs on the interval
$[-2,4]$, we now have a complete sketch of $f(x)$ on this interval,
see the figure below.
\begin{marginfigure}[0in]
\begin{tikzpicture}
	\begin{axis}[
            axis on top=true,
            domain=-2:4,
            xmin=-2,
            xmax=4,
            ymax=25,
            ymin=-25,
            axis lines =middle, xlabel=$x$, ylabel=$y$,
            every axis y label/.style={at=(current axis.above origin),anchor=south},
            every axis x label/.style={at=(current axis.right of origin),anchor=west}
          ]
          \addplot [->, line width=10, penColor!10!background] plot coordinates {(-2,-25) (-1,7)}; 
          \addplot [->, line width=10, penColor!10!background] plot coordinates {(-1,7) (2,-20)}; 
          \addplot [->, line width=10, penColor!10!background] plot coordinates {(2,-20) (4,25)}; 
          \addplot [dashed, penColor2] plot coordinates {(-1,-25) (-1,25)}; 
          \addplot [dashed, penColor2] plot coordinates {(2,-25) (2,25)}; 
          \addplot [dashed, penColor4] plot coordinates {(1/2,-25) (1/2,25)}; 
          \addplot [color=penColor,fill=penColor,only marks,mark=*] coordinates{(1/2,-6.5)};  %% closed hole
          \addplot [color=penColor,fill=penColor,only marks,mark=*] coordinates{(0,0)};  %% closed hole
          \addplot [color=penColor,fill=penColor,only marks,mark=*] coordinates{(-1,7)};  %% closed hole
          \addplot [color=penColor,fill=penColor,only marks,mark=*] coordinates{(2,-20)};  %% closed hole
          \addplot [very thick, penColor, samples=100, smooth,domain=(-1.5:3)] {2*x^3-3*x^2-12*x};
        \end{axis}
\end{tikzpicture}
\caption{We identify the inflection point and note that the curve is
  concave down when $x<1/2$ and concave up when $x>1/2$.}
\label{figure:CS4}
\end{marginfigure}

\begin{tikzpicture}
	\begin{axis}[
            axis on top=true,
            domain=-2:4,
            xmin=-2,
            xmax=4,
            ymax=25,
            ymin=-25,
            axis lines =middle, xlabel=$x$, ylabel=$y$,
            every axis y label/.style={at=(current axis.above origin),anchor=south},
            every axis x label/.style={at=(current axis.right of origin),anchor=west}
          ]
          \addplot [->, line width=10, penColor!10!background] plot coordinates {(-2,-25) (-1,7)}; 
          \addplot [->, line width=10, penColor!10!background] plot coordinates {(-1,7) (2,-20)}; 
          \addplot [->, line width=10, penColor!10!background] plot coordinates {(2,-20) (4,25)}; 
          \addplot [dashed, penColor2] plot coordinates {(-1,-25) (-1,25)}; 
          \addplot [dashed, penColor2] plot coordinates {(2,-25) (2,25)}; 
          \addplot [dashed, penColor4] plot coordinates {(1/2,-25) (1/2,25)}; 
          \addplot [color=penColor,fill=penColor,only marks,mark=*] coordinates{(1/2,-6.5)};  %% closed hole
          \addplot [color=penColor,fill=penColor,only marks,mark=*] coordinates{(0,0)};  %% closed hole
          \addplot [color=penColor,fill=penColor,only marks,mark=*] coordinates{(-1,7)};  %% closed hole
          \addplot [color=penColor,fill=penColor,only marks,mark=*] coordinates{(2,-20)};  %% closed hole
          \addplot [color=penColor,fill=penColor,only marks,mark=*] coordinates{(-1.812,0)};  %% closed hole
          \addplot [color=penColor,fill=penColor,only marks,mark=*] coordinates{(3.312,0)};  %% closed hole
          \addplot [very thick, penColor, samples=100, smooth,domain=(-2:4)] {2*x^3-3*x^2-12*x};
        \end{axis}
\end{tikzpicture}















 
\begin{exercises}

\noindent Sketch the curves via the procedure outlined in this
section. Clearly identify any interesting features, including local
maximum and minimum points, inflection points, asymptotes, and
intercepts.

\twocol


\begin{exercise} $y= x^5 - x$
\begin{answer}
$y$-intercept at $(0,0)$; no vertical asymptotes; critical points:
  $x=\pm\sqrt[4]{5}$; local max at $x=-\sqrt[4]{5}$, local min at
  $x=-\sqrt[4]{5}$; increasing on $(-\infty,-\sqrt[4]{5})$, decreasing
  on $(-\sqrt[4]{5},\sqrt[4]{5})$, increasing on
  $(\sqrt[4]{5},\infty)$; concave down on $(-\infty,0)$, concave up on
  $(0, \infty)$; root at $x=0$; no horizontal asymptotes; interval for
  sketch: $[-1.2,1.2]$ (answers may vary)
\end{answer}
\end{exercise}

\begin{exercise} $y=x(x^2+1)$
\begin{answer}
$y$-intercept at $(0,0)$; no vertical asymptotes; no critical points;
  no local extrema; increasing on $(-\infty,\infty)$; concave down on
  $(-\infty,0)$, concave up on $(0, \infty)$; roots at $x=0$; no
  horizontal asymptotes; interval for sketch: $[-3,3]$ (answers may
  vary)
\end{answer}
\end{exercise}

\begin{exercise} $y=2\sqrt{x} - x$
\begin{answer}
$y$-intercept at $(0,0)$; no vertical asymptotes; critical points: $x=
  1$; local max at $x=1$; increasing on $[0,1)$, decreasing on
    $(1,\infty)$; concave down on $[0,\infty)$; roots at $x=0$, $x=4$;
      no horizontal asymptotes; interval for sketch: $[0,6]$ (answers
      may vary)
\end{answer}
\end{exercise}

\begin{exercise} $y=x^3+6x^2 + 9x$
\begin{answer}
$y$-intercept at $(0,0)$; no vertical asymptotes; critical points:
  $x=-3$, $x= -1$; local max at $x=-3$, local min at $x=-1$;
  increasing on $(-\infty,-3)$, decreasing on $(-3,-1)$, increasing on
  $(-1,\infty)$; concave down on $(-\infty,-2)$, concave up on $(-2,
  \infty)$; roots at $x=-3$, $x=0$; no horizontal asymptotes; interval
  for sketch: $[-5,3]$ (answers may vary)
\end{answer}
\end{exercise}

\begin{exercise} $y=x^3-3x^2-9x+5$
\begin{answer}
$y$-intercept at $(0,5)$; no vertical asymptotes; critical points:
  $x=-1$, $x= 3$; local max at $x=-1$, local min at $x=3$; increasing
  on $(-\infty,-1)$, decreasing on $(-1,3)$, increasing on
  $(3,\infty)$; concave down on $(-\infty,1)$, concave up on $(1,
  \infty)$; roots are too difficult to be determined---cubic formula
  could be used; no horizontal asymptotes; interval for sketch:
  $[-2,5]$ (answers may vary)
\end{answer}
\end{exercise}


\begin{exercise} $y=x^5-5x^4+5x^3$
\begin{answer}
$y$-intercept at $(0,0)$; no vertical asymptotes; critical points:
  $x=0$, $x= \frac{10\pm\sqrt{85}}{5}$; local max at $x=
  \frac{10-\sqrt{85}}{5}$, local min at $x= \frac{10+\sqrt{85}}{5}$;
  increasing on $(-\infty,\frac{10-\sqrt{85}}{5})$, decreasing on
  $(\frac{10-\sqrt{85}}{5}, \frac{10+\sqrt{85}}{5})$, increasing on
  $(\frac{10+\sqrt{85}}{5},\infty)$; concave down on $(-\infty,0)$,
  concave up on $(0, \frac{15-\sqrt{195}}{10})$, concave down on
  $(\frac{15-\sqrt{195}}{10},\frac{15+\sqrt{195}}{10})$, concave up on
  $(\frac{15+\sqrt{195}}{10},\infty)$; roots at $x=0$, $x= \frac{5\pm
    \sqrt{21}}{2}$; no horizontal asymptotes; interval for sketch:
  $[-1,5]$ (answers may vary)
\end{answer}
\end{exercise}


\begin{exercise} $y = x+ 1/x$
\begin{answer}
no $y$-intercept; vertical asymptote at $x=0$; critical points: $x=0$,
$x=\pm 1$; local max at $x=-1$, local min at $1$; increasing on
$(-\infty,-1)$, decreasing on $(-1,0)\cup(0,1)$, increasing on
$(1,\infty)$; concave down on $(-\infty,0)$, concave up on $(0,
\infty)$; no roots; no horizontal asymptotes; interval for sketch:
$[-2,2]$ (answers may vary)
\end{answer}
\end{exercise}

\begin{exercise} $y = x^2+ 1/x$
\begin{answer}
no $y$-intercept; vertical asymptote at $x=0$; critical points: $x=0$,
$x=\frac{1}{\sqrt[3]{2}}$; local min at $x=\frac{1}{\sqrt[3]{2}}$;
decreasing on $(-\infty,0)$, decreasing on
$(0,\frac{1}{\sqrt[3]{2}})$, increasing on
$(\frac{1}{\sqrt[3]{2}},\infty)$; concave up on $(-\infty,-1)$,
concave down on $(-1,0)$, concave up on $(0,\infty)$; root at $x=-1$;
no horizontal asymptotes; interval for sketch: $[-3,2]$ (answers may
vary)
\end{answer}
\end{exercise}





\endtwocol

\end{exercises}
