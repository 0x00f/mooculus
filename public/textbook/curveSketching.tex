\chapter{Curve Sketching}



Whether we are interested in a function as a purely mathematical
object or in connection with some application to the real world, it is
often useful to know what the graph of the function looks like. We can
obtain a good picture of the graph using certain crucial information
provided by derivatives of the function and certain limits.

\section{Extrema}

Local \textit{extrema} on a function are points on the graph where the
$y$ coordinate is larger (or smaller) than all other $y$ coordinates
on the graph at points ``close to'' $(x,y)$. 

\begin{definition}\hfil
\begin{enumerate}
\item A point $(x,f(x))$ is a \textbf{local maximum}\index{local
  maximum} if there is an interval $a<x<b$ with $f(x)\ge f(z)$ for
  every $z$ in $(a,b)$.
\item A point $(x,f(x))$ is a \textbf{local minimum}\index{local
  minimum} if there is an interval $a<x<b$ with $f(x)\le f(z)$ for
  every $z$ in $(a,b)$.
\end{enumerate}
A \textbf{local extremum}\index{local extremum} is either a local
maximum or a local minimum.
\end{definition}

Local maximum and minimum points are quite distinctive on the graph of
a function, and are therefore useful in understanding the shape of the
graph. In many applied problems we want to find the largest or
smallest value that a function achieves (for example, we might want
to find the minimum cost at which some task can be performed) and so
identifying maximum and minimum points will be useful for applied
problems as well.

If $(x,f(x))$ is a point where $f(x)$ reaches a local maximum or minimum,
and if the derivative of $f$ exists at $x$, then the graph has a
tangent line and the tangent line must be horizontal. This is
important enough to state as a theorem, though we will not prove it.

\begin{mainTheorem}[Fermat's Theorem]\index{Fermat's Theorem}
If $f(x)$ has a local extremum at $x=a$ and $f(x)$ is differentiable at
$a$, then $f'(a)=0$.
\end{mainTheorem}
\begin{marginfigure}[0in]
\begin{tikzpicture}
	\begin{axis}[
            domain=-3:3,
            ymax=3,
            ymin=-3,
            %samples=100,
            axis lines =middle, xlabel=$x$, ylabel=$y$,
            every axis y label/.style={at=(current axis.above origin),anchor=south},
            every axis x label/.style={at=(current axis.right of origin),anchor=west}
          ]
          \addplot [dashed, textColor, smooth] plot coordinates {(.451,0) (.451,.631)}; %% {.451};
          \addplot [dashed, textColor, smooth] plot coordinates {(2.215,-2.113) (2.215,0)}; %% axis{2.215};
          \addplot [very thick, penColor2, smooth] {3*x^2-8*x+3};
          \addplot [very thick, penColor, smooth] {x^3-4*x^2+3*x};
          \node at (axis cs:2.5,-2) [anchor=west] {\color{penColor}$f(x)$};  
          \node at (axis cs:.2,2) [anchor=west] {\color{penColor2}$f'(x)$};
          \addplot[color=penColor2,fill=penColor2,only marks,mark=*] coordinates{(.451,0)};  %% closed hole
          \addplot[color=penColor2,fill=penColor2,only marks,mark=*] coordinates{(2.215,0)};  %% closed hole
          \addplot[color=penColor,fill=penColor,only marks,mark=*] coordinates{(.451,.631)};  %% closed hole
          \addplot[color=penColor,fill=penColor,only marks,mark=*] coordinates{(2.215,-2.113)};  %% closed hole
        \end{axis}
\end{tikzpicture}
\caption{A plot of $f(x) = x^3-4x^2+3x$ and $f'(x) = 3x^2-8x+3$.}
\label{figure:x^3-4x^2+3x}
\end{marginfigure}
Thus, the only points at which a function can have a local maximum or
minimum are points at which the derivative is zero, as in the left
hand graph in Figure~\ref{figure:x^3-4x^2+3x}, or the derivative
is undefined, as in Figure~\ref{figure:x^{2/3}}. This brings us to our next definition.
\begin{marginfigure}[0in]
\begin{tikzpicture}
	\begin{axis}[
            domain=-3:3,
            ymax=3,
            ymin=-1,
            axis lines =middle, xlabel=$x$, ylabel=$y$,
            every axis y label/.style={at=(current axis.above origin),anchor=south},
            every axis x label/.style={at=(current axis.right of origin),anchor=west}
          ]
          \addplot [very thick, penColor2, samples=100, smooth,domain=(-3:-.01)] {(2/3)*abs(x)^(-1/3)};
          \addplot [very thick, penColor2, samples=100, smooth,domain=(.01:3)] {(2/3)*abs(x)^(-1/3)};
          \addplot [very thick, penColor, smooth,domain=(-3:-.01)] {abs(x)^(2/3)};
          \addplot [very thick, penColor, smooth,domain=(.01:3)] {x^(2/3)};         
          \node at (axis cs:-2,1.7) [anchor=west] {\color{penColor}$f(x)$};  
          \node at (axis cs:2,.7) [anchor=west] {\color{penColor2}$f'(x)$};
        \end{axis}
\end{tikzpicture}
\caption{A plot of $f(x) = x^{2/3}$ and $f'(x) = \frac{2}{3x^{1/3}}$.}
\label{figure:x^{2/3}}
\end{marginfigure}

\begin{definition}\index{critical value}
Any value of $x$ for which $f'(x)$ is zero or undefined is called a
\textbf{critical value} for $f(x)$.
\end{definition}

\begin{warning} When looking for local maximum and minimum points, you are likely to
make two sorts of mistakes: 
\begin{itemize}
\item You may forget that a maximum or minimum can occur where the
  derivative does not exist, and so forget to check whether the
  derivative exists everywhere. 
\item You might assume that any place that the derivative is zero is a
  local maximum or minimum point, but this is not true. See Figure~\ref{figure:x^3}.
\end{itemize}
\end{warning}



Since the derivative is zero or undefined at both local maximum and
local minimum points, we need a way to determine which, if either,
actually occurs. The most elementary approach is to test directly
whether the $y$ coordinates near the potential maximum or minimum are
above or below the $y$ coordinate at the point of interest. 
\begin{marginfigure}[0in]
\begin{tikzpicture}
	\begin{axis}[
            domain=-3:3,
            ymax=3,
            ymin=-3,
            axis lines =middle, xlabel=$x$, ylabel=$y$,
            every axis y label/.style={at=(current axis.above origin),anchor=south},
            every axis x label/.style={at=(current axis.right of origin),anchor=west}
          ]
          \addplot [very thick, penColor2, smooth] {3*x^2};
          \addplot [very thick, penColor, smooth] {x^3};         
          \node at (axis cs:1,.9) [anchor=west] {\color{penColor}$f(x)$};  
          \node at (axis cs:-.5,1) [anchor=west] {\color{penColor2}$f'(x)$};
        \end{axis}
\end{tikzpicture}
\caption{A plot of $f(x) = x^3$ and $f'(x) = 3x^2$. While $f'(0)=0$,
  there is neither a maximum nor minimum at $(0,f(0))$.}
\label{figure:x^3}
\end{marginfigure}

It is not always easy to compute the value of a function at a
particular point. The task is made easier by the availability of
calculators and computers, but they have their own drawbacks---they do
not always allow us to distinguish between values that are very close
together. Nevertheless, because this method is conceptually simple and
sometimes easy to perform, you should always consider it.

\begin{example}
Find all local maximum and minimum points for the function 
$f(x)=x^3-x$. 
\end{example}

\begin{solution} 
Write
\[
\ddx f(x)=3x^2-1.
\] 
This is defined everywhere and is zero at $x=\pm \sqrt{3}/3$. Looking
first at $x=\sqrt{3}/3$, we see that 
\[
f(\sqrt{3}/3)=-2\sqrt{3}/9.
\] 
Now we test two points on either side of $x=\sqrt{3}/3$, making sure
that neither is farther away than the nearest critical value; since
$\sqrt{3}<3$, $\sqrt{3}/3<1$ and we can use $x=0$ and $x=1$. Since
\[
f(0)=0>-2\sqrt{3}/9\qquad\text{and}\qquad f(1)=0>-2\sqrt{3}/9,
\] 
there must be a local minimum at $x=\sqrt{3}/3$.

For $x=-\sqrt{3}/3$, we see that $f(-\sqrt{3}/3)=2\sqrt{3}/9$. This
time we can use $x=0$ and $x=-1$, and we find that $f(-1)=f(0)=0<
2\sqrt{3}/9$, so there must be a local maximum at $x=-\sqrt{3}/3$, see
Figure~\ref{figure:x^3-x}.
\end{solution}
\begin{marginfigure}[0in]
\begin{tikzpicture}
	\begin{axis}[
            domain=-2:2,
            ymax=2,
            ymin=-2,
            %samples=100,
            axis lines =middle, xlabel=$x$, ylabel=$y$,
            every axis y label/.style={at=(current axis.above origin),anchor=south},
            every axis x label/.style={at=(current axis.right of origin),anchor=west}
          ]
          \addplot [dashed, textColor, smooth] plot coordinates {(-.577,0) (-.577,.385)}; %% {.451};
          \addplot [dashed, textColor, smooth] plot coordinates {(.577,-.385) (.577,0)}; %% axis{2.215};
          \addplot [very thick, penColor2, smooth] {3*x^2-1};
          \addplot [very thick, penColor, smooth] {x^3-x};
          \node at (axis cs:1,1) [anchor=west] {\color{penColor}$f(x)$};  
          \node at (axis cs:-.75,1) [anchor=west] {\color{penColor2}$f'(x)$};
          \addplot[color=penColor2,fill=penColor2,only marks,mark=*] coordinates{(-.577,0)};  %% closed hole
          \addplot[color=penColor2,fill=penColor2,only marks,mark=*] coordinates{(.577,0)};  %% closed hole
          \addplot[color=penColor,fill=penColor,only marks,mark=*] coordinates{(-.577,.385)};  %% closed hole
          \addplot[color=penColor,fill=penColor,only marks,mark=*] coordinates{(.577,-.385)};  %% closed hole
        \end{axis}
\end{tikzpicture}
\caption{A plot of $f(x) = x^3-x$ and $f'(x) = 3x^2-1$.}
\label{figure:x^3-x}
\end{marginfigure}


\begin{exercises} 
\noindent In problems 1--12, find all local maximum and minimum points
$(x,y)$ by the method of this section.

\twocol
\begin{exercise} $y=x^2-x$ 
\begin{answer} min at $x=1/2$
\end{answer}\end{exercise}

\begin{exercise} $y=2+3x-x^3$ 
\begin{answer} min at $x=-1$, max at $x=1$
\end{answer}\end{exercise}

\begin{exercise} $y=x^3-9x^2+24x$
\begin{answer} max at $x=2$, min at $x=4$
\end{answer}\end{exercise}

\begin{exercise} $y=x^4-2x^2+3$ 
\begin{answer} min at $x=\pm 1$, max at $x=0$.
\end{answer}\end{exercise}

\begin{exercise} $y=3x^4-4x^3$
\begin{answer} min at $x=1$
\end{answer}\end{exercise}

\begin{exercise} $y=(x^2-1)/x$
\begin{answer} none
\end{answer}\end{exercise}

\begin{exercise} $y=3x^2-(1/x^2)$ 
\begin{answer} none
\end{answer}\end{exercise}

\begin{exercise} $y=\cos(2x)-x$ 
\begin{answer} min at $x=7\pi/12+k\pi$, max at $x=-\pi/12+k\pi$, for integer $k$.
\end{answer}\end{exercise}

\begin{exercise} $f(x) = \begin{cases} x-1 & x < 2  \\
x^2 & x\geq 2 \end{cases}$
\begin{answer} none
\end{answer}\end{exercise}

 \begin{exercise} $f(x) = \begin{cases} x-3 & x < 3  \\
x^3  & 3\leq x \leq 5 \\
1/x  & x>5 \end{cases}$
\begin{answer} local max at $x=5$
\end{answer}\end{exercise}

\begin{exercise} $f(x) = x^2 - 98x + 4$
%(Hint: Complete the square.)
\begin{answer} local min at $x=49$
\end{answer}\end{exercise}

\begin{exercise} $f(x) =\begin{cases} -2 & x = 0  \\
1/x^2 & x \neq 0 \end{cases}$
\begin{answer} local min at $x=0$
\end{answer}\end{exercise}

\endtwocol

\begin{exercise}  For any real number $x$ there is a unique
  integer $n$ such that $n \leq x < n +1$, and the greatest
  integer\index{greatest integer} function is defined as $\ds\lfloor
  x\rfloor = n$. Where
  are the critical values of the greatest integer function?  Which are
  local maxima and which are local minima?
\end{exercise}

\begin{exercise} Explain why the function $f(x) =1/x$ has no local
maxima or minima.
\end{exercise}

\begin{exercise} How many critical points can a quadratic polynomial function have?
\begin{answer} one
\end{answer}\end{exercise}

\begin{exercise} Show that a cubic polynomial can have at most two critical
points. Give examples to show that a cubic polynomial can have zero,
one, or two critical points.
\end{exercise}

\begin{exercise} Explore the family of functions $f(x) = x^3 + cx +1$ where $c$
 is a constant.  How many and what types of local extremes are there?
 Your answer should depend on the value of $c$, that is, different
 values of $c$ will give different answers.
\end{exercise}

\begin{exercise} We generalize the preceding two questions. Let $n$ be a
positive integer and let $f$ be a polynomial of degree $n$. How many
critical points can $f$ have? (Hint: Recall the {\dfont Fundamental
  Theorem of Algebra\index{Fundamental Theorem of Algebra}}, 
which says that a polynomial of degree $n$ has
  at most $n$ roots.)
\end{exercise}

\end{exercises}









\section{The First Derivative Test}

The method of the previous section for deciding whether there is a
local maximum or minimum at a critical value by testing ``near-by''
points is not always convenient. Instead, since we have already had to
compute the derivative to find the critical values, we can use
information about the derivative to decide.

\begin{mainTheorem}[First Derivative Test]\index{first derivative test}\hfil
\begin{enumerate}
\item If $f'(x) >0$ on an interval, then $f(x)$ is increasing on that interval
\item If $f'(x) <0$ on an interval, then $f(x)$ is decreasing on that interval.
\end{enumerate}
\end{mainTheorem}


How can the derivative tell us whether there is a maximum, minimum, or
neither at a point? Suppose that $f'(a)=0$:
\begin{itemize}
\item If there is a local maximum when $x=a$, $f(x)$ must be
  increasing before $x=a$ and decreasing after $x=a$.
\item If there is a local minimum when $x=a$, $f(x)$ must be
  increasing before $x=a$ and decreasing after $x=a$.
\end{itemize}

\begin{example}
Consider the function 
\[
f(x) = \frac{x^4}{4}+\frac{x^3}{3}-x^2
\]
\begin{enumerate}
\item Find the intervals on which $f(x)$ is increasing and decreasing.
\item Identify the local extrema of $f(x)$. 
\end{enumerate}
\end{example}

\begin{solution}
Start by computing
\[
\ddx f(x) = x^3+x^2-2x.
\]
Now we need to find when this function is postive and when it is
negative. To do this, solve 
\[
f'(x) = x^3+x^2-2x =0.
\]
Factor $f'(x)$
\begin{align*}
f'(x) &= x^3+x^2-2x \\
&=x(x^2+x-2)\\
&=x(x+2)(x-1).
\end{align*}
So the critical values (when $f'(x)=0$) are when $x=-2$, $x=0$, and
$x=1$. Now we can check points between the critical values to find
when $f'(x)$ is increasing and decreasing and make a sign table:
\begin{tikzpicture}
	\begin{axis}[
            %trim axis left,
            domain=-3:3,
            ymax=2,
            ymin=-2,
            axis lines=none,
            height=4cm, %% Hard coded height! 
            %scale only axis,
            width=\textwidth, %% width
          ]
          \addplot [draw=none, fill=fill1, domain=(-3:-2)] {2} \closedcycle;
          \addplot [draw=none, fill=fill2, domain=(-2:0)] {2} \closedcycle;
          \addplot [draw=none, fill=fill1, domain=(0:1)] {2} \closedcycle;
          \addplot [draw=none, fill=fill2, domain=(1:3)] {2} \closedcycle;
          
          \addplot [->,textColor] plot coordinates {(-3,0) (3,0)}; %% axis{0};
          
          \addplot [dashed, textColor] plot coordinates {(-2,0) (-2,2)};
          \addplot [dashed, textColor] plot coordinates {(0,0) (0,2)};
          \addplot [dashed, textColor] plot coordinates {(1,0) (1,2)};
          
          \node at (axis cs:-2,0) [anchor=north,textColor] {\footnotesize$-2$};
          \node at (axis cs:0,0) [anchor=north,textColor] {\footnotesize$0$};
          \node at (axis cs:1,0) [anchor=north,textColor] {\footnotesize$1$};

          \node at (axis cs:-2.5,1) [anchor=south,textColor] {\footnotesize$f'(x)<0$};
          \node at (axis cs:.5,1) [anchor=south,textColor] {\footnotesize$f'(x)<0$};
          \node at (axis cs:-1,1) [anchor=south,textColor] {\footnotesize$f'(x)>0$};
          \node at (axis cs:2,1) [anchor=south,textColor] {\footnotesize$f'(x)>0$};

          \node at (axis cs:-2.5,.5) [anchor=south,textColor] {\footnotesize$f'(-3)=-12$};
          \node at (axis cs:.5,.5) [anchor=south,textColor] {\footnotesize$f'(.5)=-0.625$};
          \node at (axis cs:-1,.5) [anchor=south,textColor] {\footnotesize$f'(-1)=2$};
          \node at (axis cs:2,.5) [anchor=south,textColor] {\footnotesize$f'(2)=8$};

          \node at (axis cs:-2.5,-.5) [anchor=north,textColor] {\footnotesize Decreasing};
          \node at (axis cs:.5,-.5) [anchor=north,textColor] {\footnotesize Decreasing};
          \node at (axis cs:-1,-.5) [anchor=north,textColor] {\footnotesize Increasing};
          \node at (axis cs:2,-.5) [anchor=north,textColor] {\footnotesize Increasing};

        \end{axis}
\end{tikzpicture}


\begin{tikzpicture}
	\begin{axis}[
            domain=-4:4,
            ymax=5,
            ymin=-5,
            %samples=100,
            axis lines =middle, xlabel=$x$, ylabel=$y$,
            every axis y label/.style={at=(current axis.above origin),anchor=south},
            every axis x label/.style={at=(current axis.right of origin),anchor=west}
          ]

          \addplot [very thick, penColor, smooth] {(x^4)/4 + (x^3)/3 -x^2};
        \end{axis}
\end{tikzpicture}

\end{solution}

\begin{exercises}
In 1--13,
find all critical points and identify them as
local maximum points, local minimum points, or neither.

\twocol

\begin{exercise} $y=x^2-x$ 
\begin{answer} min at $x=1/2$
\end{answer}\end{exercise}

\begin{exercise} $y=2+3x-x^3$ 
\begin{answer} min at $x=-1$, max at $x=1$
\end{answer}\end{exercise}

\begin{exercise} $y=x^3-9x^2+24x$
\begin{answer} max at $x=2$, min at $x=4$
\end{answer}\end{exercise}

\begin{exercise} $y=x^4-2x^2+3$ 
\begin{answer} min at $x=\pm 1$, max at $x=0$.
\end{answer}\end{exercise}

\begin{exercise} $y=3x^4-4x^3$
\begin{answer} min at $x=1$
\end{answer}\end{exercise}

\begin{exercise} $y=(x^2-1)/x$
\begin{answer} none
\end{answer}\end{exercise}

\begin{exercise} $y=3x^2-(1/x^2)$ 
\begin{answer} none
\end{answer}\end{exercise}

\begin{exercise} $y=\cos(2x)-x$ 
\begin{answer} min at $x=7\pi/12+k\pi$, max at $x=-\pi/12+k\pi$, for integer $k$.
\end{answer}\end{exercise}

\begin{exercise}
$f(x) = (5-x)/(x+2)$
\begin{answer} none
\end{answer}\end{exercise}

\begin{exercise} $f(x) = |x^2 - 121|$
\begin{answer} max at $x=0$, min at $x=\pm 11$
\end{answer}\end{exercise}

\begin{exercise} $f(x) = x^3/(x+1)$
\begin{answer} min at $x=-3/2$, neither at $x=0$
\end{answer}\end{exercise}

\begin{exercise} $f(x)= \begin{cases}
x^2 \sin(1/x)  & x\neq 0  \\
 0  & x=0 \end{cases}$
\end{exercise}

\begin{exercise} $f(x) = \sin ^2 x$
\begin{answer} min at $n\pi$, max at $\pi/2+n\pi$
\end{answer}\end{exercise}

\endtwocol
\bsk
\begin{exercise} Find the maxima and minima of $f(x)=\sec x$.
\begin{answer} min at $2n\pi$, max at $(2n+1)\pi$
\end{answer}\end{exercise}

\begin{exercise}  Let $f(\theta) = \cos^2(\theta) -
 2\sin(\theta)$.  Find the intervals where $f$ is increasing and the
 intervals where $f$ is decreasing in $[0,2\pi]$.  Use this
 information to classify the critical points of $f$ as either local
 maximums, local minimums, or neither.
\begin{answer} min at $\pi/2+2n\pi$, max at $3\pi/2+2n\pi$
\end{answer}\end{exercise}

\begin{exercise} Let $r>0$. Find the local
maxima and minima of the function $f(x)
=\sqrt{r^2 -x^2 }$ on its domain $[-r,r]$.
\end{exercise}

\begin{exercise} Let $f(x) =a x^2 + bx + c$ with $a\neq 0$. Show that $f$
has exactly one critical point using the first derivative test. Give
conditions on $a$ and $b$ which guarantee that the critical point will
be a maximum. It is possible to see this without using calculus at
all; explain.
\end{exercise}

\end{exercises}





\section{The Second Derivative Test}

The basis of the first derivative test is that if the derivative
changes from positive to negative at a point at which the derivative
is zero then there is a local maximum at the point, and similarly for
a local minimum. If $f'$ changes from positive to negative it is
decreasing; this means that the derivative of $f'$, $f''$, might be negative,
and if in fact $f''$ is negative then $f'$ is definitely
decreasing, so there is a local maximum at the point in question. Note
well that $f'$ might change from positive to negative while $f''$ is
zero, in which case $f''$ gives us no information about the critical
value. Similarly, if $f'$ changes from negative to positive there is a
local minimum at the point, and $f'$ is increasing. If $f''>0$ at the
point, this tells us that $f'$ is increasing, and so there is a local
minimum. 

\begin{example}
Consider again $f(x)=\sin x + \cos x$, with $f'(x)=\cos x-\sin x$ and
$ f''(x)=-\sin x -\cos x$. Since $f''(\pi/4)=-\sqrt{2}/2-\sqrt2/2=-\sqrt2<0$,
we know there is a local maximum at $\pi/4$. Since
$f''(5\pi/4)=--\sqrt{2}/2--\sqrt2/2=\sqrt2>0$, there is a local
minimum at $5\pi/4$.
\end{example}

When it works, the second derivative test is often the easiest way to
identify local maximum and minimum points. Sometimes the test fails,
and sometimes the second derivative is quite difficult to evaluate; in
such cases we must fall back on one of the previous tests.

\begin{example}
Let $f(x)=x^4$. The derivatives are $f'(x)=4x^3$ and
$f''(x)=12x^2$. Zero is the only critical value, but $f''(0)=0$, so
the second derivative test tells us nothing. However, $f(x)$ is
positive everywhere except at zero, so clearly $f(x)$ has a local
minimum at zero. On the other hand, $f(x)=-x^4$ also has zero as its
only critical value, and the second derivative is again zero, but
$-x^4$ has a local maximum at zero.
\end{example}

\begin{exercises}
Find all local maximum and minimum points by the second derivative
test. 

\twocol
\begin{exercise} $y=x^2-x$ 
\begin{answer} min at $x=1/2$
\end{answer}\end{exercise}

\begin{exercise} $y=2+3x-x^3$ 
\begin{answer} min at $x=-1$, max at $x=1$
\end{answer}\end{exercise}

\begin{exercise} $y=x^3-9x^2+24x$
\begin{answer} max at $x=2$, min at $x=4$
\end{answer}\end{exercise}

\begin{exercise} $y=x^4-2x^2+3$ 
\begin{answer} min at $x=\pm 1$, max at $x=0$.
\end{answer}\end{exercise}

\begin{exercise} $y=3x^4-4x^3$
\begin{answer} min at $x=1$
\end{answer}\end{exercise}

\begin{exercise} $y=(x^2-1)/x$
\begin{answer} none
\end{answer}\end{exercise}

\begin{exercise} $y=3x^2-(1/x^2)$ 
\begin{answer} none
\end{answer}\end{exercise}

\begin{exercise} $y=\cos(2x)-x$ 
\begin{answer} min at $x=7\pi/12+n\pi$, max at $x=-\pi/12+n\pi$, for integer $n$.
\end{answer}\end{exercise}

\begin{exercise} $y = 4x+\sqrt{1-x}$
\begin{answer} max at $x=63/64$
\end{answer}\end{exercise}

\begin{exercise} $y = (x+1)/\sqrt{5x^2 + 35}$
\begin{answer} max at $x=7$
\end{answer}\end{exercise}

\begin{exercise} $y= x^5 - x$
\begin{answer} max at $-5^{-1/4}$, min at $5^{-1/4}$
\end{answer}\end{exercise}

\begin{exercise} $y = 6x +\sin 3x$
\begin{answer} none
\end{answer}\end{exercise}

\begin{exercise} $y = x+ 1/x$
\begin{answer} max at $-1$, min at $1$
\end{answer}\end{exercise}

\begin{exercise} $y = x^2+ 1/x$
\begin{answer} min at $2^{-1/3}$
\end{answer}\end{exercise}

\begin{exercise} $y = (x+5)^{1/4}$
\begin{answer} none
\end{answer}\end{exercise}

\begin{exercise} $y = \tan^2 x$
\begin{answer} min at $n\pi$
\end{answer}\end{exercise}

\begin{exercise} $y =\cos^2 x - \sin^2 x$
\begin{answer} max at $n\pi$, min at $\pi/2+n\pi$
\end{answer}\end{exercise}

\begin{exercise} $y = \sin^3 x$
\begin{answer} max at $\pi/2+2n\pi$, min at $3\pi/2+2n\pi$
\end{answer}\end{exercise}

\endtwocol
\end{exercises}

\section{Concavity and inflection points} {}{}
\nobreak
We know that the sign of the derivative tells us whether a function is
increasing or decreasing; for example, when $f'(x)>0$,
$f(x)$ is increasing. The sign of the second derivative
$f''(x)$ tells us whether $f'$ is increasing or decreasing; we have
seen that if $f'$ is zero and increasing at a point then there is a
local minimum at the point, and if $f'$ is zero and decreasing at a
point then there is a local maximum at the point. Thus, we extracted
information about $f$ from information about $f''$. 

We can get information from the sign of $f''$ even when $f'$ is not
zero. Suppose that $f''(a)>0$. This means that near $x=a$, $f'$ is
increasing. If $f'(a)>0$, this means that $f$ slopes up and is getting
steeper; if $f'(a)<0$, this means that $f$ slopes down and is getting
{\it less\/} steep. The two situations are shown in
figure~\xrefn{fig:concave up}. A curve that is shaped like this is
called {\dfont concave\index{concave up} up.}

% BADBAD
% \figure
% \vbox{\beginpicture
% \normalgraphs
% \ninepoint
% \setcoordinatesystem units <2truecm,2truecm>
% \setplotarea x from 0 to 1, y from 0 to 1
% \axis left shiftedto x=0 /
% \axis bottom shiftedto y=0 ticks withvalues {$a$} / at 0.5 / /
% \setquadratic
% \plot 0.1 0.2 0.5 0.4 0.9 0.9 /
% \setcoordinatesystem units <2truecm,2truecm> point at -2 0
% \setplotarea x from 0 to 1, y from 0 to 1
% \axis left shiftedto x=0 /
% \axis bottom shiftedto y=0 ticks withvalues {$a$} / at 0.5 / /
% \setquadratic
% \plot 0.1 0.9 0.5 0.3 0.9 0.1 /
% \endpicture}
% \figrdef{fig:concave up}
% \endfigure{$f''(a)>0$: $f'(a)$ positive and increasing, $f'(a)$ negative and
%   increasing.}

Now suppose that $f''(a)<0$. This means that near $x=a$, $f'$ is
decreasing. If $f'(a)>0$, this means that $f$ slopes up and is getting
less steep; if $f'(a)<0$, this means that $f$ slopes down and is getting
steeper. The two situations are shown in
figure~\xrefn{fig:concave down}. A curve that is shaped like this is
called {\dfont concave\index{concave down} down.}

% BADBAD
% \figure
% \vbox{\beginpicture
% \normalgraphs
% \ninepoint
% \setcoordinatesystem units <2truecm,2truecm>
% \setplotarea x from 0 to 1, y from 0 to 1
% \axis left shiftedto x=0 /
% \axis bottom shiftedto y=0 ticks withvalues {$a$} / at 0.5 / /
% \setquadratic
% \plot 0.1 0.2 0.5 0.7 0.9 0.9 /
% \setcoordinatesystem units <2truecm,2truecm> point at -2 0
% \setplotarea x from 0 to 1, y from 0 to 1
% \axis left shiftedto x=0 /
% \axis bottom shiftedto y=0 ticks withvalues {$a$} / at 0.5 / /
% \setquadratic
% \plot 0.1 0.9 0.5 0.6 0.9 0.1 /
% \endpicture}
% \figrdef{fig:concave down}
% \endfigure{$f''(a)<0$: $f'(a)$ positive and decreasing, $f'(a)$ negative and
%   decreasing.}

If we are trying to understand the shape of the graph of a function,
knowing where it is concave up and concave down helps us to get a more
accurate picture. Of particular interest are points at which the
concavity changes from up to down or down to up; such points are
called {\dfont inflection\index{inflection point} points.} If the
concavity changes from up to down at $x=a$, $f''$ changes from
positive to the left of $a$ to negative to the right of $a$, and
usually $f''(a)=0$. We can identify such points by first finding where
$f''(x)$ is zero and then checking to see whether $f''(x)$ does in
fact go from positive to negative or negative to positive at these
points. Note that it is possible that $f''(a)=0$ but the concavity is
the same on both sides; $f(x)=x^4$ at $x=0$ is an example.

\begin{example}
Describe the concavity of $f(x)=x^3-x$. $f'(x)=3x^2-1$, $f''(x)=6x$.
Since $f''(0)=0$, there is potentially an inflection point at
zero. Since $f''(x)>0$ when $x>0$ and $f''(x)<0$ when $x<0$ the
concavity does change from down to up at zero, and the curve is
concave down for all $x<0$ and concave up for all $x>0$.
\end{example}

Note that we need to compute and analyze the second derivative to
understand concavity, so we may as well try to use the second
derivative test for maxima and minima. If for some reason this fails
we can then try one of the other tests.

\begin{exercises}
Describe the concavity of the functions in 1--18.

\twocol

\begin{exercise} $y=x^2-x$ 
\begin{answer} concave up everywhere
\end{answer}\end{exercise}

\begin{exercise} $y=2+3x-x^3$ 
\begin{answer} concave up when $x<0$, concave down when $x>0$
\end{answer}\end{exercise}

\begin{exercise} $y=x^3-9x^2+24x$
\begin{answer} concave down when $x<3$, concave up when $x>3$
\end{answer}\end{exercise}

\begin{exercise} $y=x^4-2x^2+3$ 
\begin{answer} concave up when $x<-1/\sqrt3$ or $x>1/\sqrt3$,
concave down when $-1/\sqrt3<x<1/\sqrt3$
\end{answer}\end{exercise}

\begin{exercise} $y=3x^4-4x^3$
\begin{answer} concave up when $x<0$ or $x>2/3$,
concave down when $0<x<2/3$
\end{answer}\end{exercise}

\begin{exercise} $y=(x^2-1)/x$
\begin{answer} concave up when $x<0$, concave down when $x>0$
\end{answer}\end{exercise}

\begin{exercise} $y=3x^2-(1/x^2)$ 
\begin{answer} concave up when $x<-1$ or $x>1$, concave down when
$-1<x<0$ or $0<x<1$
\end{answer}\end{exercise}

\begin{exercise} $y=\sin x + \cos x$ 
\begin{answer} concave down on $((8n-1)\pi/4,(8n+3)\pi/4)$,
concave up on $((8n+3)\pi/4,(8n+7)\pi/4)$, for integer $n$
\end{answer}\end{exercise}

\begin{exercise} $y = 4x+\sqrt{1-x}$
\begin{answer} concave down everywhere
\end{answer}\end{exercise}

\begin{exercise} $y = (x+1)/\sqrt{5x^2 + 35}$
\begin{answer} concave up on $(-\infty,(21-\sqrt{497})/4)$ and 
$(21+\sqrt{497})/4,\infty)$
\end{answer}\end{exercise}

\begin{exercise} $y= x^5 - x$
\begin{answer} concave up on $(0,\infty)$
\end{answer}\end{exercise}

\begin{exercise} $y = 6x + \sin 3x$
\begin{answer} concave down on $(2n\pi/3,(2n+1)\pi/3)$
\end{answer}\end{exercise}

\begin{exercise} $y = x+ 1/x$
\begin{answer} concave up on $(0,\infty)$
\end{answer}\end{exercise}

\begin{exercise} $y = x^2+ 1/x$
\begin{answer} concave up on $(-\infty,-1)$ and $(0,\infty)$
\end{answer}\end{exercise}

\begin{exercise} $y = (x+5)^{1/4}$
\begin{answer} concave down everywhere
\end{answer}\end{exercise}

\begin{exercise} $y = \tan^2 x$
\begin{answer} concave up everywhere
\end{answer}\end{exercise}

\begin{exercise} $y =\cos^2 x - \sin^2 x$
\begin{answer} concave up on $(\pi/4+n\pi,3\pi/4+n\pi)$
\end{answer}\end{exercise}

\begin{exercise} $y = \sin^3 x$
\begin{answer} inflection points at $n\pi$, $\pm\arcsin(\sqrt{2/3})+n\pi$
\end{answer}\end{exercise}

\endtwocol

\msk \begin{exercise} Identify the intervals on which the graph of the function
$f(x) = x^4-4x^3 +10$ is of one of these four
shapes: concave up and increasing; concave up and decreasing; concave
down and increasing; concave down and decreasing.
\begin{answer} up/incr: $(3,\infty)$, up/decr: $(-\infty,0)$, $(2,3)$,
down/decr: $(0,2)$
\end{answer}\end{exercise}

\begin{exercise} Describe the concavity of $y =  x^3 + bx^2 + cx + d$.
You will need to consider different cases, depending on the values of
the coefficients.
\end{exercise}

\begin{exercise} Let $n$ be an integer greater than or equal to
two, and suppose $f$ is a polynomial of degree $n$. How many inflection points
can $f$ have?  Hint: Use the second derivative test and the
fundamental theorem of algebra.
\end{exercise}

% Mike Wills stuff
% \iflatetranscendentals
% 
% \begin{remark}{Definition} Let $f: I \to \R$ be a
% function. Let $a$ and $b$ be distinct points in $I$.  The
% {\dfont secant line\/} of $f$ from $a$ to $b$ is the (unique) line that
% passes through the points $(a,f(a))$ and $(b, f(b))$.
% 
% We now give a formal definition of concavity.
% 
% \end{remark}
% 
% \begin{remark}{Definition} 
% Let $I$ be an (open or closed) interval. Let $f:I \to \R$ be a given function.
% Suppose that for every $t$ in $[0,1]$ and every $a, b$ in $I$ the following inequality holds:
% $$ f(ta + (1-t)b ) \leq t f(a) + (1-t) f(b) .$$
% Then $f$ is said to  be {\dfont concave up\/} on $I$.
% If $g:I\to \R$ is a function such that $-g$ is concave up, then $g$
% is {\dfont concave down}.
% 
%  In more advanced texts concave up functions are called
%  {\dfont convex\/}, while concave down functions are called
%  {\dfont concave\/}. Additionally, several texts define concave
%  up/concave down in terms of tangent lines which presupposes that the
%  function is differentiable; taking that point of view, requiring a
%  function to be concave up is a stronger requirement than requiring
%  the function to be convex.
% 
% \end{remark}
% 
% \begin{exercise} Illustrate the definition of concave up with a picture.
% % (Hint: what goes the graph of
% %$h(t) = tf(a) + (1-t)f(b) $ look like?)
% Conclude that if a function is concave up on an interval then the
% graph of each secant line lies on or above the graph of the function.
% 
% \begin{exercise} It can be shown that if $f$ is concave up on an open
% interval then $f$ is continuous.  Give an example of a function that
% is concave up on the closed interval $[-1, 1]$ but fails to be
% continuous.  (Hint: note that the function must be continuous on $(-1,
% 1)$ so the function will be discontinuous only at one or both of the
% endpoints.
% 
% 
% \begin{exercise} Let $f(x) = mx+b$. Show that $f$ is concave up
% and concave down.
% \label{exer:linear concavity} 
% 
% \begin{exercise} Let $f(x) = ax^2 + bx +c$ with $a \neq 0$. Use the second
% derivative test to show that $f$ is concave up on all of $\R$
% if $a>0$ and that $f$ is concave down on all of $\R$ if $a<0$.
% \label{exer:quadratic concavity}
% 
% \begin{exercise} Give an example of a function $f$ on the interval $(-1,1)$ which is
% concave up but is not differentiable at $0$.
% 
% \begin{exercise} Let $f,g :I \to \R$ be concave up functions. Let $c\geq
% 0$. Show that $f+g$ and $cf$ are both concave up.
%  
% 
%  \begin{exercise} Let $f,g :I \to \R$ be concave up
%  functions. Show by means of an example that $fg$ need not be concave
%  up. Hint: Use exercises \xrefn{exer:linear concavity} and 
% \xrefn{exer:quadratic concavity}.
% 
% \fi

\end{exercises}

\section{Asymptotes and Other Things to Look For}  {}{}
\nobreak
A vertical asymptote\index{asymptote} is a place where the function
becomes infinite, typically because the formula for the function has a
denominator that becomes zero.  For example, the reciprocal function
$f(x)=1/x$ has a vertical asymptote at $x=0$, and the function $\tan
x$ has a vertical asymptote at $x=\pi/2$ (and also at $x=-\pi/2$,
$x=3\pi/2$, etc.).  Whenever the formula for a function contains a
denominator it is worth looking for a vertical asymptote by 
checking to see if the denominator can ever be zero, and then checking
the limit at such points. Note that there is not always a vertical
asymptote where the derivative is zero: $f(x)=(\sin x)/x$ has a zero
denominator at $x=0$, but since $\lim_{x\to 0}(\sin x)/x=1$ there is
no asymptote there.

A horizontal asymptote is a horizontal line to which $f(x)$ gets closer and
closer as $x$ approaches $\infty$ (or as $x$ approaches $-\infty$).  For
example, the reciprocal function has the $x$-axis for a horizontal
asymptote.  Horizontal asymptotes can be identified by computing 
the limits $\lim_{x \to \infty}f(x)$ and $\lim_{x \to -\infty}f(x)$.
Since $\lim_{x \to \infty}1/x=\lim_{x \to -\infty}1/x=0$, the line
$y=0$ (that is, the $x$-axis) is a horizontal asymptote in both directions.

Some functions have asymptotes that are neither horizontal nor
vertical, but some other line. Such asymptotes are somewhat more
difficult to identify and we will ignore them.

If the domain of the function does not extend out to infinity, we should
also ask what happens as $x$ approaches the boundary of the domain.  For
example, the function $y=f(x)=1/\sqrt{r^2-x^2}$ has domain $-r<x<r$, and
$y$ becomes infinite as $x$ approaches either $r$ or $-r$. In this
case we might also identify this behavior because when $x=\pm r$ the
denominator of the function is zero.

If there are any points where the derivative fails to exist (a cusp or
corner), then we should take special note of what the function does at such
a point.

Finally, it is worthwhile to notice any symmetry.  A function $f(x)$ that
has the same value for $-x$ as for $x$, i.e., $f(-x)=f(x)$, is called an
``even function.''  Its graph is symmetric with respect to the $y$-axis.
Some examples of even functions are: $x^n$ when $n$ is an even number,
$\cos x$, and $\sin^2x$.  On the other hand, a function that satisfies the
property $f(-x)=-f(x)$ is called an ``odd function.''  Its graph is
symmetric with respect to the origin.  Some examples of odd functions are:
$x^n$ when $n$ is an odd number, $\sin x$, and $\tan x$.  Of course, most
functions are neither even nor odd, and do not have any particular
symmetry.
 
\begin{exercises}

Sketch the curves. Identify clearly any interesting features, including
local maximum and minimum points, inflection points, asymptotes, and
intercepts. 

\twocol

\begin{exercise} $y=x^5-5x^4+5x^3$
\end{exercise}

\begin{exercise} $y=x^3-3x^2-9x+5$
\end{exercise}

\begin{exercise} $y=(x-1)^2(x+3)^{2/3}$
\end{exercise}

\begin{exercise} $x^2+x^2y^2=a^2y^2$, $a>0$.
\end{exercise}

\begin{exercise} $y = 4x+\sqrt{1-x}$
\end{exercise}

\begin{exercise} $y = (x+1)/\sqrt{5x^2 + 35}$
\end{exercise}

\begin{exercise} $y= x^5 - x$
\end{exercise}

\begin{exercise} $y = 6x + \sin 3x$
\end{exercise}

\begin{exercise} $y = x+ 1/x$
\end{exercise}

\begin{exercise} $y = x^2+ 1/x$
\end{exercise}

\begin{exercise} $y = (x+5)^{1/4}$
\end{exercise}

\begin{exercise} $y = \tan^2 x$
\end{exercise}

\begin{exercise} $y =\cos^2 x - \sin^2 x$
\end{exercise}

\begin{exercise} $y = \sin^3 x$
\end{exercise}

\begin{exercise} $y=x(x^2+1)$
\end{exercise}

\begin{exercise} $y=x^3+6x^2 + 9x$
\end{exercise}

\begin{exercise} $y=x/(x^2-9)$
\end{exercise}

\begin{exercise} $y=x^2/(x^2+9)$
\end{exercise}

\begin{exercise} $y=2\sqrt{x} - x$
\end{exercise}

\begin{exercise} $y=3\sin(x) - \sin^3(x)$, for $x\in[0,2\pi]$
\end{exercise}

\begin{exercise} $y=(x-1)/(x^2)$
\end{exercise}

\endtwocol

\msk
\noindent
For each of the following five functions, identify any vertical and horizontal
asymptotes, and identify intervals on which the function is 
concave up and increasing; concave up and decreasing; concave
down and increasing; concave down and decreasing.

\begin{exercise} $f(\theta)=\sec(\theta)$ \end{exercise}
\begin{exercise} $f(x) = 1/(1+x^2)$\end{exercise}
\begin{exercise} $f(x) = (x-3)/(2x-2)$ \end{exercise}
\begin{exercise} $f(x) = 1/(1-x^2)$\end{exercise}
\begin{exercise} $f(x) = 1+1/(x^2)$\end{exercise}

\begin{exercise} Let $f(x) = 1/(x^2-a^2)$, where $a\geq0$.  Find any
 vertical and horizontal asymptotes and the intervals upon which the
 given function is concave up and increasing; concave up and
 decreasing; concave down and increasing; concave down and decreasing.
 Discuss how the value of $a$ affects these features.
\end{exercise}



\end{exercises}

\endinput

% Example 1
\begin{example}

Sketch $y=x^3-x$.

\smallskip
\begin{rightindent}{3.5in}
First, we set $0=y'=3x^2-1$, which has solutions $x=\pm\sqrt{3}/3=\pm
0.577$.  The corresponding $y$-coordinates are $-0.385$ and $+0.385$, i.e.,
the two ``critical points'' are $(0.577,-0.385)$ and $(-0.577,0.385)$.  The
second derivative test gives $y''=6x$, which is positive for the first
point and negative for the second.  Thus, the first of the two points (in
the fourth quadrant) is a local minimum, and the second is a local maximum.
Since $y''>0$ when $x>0$ and $y''<0$ when $x<0$, it follows that the curve
is concave upward in the right half of the graph and concave down-
\end{rightindent}
\hfill
%--- figure ND (curve sketching--Example 1)--------------------
\begin{psfigure}{1.95in}{2.0in}{124.figND.ps}
\end{psfigure}
%------------------------------
ward in the left half. The two halves are
separated by the inflection point at the origin.  This curve has no
asymptotes.  It does have symmetry, however, because it is an odd function.
Its graph is shown above.

\end{example}

\begin{rightindent}{3in}
% Example 2
\begin{example}

Sketch $y=x^3+x$.

\smallskip

The only difference with Example 13.1 is the $+$ in front of the $x$.  But
this means that the derivative $y'=3x^2+1$ is {\it never} zero, and hence
there are no maxima or minima.  In fact, the function is always increasing,
because $y'$ is always positive.  The second derivative $y''=6x$ is the
same as in the last problem, and hence the concavity situation is the same.
In particular, this curve also has an inflection point at the origin.

\end{example}
\end{rightindent}
\hfill
%--- figure NE (curve sketching--Example 2)--------------------
\begin{psfigure}{2in}{2in}{124.figNE.ps}
\end{psfigure} 
%------------------------------

% Example 3
\begin{example}

Sketch $y=x^2-\cos(2x)$ for $-\pi/2\le x\le\pi/2$.

\smallskip

When we set $0=y'=2x+2\sin(2x)$, we obtain the equality $\sin(2x)=-x$.
However, from a quick sketch of the two curves $\sin(2x)$ and $-x$, we
immediately see that the only $x$ for which they are equal is $x=0$.  When
$x=0$ the $y$-coordinate is $0^2-\cos(2\cdot 0)=-1$, so our critical point
is $(0,-1)$.  Since $y''=2+4\cos(2x)$, which is positive when $x=0$, the
second derivative test tells us that $(0,1)$ is a local minimum.  To find
inflection points we set $0=y''=2+4\cos(2x)$.  This gives $\cos(2x)=-0.5$.
Looking at our table in the section on trig functions, we see that in the
range from $x=0$ to $x=\pi/2$ the equality $\cos(2x)=-0.5$ holds when
$2x=\frac{2}{ 3}\pi$, 
\begin{rightindent}{3in}
\noindent
i.e., $x=\pi/3$.  Since $\cos(-2x)=\cos(2x)$, the
equality also holds when $x=-\pi/3$.  Thus, the points $(\pi/3,1.5966)$ and
$(-\pi/3,1.5966)$ are inflection points.  Between these two inflection
points the second derivative is positive (concave up), whereas for
$x>\pi/3$ and for $x<-\pi/3$ the second derivative is negative (concave
downward).  Finally, note that $x^2-\cos(2x)$ is an even function, and so
the graph is symmetrical with respect to the $y$-axis.
\end{rightindent}
%--- figure NF (curve sketching--Example 3)--------------------
\begin{psfigure}{2.0in}{1.8in}{124.figNF.ps}
\end{psfigure}
%------------------------------
\end{example}

% Example 4
\begin{example}

Sketch $y=x^2+\frac{1}{ x}$.

\smallskip

Setting $0=y'=2x-\frac{1}{ x^2}$, we solve this by bringing $1/x^2$ to the
left and clearing denominators: $1=2x\cdot x^2=2x^3$.  So $x={\root
3\of{0.5}}= 0.7937$.  The corresponding $y$-coordinate is 1.8899.  Using
the second derivative $y''=2+2\cdot x^{-3}$, we see that this 
\begin{rightindent}{3in}
\noindent
is a local minimum.  To find inflections, we set $2+2/x^3=0$.  Clearing
denominators and solving for $x$ gives $x^3=-1$, and so $x=-1$.  Thus, the
point $(-1,0)$ is an inflection.  In this example we have an asymptote when
$x=0$.  To the right of the asymptote (i.e., for positive $x$), the second
derivative is always positive; whereas for negative $x$ the second
derivative is positive when $x<-1$ and negative when $x$ is between $-1$
and $0$.  Thus, the interval $-1<x<0$ is a region of downward concavity;
the graph is concave upward outside of this interval.  Putting all this
together leads to the graph at the right.
\end{rightindent}
\hfill
%--- figure NG (curve sketching--Example 4)--------------------
%
\begin{psfigure}{2.0in}{2.0in}{124.figNG.ps}
\end{psfigure}
%------------------------------

\smallskip

There is another way to think of this example.  Our function is the sum of
two functions $x^2$ and $1/x$.  The former function is by far the larger of
the two when $x$ is large positive or large negative, whereas the
reciprocal function is by far the more important when $x$ is near 0.  Thus,
the graph resembles $1/x$ when $x$ is near 0 and resembles $x^2$ when $x$
is far from 0.  Roughly speaking, the inflection point $(-1,0)$ and the
local minimum $(0.7937,1.8899)$ mark the transition from behaving like the
graph of $x^2$ to behaving like the graph of $1/x$.

\end{example}

% Example 5
\begin{example}

Sketch (a) $y=x^3$ and (b) $y=x^4$.

\smallskip


(a) Setting $0=y'=3x^2$, we see that the origin is a possible maximum or
minimum.  However, the second derivative test tells us nothing, since
$y''=6x$ also is zero when $x=0$.  In fact, even though $y'=0$ when $x=0$,
the origin is neither a maximum nor a minimum.  Rather, it is a point of
inflection, separating the concave downward region in the third quadrant
from the concave upward region in the first quadrant.

\smallskip


(b) Again we see that both the first derivative and the second derivative
vanish at the origin (and neither derivative is zero anywhere else).  This
time, however, the origin is a local minimum.  Even though the second derivative
test doesn't tell us this, we can see directly that, since $x^4$ is
positive for nonzero $x$, its smallest possible value is when $x=0$.  Note
that the origin is {\it not} a point of inflection, even though $y''=0$
there.  This is because $y''=12x^2>0$ both for $x>0$ and for $x<0$, so
everywhere we have upward concavity.  (It is rare for a point where $y''=0$
not to be an inflection point; this can occur only when the {\it third}
derivative $y^{\prime\prime\prime}$ is also zero at the same point.)

\begin{centering}
%--------------------------------------------------------------
%--- figure NH (curve sketching--Example 5)--------------------
\begin{psfigure}{4.5in}{1.5in}{124.figNH.ps}
\end{psfigure}\\
%------------------------------
\end{centering}

\end{example}


% Example 6
\begin{example}

Sketch $y=x^5-5x^4+5x^3$.

\smallskip

First, we set $0=y'=5x^4-20x^3+15x^2$.  To solve this, we factor out what
we can, namely $5x^2$.  This leaves a quadratic that can be factored either
by inspection or by the quadratic formula.  The result is
$0=y'=5x^2(x-1)(x-3)$.  Thus, the critical points are $(0,0)$, $(1,1)$, and
$(3,-27)$.  Using $y''=20x^3-60x^2+30x=10x(2x^2-6x+3)$, we see from the
second derivative test that $(1,1)$ is a local maximum and $(3,-27)$ is a
local minimum, but we get no information about $(0,0)$.  Setting $0=y''$
and using the quadratic formula to find the roots of $2x^2-6x+3$, we find
the following three points of inflection: $(0,0)$, $(0.634,0.569)$,
$(2.366,-16.32)$.

\begin{rightindent}{3.0in}
\noindent
  In a complicated case like
this, it is also worthwhile to see what the function is doing when $x$ is
large positive or large negative.  If $x$ is large, the $x^5$ term in our
function dominates (is greater in absolute value than all the other terms).
Thus, the function heads upward steeply into the first quadrant as
$x\longrightarrow+\infty$, and it heads steeply down into the third
quadrant as $x\longrightarrow-\infty$.  Putting this information together,
we obtain the graph shown above.  The curve is concave downward in the
third quadrant, and also between the two points of inflection
$(0.634,0.569)$ and $(2.366,-16.32)$.  For $0<x<0.634$ and for $x>2.366$
the curve is concave upward.
\end{rightindent}
\hfill
%--- figure NI (curve sketching--Example 6)--------------------
%
\begin{psfigure}{2.45in}{2.5in}{124.figNI.ps}
\end{psfigure}
%------------------------------

\end{example}

\newpage

%
% Homework on curve sketching
%
\begin{homework}\
\label{homework:13}

\medskip

\begin{exercise} %1.
Suppose that $n$ is an integer greater than 2.  On the curve $y=f(x)=x^n$,
what sort of point is the origin?  Sketch the curve, and indicate the
concavity.

\smallskip

\begin{exercise} %2.
In parts (a)-(j) below, find each max/min point (using the second
derivative test to be sure what type of point it is), point of inflection,
and asymptote (vertical, horizontal, or slanted), if there are any.  Also
indicate where the curve is concave up and concave down.  Sketch the graph.

\noindent
(a) $y=x^2-x$, (b) $y=2+3x-x^3$, (c) $y=x^3-9x^2+24x$,
(d) $y=x^4-2x^2+3$, (e) $y=3x^4-4x^3$,
(f) $y=(x^2-1)/x$,
(g) $y=3x^2-(1/x^2)$, (h) $y=\cos(2x)-x$, (i) $y=\sin x+\cos x$,
(j) $y=\tan(x/2)-x$.

\smallskip

\begin{exercise} %3.
Suppose  $\begin{cases} f'(x) > 0& \text{for $|x| > 2$;}\\
                                  f'(x) < 0& \text{for $|x| < 2$;}\\
                                  f''(x) < 0& \text{for $x  < 0$; and }\\
                                  f''(x) > 0& \text{for  $x  > 0$}\end{cases}$.

\noindent
From the given information, sketch a possible graph of $f(x)$.
How could your answer vary and still be correct?


\smallskip

\begin{exercise} %4.
Suppose $f''(x) <0$ for $x<1$ and  $f''(x)>0$ for $x>1$.
Suppose also that $f(1)=1$.  Sketch a possible graph of $f(x)$, assuming that

         i) $f'(1) = 0$,   ii) $f'(1) > 0$ and  iii) $f'(1) < 0$.

\smallskip

\begin{rightindent}{3.4in}
\begin{exercise} %5.
At the right is a sketch of $y=x^6-5x^4$.

(a) Find the exact coordinates $(x,y)$ of the local maxima and local
minima.

(b) For what values of $x$ is the function concave upward?
									
(c) Find the exact $x$-coordinates of all points of inflection. 
({\bf Hint:} Color the part of the curve that is concave up blue and color
the part that is concave down red. Points of inflection occur {\it only}
where the curve {\it changes} color!)
\end{rightindent}
\hfill
\begin{psfigure}{2.0in}{2.0in}{124.figNJ.ps}
\end{psfigure}

(d) Explain how you {\it know from your calculations} (not from the
sketch you were given) which of your answers to part (a) are minima and
which are maxima.


\end{homework}


%-----------------end of Chapter 13 (curve sketching) -----



