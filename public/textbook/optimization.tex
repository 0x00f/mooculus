\chapter{Optimization}




Many important applied problems involve finding the best way to
accomplish some task. Often this involves finding the maximum or
minimum value of some function: The minimum time to make a certain
journey, the minimum cost for doing a task, the maximum power that can
be generated by a device, and so on. Many of these problems can be
solved by finding the appropriate function and then using techniques
of calculus to find the maximum or the minimum value required.


\section{Maximum and Minimum Values of Curves}

We already know how to find local extrema. We wish to find
\textit{absolute extrema}.

\marginnote{It is common to leave off the word ``absolute'' when
  asking for absolute extrema. Hence a ``maximum'' or a ``minimum''
  refers to an absolute extremum. On the other hand, local extrema are
  always specified as such.}
\begin{definition}\hfil\index{maximum/minimum!absolute}
\begin{enumerate}
\item A point $(x,f(x))$ is an \textbf{absolute maximum} on an interval
  if $f(x)\ge f(z)$ for every $z$ in that interval.
\item A point $(x,f(x))$ is an \textbf{absolute minimum} on an interval if
  $f(x)\le f(z)$ for every $z$ in that interval.
\end{enumerate}
A \textbf{absolute extremum}\index{extremum!absolute} is either an
absolute maximum or an absolute minimum.
\end{definition}

If we are working on an finite closed interval, then we have the
following theorem.

\begin{mainTheorem}[Extreme Value Theorem]\label{theorem:evt}\index{Extreme Value Theorem}
If $f(x)$ is a continuous function for all $x$ in the closed interval
$[a,b]$, then there are points $c$ and $d$ in $[a,b]$, such that
$(c,f(c))$ is an absolute maximum and $(d,f(d))$ is an absolute
minimum on $[a, b]$.
\end{mainTheorem}
In Figure~\ref{figure:extreme-value}, we see a geometric
interpretation of this theorem. 

\begin{marginfigure}
\begin{tikzpicture}
	\begin{axis}[
            domain=0:6, xmin=0, xmax=6, ymin=0, ymax=2.5,
            axis lines =left, xlabel=$x$, ylabel=$y$,
            every axis y label/.style={at=(current axis.above origin),anchor=south},
            every axis x label/.style={at=(current axis.right of origin),anchor=west},
            xtick={1,2,4,5}, ytick={.2,2.2},
            xticklabels={$a$,$c$,$d$,$b$}, yticklabels={$f(d)$,$f(c)$},
            axis on top,
          ]
          \addplot [draw=none, fill=fill2, domain=(1:5)] {2.5} \closedcycle;

          \addplot [textColor,dashed] plot coordinates {(0,2.2) (2,2.2)};
          \addplot [textColor,dashed] plot coordinates {(0,.2) (4,.2)};
          \addplot [textColor,dashed] plot coordinates {(2,0) (2,2.2)};
          \addplot [textColor,dashed] plot coordinates {(4,0) (4,.2)};

          \addplot [fill1,very thick] plot coordinates {(1,0) (1,2.5)};
          \addplot [fill1,very thick] plot coordinates {(5,0) (5,2.5)};

          \addplot [very thick,penColor, smooth,domain=(1.5:2.5)] {sin(deg(x*1.57-1.57)) + 1.2};%max
          \addplot [very thick,penColor, smooth,domain=(3.5:4.5)] {sin(deg(x*1.57-1.57)) + 1.2};%min
          \addplot [very thick,dashed,penColor!50!background, smooth,domain=(2.5:3.5)] {sin(deg(x*1.57 - 1.57)) + 1.2}; 
          \addplot [very thick,dashed,penColor!50!background, smooth,domain=(1:1.5)] {sin(deg(x*1.57 - 1.57)) + 1.2}; 
          \addplot [very thick,dashed,penColor!50!background, smooth,domain=(4.5:5)] {sin(deg(x*1.57 - 1.57)) + 1.2}; 
          
          \addplot [color=penColor,fill=penColor,only marks,mark=*] coordinates{(2,2.2)};  %% closed hole          
          \addplot [color=penColor,fill=penColor,only marks,mark=*] coordinates{(4,.2)};  %% closed hole          
        \end{axis}
\end{tikzpicture}
\caption{A geometric interpretation of the Extreme Value Theorem. A
  continuous function $f(x)$ attains both an absolute maximum and an
  absolute minimum on an interval $[a,b]$. Note, it may be the case
  that $a=c$, $b=d$, or that $d<c$.}
\label{figure:extreme-value}
\end{marginfigure}

\begin{example} 
Find the (absolute) maximum and minimum values of $f(x)=x^2$ on the
interval $[-2,1]$.
\end{example}

\begin{solution}
To start, write
\[
\ddx x^2 = 2x.
\]
The critical point is at $x=0$. By the Extreme Value Theorem,
Theorem~\ref{theorem:evt}, we must also consider the endpoints of the
closed interval, $x=-2$ and $x=1$. Check
\[
f(-2) = 4, \qquad f(0) = 0, \qquad f(1) = 1.
\]
So on the interval $[-2,1]$, the absolute maximum of $f(x)$ is 4 at
$x=-2$ and the absolute minimum is 0 at $x=0$, see Figure~\ref{figure:evt x^2}.
\end{solution}\begin{marginfigure}
\begin{tikzpicture}
	\begin{axis}[
            domain=-3:2, ymin=-1, ymax=6, xmin=-3, xmax=2,
            axis lines =center, xlabel=$x$, ylabel=$y$,
            every axis y label/.style={at=(current axis.above origin),anchor=south},
            every axis x label/.style={at=(current axis.right of origin),anchor=west},
            axis on top,
          ]
          \addplot [draw=none, fill=fill2, domain=(-2:1)] {6} \closedcycle;
          \addplot [draw=none, fill=fill2, domain=(-2:1)] {-1} \closedcycle;
          
          \addplot [fill1,very thick] plot coordinates {(-2,-1) (-2,6)};
          \addplot [fill1,very thick] plot coordinates {(1,-1) (1,6)};

          \addplot [very thick,penColor, smooth,domain=(-2:1)] {x^2};
                    
          \addplot [color=penColor,fill=penColor,only marks,mark=*] coordinates{(-2,4)};  %% closed hole          
          \addplot [color=penColor,fill=penColor,only marks,mark=*] coordinates{(0,0)};  %% closed hole          
        \end{axis}
\end{tikzpicture}
\caption{A plot of the function $f(x) = x^2$ on the interval $[-2,1]$.}
\label{figure:evt x^2}
\end{marginfigure}

It is possible that there is no global maximum or minimum. It is
difficult, and not particularly useful, to express a complete procedure for
determining whether this is the case. Generally, the best approach is
to gain enough understanding of the shape of the graph to
decide. 

\begin{example}
Find the (absolute) maximum and minimum values of the function $f(x)=
|x-2|$ on the interval $[1,4]$.
\end{example}

\begin{solution}
To start, rewrite $f(x)$ as 
\[
f(x) = \sqrt{(x-2)^2},
\]
now
\[
\ddx f(x) = \frac{2(x-2)}{2\sqrt{(x-2)^2}}= \frac{x-2}{|x-2|}.
\]
The derivative $f'(x)$ is never zero, but $f'(x)$ is undefined at
$x=2$, so we have a critical point at $x=2$. Compute $f(2)=
0$. Checking the endpoints we get $f(1)=1$ and $f(4)=2$. The smallest
of these numbers is $f(2)=0$, which is, therefore, the minimum value
of $f(x)$ on the interval and the maximum is $f(4)=2$, see Figure~\ref{figure:evt |x-2|}.
\end{solution}\begin{marginfigure}
\begin{tikzpicture}
	\begin{axis}[
            domain=-3:2, ymin=-1, ymax=3, xmin=0, xmax=5,
            axis lines =center, xlabel=$x$, ylabel=$y$,
            every axis y label/.style={at=(current axis.above origin),anchor=south},
            every axis x label/.style={at=(current axis.right of origin),anchor=west},
            axis on top,
          ]
          \addplot [draw=none, fill=fill2, domain=(1:4)] {3} \closedcycle;
          \addplot [draw=none, fill=fill2, domain=(1:4)] {-1} \closedcycle;
          
          \addplot [fill1,very thick] plot coordinates {(1,-1) (1,3)};
          \addplot [fill1,very thick] plot coordinates {(4,-1) (4,3)};

          \addplot [very thick,penColor,domain=(1:4)] {abs(x-2)};
                    
          \addplot [color=penColor,fill=penColor,only marks,mark=*] coordinates{(4,2)};  %% closed hole          
          \addplot [color=penColor,fill=penColor,only marks,mark=*] coordinates{(2,0)};  %% closed hole          
        \end{axis}
\end{tikzpicture}
\caption{A plot of the function $f(x) = |x-2|$ on the interval $[1,4]$.}
\label{figure:evt |x-2|}
\end{marginfigure}

\begin{warning}
The Extreme Value Theorem, Theorem~\ref{theorem:evt}, requires that
the function in question be \textbf{continuous} on a \textbf{closed}
interval.  For example consider $f(x)=\tan(x)$ on $(-\pi/2,
\pi/2)$. In this case, the function is continuous on $(-\pi/2,\pi/2)$,
but the interval is not closed. Hence, the Extreme Value Theorem
\textbf{does not apply}, see Figure~\ref{figure:NOevt tan}.
\end{warning}\begin{marginfigure}
\begin{tikzpicture}
	\begin{axis}[
            domain=-2:2, ymin=-10, ymax=10, xmin=-2, xmax=2,
            axis lines =center, xlabel=$x$, ylabel=$y$,
            xtick={-1.57,1.57},
            xticklabels={$-\pi/2$,$\pi/2$},
            every axis y label/.style={at=(current axis.above origin),anchor=south},
            every axis x label/.style={at=(current axis.right of origin),anchor=west},
            axis on top,
          ]
          \addplot [draw=none, fill=fill2, domain=(-1.57:1.57)] {10} \closedcycle;
          \addplot [draw=none, fill=fill2, domain=(-1.57:1.57)] {-10} \closedcycle;
          
          \addplot [very thick,penColor,samples=100,smooth,domain=(-1.54:1.54)] {tan(deg(x))};
        \end{axis}
\end{tikzpicture}
\caption{A plot of the function $f(x) = \tan(x)$ on the interval
  $(-\pi/2,\pi/2)$. Here the Extreme Value Theorem does not apply.}
\label{figure:NOevt tan}
\end{marginfigure}

Finally, if there are several critical points in the interval, then
the mathematician might want to use the second derivative test,
Theorem~\ref{T:sdt}, to identify if the critical points are local
maxima or minima, rather than simply evaluating the function at these
points. Regardless, it depends on the situation, and we will leave it
up to you---our capable reader.





\begin{exercises}
\noindent Find the maximum value and minimum values of $f(x)$ for $x$
on the given interval.

\begin{exercise}
$f(x) = x-2x^2$ on the interval $[0,1]$
\begin{answer}
max at $(1/4,1/8)$, min at $(1,-1)$
\end{answer}
\end{exercise}

\begin{exercise}
$f(x) = x-2x^3$ on the interval $[-1,1]$
\begin{answer}
max at $(-1,1)$, min at $(1,-1)$
\end{answer}
\end{exercise}

\begin{exercise}
$f(x) = x^3-6x^2+12x-8$ on the interval $[1,3]$
\begin{answer}
max at $(3,1)$, min at $(1,-1)$
\end{answer}
\end{exercise}

\begin{exercise}
$f(x) = -x^3-3x^2-2x$ on the interval $[-2,0]$
\begin{answer}
max at $(-1+1/\sqrt{3},2/(3\sqrt{3}))$, min at
$(-1-1/\sqrt{3},-2/(3\sqrt{3}))$
\end{answer}
\end{exercise}

\begin{exercise}
$f(x) = \sin^2(x)$ on the interval $[\pi/4,5\pi/3]$
\begin{answer}
max at $(3\pi/2,1)$, min at
$(\pi,0)$
\end{answer}
\end{exercise}

\begin{exercise}
$f(x) = \arctan(x)$ on the interval $[-\pi,\pi]$
\begin{answer}
max at $(\pi,\pi/4)$, min at
$(-\pi,-\pi/4)$
\end{answer}
\end{exercise}

\begin{exercise}
$f(x) = e^{\sin(x)}$ on the interval $[-\pi,\pi]$
\begin{answer}
max at $(\pi/2,e)$, min at
$(-\pi/2,1/e)$
\end{answer}
\end{exercise}

\begin{exercise}
$f(x) = \ln(\cos(x))$ on the interval $[-\pi/6,\pi/3]$
\begin{answer}
max at $(0,0)$, min at
$(\pi/3,-\ln(2)$
\end{answer}
\end{exercise}

\begin{exercise}
$f(x) = \begin{cases} 1 + 4 x -x^2 & \text{if  $x\leq 3$}, \\ 
(x+5)/2 &\text{if $x>3$},
\end{cases}$ on the interval $[0,4]$
\begin{answer} max at $(2,5)$, min at $(0,1)$
\end{answer}\end{exercise}


\begin{exercise}
$f(x) = \begin{cases} (x+5)/2 &\text{if $x<3$},\\
1 + 4 x -x^2 & \text{if  $x\ge 3$},
\end{cases}$ on the interval $[0,4]$
\begin{answer} max at $(3,4)$, min at $(0,5/2)$
\end{answer}\end{exercise}

%% two piece wise

\end{exercises}

















\section{Basic Optimization Problems}


In this section, we will present several worked examples of
optimization problems. Our method for solving these problems is
essentially the following:


\begin{guidelinesForOptimization}\hfil
\begin{itemize}
\item[\textbf{Draw a picture.}] If possible, draw a schematic picture with all the relevant information. 
\item[\textbf{Determine your goal.}] We need identify what needs to be
  optimized.
\item[\textbf{Find constraints.}] What limitations are set on our
  optimization?
\item[\textbf{Solve for a single variable.}] Now you should have a function to optimize.
\item[\textbf{Use calculus to find the extreme values.}] Be sure to check your answer!
\end{itemize}
\end{guidelinesForOptimization}

\begin{example}
Of all rectangles of area $100$ cm$^3$, which has the smallest
perimeter?
\end{example}

\begin{marginfigure}
\begin{tikzpicture}
\draw [penColor,very thick,fill=fill2] (0,0) rectangle (5,4);
\node [penColor] at (2.5,2) {$A=100$ cm$^3$};
\node [right,penColor] at (5,2) {$\frac{100}{x}$ cm};
\node [below,penColor] at (2.5,0) {$x$ cm};
\end{tikzpicture}
\caption{A rectangle with an area of $100$ cm$^3$.}
\label{fig:rect100}
\end{marginfigure}

\begin{solution}
First we draw a picture, see Figure~\ref{fig:rect100}.  If $x$ denotes
one of the sides of the rectangle, then the adjacent side must be
$100/x$.


The perimeter of this rectangle is given by
\[
p(x)=2x+2\frac{100}{x}.
\]
We wish to maximize $p(x)$.  Note, not all values of $x$ make sense in this
problem: lengths of sides of rectangles must be positive, so $x>0$. If
$x>0$ then so is $100/x$, so we need no second condition on $x$.

We next find $p'(x)$ and set it equal to zero. Write
\[
p'(x)=2-200/x^2 = 0.
\]
Solving for $x$ gives us $x=\pm 10$. We are interested only in $x>0$,
so only the value $x=10$ is of interest. Since $p'(x)$ is defined
everywhere on the interval $(0,\infty)$, there are no more critical
values, and there are no endpoints. Is there a local maximum, minimum,
or neither at $x=10$? The second derivative is $p''(x)=400/x^3$, and
$f''(10)>0$, so there is a local minimum. Since there is only one
critical value, this is also the global minimum, so the rectangle with
smallest perimeter is the $10$ cm$\times10$ cm square.
\end{solution}

\begin{example}
You want to sell a certain number $n$ of items in order to maximize your
profit.  Market research tells you that if you set the price at \$$1.50$, you
will be able to sell $5000$ items, and for every $10$ cents you lower the price
below \$$1.50$ you will be able to sell another $1000$ items.  Suppose that
your fixed costs (``start-up costs'') total \$$2000$, and the per item cost
of production (``marginal cost'') is \$$0.50$.  Find the price to set per
item and the number of items sold in order to maximize profit, and also
determine the maximum profit you can get.
\end{example}

\begin{solution}
The first step is to convert the problem into a function maximization
problem. The revenue for selling $n$ items at $x$ dollars is given by
\[
r(x) = nx
\]
and the cost of producing $n$ items is given by
\[
c(x) = 2000+0.5 n. 
\]
However, from the problem we see that the number of items sold is
itself a function of $x$,
\[
n(x) =5000+1000(1.5-x)/0.10
\]
So profit is give by:
\begin{align*}
P(x) &= r(x) - c(x)\\
&= nx - (2000+0.5 n)\\
&= (5000+1000(1.5-x)/0.10)x - 2000 - 0.05 (5000+1000(1.5-x)/0.10)\\
&=-10000x^2+25000x-12000. 
\end{align*}
We want to know the maximum value of this function when $x$ is between
0 and $1.5$. The derivative is $P'(x)=-20000x+25000$, which is zero
when $x=1.25$. Since $P''(x)=-20000<0$, there must be a local maximum
at $x=1.25$, and since this is the only critical value it must be a
global maximum as well. Alternately, we could compute $P(0)=-12000$,
$P(1.25)=3625$, and $P(1.5)=3000$ and note that $P(1.25)$ is the
maximum of these. Thus the maximum profit is \$$3625$, attained when we
set the price at \$$1.25$ and sell $7500$ items. 
\end{solution}


\begin{example} 
Find the rectangle with largest area that fits inside the graph of the
parabola $y=x^2$ below the line $y=a$, where $a$ is an unspecified
constant value, with the top side of the rectangle on the horizontal
line $y=a$. See Figure~\ref{fig:rectangle parabola}.
\end{example}

\begin{marginfigure}
\begin{tikzpicture}
	\begin{axis}[
            domain=-3:3, ymin=0, ymax=9, xmin=-3, xmax=3,
            axis lines =center, xlabel=$x$, ylabel=$y$,
            ticks=none,
            every axis y label/.style={at=(current axis.above origin),anchor=south},
            every axis x label/.style={at=(current axis.right of origin),anchor=west},
            axis on top,
          ]
          \addplot [draw=none, fill=fill2, domain=(-1.5:1.5)] {7} \closedcycle;
          \addplot [draw=none, fill=background, domain=(-1.5:1.5)] {2.25} \closedcycle;

          \addplot [very thick,penColor2,domain=(-3:3)] {x^2};
          \addplot [very thick,penColor5,domain=(-3:3)] {7};

          \addplot [very thick, penColor] plot coordinates {(1.5,2.25) (-1.5,2.25)};
          \addplot [very thick, penColor] plot coordinates {(1.5,2.25) (1.5,7)};
          \addplot [very thick, penColor] plot coordinates {(1.5,7) (-1.5,7)};
          \addplot [very thick, penColor] plot coordinates {(-1.5,7) (-1.5,2.25)};
          
          \addplot [color=penColor,fill=penColor,only marks,mark=*] coordinates{(1.5,2.25)};  %% closed hole          
          \addplot [color=penColor,fill=penColor,only marks,mark=*] coordinates{(-1.5,2.25)};  %% closed hole          
          \addplot [color=penColor,fill=penColor,only marks,mark=*] coordinates{(1.5,7)};  %% closed hole          
          \addplot [color=penColor,fill=penColor,only marks,mark=*] coordinates{(-1.5,7)};  %% closed hole   

          \node at (axis cs:0,4.625) [penColor] {$A(x) =$~area};
          \node at (axis cs:1.5,2.25) [anchor=west,penColor] {$(x,x^2)$};
          \node at (axis cs:1.5,7.2) [anchor=west,penColor] {$(x,a)$};
        \end{axis}
\end{tikzpicture}
\caption{A plot of the parabola $y=x^2$ along with the line $y=a$ and the rectangle in question.}
\label{fig:rectangle parabola}
\end{marginfigure}

\begin{solution}
We want to maximize value of $A(x)$.  The lower right corner of the
rectangle is at $(x,x^2)$, and once this is chosen the rectangle is
completely determined. Then the area is
\[
A(x)=(2x)(a-x^2)=-2x^3+2ax.
\] 
We want the maximum value of $A(x)$ when $x$ is in $[0,\sqrt{a}]$. You
might object to allowing $x=0$ or $x=\sqrt{a}$, since then the
``rectangle'' has either no width or no height, so is not ``really'' a
rectangle. But the problem is somewhat easier if we simply allow such
rectangles, which have zero area as we may then apply the Extreme
Value Theorem, Theorem~\ref{theorem:evt}.

Setting $0=A'(x)=-6x^2+2a$ we find $x=\sqrt{a/3}$ as the only critical
point. Testing this and the two endpoints, we have
$A(0)=A(\sqrt{a})=0$ and $A(\sqrt{a/3})=(4/9)\sqrt{3}a^{3/2}$. Hence,
the maximum area thus occurs when the rectangle has dimensions
$2\sqrt{a/3}\times (2/3)a$.
\end{solution}

%% BADBAD
%% \figure
%% \vbox{\beginpicture
%% \normalgraphs
%% \ninepoint
%% \setcoordinatesystem units <2cm,2cm>
%% \setplotarea x from -1.2 to 1.2, y from -1.2 to 1.2
%% \axis left shiftedto x=0 /
%% \axis bottom shiftedto y=0 /
%% \setquadratic
%% \circulararc 360 degrees from 1 0 center at 0 0
%% \setlinear
%% \plot -1 0 0.5 .866 0.5 -0.866 -1 0 /
%% \put {$(h-R,r)$} [bl] <2pt,2pt> at 0.5 .866
%% \endpicture}
%% \figrdef{fig:cone in sphere}
%% \endfigure{Cone in a sphere.}

\begin{example}
If you fit the largest possible cone inside a sphere, what fraction of the
volume of the sphere is occupied by the cone?  (Here by ``cone'' we mean a
right circular cone, i.e., a cone for which the base is perpendicular to
the axis of symmetry, and for which the cross-section cut perpendicular to
the axis of symmetry at any point is a circle.)
\end{example}

\begin{marginfigure}
\begin{tikzpicture}
\draw[very thick,penColor2!20!background] (2,0) arc (0:180:2 and .7);% top half of ellipse
\draw [penColor, very thick] (0,1) ellipse (1.7 and .5);
\draw[penColor, very thick] (1.7,.95) -- (0,-2);
\draw[penColor, very thick] (-1.7,.95) -- (0,-2);
\draw[very thick,penColor2] (-2,0) arc (180:360:2 and .7);% bottom half of ellipse


\draw [penColor2, very thick] (0,0) circle [radius=2];

\draw[penColor2, dashed, very thick] (0,0) -- (2,0);
\draw[penColor, dashed, very thick] (0,1) -- (1.7,1);
\draw[penColor, dashed, very thick] (0,1) -- (0,-2);

\node [below,penColor2] at (1.5,0) {$R$};
\node [above,penColor] at (.85,1) {$r$};
\node [left,penColor] at (0,-.33) {$h$};

\node [penColor,left] at (-1.5,1.42) {$V_{\text{c}} = \frac{\pi r^2h}{3}$};
\node [penColor2, right] at (1.5,-1.42) {$V_{s} = \frac{4\pi r^3}{3}$};
\end{tikzpicture}
\caption{A cone inside a sphere.}
\label{fig:cone-sphere}
\end{marginfigure}

\begin{solution}
Let $R$ be the radius of the sphere, and let $r$ and $h$ be the base
radius and height of the cone inside the sphere.  Our goal is to
maximize the volume of the cone: $V_c=\pi r^2h/3$.  The largest $r$
could be is $R$ and the largest $h$ could be is $2R$.

Notice that the function we want to maximize, $\pi r^2h/3$, depends on
\textit{two} variables.  Our next step is to find the relationship and
use it to solve for one of the variables in terms of the other, so as
to have a function of only one variable to maximize.  In this problem,
the condition is apparent in the figure, as the upper corner of the
triangle, whose coordinates are $(h-R,r)$, must be on the circle of
radius $R$. Write
\[
(h-R)^2+r^2=R^2.
\] 
Solving for $r^2$, since $r^2$ is found in the formula for the volume
of the cone, we find 
\[
r^2=R^2-(h-R)^2.
\]  
Substitute this into the formula for the volume of the cone to find

\begin{align*}
 V_{\text{c}}(h)&=\pi(R^2-(h-R)^2)h/3 \\
&=-{\pi\over3}h^3+{2\over3}\pi h^2R
\end{align*}

We want to maximize $V_{\text{c}}(h)$ when $h$ is between 0 and $2R$.  We
solve 
\[
V_{\text{c}}'(h)=-\pi h^2+(4/3)\pi h R=0,
\] 
finding $h=0$ or $h=4R/3$.  We compute
\[
V_{\text{c}}(0)=V_{\text{c}}(2R)=0\qquad\text{and}\qquad V_{\text{c}}(4R/3)=(32/81)\pi R^3.
\] 
The maximum is the latter. Since the volume of the sphere is $(4/3)\pi
R^3$, the fraction of the sphere occupied by the cone is
\[
\frac{(32/81)\pi R^3}{(4/3)\pi R^3}=\frac{8}{27}\approx 30\%.
\]
\end{solution}






\begin{example}
You are making cylindrical containers to contain a given volume.  Suppose
that the top and bottom are made of a material that is $N$ times as
expensive (cost per unit area) as the material used for the lateral side of
the cylinder.  Find (in terms of $N$) the ratio of height to base radius of
the cylinder that minimizes the cost of making the containers.
\end{example}

\begin{marginfigure}
\begin{tikzpicture}
\draw[penColor,very thick] (0,2) ellipse (2 and .7);
\draw[very thick,penColor!20!background] (2,-2) arc (0:180:2 and .7);% top half of ellipse
\draw[very thick,penColor] (-2,-2) arc (180:360:2 and .7);% bottom half of ellipse

\draw[penColor, very thick] (2,2) -- (2,-2);
\draw[penColor, very thick] (-2,2) -- (-2,-2);

\draw[penColor, dashed, very thick] (0,2) -- (2,2);
\draw[penColor, dashed, very thick] (0,2) -- (0,-2);

\node [above,penColor] at (1,2) {$r$};
\node [left,penColor] at (0,-.33) {$h$};
\node [penColor,right] at (2,-1.42) {$V = 2\pi r^2h$};
\end{tikzpicture}
\caption{A cylinder with radius $r$, height $h$, volume $V$, $c$ for
  the cost per unit area of the lateral side of the cylinder.}
\label{fig:cylinder}
\end{marginfigure}

\begin{solution}
First we draw a picture, see Figure~\ref{fig:cylinder}.  Now we can
write an expression for the cost of materials:
\[
  C = 2\pi crh+2\pi r^2Nc.
\]
Since we know that $V=\pi r^2h$, we can use this relationship to
eliminate $h$ (we could eliminate $r$, but it's a little easier if we
eliminate $h$, which appears in only one place in the above formula
for cost).  We find
\begin{align*}
C(r)&=2c\pi r\frac{V}{\pi r^2}+2Nc\pi r^2\\
&=\frac{2cV}{r}+2Nc\pi r^2.
\end{align*}
We want to know the minimum value of this function when $r$ is in
$(0,\infty)$. Setting
\[
C'(r)=-2cV/r^2+4Nc\pi r =0
\]
we find $r=\sqrt[3]{V/(2N\pi)}$.  Since $C''(r)=4cV/r^3+4Nc\pi$ is
  positive when $r$ is positive, there is a local minimum at the
  critical value, and hence a global minimum since there is only one
  critical value.

Finally, since $h=V/(\pi r^2)$, 
\begin{align*}
\frac{h}{r}&=\frac{V}{\pi r^3}\\ 
&=\frac{V}{\pi(V/(2N\pi))}\\ 
&=2N,
\end{align*}
so the minimum cost occurs when the height $h$ is $2N$ times the
radius. If, for example, there is no difference in the cost of
materials, the height is twice the radius.
\end{solution}


\begin{example}\label{exam:sand and road} Suppose you want to reach a point $A$ that is located across the
sand from a nearby road, see Figure~\ref{fig:roadsand}.  Suppose that
the road is straight, and $b$ is the distance from $A$ to the closest
point $C$ on the road.  Let $v$ be your speed on the road, and let
$w$, which is less than $v$, be your speed on the sand.  Right now you
are at the point $D$, which is a distance $a$ from $C$.  At what point
$B$ should you turn off the road and head across the sand in order to
minimize your travel time to $A$?
\end{example}

\begin{marginfigure}
\begin{tikzpicture}
\draw[fill2, fill=fill2] (0,0) rectangle (6,4);
\draw[fill1, fill=fill1] (0,.4) rectangle (6,1);

\node[penColor] at (.5,.75) {\scalebox{-2}[2] \Bicycle};

\draw [penColor, fill] (1,.75) circle [radius=.07];
\draw [penColor, fill] (2.5,.75) circle [radius=.07];
\draw [penColor, fill] (5,.75) circle [radius=.07];
\draw [penColor, fill] (5,3) circle [radius=.07];

\draw[penColor2, very thick, ->] (1.2,.75) -- (2.3,.75);
\draw[penColor2, very thick, ->] (2.6,.85) -- (4.9,2.9);

\draw[penColor, very thick, dashed] (5,.75) -- (5,3);
\draw[penColor, very thick, dashed] (5,.75) -- (2.5,.75);
\draw[penColor, very thick, dashed] (1,.4) -- (5,.4);

\node [right,penColor] at (5,3) {$A$};
\node [below,penColor] at (1,.75) {$D$};
\node [below,penColor] at (2.5,.75) {$B$};
\node [below,penColor] at (5,.75) {$C$};
\node [right,penColor] at (5,2) {$b$};
\node [above,penColor] at (4,.75) {$x$};
\node [left,penColor2] at (3.8,2) {$w$};
\node [above,penColor2] at (1.75,.75) {$v$};
\node [below,penColor] at (3,.4) {$a$};
\end{tikzpicture}
\caption{A road where one travels at rate $v$, with sand where one
  travels at rate $w$. Where should one turn off of the road to
  minimize total travel time from $D$ to $A$?}
\label{fig:roadsand}
\end{marginfigure}

\begin{solution}
Let $x$ be the distance short of $C$ where you turn off, the distance
from $B$ to $C$.  We want to minimize the total travel time.  Recall
that when traveling at constant velocity, time is distance divided by
velocity.

You travel the distance from $D$ to $B$ at speed $v$, and then the
distance from $B$ to $A$ at speed $w$.  The distance from $D$ to $B$
is $a-x$. By the Pythagorean theorem, the distance from $B$ to $A$
is
\[
\sqrt{x^2+b^2}.
\] 
Hence the total time for the trip is
\[
   T(x)=\frac{a-x}{v}+\frac{\sqrt{x^2+b^2}}{w}.
\]
We want to find the minimum value of $T$ when $x$ is between 0 and
$a$.  As usual we set $T'(x)=0$ and solve for $x$. Write
\[
  T'(x)=-\frac{1}{v}+\frac{x}{w\sqrt{x^2+b^2}} =0.
\]
We find that 
\[
x=\frac{wb}{\sqrt{v^2-w^2}}
\]
Notice that $a$ does not appear in the last expression, but $a$ is not
irrelevant, since we are interested only in critical values that are
in $[0,a]$, and $wb/\sqrt{v^2-w^2}$ is either in this interval or not.
If it is, we can use the second derivative to test it:
\[
T''(x) = {b^2\over (x^2+b^2)^{3/2}w}.
\]
Since this is always positive there is a local minimum at the critical
point, and so it is a global minimum as well.

If the critical value is not in $[0,a]$ it is larger than $a$. In this
case the minimum must occur at one of the endpoints. We can compute
\begin{align*}
T(0)&={a\over v}+{b\over w} \\
T(a)&={\sqrt{a^2+b^2}\over w} 
\end{align*}
but it is difficult to determine which of these is smaller by direct
comparison. If, as is likely in practice, we know the values of $v$,
$w$, $a$, and $b$, then it is easy to determine this. With a little
cleverness, however, we can determine the minimum in general. We have seen that
$T''(x)$ is always positive, so the derivative $T'(x)$ is always increasing.
We know that at $wb/\sqrt{v^2-w^2}$ the derivative is zero, so for
values of $x$ less than that critical value, the derivative is
negative. This means that $T(0)>T(a)$, so the minimum occurs when $x=a$.

So the upshot is this: If you start farther away from $C$ than
$wb/\sqrt{v^2-w^2}$ then you always want to cut across the sand 
when you are a distance $wb/\sqrt{v^2-w^2}$ from point $C$. If you
start closer than this to $C$, you should cut directly across the sand.
\end{solution}



\begin{exercises}


\begin{exercise}
Find the dimensions of the rectangle of largest area having fixed perimeter
$100$.
\begin{answer} $25\times 25$
\end{answer}\end{exercise}


\begin{exercise}
Find the dimensions of the rectangle of largest area having fixed perimeter
$P$.
\begin{answer} $P/4\times P/4$
\end{answer}\end{exercise}

\begin{exercise}
A box with square base and no top is to hold a volume $100$.  Find
the dimensions of the box that requires the least material for the
five sides.  Also find the ratio of height to side of the base.
\begin{answer} $w=l=2\cdot 5^{2/3}$, $h=5^{2/3}$, $h/w=1/2$
\end{answer}\end{exercise}


\begin{exercise} A box with square base is to hold a volume
$200$. The bottom and top are formed by folding in flaps from all four
sides, so that the bottom and top consist of two layers of cardboard.
Find the dimensions of the box that requires the least material.
Also find the ratio of height to side of the base.
\begin{answer} $\root 3\of {100}\times\root 3\of {100}\times 2\root 3\of
{100}$, $h/s=2$
\end{answer}\end{exercise}

\begin{exercise}
A box with square base and no top is to hold a volume $V$.  Find (in terms
of $V$) the dimensions of the box that requires the least material for the
five sides.  Also find the ratio of height to side of the base.  (This
ratio will not involve $V$.)
\begin{answer} $w=l=2^{1/3}V^{1/3}$, $h=V^{1/3}/2^{2/3}$, $h/w=1/2$
\end{answer}\end{exercise}

\begin{exercise}
You have $100$ feet of fence to make a rectangular play area alongside the
wall of your house.  The wall of the house bounds one side.  What is the
largest size possible (in square feet) for the play area?
\begin{answer} $1250$ square feet
\end{answer}\end{exercise}

\begin{exercise}
You have $l$ feet of fence to make a rectangular play area alongside the
wall of your house.  The wall of the house bounds one side.  What is the
largest size possible (in square feet) for the play area?
\begin{answer} $l^2/8$ square feet
\end{answer}\end{exercise}

\begin{exercise}
Marketing tells you that if you set the price of an item at \$10
then you will be unable to sell it, but that you can sell 500 items for
each dollar below \$10 that you set the price.  Suppose your fixed costs total
\$3000, and your marginal cost is \$2 per item.  What is the most profit
you can make?
\label{ex:manufacturing}
\begin{answer} \$5000
\end{answer}\end{exercise}

\begin{exercise}
Find the area of the largest rectangle that fits inside a semicircle of
radius $10$ (one side of the rectangle is along the diameter of the
semicircle).
\begin{answer} $100$
\end{answer}\end{exercise}

\begin{exercise}
Find the area of the largest rectangle that fits inside a semicircle of
radius $r$ (one side of the rectangle is along the diameter of the
semicircle).
\begin{answer} $r^2$
\end{answer}\end{exercise}

\begin{exercise}
For a cylinder with surface area $50$, including 
the top and the bottom, find the ratio of height to
base radius that maximizes the volume.
\begin{answer} $h/r=2$
\end{answer}\end{exercise}

\begin{exercise}
For a cylinder with given surface area $S$, including 
the top and the bottom, find the ratio of height to
base radius that maximizes the volume.
\begin{answer} $h/r=2$
\end{answer}\end{exercise}

\begin{exercise}
You want to make cylindrical containers to hold 1 liter using the
least amount of construction material.  The side is made from a
rectangular piece of material, and this can be done with no material
wasted.  However, the top and bottom are cut from squares of side $2r$, so
that $2(2r)^2=8r^2$ of material is needed (rather than $2\pi r^2$, which is
the total area of the top and bottom).  Find the dimensions of the
container using the least amount of material, and also find the
ratio of height to
radius for this container.
\begin{answer} $r=5$, $h=40/\pi$, $h/r=8/\pi$
\end{answer}\end{exercise}

\begin{exercise}
You want to make cylindrical containers of a given volume $V$ using the
least amount of construction material.  The side is made from a
rectangular piece of material, and this can be done with no material
wasted.  However, the top and bottom are cut from squares of side $2r$, so
that $2(2r)^2=8r^2$ of material is needed (rather than $2\pi r^2$, which is
the total area of the top and bottom).  Find the optimal ratio of height to
radius.
\begin{answer} $8/\pi$
\end{answer}\end{exercise}

\begin{exercise}
Given a right circular cone, you put an upside-down cone inside it so that
its vertex is at the center of the base of the larger cone and its base is
parallel to the base of the larger cone.  If you choose the upside-down
cone to have the largest possible volume, what fraction of the volume of
the larger cone does it occupy?  (Let $H$ and $R$ be the height and base
radius of the larger cone, and let $h$ and $r$ be the height and base
radius of the smaller cone.  Hint: Use similar triangles to get an equation
relating $h$ and $r$.)
\begin{answer} $4/27$
\end{answer}\end{exercise}

\begin{exercise}
In Example~\ref{exam:sand and road}, what happens if
$w\ge v$ (i.e., your speed on sand is at least your speed on the
road)?
\begin{answer} Go direct from $A$ to $D$.
\end{answer}\end{exercise}

\begin{exercise}
A container holding a fixed volume is being made in the shape of a cylinder
with a hemispherical top.  (The hemispherical top has the same radius
as the cylinder.)  Find the ratio of height to radius of the cylinder which
minimizes the cost of the container if (a) the cost per unit area of the
top is twice as great as the cost per unit area of the side, and the
container is made with no bottom; (b) the same as in (a), except that the
container is made with a circular bottom, for which the cost per unit area is
1.5 times the cost per unit area of the side.
\begin{answer} (a) 2, (b) $7/2$
\end{answer}\end{exercise}

\begin{exercise} A piece of cardboard is 1 meter by $1/2$ meter. A square is
to be cut from each corner and the sides folded up to make an open-top
box. What are the dimensions of the box with maximum possible volume?
\begin{answer} $\ds{\sqrt3\over6}\times{\sqrt3\over6}+{1\over2}\times
{1\over4}-{\sqrt3\over 12}$
\end{answer}\end{exercise}


\begin{exercise} (a) A square piece of cardboard of side $a$ is used to make
an open-top box by cutting out a small square from each corner and
bending up the sides.  How large a square should be cut from each
corner in order that the box have maximum volume? (b) What if the
piece of cardboard used to make the box is a rectangle of sides $a$
and $b$?  
\begin{answer} (a) $a/6$, (b) $(a+b-\sqrt{a^2-ab+b^2})/6$
\end{answer}\end{exercise} 
\label{exercise: cardboard box}

\begin{exercise} A window consists of a rectangular piece of clear glass with
a semicircular piece of colored glass on top; the
colored glass transmits only $1/2$ as much light per unit area as the
the clear glass.  If the distance from
top to bottom (across both the rectangle and the semicircle) is
2 meters and the window may be no more than 1.5 meters wide, find the
dimensions of the rectangular portion of the window that lets through
the most light.
\begin{answer} $1.5$ meters wide by $1.25$ meters tall
\end{answer}\end{exercise} 

\begin{exercise} A window consists of a rectangular piece of clear glass with
a semicircular piece of colored glass on top.  Suppose that the
colored glass transmits only $k$ times as much light per unit area as
the clear glass ($k$ is between $0$ and $1$).  If the distance from
top to bottom (across both the rectangle and the semicircle) is a
fixed distance $H$,
find (in terms of $k$) the ratio of vertical side to horizontal side
of the rectangle for which the window lets through the most light.
\begin{answer} If $k\le 2/\pi$ the ratio is $(2-k\pi)/4$; if $k\ge 2/\pi$,
the ratio is zero: the window should be semicircular with no
rectangular part.
\end{answer}\end{exercise}

\begin{exercise} You are designing a poster to contain a fixed amount $A$ of
printing (measured in square centimeters) and have margins of $a$
centimeters at the top and bottom and $b$ centimeters at the sides.
Find the ratio of vertical dimension to horizontal dimension of the
printed area on the poster if you want to minimize the amount of
posterboard needed.
\begin{answer} $a/b$
\end{answer}\end{exercise}

\begin{exercise}
The strength of a rectangular beam is proportional to the product of its
width $w$ times the square of its depth $d$.  
Find the dimensions of the strongest
beam that can be cut from a cylindrical log of radius $r$.
\begin{answer} $w=2r/\sqrt3$, $h=2\sqrt2r/\sqrt3$
\end{answer}\end{exercise}

%% BADBAD
%% \figure
%% \vbox{\beginpicture
%% \normalgraphs
%% \ninepoint
%% \setcoordinatesystem units <1.5truecm,1.5truecm>
%% \setplotarea x from -1 to 1, y from -1 to 1
%% \circulararc 360 degrees from 1 0 center at 0 0
%% \setlinear
%% \plot 0.5 0.866 -0.5 0.866 -0.5 -0.866 0.5 -0.866 0.5 0.866 /
%% \betweenarrows {$d$} <-4pt,0pt> from 0.5 -0.866 to 0.5 0.866
%% \betweenarrows {$w$} <0pt,6pt> from -0.5 -0.866 to 0.5 -0.866
%% \endpicture}
%% \figrdef{fig:beam strength}
%% \endfigure{Cutting a beam.}



\begin{exercise}
What fraction of the volume of a sphere is taken up by the largest cylinder
that can be fit inside the sphere?
\begin{answer} $1/\sqrt3\approx 58\%$
\end{answer}\end{exercise}

\begin{exercise}
The U.S.~post office will accept a box for shipment only if the sum of the
length and girth (distance around) is at most 108 in.  Find the dimensions
of the largest acceptable box with square front and back.
\begin{answer} $18\times18\times36$
\end{answer}\end{exercise}

\begin{exercise}
Find the dimensions of the lightest cylindrical can containing 0.25 liter
(=250 cm${}^3$) if the top and bottom are made of a material that is twice
as heavy (per unit area) as the material used for the side.
\begin{answer} $r=5/(2\pi)^{1/3}\approx 2.7\hbox{ cm}$,\hfill\break
$h=5\cdot2^{5/3}/\pi^{1/3}=4r\approx 10.8 \hbox{ cm}$
\end{answer}\end{exercise}

\begin{exercise} A conical paper cup is to hold $1/4$ of a liter. Find the
height and radius of the cone which minimizes
the amount of paper needed to make the cup.  Use the formula $\pi
r\sqrt{r^2+h^2}$ for the area of the side of a cone.
\begin{answer} $h={750\over\pi}\left({2\pi^2\over 750^2}\right)^{1/3}$, 
$r=\left({750^2\over 2\pi^2}\right)^{1/6}$
\end{answer}\end{exercise}

\begin{exercise} A conical paper cup is to hold a fixed volume of water.
Find the ratio of height to base radius of the cone which minimizes
the amount of paper needed to make the cup.  Use the formula $\pi
r\sqrt{r^2+h^2}$ for the area of the side of a cone, called the
{\dfont lateral area\index{lateral area of a cone}\/} of the cone.
\begin{answer} $h/r=\sqrt2$
\end{answer}\end{exercise}

% Maple only?
\begin{exercise}
If you fit the cone with the largest possible surface area (lateral area
plus area of base) into a sphere, what percent of the volume of the
sphere is occupied by the cone?  
\begin{answer} The ratio of the volume of the sphere to the volume of the
cone is $1033/4096+33/4096\sqrt{17}\approx 0.2854$, so the cone
occupies approximately $28.54\%$ of the sphere.
\end{answer}\end{exercise}

% Maple only?
\begin{exercise}
Two electrical charges, one a positive charge $A$ of magnitude $a$ and the
other a negative charge $B$ of magnitude $b$, are located a distance $c$
apart.  A positively charged particle $P$ is situated on the line between $A$
and $B$.  Find where $P$ should be put so that the pull away from $A$ towards
$B$ is minimal.  Here assume that the force from each charge is
proportional to the strength of the source and inversely proportional to
the square of the distance from the source.
\begin{answer} $P$ should be at distance $c\root 3\of {a} /
(\root 3\of {a} + \root 3\of {b})$ from charge $A$.
\end{answer}\end{exercise}

\begin{exercise}
Find the fraction of the area of a triangle that is occupied by the largest
rectangle that can be drawn in the triangle (with one of its sides along a
side of the triangle).  Show that this fraction does not depend on the
dimensions of the given triangle.
\begin{answer} $1/2$
\end{answer}\end{exercise}

\begin{exercise}
How are your answers to Problem~\ref{ex:manufacturing} affected if
the cost per item for the $x$ items, instead of being simply \$2,
decreases below \$2 in proportion to $x$ (because of economy of scale
and volume discounts) by 1 cent for each 25 items produced?
\begin{answer} \$7000
\end{answer}\end{exercise}

%% \begin{exercise}
%% You are standing near the side of a large wading pool of uniform depth when
%% you see a child in trouble.  You can run at a speed $v_1$ on land and at a
%% slower speed $v_2$ in the water.  Your perpendicular distance from the side
%% of the pool is $a$, the child's perpendicular distance is $b$, and the
%% distance along the side of the pool between the closest point to you and
%% the closest point to the child is $c$ (see the figure below). 
%% Without stopping to do any calculus, you instinctively choose the
%% quickest route (shown in the figure) and save the child.  Our
%% purpose is to derive a relation between the angle $\theta_1$ your path
%% makes with the perpendicular to the side of the pool when you're on land,
%% and the angle $\theta_2$ your path makes with the perpendicular when you're
%% in the water.  To do this, let $x$ be the distance between the closest
%% point to you at the side of the pool and the point where you enter the
%% water.  Write the total time you run (on land and in the water) in 
%% terms of $x$ (and also the constants $a,b,c,v_1,v_2$).  Then set the
%% derivative equal to zero.  The result, called ``Snell's law'' or the ``law
%% of refraction,'' also governs the bending of light when it goes into water.
%% \begin{answer} There is a critical point when
%% $\sin\theta_1/v_1=\sin\theta_2/v_2$, and the second derivative is
%% positive, so there is a minimum at the critical point.
%% \end{answer}\end{exercise}

%% %% BADBAD
%% %% \figure
%% %% \vbox{\beginpicture
%% %% \normalgraphs
%% %% \ninepoint
%% %% \setcoordinatesystem units <.75truecm,.75truecm>
%% %% \setplotarea x from 0 to 6, y from -3 to 3
%% %% \axis bottom shiftedto y=0 /
%% %% \circulararc 53.13 degrees from 2.667 -1 center at 4 0
%% %% \put {$\theta_1$} [tr] <-3pt,-3pt> at 3.6 -1.4
%% %% \circulararc 33.69 degrees from 5 1.5 center at 4 0
%% %% \put {$\theta_2$} at 4.6 2
%% %% \setlinear
%% %% \plot 0 -3 4 0 6 3 /
%% %% \setdashes
%% %% \putrule from 4 -3 to 4 3
%% %% \putrule from 6 0 to 6 3
%% %% \putrule from 0 -3 to 0 0
%% %% \put {$x$} [b] <0pt,4pt> at 2 0
%% %% \put {$c-x$} [b] <0pt,4pt> at 5 0
%% %% \put {$a$} [l] <4pt,0pt> at 0 -1.5
%% %% \put {$b$} [l] <4pt,0pt> at 6 1.5
%% %% \endpicture}
%% %% \figrdef{fig:rescue}
%% %% \endfigure{Wading pool rescue.}

\end{exercises}





