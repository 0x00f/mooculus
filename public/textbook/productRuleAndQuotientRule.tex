\chapter{The Product Rule and Quotient Rule}

\section{The Product Rule}
\index{product rule}

Consider the product of two simple functions, say
\[
f(x)\cdot g(x)
\]
where $f(x)=x^2+1$ and $g(x)=x^3-3x$. An obvious guess for the
derivative of $f(x)g(x)$ is the product of the derivatives:
\begin{align*}
f'(x)g'(x) &= (2x)(3x^2-3)\\
&= 6x^3-6x.
\end{align*}
Is this guess correct? We can check by rewriting $f(x)$
and $g(x)$ and doing the calculation in a way that is known to
work. Write 
\begin{align*}
f(x)g(x) &= (x^2+1)(x^3-3x)\\
&=x^5-3x^3+x^3-3x\\
&=x^5-2x^3-3x.
\end{align*} 
Hence
\[
\ddx f(x) g(x) = 5x^4-6x^2-3, 
\]
so we see that 
\[
\ddx f(x) g(x) \ne  f'(x)g'(x).
\]
So the derivative of $f(x)g(x)$ is \textbf{not} as simple as
$f'(x)g'(x)$. Never fear, we have a rule for exactly this
situation.
\begin{mainTheorem}[The Product Rule]\index{derivative rules!product}\index{product rule}\label{theorem:product-rule}
If $f(x)$ and $g(x)$ are differentiable, then
\[
\ddx f(x)g(x) = f(x)g'(x)+f'(x)g(x).
\]
\end{mainTheorem}

\begin{marginfigure}
\begin{tikzpicture}
	\begin{axis}[
            clip=false,
            domain=0:6, 
            ticks=none,
            ymin=0, ymax=6,
            xlabel=$x$, ylabel=$y$,
            axis lines=center,
            every axis y label/.style={at=(current axis.above origin),anchor=south},
            every axis x label/.style={at=(current axis.right of origin),anchor=west},
            axis on top,
          ]          
          %\addplot [dashed, textColor] plot coordinates {(4,0) (4,3.08)};
          %\node at (axis cs:4,0) [anchor=north] {$x$};

          \addplot [penColor5,very thick] plot coordinates {(5,1.4) (5,1.9)};
          \addplot [dashed, very thick, textColor] plot coordinates {(4,1.4) (5,1.4)};

          \addplot [penColor4,very thick] plot coordinates {(5,2.2) (5,2.6)};
          \addplot [dashed, very thick, textColor] plot coordinates {(4,2.2) (5,2.2)};

          \addplot [very thick, penColor5!50!penColor2] plot coordinates {(5,3.08) (5,4.18)};
          \addplot [very thick, penColor4!50!penColor] plot coordinates {(5,4.18) (5,4.74)};
          \addplot [dashed, very thick, textColor] plot coordinates {(4,3.08) (5,3.08)};
        
          \addplot [very thick,penColor,smooth] {-.6+.5*x};
          \addplot [very thick,penColor2,smooth] {.6+.4*x};         
          \addplot [very thick,penColor3,smooth,domain=1:5.5] {-.36+.06*x+.2*x^2};      
          \addplot [penColor3!70!background,smooth, domain=1.6:5.6] {-3.56+1.66*x};      


          \node at (axis cs:3.5,1.1) [anchor=north,penColor] {$f(x)$};
          \node at (axis cs:2,1.95) [anchor=north,penColor2] {$g(x)$};
          \node at (axis cs:4.7,4.3) [anchor=east,penColor3] {$f(x)+g(x)$};

          \node at (axis cs:5,3.91) [anchor=west] {${\color{penColor}f(x)}{\color{penColor4}g'(x)}+{\color{penColor5}f'(x)}{\color{penColor2}g(x)}$};
          \node at (axis cs:5,1.65) [anchor=west,penColor5] {$f'(x)$};
          \node at (axis cs:5,2.4) [anchor=west,penColor4] {$g'(x)$};
          \node at (axis cs:4.5,1.5) [anchor=north] {$\underbrace{\hspace{.40in}}_{1}$};
        \end{axis}
\end{tikzpicture}
\caption{A geometric interpretation of the product rule. Since every
  point on $f(x)g(x)$ is the product of the corresponding points on
  $f(x)$ and $g(x)$, the instantaneous growth rate of $f(x)g(x)$ is the
  sum of $f(x)g'(x)$ and $f'(x)g(x)$.}
\end{marginfigure}

\begin{proof}
From the limit definition of the derivative, write
\[
\ddx (f(x)g(x)) = \lim_{h \to0} \frac{f(x+h)g(x+h) - f(x)g(x)}{h}
\]
Now we use the exact same trick we used in the proof of
Theorem~\ref{theorem:limit-product}, we add $0 = -f(x+h)g(x) + f(x+h)g(x)$:
\begin{align*}
&=\lim_{h \to0} \frac{f(x+h)g(x+h){\color{penColor2}-f(x+h)g(x) + f(x+h)g(x)}- f(x)g(x)}{h} \\ 
&=\lim_{h \to0} \frac{f(x+h)g(x+h)-f(x+h)g(x)}{h} + \lim_{h \to0} \frac{f(x+h)g(x)- f(x)g(x)}{h}.
\end{align*}
Now since both $f(x)$ and $g(x)$ are differentiable, they are
continuous, see Theorem~\ref{theorem:diff-cont}. Hence
\begin{align*}
&=\lim_{h \to0} f(x+h)\frac{g(x+h)-g(x)}{h} + \lim_{h \to0} \frac{f(x+h)- f(x)}{h}g(x) \\ 
&=\lim_{h \to0} f(x+h)\lim_{h \to0}\frac{g(x+h)-g(x)}{h} + \lim_{h \to0} \frac{f(x+h)- f(x)}{h}\lim_{h \to0}g(x) \\ 
&=f(x)g'(x) + f'(x)g(x).
\end{align*}
\end{proof}



Let's return to the example with which we started.
\begin{example} 
Let $f(x)=(x^2+1)$ and $g(x)=(x^3-3x)$. Compute:
\[
\ddx f(x)g(x).
\]
\end{example}
\begin{solution}
Write
\begin{align*}
\ddx f(x)g(x) &= f(x)g'(x) + f'(x)g(x)\\
&=(x^2+1)(3x^2-3) + 2x(x^3-3x).
\end{align*}
We could stop here---but we should show that expanding this out recovers our previous result. Write
\begin{align*}
(x^2+1)(3x^2-3) + 2x(x^3-3x) &= 3x^4-3x^2 +3x^2 +3 + 2x^4-6x^2\\
&=5x^4-6x^2+2,
\end{align*}
which is precisely what we obtained before.
\end{solution}


\begin{example} Compute the derivative of $\ds f(x)=x^2\sqrt{625-x^2}$.  We have
already computed $\ds {d\over
  dx}\sqrt{625-x^2}={-x\over\sqrt{625-x^2}}$.  Now
$$f'(x)=x^2{-x\over\sqrt{625-x^2}}+2x\sqrt{625-x^2}=
{-x^3+2x(625-x^2)\over \sqrt{625-x^2}}=
{-3x^3+1250x\over \sqrt{625-x^2}}.
$$
\end{example}

\begin{exercises}

In 1--4, find the derivatives of the functions using the product rule.

\begin{exercise} $\ds x^3(x^3-5x+10)$
\begin{answer} $\ds 3x^2(x^3-5x+10)+x^3(3x^2-5)$
\end{answer}\end{exercise}

\begin{exercise} $\ds (x^2+5x-3)(x^5-6x^3+3x^2-7x+1)$
\begin{answer} $\ds (x^2+5x-3)(5x^4-18x^2+6x-7)+(2x+5)(x^5-6x^3+3x^2-7x+1)$
\end{answer}\end{exercise}

\begin{exercise} $\ds \sqrt{x}\sqrt{625-x^2}$
\begin{answer} $\ds \ds{\sqrt{625-x^2}\over 2\sqrt{x}}-{x\sqrt{x}\over\sqrt{625-x^2}}$
\end{answer}\end{exercise}

\begin{exercise} $\displaystyle {\sqrt{625-x^2}\over x^{20}}$
\begin{answer} $\ds{-1\over x^{19}\sqrt{625-x^2}}-{20\sqrt{625-x^2}\over x^{21}}$
\end{answer}\end{exercise}

\begin{exercise} Use the product rule to compute the derivative of $\ds f(x)=(2x-3)^2$.
 Sketch the function.  Find an equation of the tangent line to the curve at
 $x=2$.  Sketch the tangent line at $x=2$.
\begin{answer} $f'=4(2x-3)$, $y=4x-7$
\end{answer}\end{exercise}

\begin{exercise}
Suppose that $f$, $g$, and $h$ are differentiable functions.
Show that $(fgh)'(x) = f'(x) g(x)h(x) + f(x)g'(x) h(x) + f(x) g(x)
h'(x)$.
\end{exercise}

\begin{exercise}
State and prove a rule to compute $(fghi)'(x)$, 
similar to the rule in the previous problem.
\end{exercise}

\begin{remark}{Product notation}
Suppose $\ds f_1 , f_2 , \ldots f_n$ are functions.
The product of all these functions can be written
$$ \prod _{k=1 } ^n f_k.$$
This is similar to the use of $\ds \sum$ to denote a 
sum.
For example,
$$\prod _{k=1 } ^5 f_k =f_1 f_2 f_3 f_4 f_5$$
and
$$
\prod _ {k=1 } ^n k = 1\cdot 2 \cdot \ldots \cdot n = n!.$$
We sometimes use somewhat more complicated conditions; for example
$$\prod _{k=1 , k\neq j } ^n f_k$$
denotes the product of $\ds f_1$ through $\ds f_n$ except for $\ds f_j$.
For example, 
$$\prod _{k=1 , k\neq 4} ^5 x^k = x\cdot x^2 \cdot x^3 \cdot x^5 =
x^{11}.$$
\end{remark}

\begin{exercise}
  The {\dfont generalized product rule\index{product rule!generalized}\/} 
says that if $\ds f_1 , f_2 ,\ldots ,f_n$ are differentiable functions at
  $x$ then
$${d\over dx}\prod _{k=1 } ^n f_k(x) = 
\sum _{j=1 } ^n \left(f'_j (x) \prod _{k=1 , k\neq j} ^n
   f_k (x)\right).$$
Verify that this is the same as your answer to the previous problem
when $n=4$,
and write out what this says when $n=5$.
\end{exercise}

%% Wills
%% \begin{exercise} 
%% The generalized product rule is trivial when $n=1$. The case
%% when $n=2$ is shown in the text and the case when $n=3$ is assigned
%% in an earlier exercise. Use mathematical induction to show that the
%% generalized product rule is true for any positive integer $n$.
%%  \end{exer}

\end{exercises}





\section{The Quotient Rule}

\index{quotient rule} 

We'd like to have a formula to compute
\[
\ddx \frac{f(x)}{g(x)}
\]
if we already know $f'(x)$ and $g'(x)$. Instead of attacking this
problem head-on, let's notice that we've already done part of the
problem: $f(x)/g(x)= f(x)\cdot(1/g(x))$, that is, this is ``really'' a
product, and we can compute the derivative if we know $f'(x)$ and
$(1/g(x))'$. So really the only new bit of information we need is
$(1/g(x))'$ in terms of $g'(x)$. As with the product rule, let's set
this up and see how far we can get:
\begin{align*}
{d\over dx}{1\over g(x)}&=\lim_{h\to0} 
{{1\over g(x+h)}-{1\over g(x)}\over h} \\
&=\lim_{h\to0} {{g(x)-g(x+h)\over g(x+h)g(x)}\over h} \\
&=\lim_{h\to0} {g(x)-g(x+h)\over g(x+h)g(x)h} \\
&=\lim_{h\to0} -{g(x+h)-g(x)\over h}
 {1\over g(x+h)g(x)} \\
&=-{g'(x)\over g(x)^2} \\
\end{align*}
Now we can put this together with the product rule:
$${d\over dx}{f(x)\over g(x)}=f(x){-g'(x)\over g(x)^2}+f'(x){1\over
  g(x)}={-f(x)g'(x)+f'(x)g(x)\over g(x)^2}=
  {f'(x)g(x)-f(x)g'(x)\over g(x)^2}.
$$

\begin{example}
Compute the derivative of $\ds (x^2+1)/(x^3-3x)$.
$${d\over dx}{x^2+1\over
  x^3-3x}={2x(x^3-3x)-(x^2+1)(3x^2-3)\over(x^3-3x)^2}=
  {-x^4-6x^2+3\over (x^3-3x)^2}.
$$

\end{example}

It is often possible to calculate derivatives in more than one way, as
we have already seen. Since every quotient can be written as a
product, it is always possible to use the product rule to compute the
derivative, though it is not always simpler.

\begin{example}
Find the derivative of $\ds \sqrt{625-x^2}/\sqrt{x}$ in two ways: using the
quotient rule, and using the product rule.

Quotient rule:
$${d\over dx}{\sqrt{625-x^2}\over\sqrt{x}} = 
{\sqrt{x}(-x/\sqrt{625-x^2})-\sqrt{625-x^2}\cdot 1/(2\sqrt{x})\over
x}.$$
Note that we have used $\ds \sqrt{x}=x^{1/2}$ to compute the derivative of
$\ds \sqrt{x}$ by the power rule.

Product rule:
$${d\over dx}\sqrt{625-x^2} x^{-1/2} = 
\sqrt{625-x^2} {-1\over 2}x^{-3/2}+{-x\over \sqrt{625-x^2}}x^{-1/2}.
$$

With a bit of algebra, both of these simplify to
$$-{x^2+625\over 2\sqrt{625-x^2}x^{3/2}}.$$

\end{example}

Occasionally you will need to compute the derivative of a quotient
with a constant numerator, like $\ds 10/x^2$. Of course you can use the
quotient rule, but it is usually not the easiest method. If we do use
it here, we get 
$${d\over dx}{10\over x^2}={x^2\cdot 0-10\cdot 2x\over x^4}=
{-20\over x^3},$$
since the derivative of 10 is 0. But it is simpler to do this:
$${d\over dx}{10\over x^2}={d\over dx}10x^{-2}=-20x^{-3}.$$
Admittedly, $\ds x^2$ is a particularly simple denominator, but we will
see that a similar calculation is usually possible. Another approach
is to remember that
$${d\over dx}{1\over g(x)}={-g'(x)\over g(x)^2},$$
but this requires extra memorization. Using this formula,
$${d\over dx}{10\over x^2}=10{-2x\over x^4}.$$
Note that we first use linearity of the derivative to pull the 10 out
in front.

% Hack
% \vfill\eject

\begin{exercises}

Find the derivatives of the functions in 1--4
using the quotient rule.

\twocol

\begin{exercise} $\ds {x^3\over x^3-5x+10}$
\begin{answer} $\ds {3x^2\over x^3-5x+10}-{x^3(3x^2-5)\over (x^3-5x+10)^2}$
\end{answer}\end{exercise}

\begin{exercise} $\ds {x^2+5x-3\over x^5-6x^3+3x^2-7x+1}$
\begin{answer} $\ds {2x+5\over x^5-6x^3+3x^2-7x+1}-
{(x^2+5x-3)(5x^4-18x^2+6x-7)\over(x^5-6x^3+3x^2-7x+1)^2}$
\end{answer}\end{exercise}

\ssk
\begin{exercise} $\ds {\sqrt{x}\over\sqrt{625-x^2}}$
\begin{answer} $\ds {1\over 2\sqrt{x}\sqrt{625-x^2}}+{x^{3/2}\over(625-x^2)^{3/2}}$
\end{answer}\end{exercise}

\begin{exercise} $\ds {\sqrt{625-x^2}\over x^{20}}$
\begin{answer} $\ds {-1\over x^{19}\sqrt{625-x^2}}-{20\sqrt{625-x^2}\over x^{21}}$
\end{answer}\end{exercise}

\endtwocol
\bsk

\begin{exercise} Find an equation for the tangent line to $\ds f(x) = (x^2 -
4)/(5-x)$ at $x= 3$.  
\begin{answer} $\ds y=17x/4-41/4$ 
\end{answer}\end{exercise}

\begin{exercise}  Find an equation for the tangent line to 
$\ds f(x) = (x-2)/(x^3 + 4x - 1)$ at $x=1$.
\begin{answer} $y=11x/16-15/16$
\end{answer}\end{exercise}

\begin{exercise} Let $P$ be a polynomial of degree $n$ and let $Q$ be a
polynomial of degree $m$ (with $Q$ not the zero polynomial). 
Using sigma notation we can write
$$P=\sum _{k=0 } ^n a_k x^k,\qquad
Q=\sum_{k=0}^m b_k x^k.
$$
Use sigma notation to write the derivative of the 
{\dfont rational function\index{rational function}}
$P/Q$.
%% \begin{answer} $\left(\sum_{k=0}^m b_k x^k\sum _{k=1}^n ka_k x^{k-1}-
%% \sum _{k=0 } ^n a_k x^k\sum_{k=1}^m kb_k x^k\right)/
%% \left(\sum_{k=0}^m b_k x^k\right)^2$
%% \end{answer}
\end{exercise}

\begin{exercise} The curve $\ds y=1/(1+x^2)$ is an example of a class of
curves each of which is called a {\dfont witch of
Agnesi\index{witch of Agnesi}}. 
Sketch the curve and find the tangent line to the curve at
$x= 5$. (The word {\em witch\/} here is a mistranslation of the
original Italian, as described at
% BADBAD
% $$\hbox{\url{http://mathworld.wolfram.com/WitchofAgnesi.html} 
% \vb|http://mathworld.wolfram.com/WitchofAgnesi.html|\endurl}$$
% and 
% \begin{align*}
% \hbox{\url{http://instructional1.calstatela.edu/sgray/Agnesi/WitchHistory/Historynamewitch.html} 
% \vb|http://|\endurl}%
% &\!\!\hbox{\url{http://instructional1.calstatela.edu/sgray/Agnesi/WitchHistory/Historynamewitch.html} 
% \vb|instructional1.calstatela.edu/sgray/Agnesi/|\endurl} \\
% &\hbox{\url{http://instructional1.calstatela.edu/sgray/Agnesi/WitchHistory/Historynamewitch.html} 
% \vb|WitchHistory/Historynamewitch.html|\endurl.)} \\
% \end{align*}
\begin{answer} $y=19/169-5x/338$
\end{answer}\end{exercise}
%% \footnote{Due to a mistranslation of the Italian word
%%   \emph{versiera} which actually refers to a rope that turns the
%%   sail.}.  
 
\begin{exercise} If $f'(4) = 5$, $g'(4) = 12$, $(fg)(4)= f(4)g(4)=2$, and $g(4) = 6$,
compute $f(4)$ and $\ds{d\over dx}{f\over g}$ at 4.
\begin{answer} $13/18$
\end{answer}\end{exercise}

\end{exercises}















\section{The Derivative of Trigonometric Functions}

\section{The Chain Rule}


\section{Rates of Change}

\section{Implicit Differentiation}

\section{Applications}

\subsection{Related Rates}
