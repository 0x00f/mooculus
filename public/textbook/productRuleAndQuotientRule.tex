\chapter{The Product Rule and Quotient Rule}

\section{The Product Rule}


Consider the product of two simple functions, say
\[
f(x)\cdot g(x)
\]
where $f(x)=x^2+1$ and $g(x)=x^3-3x$. An obvious guess for the
derivative of $f(x)g(x)$ is the product of the derivatives:
\begin{align*}
f'(x)g'(x) &= (2x)(3x^2-3)\\
&= 6x^3-6x.
\end{align*}
Is this guess correct? We can check by rewriting $f(x)$
and $g(x)$ and doing the calculation in a way that is known to
work. Write 
\begin{align*}
f(x)g(x) &= (x^2+1)(x^3-3x)\\
&=x^5-3x^3+x^3-3x\\
&=x^5-2x^3-3x.
\end{align*} 
Hence
\[
\ddx f(x) g(x) = 5x^4-6x^2-3, 
\]
so we see that 
\[
\ddx f(x) g(x) \ne  f'(x)g'(x).
\]
So the derivative of $f(x)g(x)$ is \textbf{not} as simple as
$f'(x)g'(x)$. Never fear, we have a rule for exactly this
situation.
\begin{mainTheorem}[The Product Rule]\index{derivative rules!product}\index{product rule}\label{theorem:product-rule}
If $f(x)$ and $g(x)$ are differentiable, then
\[
\ddx f(x)g(x) = f(x)g'(x)+f'(x)g(x).
\]
\end{mainTheorem}

\begin{marginfigure}
\begin{tikzpicture}
	\begin{axis}[
            clip=false,
            domain=0:6, 
            ticks=none,
            ymin=0, ymax=6,
            xlabel=$x$, ylabel=$y$,
            axis lines=center,
            every axis y label/.style={at=(current axis.above origin),anchor=south},
            every axis x label/.style={at=(current axis.right of origin),anchor=west},
            axis on top,
          ]          
          %\addplot [dashed, textColor] plot coordinates {(4,0) (4,3.08)};
          %\node at (axis cs:4,0) [anchor=north] {$x$};

          \addplot [penColor5,very thick] plot coordinates {(5,1.4) (5,1.9)};
          \addplot [dashed, very thick, textColor] plot coordinates {(4,1.4) (5,1.4)};

          \addplot [penColor4,very thick] plot coordinates {(5,2.2) (5,2.6)};
          \addplot [dashed, very thick, textColor] plot coordinates {(4,2.2) (5,2.2)};

          \addplot [very thick, penColor5!50!penColor2] plot coordinates {(5,3.08) (5,4.18)};
          \addplot [very thick, penColor4!50!penColor] plot coordinates {(5,4.18) (5,4.74)};
          \addplot [dashed, very thick, textColor] plot coordinates {(4,3.08) (5,3.08)};
        
          \addplot [very thick,penColor,smooth] {-.6+.5*x};
          \addplot [very thick,penColor2,smooth] {.6+.4*x};         
          \addplot [very thick,penColor3,smooth,domain=1:5.5] {-.36+.06*x+.2*x^2};      
          \addplot [penColor3!70!background,smooth, domain=1.6:5.6] {-3.56+1.66*x};      


          \node at (axis cs:3.5,1.1) [anchor=north,penColor] {$f(x)$};
          \node at (axis cs:2,1.95) [anchor=north,penColor2] {$g(x)$};
          \node at (axis cs:4.7,4.3) [anchor=east,penColor3] {$f(x)+g(x)$};

          \node at (axis cs:5,3.91) [anchor=west] {${\color{penColor}f(x)}{\color{penColor4}g'(x)}+{\color{penColor5}f'(x)}{\color{penColor2}g(x)}$};
          \node at (axis cs:5,1.65) [anchor=west,penColor5] {$f'(x)$};
          \node at (axis cs:5,2.4) [anchor=west,penColor4] {$g'(x)$};
          \node at (axis cs:4.5,1.5) [anchor=north] {$\underbrace{\hspace{.40in}}_{1}$};
        \end{axis}
\end{tikzpicture}
\caption{A geometric interpretation of the product rule. Since every
  point on $f(x)g(x)$ is the product of the corresponding points on
  $f(x)$ and $g(x)$, the instantaneous growth rate of $f(x)g(x)$ is the
  sum of $f(x)g'(x)$ and $f'(x)g(x)$.}
\end{marginfigure}

\begin{proof}
From the limit definition of the derivative, write
\[
\ddx (f(x)g(x)) = \lim_{h \to0} \frac{f(x+h)g(x+h) - f(x)g(x)}{h}
\]
Now we use the exact same trick we used in the proof of
Theorem~\ref{theorem:limit-product}, we add $0 = -f(x+h)g(x) + f(x+h)g(x)$:
\begin{align*}
&=\lim_{h \to0} \frac{f(x+h)g(x+h){\color{penColor2}-f(x+h)g(x) + f(x+h)g(x)}- f(x)g(x)}{h} \\ 
&=\lim_{h \to0} \frac{f(x+h)g(x+h)-f(x+h)g(x)}{h} + \lim_{h \to0} \frac{f(x+h)g(x)- f(x)g(x)}{h}.
\end{align*}
Now since both $f(x)$ and $g(x)$ are differentiable, they are
continuous, see Theorem~\ref{theorem:diff-cont}. Hence
\begin{align*}
&=\lim_{h \to0} f(x+h)\frac{g(x+h)-g(x)}{h} + \lim_{h \to0} \frac{f(x+h)- f(x)}{h}g(x) \\ 
&=\lim_{h \to0} f(x+h)\lim_{h \to0}\frac{g(x+h)-g(x)}{h} + \lim_{h \to0} \frac{f(x+h)- f(x)}{h}\lim_{h \to0}g(x) \\ 
&=f(x)g'(x) + f'(x)g(x).
\end{align*}
\end{proof}



Let's return to the example with which we started.
\begin{example} 
Let $f(x)=(x^2+1)$ and $g(x)=(x^3-3x)$. Compute:
\[
\ddx f(x)g(x).
\]
\end{example}
\begin{solution}
Write
\begin{align*}
\ddx f(x)g(x) &= f(x)g'(x) + f'(x)g(x)\\
&=(x^2+1)(3x^2-3) + 2x(x^3-3x).
\end{align*}
We could stop here---but we should show that expanding this out recovers our previous result. Write
\begin{align*}
(x^2+1)(3x^2-3) + 2x(x^3-3x) &= 3x^4-3x^2 +3x^2 -3 + 2x^4-6x^2\\
&=5x^4-6x^2-3,
\end{align*}
which is precisely what we obtained before.
\end{solution}




\begin{exercises}

\noindent Compute:

\twocol

\begin{exercise} $\ddx x^3(x^3-5x+10)$
\begin{answer} $3x^2(x^3-5x+10)+x^3(3x^2-5)$
\end{answer}\end{exercise}

\begin{exercise} $\ddx (x^2+5x-3)(x^5-6x^3+3x^2-7x+1)$
\begin{answer} $(x^2+5x-3)(5x^4-18x^2+6x-7)+(2x+5)(x^5-6x^3+3x^2-7x+1)$
\end{answer}\end{exercise}

\begin{exercise} $\ddx \sqrt{x}\sqrt{625-x^2}$
\begin{answer} $\frac{\sqrt{625-x^2}}{2\sqrt{x}}-\frac{x\sqrt{x}}{\sqrt{625-x^2}}$
\end{answer}\end{exercise}

\begin{exercise} $\ddx\frac{\sqrt{625-x^2}}{x^{20}}$
\begin{answer} $\frac{-1}{x^{19}\sqrt{625-x^2}}-\frac{20\sqrt{625-x^2}}{x^{21}}$
\end{answer}\end{exercise}

\begin{exercise} $\ddx e^{2x} = \ddx \left(e^x \cdot e^x\right)$
\begin{answer}
$2e^{2x}$
\end{answer}
\end{exercise}

\begin{exercise} $\ddx e^{3x}$
\begin{answer} $3e^{3x}$
\end{answer}\end{exercise}


\begin{exercise} $\ddx 3x^2e^{4x}$
\begin{answer} $6xe^{4x}+12x^2e^{4x}$
\end{answer}\end{exercise}


\begin{exercise} $\ddx \frac{3e^x}{x^{16}}$
\begin{answer} $\frac{-48e^x}{x^{17}}+\frac{3e^x}{x^{16}}$
\end{answer}\end{exercise}


\endtwocol


\begin{exercise} 
Use the product rule to compute the derivative of $f(x)=(2x-3)^2$ with
respect to $x$.  Sketch the function.  Find an equation of the tangent
line to the curve at $x=2$.  Sketch the tangent line at $x=2$.
\begin{answer} $f'=4(2x-3)$, $y=4x-7$
\end{answer}\end{exercise}

\begin{exercise}
Suppose that $f$, $g$, and $h$ are differentiable functions.
Show that 
\[
\ddx f(x)\cdot g(x) \cdot h(x) = f(x) g(x)h'(x) + f(x)g'(x) h(x) + f'(x) g(x) h(x).
\]
\begin{answer}
$\ddx f(x)g(x)h(x) = \ddx f(x)(g(x)h(x)) = f(x) \ddx (g(x) h(x)) + f'(x) g(x) h(x) = f(x) (g(x) h'(x)+ g'(x)h(x)) + f'(x) g(x) h(x) = f(x)g(x)h'(x)+ f(x) g'(x)h(x)) + f'(x) g(x) h(x)$
\end{answer}
\end{exercise}
\end{exercises}





\section{The Quotient Rule}

\index{quotient rule} 

We'd like to have a formula to compute
\[
\ddx \frac{f(x)}{g(x)}
\]
if we already know $f'(x)$ and $g'(x)$. Instead of attacking this
problem head-on, let's notice that we've already done part of the
problem: $f(x)/g(x)= f(x)\cdot(1/g(x))$, that is, this is really a
product, and we can compute the derivative if we know $f'(x)$ and
$(1/g(x))'$. This brings us to our next derivative rule.

\begin{mainTheorem}[The Quotient Rule]\index{derivative rules!quotient}\index{quotient rule}\label{theorem:quotient-rule}
If $f(x)$ and $g(x)$ are differentiable, then
\[
\ddx \frac{f(x)}{g(x)} = \frac{f'(x)g(x)-f(x)g'(x)}{g(x)^2}.
\]
\end{mainTheorem}
\begin{proof}
First note that if we knew how to compute
\[
\ddx \frac{1}{g(x)}
\]
then we could use the product rule to complete our proof.  Write
\begin{align*}
\ddx\frac{1}{g(x)}&=\lim_{h\to0} \frac{\frac{1}{g(x+h)}-\frac{1}{g(x)}}{h} \\
&=\lim_{h\to0} \frac{\frac{g(x)-g(x+h)}{g(x+h)g(x)}}{h} \\
&=\lim_{h\to0} \frac{g(x)-g(x+h)}{g(x+h)g(x)h} \\
&=\lim_{h\to0} -\frac{g(x+h)-g(x)}{h} \frac{1}{g(x+h)g(x)} \\
&=-\frac{g'(x)}{g(x)^2}.
\end{align*}
Now we can put this together with the product rule:
\begin{align*}
\ddx\frac{f(x)}{g(x)} &=f(x)\frac{-g'(x)}{g(x)^2}+f'(x)\frac{1}{g(x)}\\
&=\frac{-f(x)g'(x)+f'(x)g(x)}{g(x)^2}\\
&=\frac{f'(x)g(x)-f(x)g'(x)}{g(x)^2}.
\end{align*}

\end{proof}


\begin{example}
Compute:
\[
\ddx \frac{x^2+1}{x^3-3x}.
\]
\end{example}

\begin{solution}
Write
\begin{align*}
\ddx \frac{x^2+1}{x^3-3x} &= \frac{2x(x^3-3x)-(x^2+1)(3x^2-3)}{(x^3-3x)^2}\\
&=\frac{-x^4-6x^2+3}{(x^3-3x)^2}.
\end{align*}
\end{solution}

It is often possible to calculate derivatives in more than one way, as
we have already seen. Since every quotient can be written as a
product, it is always possible to use the product rule to compute the
derivative, though it is not always simpler.

\begin{example}
Compute 
\[
\ddx \frac{625-x^2}{\sqrt{x}}
\] 
in two ways. First using the quotient rule and then using the product
rule.
\end{example}

\begin{solution} 
First, we'll compute the derivative using the quotient rule. Write
\[
\ddx \frac{625-x^2}{\sqrt{x}} = \frac{\left(-2x\right)\left(\sqrt{x}\right) - (625-x^2)\left(\frac{1}{2}x^{-1/2}\right)}{x}.
\]
Second, we'll compute the derivative using the product rule:
\begin{align*}
\ddx \frac{625-x^2}{\sqrt{x}} &= \ddx \left(625-x^2\right)x^{-1/2}\\
&=\left(625-x^2\right)\left(\frac{-x^{-3/2}}{2}\right)+ (-2x)\left(x^{-1/2}\right).
\end{align*}
With a bit of algebra, both of these simplify to
\[
-\frac{3x^2+625}{2x^{3/2}}.
\]
\end{solution}


\begin{exercises}

\noindent Find the derivatives of the following functions using the
quotient rule.

\twocol

\begin{exercise} ${x^3\over x^3-5x+10}$
\begin{answer} ${3x^2\over x^3-5x+10}-{x^3(3x^2-5)\over (x^3-5x+10)^2}$
\end{answer}\end{exercise}

\begin{exercise} ${x^2+5x-3\over x^5-6x^3+3x^2-7x+1}$
\begin{answer} ${2x+5\over x^5-6x^3+3x^2-7x+1}-
{(x^2+5x-3)(5x^4-18x^2+6x-7)\over(x^5-6x^3+3x^2-7x+1)^2}$
\end{answer}\end{exercise}

\ssk
\begin{exercise} ${\sqrt{x}\over\sqrt{625-x^2}}$
\begin{answer} ${1\over 2\sqrt{x}\sqrt{625-x^2}}+{x^{3/2}\over(625-x^2)^{3/2}}$
\end{answer}\end{exercise}

\begin{exercise} ${\sqrt{625-x^2}\over x^{20}}$
\begin{answer} ${-1\over x^{19}\sqrt{625-x^2}}-{20\sqrt{625-x^2}\over x^{21}}$
\end{answer}\end{exercise}

\begin{exercise} $\frac{e^x-4}{2x}$
\begin{answer} $NONE YET$
\end{answer}\end{exercise}
\endtwocol
\bsk

\begin{exercise} Find an equation for the tangent line to $f(x) = (x^2 -
4)/(5-x)$ at $x= 3$.  
\begin{answer} $y=17x/4-41/4$ 
\end{answer}\end{exercise}

\begin{exercise}  Find an equation for the tangent line to 
$f(x) = (x-2)/(x^3 + 4x - 1)$ at $x=1$.
\begin{answer} $y=11x/16-15/16$
\end{answer}\end{exercise}

\begin{exercise} Let $P$ be a polynomial of degree $n$ and let $Q$ be a
polynomial of degree $m$ (with $Q$ not the zero polynomial). 
Using sigma notation we can write
$$P=\sum _{k=0 } ^n a_k x^k,\qquad
Q=\sum_{k=0}^m b_k x^k.
$$
Use sigma notation to write the derivative of the 
{\dfont rational function\index{rational function}}
$P/Q$.
%% \begin{answer} $\left(\sum_{k=0}^m b_k x^k\sum _{k=1}^n ka_k x^{k-1}-
%% \sum _{k=0 } ^n a_k x^k\sum_{k=1}^m kb_k x^k\right)/
%% \left(\sum_{k=0}^m b_k x^k\right)^2$
%% \end{answer}
\end{exercise}

\begin{exercise} The curve $y=1/(1+x^2)$ is an example of a class of
curves each of which is called a {\dfont witch of
Agnesi\index{witch of Agnesi}}. 
Sketch the curve and find the tangent line to the curve at
$x= 5$. (The word {\em witch\/} here is a mistranslation of the
original Italian, as described at
% BADBAD
% $$\hbox{\url{http://mathworld.wolfram.com/WitchofAgnesi.html} 
% \vb|http://mathworld.wolfram.com/WitchofAgnesi.html|\endurl}$$
% and 
% \begin{align*}
% \hbox{\url{http://instructional1.calstatela.edu/sgray/Agnesi/WitchHistory/Historynamewitch.html} 
% \vb|http://|\endurl}%
% &\!\!\hbox{\url{http://instructional1.calstatela.edu/sgray/Agnesi/WitchHistory/Historynamewitch.html} 
% \vb|instructional1.calstatela.edu/sgray/Agnesi/|\endurl} \\
% &\hbox{\url{http://instructional1.calstatela.edu/sgray/Agnesi/WitchHistory/Historynamewitch.html} 
% \vb|WitchHistory/Historynamewitch.html|\endurl.)} \\
% \end{align*}
\begin{answer} $y=19/169-5x/338$
\end{answer}\end{exercise}
%% \footnote{Due to a mistranslation of the Italian word
%%   \emph{versiera} which actually refers to a rope that turns the
%%   sail.}.  
 
\begin{exercise} If $f'(4) = 5$, $g'(4) = 12$, $(fg)(4)= f(4)g(4)=2$, and $g(4) = 6$,
compute $f(4)$ and $\ds{d\over dx}{f\over g}$ at 4.
\begin{answer} $13/18$
\end{answer}\end{exercise}

\end{exercises}
