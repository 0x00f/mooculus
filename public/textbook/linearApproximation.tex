\chapter{Linear Approximation}

\section{Linear Approximation and Differentials}



Given a function, a \textit{linear approximation} is a fancy phrase
for something you already know.

\begin{definition}\index{linear approximation}
If $f(x)$ is a differentiable function at $x=a$, then a \textbf{linear
  approximation} for $f(x)$ at $x=a$ is given by
\[
\l(x) = f'(a)(x-a) +f(a).
\]
\end{definition}

A linear approximation of $f(x)$ is a good approximation of $f(x)$ as
long as $x$ is ``not too far'' from $a$.  As we see from
Figure~\ref{figure:informal-tangent}, if one can ``zoom in'' on $f(x)$
sufficiently, then $f(x)$ and the linear approximation are nearly
indistinguishable. Linear approximations allow us to make approximate
``difficult'' computations.

\begin{example}
Use a linear approximation of $f(x) =\sqrt[3]{x}$ at $x=64$ to
approximate $\sqrt[3]{50}$.
\end{example}



\begin{marginfigure}
\begin{tikzpicture}
	\begin{axis}[
            xmin=1,xmax=100,ymin=0,ymax=5,
            axis lines=center,
            xlabel=$x$, ylabel=$y$,
            every axis y label/.style={at=(current axis.above origin),anchor=south},
            every axis x label/.style={at=(current axis.right of origin),anchor=west},
          ]        
          \addplot [very thick, penColor, samples=150,smooth,domain=(0:100)] {x^(1/3))};
          \addplot [very thick, penColor2, domain=(0:100)] {x/48+8/3};
          \addplot [textColor,dashed] plot coordinates {(64,0) (64,4)};
          \addplot [textColor,dashed] plot coordinates {(0,4) (64,4)};
          \node at (axis cs:20,2.3) [penColor] {$f(x)$};
          \node at (axis cs:20,3.3) [penColor2] {$\l(x)$};
          \addplot[color=penColor3,fill=penColor3,only marks,mark=*] coordinates{(64,4)};  %% closed hole            
        \end{axis}
\end{tikzpicture}
\caption{A linear approximation of $f(x) = \sqrt[3]{x}$ at $x=4$.}
\label{figure:la sqrt3x}
\end{marginfigure}


\begin{solution}
To start, write
\[
\ddx f(x) = \ddx x^{1/3} = \frac{1}{3x^{2/3}}.
\]
so our linear approximation is
\begin{align*}
\l(x) &= \frac{1}{3\cdot 64^{2/3}} (x-64) + 4 \\
&= \frac{1}{48} (x-64) + 4\\
&= \frac{x}{48} +\frac{8}{3}.
\end{align*}
Now we evaluate $\l(50) \approx 3.71$ and compare it to
$\sqrt[3]{50}\approx 3.68$, see Figure~\ref{figure:la sqrt3x}. From
this we see that the linear approximation, while perhaps inexact, is
computationally \textbf{easier} than computing the cube root.
\end{solution}

With modern calculators and computing software it may not appear
necessary to use linear approximations. But in fact they are quite
useful. In cases requiring an explicit numerical approximation, they
allow us to get a quick rough estimate which can be used as a
``reality check'' on a more complex calculation. In some complex
calculations involving functions, the linear approximation makes an
otherwise intractable calculation possible, without serious loss of
accuracy.




\begin{example}%\label{exam:linear approximation of sine}
Use a linear approximation of $f(x) =\sin(x)$ at $x=0$ to approximate
$\sin(0.3)$.
\end{example}

\begin{marginfigure}
\begin{tikzpicture}
	\begin{axis}[
            xmin=-1.6,xmax=1.6,ymin=-1.5,ymax=1.5,
            axis lines=center,
            xtick={-1.57, 0, 1.57},
            xticklabels={$-\pi/2$, $0$, $\pi/2$},
            ytick={-1,1},
            %ticks=none,
            %width=3in,
            %height=2in,
            unit vector ratio*=1 1 1,
            xlabel=$x$, ylabel=$y$,
            every axis y label/.style={at=(current axis.above origin),anchor=south},
            every axis x label/.style={at=(current axis.right of origin),anchor=west},
          ]        
          \addplot [very thick, penColor, samples=100,smooth, domain=(-1.6:1.6)] {sin(deg(x))};
          \addplot [very thick, penColor2, samples=100,smooth] {x};
          \node at (axis cs:1,.6) [penColor] {$f(x)$};
          \node at (axis cs:-1,-1.2) [penColor2] {$\l(x)$};
          \addplot[color=penColor3,fill=penColor3,only marks,mark=*] coordinates{(0,0)};  %% closed hole          
        \end{axis}
\end{tikzpicture}
\caption{A linear approximation of $f(x) = \sin(x)$ at $x=0$.}
\label{figure:la sin}
\end{marginfigure}

\begin{solution}
To start, write
\[
\ddx f(x) = \cos(x),
\]
so our linear approximation is
\begin{align*}
\l(x) &= \cos(0)\cdot(x-0) + 0\\
&= x.
\end{align*}
Hence a linear approximation for $\sin(x)$ at $x=0$ is $\l(x) = x$,
and so $\l(0.3) = 0.3$.  Comparing this to $\sin(.3) \approx 0.295$. As
we see the approximation is quite good. For this reason, it is common
to approximate $\sin(x)$ with its linear approximation $\l(x) = x$
when $x$ is near zero, see Figure~\ref{figure:la sin}.
\end{solution}


\subsection*{Differentials}

The notion of a \textit{differential} goes back to the origins of
calculus, though our modern conceptualization of a differential is
somewhat different than how they were initially understood.

\begin{definition}\index{differential}
Let $f(x)$ be a differentiable function. We define a new
  independent variable $dx$, and a new dependent variable
\[
dy=f'(x)\cdot dx.
\] 
The variables $dx$ and $dy$ are called \textbf{differentials}, see
Figure~\ref{figure:differentials}.
\end{definition}
\begin{marginfigure}[0in]
\begin{tikzpicture}
	\begin{axis}[
            xmin=1, xmax=2, range=0:6,ymax=6,ymin=0,
            axis lines =left, xlabel=$x$, ylabel=$y$,
            every axis y label/.style={at=(current axis.above origin),anchor=south},
            every axis x label/.style={at=(current axis.right of origin),anchor=west},
            ticks=none,
            axis on top,
          ]         
	  \addplot [penColor2,very thick] plot coordinates {(1.4,10/6) (1.7,10/6)};
          \addplot [penColor2,very thick] plot coordinates {(1.7,10/6) (1.7,10/6 +.3/.36)};
          \addplot [penColor,dashed] plot coordinates {(1.4,10/6) (1.7,10/6 +.3/.36)};
          \addplot [very thick,penColor, smooth,samples=100,domain=(0:1.833)] {-1/(x-2)};
          \addplot[color=penColor3,fill=penColor3,only marks,mark=*] coordinates{(1.4,10/6)};  %% closed hole            
          \node at (axis cs:1.55,1.67) [below, penColor2] {$dx$};
          \node at (axis cs:1.7,2.08) [right, penColor2] {$dy$};
        \end{axis}
\end{tikzpicture}
\caption{While $dy$ and $dx$ are both variables, $dy$ depends on $dx$,
  and approximates how much a function grows after a change of size
  $dx$ from a given point.}
\label{figure:differentials}
\end{marginfigure}

Note, it is now the case (by definition!) that 
\[
\frac{dy}{dx} = f'(x).
\]

Essentially, differentials allow us to solve the problems presented in
the previous examples from a slightly different point of view. Recall,
when $h$ is near but not equal zero,
\[
f'(x) \approx \frac{f(x+h)-f(x)}{h}
\]
hence, 
\[
f'(x)h \approx f(x+h)-f(x)
\]
since $h$ is simply a variable, and $dx$ is simply a variable, we can replace $h$ with $dx$ to write
\begin{align*}
f'(x)\cdot dx &\approx f(x+dx)-f(x)\\
dy &\approx f(x+dx)-f(x).
\end{align*}
From this we see that 
\[
f(x+dx)\approx dy + f(x).
\]
While this is something of a ``sleight of hand'' with variables, there
are contexts where the language of differentials is common. We will
repeat our previous examples using differentials.

\begin{example}
Use differentials to approximate $\sqrt[3]{50}$.
\end{example}

\begin{marginfigure}
\begin{tikzpicture}
	\begin{axis}[
            xmin=1,xmax=100,ymin=2,ymax=5,
            axis lines=center,
            xlabel=$x$, ylabel=$y$,
            every axis y label/.style={at=(current axis.above origin),anchor=south},
            every axis x label/.style={at=(current axis.right of origin),anchor=west},
          ]        
          \addplot [very thick, penColor, samples=150,smooth,domain=(0:100)] {x^(1/3))};
          %\addplot [very thick, penColor2, domain=(50:64)] {x/48+8/3};
          %\addplot [textColor,dashed] plot coordinates {(64,0) (64,4)};
          \node at (axis cs:20,2.3) [penColor] {$f(x)$};
          \addplot [penColor2,very thick] plot coordinates {(64,4) (64,3.71)};
          \addplot [penColor2,very thick] plot coordinates {(50,3.71) (64,3.71)};
          \node [below] at (axis cs:57,3.7) [penColor2] {$dx$};
          \node [right] at (axis cs:64,3.8) [penColor2] {$dy$};
          \addplot[color=penColor3,fill=penColor3,only marks,mark=*] coordinates{(64,4)};  %% closed hole            
        \end{axis}
\end{tikzpicture}
\caption{A plot of $f(x) = \sqrt[3]{x}$  along with the differentials $dx$ and
  $dy$.}
\label{figure:diff sqrt3x}
\end{marginfigure}


\begin{solution}
Since $4^3 = 64$ is a perfect cube near $50$, we will set $dx=-14$. In this case
\[
\frac{dy}{dx} = f'(x)  = \frac{1}{3x^{2/3}}
\]
hence 
\begin{align*}
dy &= \frac{1}{3x^{2/3}} \cdot dx\\
&= \frac{1}{3\cdot64^{2/3}} \cdot(-14)\\
&= \frac{1}{3\cdot64^{2/3}} \cdot(-14)\\
&= \frac{-7}{24}
\end{align*}
Now $f(50) \approx f(64) + \frac{-7}{24} \approx 3.71$.
\end{solution}


\begin{example}
Use differentials to approximate $\sin(0.3)$.
\end{example}

\begin{marginfigure}
\begin{tikzpicture}
	\begin{axis}[
            xmin=-1.6,xmax=1.6,ymin=-1.5,ymax=1.5,
            axis lines=center,
            xtick={-1.57, 0, 1.57},
            xticklabels={$-\pi/2$, $0$, $\pi/2$},
            ytick={-1,1},
            %ticks=none,
            %width=3in,
            %height=2in,
            unit vector ratio*=1 1 1,
            xlabel=$x$, ylabel=$y$,
            every axis y label/.style={at=(current axis.above origin),anchor=south},
            every axis x label/.style={at=(current axis.right of origin),anchor=west},
          ]        
          \addplot [very thick, penColor, samples=100,smooth, domain=(-1.6:1.6)] {sin(deg(x))};
          %\addplot [very thick, penColor2, domain={0:0.3}] {x};
          \addplot [penColor2,very thick] plot coordinates {(0,0) (.3,0)};
          \addplot [penColor2,very thick] plot coordinates {(.3,0) (.3,.3)};
          \node at (axis cs:1,.6) [penColor] {$f(x)$};
          \node [below] at (axis cs:.15,.0) [penColor2] {$dx$};
          \node [right] at (axis cs:.3,.15) [penColor2] {$dy$};
          \addplot[color=penColor3,fill=penColor3,only marks,mark=*] coordinates{(0,0)};  %% closed hole          
        \end{axis}
\end{tikzpicture}
\caption{A plot of $f(x) = \sin(x)$ along with the differentials $dx$ and
  $dy$.}
\label{figure:diff sin}
\end{marginfigure}


\begin{solution}
Since $\sin(0) = 0$, we will set $dx=0.3$. In this case
\[
\frac{dy}{dx} = f'(x)  = \cos(x)
\]
hence 
\begin{align*}
dy &= \cos(0) \cdot dx\\
&= 1 \cdot (0.3)\\
&= 0.3
\end{align*}
Now $f(.3) \approx f(0) + 0.3 \approx 0.3$.
\end{solution}

The upshot is that linear approximations and differentials are simply
two slightly different ways of doing the exact same thing.





\begin{exercises}

\begin{exercise}
Use a linear approximation of $f(x) =\sin(x/2)$ at $x=0$ to approximate
$f(0.1)$.
\begin{answer}
$\sin(0.1/2)\approx 0.05$
\end{answer}
\end{exercise}

\begin{exercise}
Use a linear approximation of $f(x) =\sqrt[3]{x}$ at $x=8$ to approximate
$f(10)$.
\begin{answer}
$\sqrt[3]{10}\approx 2.17$
\end{answer}
\end{exercise}

\begin{exercise}
Use a linear approximation of $f(x) =\sqrt[5]{x}$ at $x=243$ to approximate
$f(250)$.
\begin{answer}
$\sqrt[5]{250}\approx 3.017$
\end{answer}
\end{exercise}


\begin{exercise}
Use a linear approximation of $f(x) =\ln(x)$ at $x=1$ to approximate
$f(1.5)$.
\begin{answer}
$\ln(1.5)\approx 0.5$
\end{answer}
\end{exercise}

\begin{exercise}
Use a linear approximation of $f(x) =\ln(\sqrt{x})$ at $x=1$ to approximate
$f(1.5)$.
\begin{answer}
$\ln(\sqrt{1.5})\approx 0.25$
\end{answer}
\end{exercise}


\begin{exercise} 
Let $f(x) = \sin(x/2)$. If $x=1$ and $dx=1/2$, what is $dy$?
\begin{answer} $dy=0.22$
\end{answer}\end{exercise}

\begin{exercise} 
Let $f(x) = \sqrt{x}$. If $x=1$ and $dx =1/10$, what is $dy$?
\begin{answer} $dy=0.05$
\end{answer}\end{exercise}

\begin{exercise} 
Let $f(x) = \ln(x)$. If $x=1$ and $dx =1/10$, what is $dy$?
\begin{answer} $dy=0.1$
\end{answer}\end{exercise}


\begin{exercise} 
Let $f(x) = \sin (2x)$. If $x=\pi$ and $dx=\pi/100$, what is $dy$?
\begin{answer} $dy=\pi/50$
\end{answer}\end{exercise}

\begin{exercise} Use differentials to estimate the amount of paint needed to
 apply a coat of paint 0.02 cm thick to a sphere with diameter $40$
 meters. Hint: Recall that the volume of a sphere of radius $r$ is $V
 =(4/3)\pi r^3$. Note that you are given that $dr=0.02$ cm.
\begin{answer} $dV=8\pi/25 \text{m}^3$
\end{answer}\end{exercise}

\end{exercises}












\section{Iterative Methods}

\subsection*{Newton's Method}

Suppose you have a function $f(x)$, and you want to solve $f(x)=0$.
Solving equations symbolically, is difficult. However, Newton's method
gives us a procedure, for finding a solution to many equations to as
many decimal places as you want.  

\marginnote[.5in]{The point
\[
a_{n+1} = a_n - \frac{f(a_n)}{f'(a_n)}
\]
is the solution to the equation $\l_n(x) = 0$, where $\l_n(x)$ is the
linear approximation of $f(x)$ at $x=a_n$.}

\begin{newtonsMethod}
Let $f(x)$ be a differentiable function and let $a_0$ be a
guess for a solution to the equation
\[
f(x) = 0.
\]
We can produce a sequence of points $x=a_0, a_1, a_2, a_3, \dots$ via
the recursive formula
\[
a_{n+1} = a_n -\frac{f(a_n)}{f'(a_n)}
\]
that (hopefully!) are successively better approximations of a solution
to the equation $f(x) = 0$.
\end{newtonsMethod}
Let's see if we can explain the logic behind this method. Consider the
following cubic function
\[
f(x) = x^3 - 4 x^2 - 5 x - 7.
\]
While there is a ``cubic formula'' for finding roots, it can be
difficult to implement. Since it is clear that $f(10)$ is positive,
and $f(0)$ is negative, by the Intermediate Value
Theorem~\ref{theorem:IVT}, there is a solution to the equation $f(x) =
0$ in the interval $[0,10]$. Let's compute $f'(x) = 3x^2 -8x-5$ and
guess that $a_0=7$ is a solution. We can easily see that
\[
f(a_0) = f(7) = 105\qquad\text{and}\qquad f'(a_0) = f'(7) = 86.
\]
This might seem pretty bad, but if we look at the linear approximation
of $f(x)$ at $x=7$, we find
\begin{marginfigure}[-1in]
\begin{tikzpicture}
	\begin{axis}[
            xmin=4, xmax=7.5,ymin=-40,ymax=125,domain=(4:8),
            axis lines =center, xlabel=$x$, ylabel=$y$,
            every axis y label/.style={at=(current axis.above origin),anchor=south},
            every axis x label/.style={at=(current axis.right of origin),anchor=west},
            xtick={7}, ytickmin=1,ytickmax=0,
            xticklabels={$a_0$},
            axis on top,
          ]         
	  \addplot [penColor2,very thick] {86*(x-7) + 105};
          \addplot [textColor,dashed] plot coordinates {(7,0) (7,105)};
          \addplot [very thick,penColor, smooth,samples=100] {x^3 - 4*x^2 - 5*x - 7};
          \addplot[color=penColor3,fill=penColor3,only marks,mark=*] coordinates{(7,105)};  %% closed hole          
        \end{axis}
\end{tikzpicture}
\caption{Here we see our first guess, along with the linear
  approximation at that point.}
\label{figure:newtonCubic1}
\end{marginfigure}
\[
\l_0(x) = 86(x-7) + 105 \qquad\text{which is the same as}\qquad \l_0(x) = f'(a_0)(x-a_0) + f(a_0).
\]
Now $\l_0(a_1) = 0$ when
\[
a_1 = 7 - \frac{105}{86} \qquad\text{which is the same as}\qquad a_1 = a_0
-\frac{f(a_0)}{f'(a_0)}.
\]
To remind you what is going on geometrically see
Figure~\ref{figure:newtonCubic1}. Now we repeat the procedure letting
$a_1$ be our new guess. Now
\[
f(a_1) \approx 23.5.
\]
We see our new guess is better than our first. If we look at the
linear approximation of $f(x)$ at $x=a_1$, we find
\[
\l_1(x) = f'(a_0)(x-a_0) + f(a_0).
\]
Now $\l_1(a_2) = 0$ when
\[
a_2 = a_1 - \frac{f(a_1)}{f'(a_1)}.
\]

\begin{marginfigure}[-2in]
\begin{tikzpicture}
	\begin{axis}[
            xmin=4, xmax=7.5,ymin=-40,ymax=125,domain=(4:8),
            axis lines =center, xlabel=$x$, ylabel=$y$,
            every axis y label/.style={at=(current axis.above origin),anchor=south},
            every axis x label/.style={at=(current axis.right of origin),anchor=west},
            xtick={5.78,7}, ytickmin=1,ytickmax=0,
            xticklabels={$a_1$,$a_0$},
            axis on top,
          ]         
	  \addplot [penColor2!40!background,very thick] {86*(x-7) + 105};
          \addplot [textColor!40!background,dashed] plot coordinates {(7,0) (7,105)};
          \addplot [penColor2,very thick] {49*x-259.4};
          \addplot [textColor,dashed] plot coordinates {(5.78,0) (5.78,23.57)};
          \addplot [very thick,penColor, smooth,samples=100] {x^3 - 4*x^2 - 5*x - 7};
          \addplot[color=penColor3,fill=penColor3,only marks,mark=*] coordinates{(5.78,23.57)};  %% closed hole          
        \end{axis}
\end{tikzpicture}
\caption{Here we see our second guess, along with the linear
  approximation at that point.}
\label{figure:newtonCubic2}
\end{marginfigure}
See Figure~\ref{figure:newtonCubic2} to see what is going on
geometrically.  Again, we repeat our procedure letting $a_2$ be our
next guess, note
\[
f(a_2) \approx 2.97,
\]
we are getting much closer to a root of $f(x)$. Looking at the linear
approximation of $f(x)$ at $x=a_2$, we find
\[
\l_2(x) = f'(a_2)(x-a_2) + f(a_2).
\]
Setting $a_3 = a_2 - \frac{f(a_2)}{f'(a_2)}$, $a_3\approx 5.22$. We now
have $\l_2(a_3) = 0$. Checking by evaluating $f(x)$ at $a_3$, we find
\[
f(a_3) \approx 0.14.
\]

\begin{marginfigure}[0in]
\begin{tikzpicture}
	\begin{axis}[
            xmin=4, xmax=7.5,ymin=-40,ymax=125,domain=(4:8),
            axis lines =center, xlabel=$x$, ylabel=$y$,
            every axis y label/.style={at=(current axis.above origin),anchor=south},
            every axis x label/.style={at=(current axis.right of origin),anchor=west},
            xtick={5.22,5.78,7}, ytickmin=1,ytickmax=0,
            xticklabels={$a_2$,$a_1$,$a_0$},
            axis on top,
          ]         
	  \addplot [penColor2!40!background,very thick] {86*(x-7) + 105};
          \addplot [textColor!40!background,dashed] plot coordinates {(7,0) (7,105)};

          \addplot [penColor2!40!,very thick] {49*x-259.4};
          \addplot [textColor!40!,dashed] plot coordinates {(5.78,0) (5.78,23.57)};

          \addplot [penColor2,very thick] {36.8*x-192.2};
          \addplot [textColor,dashed] plot coordinates {(5.22,0) (5.22,.14)};

          \addplot [very thick,penColor, smooth,samples=100] {x^3 - 4*x^2 - 5*x - 7};
          \addplot[color=penColor3,fill=penColor3,only marks,mark=*] coordinates{(5.22,.14)};  %% closed hole          
        \end{axis}
\end{tikzpicture}
\caption{Here we see our third guess, along with the linear
  approximation at that point.}
\label{figure:newtonCubic3}
\end{marginfigure}
We are now very close to a root of $f(x)$, see
Figure~\ref{figure:newtonCubic3}. This process, Newton's Method, could
be repeated indefinitely to obtain closer and closer approximations to
a root of $f(x)$.


\begin{example}
Use Newton's Method to approximate the solution to 
\[
x^3= 50
\]
to two decimal places. 
\end{example}

\begin{solution} 
To start, set $f(x) = x^3 - 50$. We will use Newton's Method to
approximate a solution to the equation
\[
f(x) = x^3-50 = 0. 
\]
Let's choose $a_0=4$ as our first guess. Now compute 
\[
f'(x) = 3x^2.
\]
At this point we can make a table:
\[
\begin{tchart}{llll}
n &  a_n  & f(a_n)         & a_n - f(a_n)/f'(a_n)\\ \hline
0 & 4    & 14             & \approx 3.708 \\
1 & 3.708 & \approx 0.982  & \approx 3.684 \\
2 & 3.684 & \approx -0.001 & \approx 3.684
\end{tchart}
\]
Hence after only two iterations, we have the solution to three (and
hence two) decimal places.
\end{solution}

In practice, which is to say, if you need to approximate a value in
the course of designing a bridge or a building or an airframe, you
will need to have some confidence that the approximation you settle on
is accurate enough. As a rule of thumb, once a certain number of
decimal places stop changing from one approximation to the next it is
likely that those decimal places are correct. Still, this may not be
enough assurance, in which case we can test the result for accuracy.


Sometimes questions involving Newton's Method do not mention an
equation that needs to be solved. Here you must reinterpret the
question as one that is asking for a solution to an equation of the
form $f(x) = 0$.

\begin{example}
Use Newton's Method to approximate $\sqrt[3]{50}$ to two decimal
places.
\end{example}

\begin{solution}
The $\sqrt[3]{50}$ is simply a solution to the equation
\[
x^3 -50 = 0.
\]
Since we did this in the previous example, we have found
$\sqrt[3]{50}\approx 3.68$.
\end{solution}

\begin{warning}
Sometimes a bad choice for $a_0$ will not lead to a root. Consider 
\[
f(x) = x^3 - 3 x^2 - x - 4.
\]
If we choose our initial guess to be $a_0=1$ and make a table we find:
\[
\begin{tchart}{llll}
n &  a_n  & f(a_n)         & a_n - f(a_n)/f'(a_n)\\ \hline
0 & 1    & -7             & -0.75 \\
1 & -0.75 & \approx -5.359  & \approx 0.283 \\
2 & 0.283 & \approx -4.501 & \approx -1.548 \\
3 & -1.548 & \approx -13.350 & \approx -0.685 \\
4 & -0.685 & \approx -5.044 & \approx 0.432 \\
  \hdotsfor{4}
\end{tchart}
\]
As you can see, we are not converging to a root, which is
approximately $x=3.589$.
\end{warning}



Iterative procedures like Newton's method are well suited for
computers. It enables us to solve equations that are otherwise
impossible to solve through symbolic methods.






\subsection*{Euler's Method}


\marginnote[.2in]{The name ``Euler'' is pronounced ``Oiler.''}
While Newton's Method allows us to solve equations that are otherwise
impossible to solve, and hence is of computational importance,
Euler's Method is more of theoretical importance to us.

\break

\begin{eulersMethod}
Given a function $f(x)$, and an initial value $(x_0,y_0)$ we wish to
find a polygonal curve defined by $(x_n,y_n)$ such that this polygonal
curve approximates $F(x)$ where $F'(x) =f(x)$, and $F(x_0)= y_0$.
\begin{enumerate}
\item Choose a step size, call it $h$.
\item Our polygonal curve defined by connecting the points as
  described by the iterative process below:
\[
\begin{tchart}{lll}
n & x_n     & y_n \\ \hline
0 & x_0     & y_0 \\
1 & x_0 + h & y_0+h\cdot f(x_0)\\
2 & x_1 + h & y_1+h\cdot f(x_1)\\
3 & x_2 + h & y_2+h\cdot f(x_2)\\
4 & x_3 + h & y_3+h\cdot f(x_3)\\
  \hdotsfor{3}
\end{tchart}
\]
\end{enumerate}
\end{eulersMethod}


Let's see an example of Euler's Method in action.
x
\begin{example}
Suppose that the velocity in meters per second of a ball tossed from a
height of 1 meter is given by
\[
v(t) = -9.8t + 6.
\]
Rounding to two decimals at each step, use Euler's Method with $h=0.2$
to approximate the height of the ball after 1 second.
\end{example}
\begin{marginfigure}[0in]
\begin{tikzpicture}
	\begin{axis}[
            xmin=0, xmax=1,ymin=0,ymax=3.5,
            axis lines =center, xlabel=$t$, ylabel=$y$,
            every axis y label/.style={at=(current axis.above origin),anchor=south},
            every axis x label/.style={at=(current axis.right of origin),anchor=west},
            xtick={0,.2,.4,.6,.8,1},
            axis on top,
          ]         
	  \addplot [penColor2, very thick] plot coordinates {
            (0,1) (.2,2.2) (.4,3) (.6,3.424) (.8,3.45) (1,3.08)
          };
          \addplot [very thick,penColor, smooth,samples=100] {6*x-4.9*x^2+1};
          \node at (axis cs:.3,2) [penColor] {$f(x)$};          
        \end{axis}
\end{tikzpicture}
\caption{Here we see our polygonal curve found via Euler's Method and
  the (unknown) function $F(x)$. Choosing a smaller step-size $h$
  would yield a better approximation.}
\label{figure:eulerMethod-ball}
\end{marginfigure}

\begin{solution}
We simply need to make a table and use Euler's Method.
\[
\begin{tchart}{lll}
n & t_n & y_n \\ \hline
0 & 0   & 1 \\
1 & 0.2 & 2.2\\
2 & 0.4 & 3.01\\
3 & 0.6 & 3.42\\
4 & 0.8 & 3.45\\
5 & 1 & 3.08
\end{tchart}
\]
Hence the ball is at a height of about $3.08$ meters, see
Figure~\ref{figure:eulerMethod-ball}.
\end{solution}



\begin{exercises}

\begin{exercise} The function 
$f(x)=x^2-2x-5$ has a root between 3 and 4, because $f(3)=-2$ and
  $f(4)=3$. Use Newton's Method to approximate the root to two decimal
  places.
\begin{answer}  $3.45$
\end{answer}\end{exercise}

\begin{exercise} The function 
$f(x)=x^3-3x^2-3x+6$ has a root between 3 and 4, because $f(3)=-3$ and
  $f(4)=10$. Use Newton's Method to approximate the root to two
  decimal places.
\begin{answer}  $3.36$
\end{answer}\end{exercise}

\begin{exercise} The function 
$f(x)=x^5-2x^3+5$ has a root between $-2$ and $-1$, because
  $f(-2)=-11$ and $f(-1)=6$. Use Newton's Method to approximate the
  root to two decimal places.
\begin{answer}  $-1.72$
\end{answer}\end{exercise}


\begin{exercise} The function 
$f(x)=x^5-5x^4+5x^2-6$ has a root between $4$ and $5$, because
  $f(4)=-182$ and $f(5)=119$. Use Newton's Method to approximate the
  root to two decimal places.
\begin{answer}  $4.79$
\end{answer}\end{exercise}

\begin{exercise} Approximate the fifth root of 7, using $x_0=1.5$ as a
first guess. Use Newton's method to find $x_3$ as your
approximation.  
\begin{answer} $x_3=1.475773162$ 
\end{answer}\end{exercise}

\begin{exercise} Use Newton's Method to approximate the cube root of 10 to
two decimal places.
\begin{answer} $2.15$
\end{answer}\end{exercise}


\begin{exercise} A rectangular piece of cardboard of dimensions $8\times 17$
is used to make an open-top box by cutting out a small square of side
$x$ from each corner and bending up the sides.  If $x=2$, then the
volume of the box is $2\cdot 4\cdot 13=104$.  Use Newton's method to
find a value of $x$ for which the box has volume 100, accurate to two
decimal places.
\begin{answer} $2.19$ or $1.26$
\end{answer}
\end{exercise}

\begin{exercise}
Given $f(x) = 3x-4$, use Euler's Method with a step size $0.2$ to
estimate $F(2)$ where $F'(x) = f(x)$ and $F(1)=5$, to two decimal
places.
\begin{answer} $5.2$
\end{answer}
\end{exercise}


\begin{exercise}
Given $f(x) = x^2+2x+1$, use Euler's Method with a step size $0.2$ to
estimate $F(3)$ where $F'(x) = f(x)$ and $F(2)=3$, to two decimal
places.
\begin{answer} $14.64$
\end{answer}
\end{exercise}

\begin{exercise}
Given $f(x) = x^2-5x+7$, use Euler's Method with a step size $0.2$ to
estimate $F(2)$ where $F'(x) = f(x)$ and $F(1)=-4$, to two decimal
places.
\begin{answer} $-1.96$
\end{answer}
\end{exercise}



% Mike Wills stuff
% \iflatetranscendentals
% 
% \begin{remark} {Further investigation} Newton's method does not always
% work. The following result gives sufficient conditions for when it
% does.  \end{remark}
% 
% \begin{theorem} (Newton) Let $f:[a,b] \rightarrow \R$ be continuous. Suppose that
% $f(c) =0$ for some $c$ in $(a,b)$.  Suppose that $f''$ exists and is
% bounded on $(a,b)$; that is, there exists $M>0$ such that
% $$f''(x) \leq M $$ 
% for every $x\in (a,b)$.
% 
% Suppose that $f'$ is bounded away from zero; that is, there exists
% $\epsilon >0$ such that
% $$f'(x) \geq \epsilon $$ for every $x \in (a,b)$. Then there is a
% closed interval $I:=[d,e] \subset (a,b)$ containing $c$ such that
% the sequence given recursively by
% $$x_n = x_{n-1} -f(x_{n-1})/f'(x_{n -1}) $$ converges to
% $c$. Moreover, $x_n$ is in $I$ for each $n$.
% \end{proof}
% 
% \noindent
% For this to make sense it helps if we have a notion of convergence of
% sequences. We will do this in second semester calculus. For now, we
% will just use the intuitive statement ``as $n$ gets large, $x_n$ gets
% arbitrarily close to $c$" and write 
% $\displaystyle{\lim _{n\rightarrow\infty } x_n }= c$.
% 
% \noindent
% Let us now analyze the various conditions in the statement.
% 
% \noindent
% Bounding $f'$ away from zero should seem reasonable from the
% definition of $x_n$ but it is not necessary. The following
% exercise shows that Newton's method may work even if $f'$ is not
% bounded away from zero.
% 
% \begin{exercise} Let $f(x) = x^2$. Adopt the notation of Newton's
% theorem. Show that if $x_{n-1 } \neq 0$ then $x_n
% =x_{n-1}/2$. Suppose that $x_0 = 1.$ What is $x_1$? How about
% $x_2$? How about $x_n$? Argue that $x_n$ converges to zero
% as $n$ goes to $\infty$ when $x_0 =1$.
% 
% \noindent
% In the above example, any $x_0$ except zero will do.  The next
% exercise illustrates what can go wrong if we do not bound $f'$ away
% from zero.
% 
% \begin{exercise} Let $f(x) = x^4 - 4x - 11$. Then $f(0)<0 < f(3)$ so by
% the intermediate value theorem there exists a root between $x=0$ and
% $x=3$. What happens if we apply Newton's method when we start at $\ds
% x_0 =1$?  Sketch the graph of $f$ on the interval $[0,3]$.
%  
% \noindent
% We also require a bound on the second derivative. The following
% exercise shows what can go wrong.
% 
% \begin{exercise} Let $f(x) = x^{1/3}$. Observe that $x=0$ is a root of
% $f$.  Compute $f''(x)$ and explain why $f''$ is unbounded near
% $x=0$. Attempt to use Newton's method with $x_0 = 1$.  What
% happens? Illustrate your conclusion with a diagram.
%  
% \noindent
% Newton's method, when it works, is extremely fast. The number of
% decimal places of accuracy approximately doubles with each successive
% iteration.
%  
% \noindent
% The statement of Newton's theorem does not tell us how to how to find
% the closed interval $I$. The proof gives some indication but is beyond
% the scope of the course.
%  
% \noindent
% For our purposes, we can use a graphing calculator to give us an idea
% of where to look for the roots of a given function and then apply
% Newton's method to get a good decimal approximation.
%  
% \begin{exercise} Use Newton's method to find the coordinates of the
% inflection point of $y=x^3/12 + \sin x$ correct to six decimal
% places.
%  
% \begin{exercise} Let $a$ be a non-zero number. Suppose we want to compute a
% decimal expansion of $1/a$.  This could be difficult if $a$ is big or
% has a large number of decimal places. Newton's method gives us a way
% of computing $a$ without division.  Let $f(x) = 1/x - a$. Notice that
% $f$ has a root at $x=1/a$.  Using the notation of Newton's theorem,
% show that $x_{n} =2x_{n-1} - ax_{n-1}^2$.
%  
% \fi

\end{exercises}











\section{The Mean Value Theorem}

Here are some  interesting questions involving derivatives:

\begin{enumerate}
\item Suppose you toss a ball into the air and then catch it. Must the
  ball's vertical velocity have been zero at some point?
\item Suppose you drive a car from toll booth on a toll road to
  another toll booth $30$ miles away in half of an hour. Must you have
  been driving at $60$ miles per hour at some point?
\item Suppose two different functions have the same derivative. What
  can you say about the relationship between the two functions?
\end{enumerate}

While these problems sound very different, it turns out that the
problems are very closely related. We'll start simply:

\begin{mainTheorem}[Rolle's Theorem]\index{Rolle's Theorem} 
Suppose that $f(x)$ is differentiable on the interval $(a,b)$, is
continuous on the interval $[a,b]$, and $f(a)=f(b)$. Then 
\[
f'(c)=0
\]
for some $a<c<b$.
\label{thm:rolle}
\end{mainTheorem}
\begin{marginfigure}[0in]
\begin{tikzpicture}
	\begin{axis}[
            xmin=0, xmax=4.5,ymin=1,ymax=5,
            axis lines =left, xlabel=$x$, ylabel=$y$,
            every axis y label/.style={at=(current axis.above origin),anchor=south},
            every axis x label/.style={at=(current axis.right of origin),anchor=west},
            xtick={1,2,3}, xticklabels={$a$,$c$,$b$},
            ytickmin=1, ytickmax=0,
            axis on top,
          ]       
          \addplot [draw=none, fill=fill2,domain=(1:3)] {5} \closedcycle;       
	  \addplot [very thick,penColor, smooth] {-(x-2)^2+4};
          \addplot [very thick,penColor2, smooth] {4};
          \node at (axis cs:.4,2.5) [penColor] {$f(x)$}; 
          \addplot [textColor,dashed] plot coordinates {(2,0) (2,4)};
          \addplot [textColor,dashed] plot coordinates {(1,3) (3,3)};
          \addplot[color=penColor3,fill=penColor3,only marks,mark=*] coordinates{(2,4)};  %% closed hole          
          \addplot[color=penColor,fill=penColor,only marks,mark=*] coordinates{(1,3)};  %% closed hole          
          \addplot[color=penColor,fill=penColor,only marks,mark=*] coordinates{(3,3)};  %% closed hole          
        \end{axis}
\end{tikzpicture}
\caption{A geometric interpretation of Rolle's Theorem.}
\label{figure:geoRolle}
\end{marginfigure}
\begin{proof}
By the Extreme Value Theorem, Theorem~\ref{theorem:evt}, we know that
$f(x)$ has a maximum and minimum value on $[a,b]$.

If maximum and minimum both occur at the endpoints, then
$f(x)=f(a)=f(b)$ at every point in $[a,b]$. Hence the function is a
horizontal line, and it has derivative zero everywhere on
$(a,b)$. We may choose any $c$ at all to get $f'(c)=0$.

If the maximum or minimum occurs at a point $c$ with $a<c<b$, then by
Fermat's Theorem, Theorem~\ref{theorem:fermat}, $f'(c)=0$.
\end{proof}

We can now answer our first question above.

\begin{example}
Suppose you toss a ball into the air and then catch it. Must the
ball's vertical velocity have been zero at some point?
\end{example}

\begin{solution}
If $p(t)$ is the position of the ball at time $t$, then we may apply
Rolle's Theorem to see at some time $c$, $p'(c)=0$. Hence the velocity
must be zero at some point.
\end{solution}

Rolle's Theorem is a special case of a more general theorem.

\begin{mainTheorem}[Mean Value Theorem]\label{thm:mvt}\index{Mean Value Theorem}
Suppose that $f(x)$ has a derivative on the interval $(a,b)$ and is
continuous on the interval $[a,b]$.  Then
\[
f'(c)=\frac{f(b)-f(a)}{b-a}
\]
for some $a<c<b$. 
\end{mainTheorem}
\begin{marginfigure}[.5in]
\begin{tikzpicture}
	\begin{axis}[
            xmin=.5, xmax=5.5,ymin=0,ymax=3.1,
            axis lines =center, xlabel=$x$, ylabel=$y$,
            every axis y label/.style={at=(current axis.above origin),anchor=south},
            every axis x label/.style={at=(current axis.right of origin),anchor=west},
            xtick={1,2.04,5}, xticklabels={$a$,$c$,$b$},
            ytickmin=1, ytickmax=0,
            axis on top,
          ] 
          \addplot [draw=none, fill=fill2,domain=(1:5)] {3.1} \closedcycle;       
          \addplot [penColor2!40!background,very thick,dashed] plot coordinates {(1,.84+1.5) (5,1.5-.96)};        
          \addplot [textColor,dashed] plot coordinates {(2.04,0) (2.04,1.5+.89)};        
	  \addplot [very thick,penColor, smooth,domain=(1:5)] {sin(deg(x))+1.5};
          \addplot [very thick,penColor2,domain=(.5:5.5)] {-.45*(x-2.04)+.89+1.5};
          %\node at (axis cs:.4,2.5) [penColor] {$f(x)$}; 
          \addplot[color=penColor,fill=penColor,only marks,mark=*] coordinates{(1,.84+1.5)};  %% closed hole          
          \addplot[color=penColor,fill=penColor,only marks,mark=*] coordinates{(5,-.96+1.5)};  %% closed hole          
          \addplot[color=penColor3,fill=penColor3,only marks,mark=*] coordinates{(2.04,.89+1.5)};  %% closed hole          
        \end{axis}
\end{tikzpicture}
\caption{A geometric interpretation of the Mean Value Theorem}
\label{figure:geoMVT}
\end{marginfigure}
\begin{proof}
Let 
\[
m=\frac{f(b)-f(a)}{b-a},
\] 
and consider a new function $g(x)=f(x) - m(x-a)-f(a)$.  We know that
$g(x)$ has a derivative on $[a,b]$, since $g'(x)=f'(x)-m$. We can
compute $g(a)=f(a)- m(a-a)-f(a) =0$ and
\begin{align*}
g(b)=f(b)-m(b-a)-f(a)&=f(b)-{f(b)-f(a)\over b-a}(b-a)-f(a) \\
&=f(b)-(f(b)-f(a))-f(a)\\
&=0. 
\end{align*}
So $g(a) = g(b) = 0$. Now by Rolle's Theorem, that at some $c$,
\[
g'(c)=0\qquad\text{for some $a<c<b$}.
\]
But we know that $g'(c)=f'(c)-m$, so
\[
0=f'(c)-m=f'(c)-\frac{f(b)-f(a)}{b-a}.
\]
Hence
\[
f'(c)=\frac{f(b)-f(a)}{b-a}.
\]
\end{proof}

We can now answer our second question above.  

\begin{example}
Suppose you drive a car from toll booth on a toll road to another toll
booth $30$ miles away in half of an hour. Must you have been driving
at $60$ miles per hour at some point?
\end{example}

\begin{solution}
If $p(t)$ is the position of the car at time $t$, and $0$ hours is
the starting time with $1/2$ hours being the final time, the Mean Value Theorem states there is a time $c$
\[
p'(c) = \frac{30-0}{1/2} = 60\qquad \text{where $0<c<1/2$.}
\]
Since the derivative of position is velocity, this says that the car
must have been driving at $60$ miles per hour at some point.
\end{solution}

Now we will address the unthinkable, could there be a function $f(x)$
whose derivative is zero on an interval that is not constant? As we will
see, the answer is ``no.''

\begin{theorem} 
If $f'(x)=0$ for all $x$ in an interval $I$, then $f(x)$ is constant
on $I$.
\end{theorem}

\begin{proof}
Let $a< b$ be two points in $I$. By the Mean Value Theorem we know
\[
\frac{f(b)-f(a)}{b-a} = f'(c)
\]
for some $c$ in the interval $(a,b)$. Since $f'(c)=0$ we see that
$f(b)=f(a)$. Moreover, since $a$ and $b$ were arbitrarily chosen,
$f(x)$ must be the constant function.
\end{proof}

Now let's answer our third question.

\begin{example}
Suppose two different functions have the same derivative. What can you
say about the relationship between the two functions?
\end{example}

\begin{solution}
Set $h'(x) = f'(x) -g'(x)$. Now $h'(x) = 0$ on the interval
$(a,b)$. This means that $h(x) = k$ where $k$ is some constant. Hence
\[
g(x) = f(x) + k.
\]
\end{solution}


\begin{example}
Describe all functions whose derivative is $\sin(x)$.
\end{example}

\begin{solution}
One such function is $-\cos(x)$, so all such functions have the form
$-\cos(x)+k$, see Figure~\ref{figure:cos+k}.
\end{solution}
\begin{marginfigure}[0in]
\begin{tikzpicture}
	\begin{axis}[
            xmin=0, xmax=6.2,ymin=-4,ymax=4,domain=(0:6.2),
            axis lines =center, xlabel=$x$, ylabel=$y$,
            every axis y label/.style={at=(current axis.above origin),anchor=south},
            every axis x label/.style={at=(current axis.right of origin),anchor=west},
            axis on top,
          ] 
          \addplot [very thick,penColor, smooth] {-cos(deg(x))};
          \addplot [very thick,penColor2!30!background, smooth] {-cos(deg(x))+1};
          \addplot [very thick,penColor3!30!background, smooth] {-cos(deg(x))-1};
          \addplot [very thick,penColor4!30!background, smooth] {-cos(deg(x))+2};
          \addplot [very thick,penColor5!30!background, smooth] {-cos(deg(x))-2};         
        \end{axis}
\end{tikzpicture}
\caption{Functions of the form $-\cos(x)+k$, each of whose derivative is $\sin(x)$.}
\label{figure:cos+k}
\end{marginfigure}

\begin{exercises}

\begin{exercise} Let $f(x) = x^2$.
Find a value $c\in (-1,2)$ so that $f'(c)$ equals the slope between
the endpoints of $f(x)$ on $[-1,2]$.
\begin{answer} $c=1/2$
\end{answer}\end{exercise}

\begin{exercise} 
Verify that $f(x) = x/(x+2)$ satisfies the hypotheses of the Mean
Value Theorem on the interval $[1,4]$ and then find all of the values,
$c$, that satisfy the conclusion of the theorem.
\begin{answer} $c=\sqrt{18}-2$
\end{answer}
\end{exercise}

\begin{exercise}
Verify that $f(x) = 3x/(x+7)$ satisfies the hypotheses of the Mean
Value Theorem on the interval $[-2 , 6]$ and then find all of the
values, $c$, that satisfy the conclusion of the theorem.
\begin{answer} $c=\sqrt{65}-7$
\end{answer}
\end{exercise}

\begin{exercise} 
Let $f(x) = \tan(x)$. Show that $f(\pi ) = f(2\pi)=0$ but there is no
number $c\in (\pi,2\pi)$ such that $f'(c) =0$. Why does this not
contradict Rolle's theorem?
\begin{answer} $f(x)$ is not continuous on $[\pi,2\pi]$
\end{answer}
\end{exercise}

\begin{exercise} Let $f(x) = (x-3)^{-2}$.  Show that there is no value 
$c\in (1,4)$ such that $f'(c) = (f(4)-f(1))/(4-1)$.  Why is
this not a contradiction of the Mean Value Theorem?
\begin{answer} $f(x)$ is not continuous on $[1,4]$
\end{answer}
\end{exercise}

\begin{exercise} Describe all functions with derivative $x^2+47x-5$.
\begin{answer} $x^3/3+47x^2/2-5x+k$
\end{answer}\end{exercise}

\begin{exercise} Describe all functions with derivative ${1\over 1+x^2}$.
\begin{answer} $\arctan(x) + k$
\end{answer}\end{exercise}

\begin{exercise} Describe all functions with derivative $x^3-{1\over x}$.
\begin{answer} $x^4/4 -\ln(x) +k$
\end{answer}\end{exercise}


\begin{exercise} Describe all functions with derivative $\sin(2x)$.
\begin{answer} $-\cos(2x)/2 +k$
\end{answer}
\end{exercise}

\begin{exercise} Show that the equation $6x^4 -7x+1 =0$ does not have more
than two distinct real roots.
\begin{answer} 
Seeking a contradiction, suppose that we have 3 real roots, call them
$a$, $b$, and $c$. By Rolle's Theorem, $24x^3-7$ must have a root on
both $(a,b)$ and $(b,c)$, but this is impossible as $24x^3-7$ has only
one real root.
\end{answer}
\end{exercise}

\begin{exercise} Let $f(x)$ be differentiable on $\R$. Suppose that $f'(x) \neq
0$ for every $x$. Prove that $f$ has at most one real root.
\begin{answer} 
Seeking a contradiction, suppose that we have 2 real roots, call them
$a$, $b$. By Rolle's Theorem, $f'(x)$ must have a root on $(a,b)$, but
this is impossible.
\end{answer}
\end{exercise}
 
%% \begin{exercise} Prove that for all real $x$ and $y$
%% $|\cos x -\cos y | \leq |x-y|$.
%% State and prove an analogous result involving sine.
%% \end{exercise}

%% \begin{exercise}
%% Show that
%% $\sqrt{1+x} \le 1 +(x/2)$ if $-1<x<1$.
%% \end{exercise}

\end{exercises}

