\section{The Chain Rule}{}{}
\index{chain rule}

So far we have seen how to compute the derivative of a function built
up from other functions by addition, subtraction, multiplication and
division. There is another very important way that we combine simple
functions to make more complicated functions: function
composition\index{composition of functions}, as discussed in
section~\xrefn{sec:limits}. For example, consider $\ds
\sqrt{625-x^2}$. This function has many simpler components, like 625
and $\ds x^2$, and then there is that square root symbol, so the
square root function $\ds \sqrt{x}=x^{1/2}$ is involved. The obvious
question is: can we compute the derivative using the derivatives of
the constituents $\ds 625-x^2$ and $\ds \sqrt{x}$? We can indeed. In
general, if $f(x)$ and $g(x)$ are functions, we can compute the
derivatives of $f(g(x))$ and $g(f(x))$ in terms of $f'(x)$ and
$g'(x)$. 

\begin{example} Form the two possible compositions of $\ds f(x)=\sqrt{x}$ and
$\ds g(x)=625-x^2$ and compute the derivatives.  First, $\ds
f(g(x))=\sqrt{625-x^2}$, and the derivative is $\ds -x/\sqrt{625-x^2}$
as we have seen. Second, $\ds g(f(x))=625-(\sqrt{x})^2=625-x$ with
derivative $-1$. Of course, these calculations do not use anything
new, and in particular the derivative of $f(g(x))$ was somewhat
tedious to compute from the definition.
\end{example}

Suppose we want the derivative of $f(g(x))$.
Again, let's set up the derivative and play some algebraic tricks:
\begin{align*}
{d\over dx}f(g(x))
&=\lim_{\Delta x\to0} {f(g(x+\Delta x))-f(g(x))\over\Delta x} \\
&=\lim_{\Delta x\to0} {f(g(x+\Delta x))-f(g(x))\over g(x+\Delta
  x))-g(x)} {g(x+\Delta x))-g(x)\over\Delta x} \\
\end{align*}
Now we see immediately that the second fraction turns into $g'(x)$
when we take the limit. The first fraction is more complicated, but it
too looks something like a derivative. The denominator, $g(x+\Delta
x))-g(x)$, is a change in the value of $g$, so let's abbreviate
it as $\Delta g=g(x+\Delta
x))-g(x)$, which also means $g(x+\Delta x)=g(x)+\Delta g$. This gives
us
$$\lim_{\Delta x\to0} {f(g(x)+\Delta g)-f(g(x))\over \Delta g}.$$
As $\Delta x$ goes to 0, it is also true that $\Delta g$ goes to 0,
because $g(x+\Delta x)$ goes to $g(x)$. So we can rewrite this limit
as
$$\lim_{\Delta g\to0} {f(g(x)+\Delta g)-f(g(x))\over \Delta g}.$$
Now this looks exactly like a derivative, namely $f'(g(x))$, that is,
the function $f'(x)$ with $x$ replaced by $g(x)$. If this all
withstands scrutiny, we then get
$${d\over dx}f(g(x))=f'(g(x))g'(x).$$
Unfortunately, there is a small flaw in the argument. Recall that what
we mean by $\lim_{\Delta x\to0}$ involves what happens when $\Delta x$
is close to 0 {\it but not equal to 0.} The qualification is very
important, since we must be able to divide by $\Delta x$.
But when $\Delta x$ is close to 0 but not equal to 0,
$\Delta g=g(x+\Delta
x))-g(x)$ is close to 0 {\it and possibly equal to 0.} This means it
doesn't really make sense to divide by $\Delta g$.
Fortunately, it is possible to recast the argument to avoid this
difficulty, but it is a bit tricky; we will not include the details,
which can be found in many calculus books. Note that many functions
$g$ do have the property that $g(x+\Delta x)-g(x)\not=0$ when $\Delta
x$ is small, and for these functions the argument above is fine.

The chain rule has a particularly simple expression if we use the
Leibniz notation for the derivative. The quantity $f'(g(x))$ is the
derivative of $f$ with $x$ replaced by $g$; this can be written 
$df/dg$. As usual, $g'(x)=dg/dx$. Then the chain rule becomes
$${df\over dx} = {df\over dg}{dg\over dx}.$$
This looks like trivial arithmetic, but it is not: $dg/dx$ is not a
fraction, that is, not literal division, but a single symbol that
means $g'(x)$. Nevertheless, it turns out that what looks like trivial
arithmetic, and is therefore easy to remember, is really true.

It will take a bit of practice to make the use of the chain rule come
naturally---it is more complicated than the earlier differentiation
rules we have seen.

\begin{example}
Compute the derivative of $\ds \sqrt{625-x^2}$. We already know that the
answer is $\ds -x/\sqrt{625-x^2}$, computed directly from the limit. In
the context of the chain rule, we have $\ds f(x)=\sqrt{x}$,
$\ds g(x)=625-x^2$. We know that $\ds f'(x)=(1/2)x^{-1/2}$, so $\ds f'(g(x))=
(1/2)(625-x^2)^{-1/2}$. Note that this is a two step computation:
first compute $f'(x)$, then replace $x$ by $g(x)$. Since $g'(x)=-2x$
we have
$$f'(g(x))g'(x)={1\over 2\sqrt{625-x^2}}(-2x)={-x\over
    \sqrt{625-x^2}}.
$$
\vskip-10pt
\end{example}

\begin{example}
Compute the derivative of $\ds 1/\sqrt{625-x^2}$. This is a quotient with
a constant numerator, so we could use the quotient rule, but it is
simpler to use the chain rule. The function is $\ds (625-x^2)^{-1/2}$, the
composition of $\ds f(x)=x^{-1/2}$ and $\ds g(x)=625-x^2$. We compute
$\ds f'(x)=(-1/2)x^{-3/2}$ using the power rule, and then
$$f'(g(x))g'(x)={-1\over 2(625-x^2)^{3/2}}(-2x)={x\over (625-x^2)^{3/2}}.
$$
\vskip-10pt
\end{example}

In practice, of course, you will need to use more than one of the
rules we have developed to compute the derivative of a complicated
function.

\begin{example}
Compute the derivative of $$f(x)={x^2-1\over x\sqrt{x^2+1}}.$$
The ``last'' operation here is division, so to get started we need to
use the quotient rule first. This gives
\begin{align*}
f'(x)&={(x^2-1)'x\sqrt{x^2+1}-(x^2-1)(x\sqrt{x^2+1})'\over
x^2(x^2+1)} \\
&={2x^2\sqrt{x^2+1}-(x^2-1)(x\sqrt{x^2+1})'\over
x^2(x^2+1)}. \\
\end{align*}
Now we need to compute the derivative of $\ds x\sqrt{x^2+1}$. This is a
product, so we use the product rule:
$${d\over dx}x\sqrt{x^2+1}=x{d\over dx}\sqrt{x^2+1}+\sqrt{x^2+1}.$$
Finally, we use the chain rule:
$${d\over dx}\sqrt{x^2+1}={d\over dx}(x^2+1)^{1/2}=
{1\over 2}(x^2+1)^{-1/2}(2x)={x\over \sqrt{x^2+1}}.$$
And putting it all together:
\begin{align*}
f'(x)&={2x^2\sqrt{x^2+1}-(x^2-1)(x\sqrt{x^2+1})'\over
x^2(x^2+1)}. \\
&={2x^2\sqrt{x^2+1}-(x^2-1)\left(x{\ds{x\over \sqrt{x^2+1}}}
+\sqrt{x^2+1}\right)\over
x^2(x^2+1)}. \\
\end{align*}
This can be simplified of course, but we have done all the calculus,
so that only algebra is left.
\end{example}

\begin{example}
Compute the derivative of $\ds \sqrt{1+\sqrt{1+\sqrt{x}}}$. Here we have a
more complicated chain of compositions, so we use the chain rule
twice.
At the outermost ``layer'' we have the function
$\ds g(x)=1+\sqrt{1+\sqrt{x}}$ plugged into $\ds f(x)=\sqrt{x}$, so applying
the chain rule once gives 
$${d\over dx}\sqrt{1+\sqrt{1+\sqrt{x}}}=
{1\over 2}\left(1+\sqrt{1+\sqrt{x}}\right)^{-1/2}{d\over dx}
\left(1+\sqrt{1+\sqrt{x}}\right).$$
Now we need the derivative of $\ds \sqrt{1+\sqrt{x}}$. Using the chain
rule again:
$${d\over dx}\sqrt{1+\sqrt{x}}={1\over
  2}\left(1+\sqrt{x}\right)^{-1/2}{1\over 2}x^{-1/2}.$$
So the original derivative is 
\begin{align*}
{d\over dx}\sqrt{1+\sqrt{1+\sqrt{x}}}&=
{1\over 2}\left(1+\sqrt{1+\sqrt{x}}\right)^{-1/2}
{1\over
  2}\left(1+\sqrt{x}\right)^{-1/2}{1\over 2}x^{-1/2}. \\
&={1\over 8 \sqrt{x}\sqrt{1+\sqrt{x}}\sqrt{1+\sqrt{1+\sqrt{x}}}}
\end{align*}
\vskip -16pt
\end{example}

Using the chain rule, the power rule, and the product rule, it is
possible to avoid using the quotient rule entirely.

\begin{example}
Compute the derivative of $\ds f(x)={x^3\over x^2+1}$. Write 
$\ds f(x)=x^3(x^2+1)^{-1}$, then
\begin{align*}
f'(x)&=x^3{d\over dx}(x^2+1)^{-1}+3x^2(x^2+1)^{-1} \\
&=x^3(-1)(x^2+1)^{-2}(2x)+3x^2(x^2+1)^{-1} \\
&=-2x^4(x^2+1)^{-2}+3x^2(x^2+1)^{-1} \\
&={-2x^4\over (x^2+1)^{2}}+{3x^2\over x^2+1} \\
&={-2x^4\over (x^2+1)^{2}}+{3x^2(x^2+1)\over (x^2+1)^{2}} \\
&={-2x^4+3x^4+3x^2\over (x^2+1)^{2}}={x^4+3x^2\over (x^2+1)^{2}} \\
\end{align*}
Note that we already had the derivative on the second line; all the
rest is simplification. It is easier to get to this answer by using
the quotient rule, so there's a trade off: more work for fewer
memorized formulas.
\end{example}

% Hack
% \vfill\eject

\begin{exercises}

Find the derivatives of the functions. For extra practice, and to
check your answers, do some of these in more than one way if
possible. 

\twocol

\begin{exercise} $\ds x^4-3x^3+(1/2)x^2+7x-\pi$
\begin{answer} $\ds 4x^3-9x^2+x+7$
\end{answer}\end{exercise}

\begin{exercise} $\ds x^3-2x^2+4\sqrt{x}$
\begin{answer} $\ds 3x^2-4x+2/\sqrt{x}$
\end{answer}\end{exercise}

\begin{exercise} $\ds (x^2+1)^3$
\begin{answer} $\ds 6(x^2+1)^2x$
\end{answer}\end{exercise}

\begin{exercise} $\ds x\sqrt{169-x^2}$
\begin{answer} $\ds \sqrt{169-x^2}-x^2/\sqrt{169-x^2}$
\end{answer}\end{exercise}

\begin{exercise} $\ds (x^2-4x+5)\sqrt{25-x^2}$
\begin{answer} $\ds  (2x-4)\sqrt{25-x^2}-$\hfill\break$(x^2-4x+5)x/\sqrt{25-x^2}$
\end{answer}\end{exercise}

\begin{exercise} $\ds \sqrt{r^2-x^2}$, $r$ is a constant
\begin{answer} $\ds -x/\sqrt{r^2-x^2}$
\end{answer}\end{exercise}

\begin{exercise} $\ds \sqrt{1+x^4}$
\begin{answer} $\ds 2x^3/\sqrt{1+x^4}$
\end{answer}\end{exercise}

\begin{exercise} $\ds \ds{1\over\sqrt{5-\sqrt{x}}}$.
\begin{answer} $\ds{1\over 4\sqrt{x}(5-\sqrt{x})^{3/2}}$
\end{answer}\end{exercise}

\begin{exercise} $\ds (1+3x)^2$
\begin{answer} $\ds  6+18x$
\end{answer}\end{exercise}

\begin{exercise} $\ds{(x^2+x+1)\over(1-x)}$
\begin{answer} $\ds {2 x + 1\over1 - x }+{x^2  + x + 1\over(1 - x)^2}$
\end{answer}\end{exercise}

\begin{exercise} $\ds{\sqrt{25-x^2}\over x}$
\begin{answer} $\ds  -1/\sqrt{25-x^2}-\sqrt{25-x^2}/x^2$
\end{answer}\end{exercise}

\begin{exercise} $\ds\sqrt{{169\over x}-x}$
\begin{answer} $\ds{1\over2}\left({-169\over x^2}-1\right)\Big/\sqrt{{169\over x}-x}$
\end{answer}\end{exercise}

\begin{exercise} $\ds \sqrt{x^3-x^2-(1/x)}$
\begin{answer} $ \ds{3x^2-2x+1/x^2\over 2\sqrt{x^3-x^2-(1/x)}}$
\end{answer}\end{exercise}

\begin{exercise} $\ds 100/(100-x^2)^{3/2}$
\begin{answer} $ \ds{300 x \over(100-x^2)^{5/2}}$
\end{answer}\end{exercise}

\begin{exercise} $\ds {\root 3 \of{x+x^3}}$
\begin{answer} $ \ds{ 1 + 3 x^2\over3(x+x^3)^{2/3}}$
\end{answer}\end{exercise}

\begin{exercise} $\ds \sqrt{(x^2+1)^2+\sqrt{1+(x^2+1)^2}}$
\begin{answer} $ \ds \left(4x(x^2+1)+{4x^3+4x\over2\sqrt{1+(x^2+1)^2}}\right)\Big/$
\hfill\break$2\sqrt{(x^2+1)^2+\sqrt{1+(x^2+1)^2}}$
\end{answer}\end{exercise}

\begin{exercise} $\ds (x+8)^5$
\begin{answer} $\ds 5(x+8)^4$
\end{answer}\end{exercise}

\begin{exercise} $\ds (4-x)^3$
\begin{answer} $\ds -3(4-x)^2$
\end{answer}\end{exercise}

\begin{exercise} $\ds (x^2+5)^3$
\begin{answer} $\ds 6x(x^2+5)^2$
\end{answer}\end{exercise}

\begin{exercise} $\ds (6-2x^2)^3$
\begin{answer} $\ds -12x(6-2x^2)^2$
\end{answer}\end{exercise}

\begin{exercise} $\ds (1-4x^3)^{-2}$
\begin{answer} $\ds 24x^2(1-4x^3)^{-3}$
\end{answer}\end{exercise}

\begin{exercise} $\ds 5(x+1-1/x)$
\begin{answer} $\ds 5+5/x^2$
\end{answer}\end{exercise}

\begin{exercise} $\ds 4(2x^2-x+3)^{-2}$
\begin{answer} $\ds -8(4x-1)(2x^2-x+3)^{-3}$
\end{answer}\end{exercise}

\begin{exercise} $\ds {1\over 1+1/x}$
\begin{answer} $\ds 1/(x+1)^2$
\end{answer}\end{exercise}

\begin{exercise} $\ds {-3\over 4x^2-2x+1}$
\begin{answer} $\ds 3(8x-2)/(4x^2-2x+1)^2$
\end{answer}\end{exercise}

\begin{exercise} $\ds (x^2+1)(5-2x)/2$
\begin{answer} $\ds -3x^2+5x-1$
\end{answer}\end{exercise}

\begin{exercise} $\ds (3x^2+1)(2x-4)^3$
%\begin{answer} $120x^4-576x^3+888x^2-480x+96$
\begin{answer} $\ds 6x(2x-4)^3+6(3x^2+1)(2x-4)^2$
\end{answer}\end{exercise}

\begin{exercise} $\ds{x+1\over x-1}$
\begin{answer} $\ds -2/(x-1)^2$
\end{answer}\end{exercise}

\begin{exercise} $\ds{x^2-1\over x^2+1}$
\begin{answer} $\ds 4x/(x^2+1)^2$
\end{answer}\end{exercise}

\begin{exercise} $\ds{(x-1)(x-2)\over x-3}$
\begin{answer} $\ds (x^2-6x+7)/(x-3)^2$
\end{answer}\end{exercise}

\begin{exercise} $\ds{2x^{-1}-x^{-2}\over 3x^{-1}-4x^{-2}}$
\begin{answer} $\ds -5/(3x-4)^2$
\end{answer}\end{exercise}

\begin{exercise} $\ds 3(x^2+1)(2x^2-1)(2x+3)$
\begin{answer} $\ds 60x^4+72x^3+18x^2+18x-6$
\end{answer}\end{exercise}

\begin{exercise} $\ds{1\over (2x+1)(x-3)}$
\begin{answer} $\ds (5-4x)/((2x+1)^2(x-3)^2)$
\end{answer}\end{exercise}

\begin{exercise} $\ds ((2x+1)^{-1}+3)^{-1}$
\begin{answer} $\ds 1/(2(2+3x)^2)$
\end{answer}\end{exercise}

\begin{exercise} $\ds (2x+1)^3(x^2+1)^2$
\begin{answer} $\ds 56x^6+72x^5+110x^4+100x^3+60x^2+28x+6$
\end{answer}\end{exercise}

\endtwocol
\bsk

\begin{exercise}  Find an equation for the tangent line to 
$\ds f(x) = (x-2)^{1/3}/(x^3 + 4x - 1)^2$ at $x=1$.
\begin{answer} $y=23x/96-29/96$
\end{answer}\end{exercise}

\begin{exercise} Find an equation for the tangent line to $\ds y=9x^{-2}$ at $(3,1)$.
\begin{answer} $y=3-2x/3$
\end{answer}\end{exercise}

\begin{exercise} Find an equation for the tangent line to $\ds (x^2-4x+5)\sqrt{25-x^2}$ 
at $(3,8)$.
\begin{answer} $y=13x/2-23/2$
\end{answer}\end{exercise}

\begin{exercise} Find an equation for the tangent line to $\ds \ds{(x^2+x+1)\over(1-x)}$ 
at $(2,-7)$.
\begin{answer} $y=2x-11$
\end{answer}\end{exercise}

\begin{exercise} Find an equation for the tangent line to 
$\ds \sqrt{(x^2+1)^2+\sqrt{1+(x^2+1)^2}}$
at $\ds (1,\sqrt{4+\sqrt{5}})$.
\begin{answer} $\ds y=
{20+2\sqrt5\over5\sqrt{4+\sqrt5}}\,x+{3\sqrt5\over5\sqrt{4+\sqrt5}}$
\end{answer}\end{exercise}

\end{exercises}
