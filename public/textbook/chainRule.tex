\chapter{The Chain Rule}




So far we have seen how to compute the derivative of a function built
up from other functions by addition, subtraction, multiplication and
division. There is another very important way that we combine
functions: composition. The \textit{chain rule} allows us to deal with
this case.

\section{The Chain Rule}


Consider
\[
h(x) = (1+2x)^5.
\] 

While there are several different ways to differentiate this function,
if we let $f(x) = x^5$ and $g(x) = 1+2x$, then we can express $h(x) =
f(g(x))$. The question is, can we compute the derivative of a
composition of functions using the derivatives of the constituents
$f(x)$ and $g(x)$? To do so, we need the \textit{chain rule}.

\begin{marginfigure}
\begin{tikzpicture}
	\begin{axis}[
            axis lines=none,
            clip=false,
          ]          
          \addplot [->,textColor] plot coordinates {(0,0) (-2,-4)}; %% x axis
          \addplot [->,textColor] plot coordinates {(0,0) (0,6)}; %% y axis
          \addplot [->,textColor] plot coordinates {(0,0) (6,0)}; %% g(x) axis

          \addplot [dashed, textColor] plot coordinates {(-.7,-1.4) (1.4,-1.4)};
          \addplot [dashed, textColor] plot coordinates {(1.4,-1.4) (2.1,0)};
          \addplot [dashed, textColor] plot coordinates {(2.1,0) (2.1,4.1)};
          
          \addplot [dashed, textColor] plot coordinates {(2.6,-2.6) (3.5,0)};
          \addplot [dashed, textColor] plot coordinates {(3.5,0) (3.5,4.1)};

          \addplot [dashed, very thick, textColor] plot coordinates {(1.4,-1.4) (.8,-2.6)};
          \addplot [dashed, very thick, textColor] plot coordinates {(2.1,4.1) (3.5,4.1)};

          \addplot [very thick, penColor5] plot coordinates {(.8,-2.6) (2.6,-2.6)};
          \addplot [very thick, penColor4] plot coordinates {(3.5,4.1) (3.5,5.5)};

          \addplot [very thick,penColor,domain=(0:4)] {2+x};
          \addplot [very thick,penColor2,domain=(0:4)] {-x};

          \node at (axis cs:3.5,4.8) [anchor=west,penColor4] {$f'(g(a)){\color{penColor5}g'(a)h}$};
          \node at (axis cs:1.7,-2.6) [anchor=north,penColor5] {$g'(a)h$};
          
          \addplot[color=penColor2,fill=penColor2,only marks,mark=*] coordinates{(1.4,-1.4)};  %% closed hole          
          \addplot[color=penColor,fill=penColor,only marks,mark=*] coordinates{(2.1,4.1)};  %% closed hole          

          \node at (axis cs:1,-2.1) [anchor=south,yslant=0,xslant=0,rotate=53] {$\overbrace{\hspace{.36in}}^{h}$};
          \node at (axis cs:7,0) [anchor=east] {$g(x)$};
          \node at (axis cs:0,6.7) [anchor=north] {$y$};
          \node at (axis cs:-2.15,-4) [anchor=north] {$x$};
          \node at (axis cs:-.7,-1.4) [anchor=east] {$a$};
        \end{axis}
\end{tikzpicture}
\caption{A geometric interpretation of the chain rule. Increasing $a$
  by a ``small amount'' $h$, increases $f(g(a))$ by $f'(g(a))g'(a)h$. Hence, 
\[
\frac{\Delta y}{\Delta x}\approx \frac{f'(g(a))g'(a)h}{h} =
f'(g(a))g'(a).
\]}
\end{marginfigure}
\begin{mainTheorem}[Chain Rule]\index{chain rule}\index{derivative rules!chain}
If $f(x)$ and $g(x)$ are differentiable, then
\[
\ddx f(g(x)) = f'(g(x))g'(x).
\]
\end{mainTheorem}

\begin{proof}
Let $g_0$ be some $x$-value and consider the following:
\[
f'(g_0) = \lim_{h\to 0}\frac{f(g_0+h)-f(g_0)}{h}.
\]

Set $h = g-g_0$ and we have
\begin{equation}\label{E:chainRuleLim}
f'(g_0) = \lim_{g\to g_0} \frac{f(g)-f(g_0)}{g-g_0}.
\end{equation}
At this point, we might like to set $g=g(x+h)$ and $g_0=g(x)$;
however, we cannot as we cannot be sure that
\[
g(x+h) - g(x) \ne 0\qquad\text{when $h\ne 0$.}
\]
To overcome this difficulty, let $E(g)$ be the ``error term'' that
gives the difference between the slope of the secant line from
$f(g_0)$ to $f(g)$ and $f'(g_0)$,
\[
E(g) = \frac{f(g)-f(g_0)}{g-g_0} - f'(g_0).
\]
In particular, $E(g)(g-g_0)$ is the difference between $f(g)$ and the
tangent line of $f(x)$ at $x=g$, see the figure below:

\vspace{1cm}
\begin{tikzpicture}
	\begin{axis}[
            clip=false,
            domain=0:2, range=0:6,ymax=4,ymin=0,
            axis lines =left, xlabel=$x$, ylabel=$y$,
            every axis y label/.style={at=(current axis.above origin),anchor=south},
            every axis x label/.style={at=(current axis.right of origin),anchor=west},
            xtick={1,1.666}, ytick={1,3},
            xticklabels={$g_0$,$g$}, yticklabels={$f(g_0)$,$f(g)$},
            axis on top,
          ]         
	  \addplot [textColor,dashed] plot coordinates {(1,0) (1,1)};
          \addplot [textColor,dashed] plot coordinates {(0,1) (1,1)};
          \addplot [textColor,dashed] plot coordinates {(0,3) (1.666,3)};
          \addplot [textColor,dashed] plot coordinates {(1.666,0) (1.666,1)};

          \addplot [textColor,dashed,very thick] plot coordinates {(1,1) (1.666,1)};
          \node at (axis cs:1.333,1) [anchor=north] {$\underbrace{\hspace{.75in}}_{g-g_0}$};

          \addplot [penColor4,very thick] plot coordinates {(1.666,1) (1.666,1.666)};
          \addplot [penColor5,very thick] plot coordinates {(1.666,1.666) (1.666,3)};

          \node at (axis cs:1.666,1.333) [anchor=west] {$f'(g_0)(g-g_0)$};
          \node at (axis cs:1.666,2.333) [anchor=west] {$E(g)(g-g_0)$};

          \addplot [very thick,penColor, smooth,domain=(0:7/4)] {-1/(x-2)};
          \addplot [very thick,penColor2, smooth,domain=(0:2)] {x};

          \addplot[color=penColor,fill=penColor,only marks,mark=*] coordinates{(1.666,3)};  %% closed hole          
          \addplot[color=penColor,fill=penColor,only marks,mark=*] coordinates{(1,1)};  %% closed hole          
        \end{axis}
\end{tikzpicture}
\vspace{2cm}

Hence we see that
\begin{equation}\label{E:chainRuleDer}
f(g)-f(g_0) = \left(f'(g_0) + E(g)\right)(g-g_0),
\end{equation}
and so
\[
\frac{f(g)-f(g_0)}{g-g_0} = f'(g_0) + E(g).
\]


Combining this with Equation~\ref{E:chainRuleLim}, we have that
\begin{align*}
f'(g_0) &= \lim_{g\to g_0}\frac{f(g)-f(g_0)}{g-g_0} \\
&= \lim_{g\to g_0}f'(g_0) + E(g)\\
&= f'(g_0) + \lim_{g\to g_0} E(g),
\end{align*}
and hence it follows that $\lim_{g\to g_0} E(g) = 0$. At this point,
we may return to the ``well-worn path.'' Starting with
Equation~\ref{E:chainRuleDer}, divide both sides by $h$ and set
$g=g(x+h)$ and $g_0=g(x)$
\[
\frac{f(g(x+h))-f(g(x))}{h} = \left(f'(g(x)) +
E(g(x))\right)\frac{g(x+h)-g(x)}{h}.
\]
Taking the limit as $h$ approaches $0$, we see 
\begin{align*}
\lim_{h\to 0}\frac{f(g(x+h))-f(g(x))}{h} &= \lim_{h\to 0}\left(f'(g(x))
+ E(g(x))\right)\frac{g(x+h)-g(x)}{h}\\
&= \lim_{h\to 0}\left(f'(g(x))
+ E(g(x))\right)\lim_{h\to 0}\frac{g(x+h)-g(x)}{h}\\
&= f'(g(x))g'(x).
\end{align*}
Hence, $\ddx f(g(x))= f'(g(x))g'(x)$.
\end{proof}


It will take a bit of practice to make the use of the chain rule come
naturally---it is more complicated than the earlier differentiation
rules we have seen. Let's return to our motivating example.

\begin{example}
Compute:
\[
\ddx (1+2x)^5
\]
\end{example}

\begin{solution}
Set $f(x) = x^5$ and $g(x) = 1+2x$, now
\[
f'(x) = 5x^4 \qquad\text{and}\qquad g'(x) = 2.
\]
Hence
\begin{align*}
\ddx (1+2x)^5 &= \ddx f(g(x))\\ 
&=f'(g(x))g'(x) \\
&= 5(1+2x)^4\cdot 2\\
&= 10(1+2x)^4.
\end{align*}
\end{solution}


Let's see a more complicated chain of compositions.

\begin{example}
Compute:
\[
\ddx \sqrt{1+\sqrt{x}}
\]
\end{example}

\begin{solution}
Set 
$f(x)=\sqrt{x}$ and $g(x)=1+x$. Hence,
\[
\sqrt{1+\sqrt{x}}=f(g(f(x))) \qquad\text{and}\qquad\ddx f(g(f(x))) = f'(g(f(x)))g'(f(x))f'(x).
\]
Since 
\[
f'(x) = \frac{1}{2\sqrt{x}} \qquad\text{and}\qquad g'(x) = 1
\]
We have that
\[
\ddx \sqrt{1+\sqrt{x}} = \frac{1}{2\sqrt{1+\sqrt{x}}}\cdot 1\cdot  \frac{1}{2\sqrt{x}}.
\]
\end{solution}

Using the chain rule, the power rule, and the product rule it is
possible to avoid using the quotient rule entirely.

\begin{example}
Compute:
\[
\ddx \frac{x^3}{x^2+1}
\]
\end{example}
\begin{solution}
Rewriting this as 
\[
\ddx x^3(x^2+1)^{-1}, 
\]
set $f(x) = x^{-1}$ and $g(x) = x^2+1$. Now
\[
x^3(x^2+1)^{-1} = x^3 f(g(x)) \qquad\text{and}\qquad \ddx x^3 f(g(x)) = 3x^2f(g(x))+ x^3 f'(g(x))g'(x).
\]
Since $f'(x) = \frac{-1}{x^2}$ and $g'(x) = 2x$, write
\[
\ddx \frac{x^3}{x^2+1} = \frac{3x^2}{x^2+1}-\frac{2x^4}{(x^2+1)^2}.
\]
\end{solution}


\begin{exercises}

\noindent Compute the derivatives of the functions. For extra practice, and to
check your answers, do some of these in more than one way if possible.

\twocol

\begin{exercise} $x^4-3x^3+(1/2)x^2+7x-\pi$
\begin{answer} $4x^3-9x^2+x+7$
\end{answer}\end{exercise}

\begin{exercise} $x^3-2x^2+4\sqrt{x}$
\begin{answer} $3x^2-4x+2/\sqrt{x}$
\end{answer}\end{exercise}

\begin{exercise} $(x^2+1)^3$
\begin{answer} $6(x^2+1)^2x$
\end{answer}\end{exercise}

\begin{exercise} $x\sqrt{169-x^2}$
\begin{answer} $\sqrt{169-x^2}-x^2/\sqrt{169-x^2}$
\end{answer}\end{exercise}

\begin{exercise} $(x^2-4x+5)\sqrt{25-x^2}$
\begin{answer} $ (2x-4)\sqrt{25-x^2}-$\hfill\break$(x^2-4x+5)x/\sqrt{25-x^2}$
\end{answer}\end{exercise}

\begin{exercise} $\sqrt{r^2-x^2}$, $r$ is a constant
\begin{answer} $-x/\sqrt{r^2-x^2}$
\end{answer}\end{exercise}

\begin{exercise} $\sqrt{1+x^4}$
\begin{answer} $2x^3/\sqrt{1+x^4}$
\end{answer}\end{exercise}

\begin{exercise} ${1\over\sqrt{5-\sqrt{x}}}$
\begin{answer} ${1\over 4\sqrt{x}(5-\sqrt{x})^{3/2}}$
\end{answer}\end{exercise}

\begin{exercise} $(1+3x)^2$
\begin{answer} $ 6+18x$
\end{answer}\end{exercise}

\begin{exercise} ${(x^2+x+1)\over(1-x)}$
\begin{answer} ${2 x + 1\over1 - x }+{x^2  + x + 1\over(1 - x)^2}$
\end{answer}\end{exercise}

\begin{exercise} ${\sqrt{25-x^2}\over x}$
\begin{answer} $ -1/\sqrt{25-x^2}-\sqrt{25-x^2}/x^2$
\end{answer}\end{exercise}

\begin{exercise} $\sqrt{{169\over x}-x}$
\begin{answer} ${1\over2}\left({-169\over x^2}-1\right)\Big/\sqrt{{169\over x}-x}$
\end{answer}\end{exercise}

\begin{exercise} $\sqrt{x^3-x^2-(1/x)}$
\begin{answer} $ {3x^2-2x+1/x^2\over 2\sqrt{x^3-x^2-(1/x)}}$
\end{answer}\end{exercise}

\begin{exercise} $100/(100-x^2)^{3/2}$
\begin{answer} $ {300 x \over(100-x^2)^{5/2}}$
\end{answer}\end{exercise}

\begin{exercise} ${\root 3 \of{x+x^3}}$
\begin{answer} $ { 1 + 3 x^2\over3(x+x^3)^{2/3}}$
\end{answer}\end{exercise}

\begin{exercise} $\sqrt{(x^2+1)^2+\sqrt{1+(x^2+1)^2}}$
\begin{answer} $ \left(4x(x^2+1)+{4x^3+4x\over2\sqrt{1+(x^2+1)^2}}\right)\Big/$
\hfill\break$2\sqrt{(x^2+1)^2+\sqrt{1+(x^2+1)^2}}$
\end{answer}\end{exercise}

\begin{exercise} $(x+8)^5$
\begin{answer} $5(x+8)^4$
\end{answer}\end{exercise}

\begin{exercise} $(4-x)^3$
\begin{answer} $-3(4-x)^2$
\end{answer}\end{exercise}

\begin{exercise} $(x^2+5)^3$
\begin{answer} $6x(x^2+5)^2$
\end{answer}\end{exercise}

\begin{exercise} $(6-2x^2)^3$
\begin{answer} $-12x(6-2x^2)^2$
\end{answer}\end{exercise}

\begin{exercise} $(1-4x^3)^{-2}$
\begin{answer} $24x^2(1-4x^3)^{-3}$
\end{answer}\end{exercise}

\begin{exercise} $5(x+1-1/x)$
\begin{answer} $5+5/x^2$
\end{answer}\end{exercise}

\begin{exercise} $4(2x^2-x+3)^{-2}$
\begin{answer} $-8(4x-1)(2x^2-x+3)^{-3}$
\end{answer}\end{exercise}

\begin{exercise} ${1\over 1+1/x}$
\begin{answer} $1/(x+1)^2$
\end{answer}\end{exercise}

\begin{exercise} ${-3\over 4x^2-2x+1}$
\begin{answer} $3(8x-2)/(4x^2-2x+1)^2$
\end{answer}\end{exercise}

\begin{exercise} $(x^2+1)(5-2x)/2$
\begin{answer} $-3x^2+5x-1$
\end{answer}\end{exercise}

\begin{exercise} $(3x^2+1)(2x-4)^3$
%\begin{answer} $120x^4-576x^3+888x^2-480x+96$
\begin{answer} $6x(2x-4)^3+6(3x^2+1)(2x-4)^2$
\end{answer}\end{exercise}

\begin{exercise} ${x+1\over x-1}$
\begin{answer} $-2/(x-1)^2$
\end{answer}\end{exercise}

\begin{exercise} ${x^2-1\over x^2+1}$
\begin{answer} $4x/(x^2+1)^2$
\end{answer}\end{exercise}

\begin{exercise} ${(x-1)(x-2)\over x-3}$
\begin{answer} $(x^2-6x+7)/(x-3)^2$
\end{answer}\end{exercise}

\begin{exercise} ${2x^{-1}-x^{-2}\over 3x^{-1}-4x^{-2}}$
\begin{answer} $-5/(3x-4)^2$
\end{answer}\end{exercise}

\begin{exercise} $3(x^2+1)(2x^2-1)(2x+3)$
\begin{answer} $60x^4+72x^3+18x^2+18x-6$
\end{answer}\end{exercise}

\begin{exercise} ${1\over (2x+1)(x-3)}$
\begin{answer} $(5-4x)/((2x+1)^2(x-3)^2)$
\end{answer}\end{exercise}

\begin{exercise} $((2x+1)^{-1}+3)^{-1}$
\begin{answer} $1/(2(2+3x)^2)$
\end{answer}\end{exercise}

\begin{exercise} $(2x+1)^3(x^2+1)^2$
\begin{answer} $56x^6+72x^5+110x^4+100x^3+60x^2+28x+6$
\end{answer}\end{exercise}

\endtwocol

\begin{exercise}  Find an equation for the tangent line to 
$f(x) = (x-2)^{1/3}/(x^3 + 4x - 1)^2$ at $x=1$.
\begin{answer} $y=23x/96-29/96$
\end{answer}\end{exercise}

\begin{exercise} Find an equation for the tangent line to $y=9x^{-2}$ at $(3,1)$.
\begin{answer} $y=3-2x/3$
\end{answer}\end{exercise}

\begin{exercise} Find an equation for the tangent line to $(x^2-4x+5)\sqrt{25-x^2}$ 
at $(3,8)$.
\begin{answer} $y=13x/2-23/2$
\end{answer}\end{exercise}

\begin{exercise} Find an equation for the tangent line to ${(x^2+x+1)\over(1-x)}$ 
at $(2,-7)$.
\begin{answer} $y=2x-11$
\end{answer}\end{exercise}

\begin{exercise} Find an equation for the tangent line to 
$\sqrt{(x^2+1)^2+\sqrt{1+(x^2+1)^2}}$
at $(1,\sqrt{4+\sqrt{5}})$.
\begin{answer} $y=
{20+2\sqrt5\over5\sqrt{4+\sqrt5}}\,x+{3\sqrt5\over5\sqrt{4+\sqrt5}}$
\end{answer}\end{exercise}

\end{exercises}





















\section{Implicit Differentiation}

The functions we've been dealing with so far have been
\textit{explicit functions}\index{explicit function}, meaning that the
dependent variable is written in terms of the independent variable. For
example:
\[
y=3x^2-2x+1,\qquad y=e^{3x}, \qquad y = \frac{x-2}{x^2-3x+2}.
\]
However, there are another type of functions, called \textit{implicit
  functions}. In this case, the dependent variable is not stated
explicitly in terms of the independent variable. For example:
\[
x^2+y^2 = 4,\qquad x^3+y^3 = 9xy, \qquad x^4+3x^2 = x^{2/3}+y^{2/3} = 1.
\]
Your inclination might be simply to solve each of these for $y$ and go
merrily on your way. However this can be difficult and it may require
two \textit{branches}, for example to explicitly plot $x^2+y^2 = 4$,
one needs both $y= \sqrt{4-x^2}$ and $y=-\sqrt{4-x^2}$. Moreover, it
may not even be possible to solve for $y$. To deal with such
situations, we use \index{implicit differentiation}\textit{implicit
  differentiation}. Let's see an illustrative example:


\begin{example}
Consider the curve defined by
\[
x^3+y^3 = 9xy.
\]
\begin{enumerate}
\item Compute $\dydx$.
\item Find the slope of the tangent line at $(4,2)$.
\end{enumerate}
\end{example}


\begin{marginfigure}
\begin{tikzpicture}
	\begin{axis}[
            xmin=-6,xmax=6,ymin=-6,ymax=6,
            axis lines=center,
            xlabel=$x$, ylabel=$y$,
            every axis y label/.style={at=(current axis.above origin),anchor=south},
            every axis x label/.style={at=(current axis.right of origin),anchor=west},
          ]        
          \addplot [very thick, penColor, smooth, samples=100, domain=(-.99:0)] ({9*x/(1+x^3)},{9*x^2/(1+x^3)});
          \addplot [very thick, penColor, smooth, samples=100, domain=(-.99:0)] ({9*x^2/(1+x^3)},{9*x/(1+x^3)});
          \addplot [very thick, penColor, smooth, samples=100, domain=(0:1)] ({9*x/(1+x^3)},{9*x^2/(1+x^3)});
          \addplot [very thick, penColor, smooth, samples=100, domain=(0:1)] ({9*x^2/(1+x^3)},{9*x/(1+x^3)});
        \end{axis}
\end{tikzpicture}
\caption{A plot of $x^3+y^3 = 9xy$. While this is not a function of
  $y$ in terms of $x$, the equation still defines a relation between
  $x$ and $y$.}
\end{marginfigure}

\begin{solution}
Starting with 
\[
x^3+y^3 = 9xy,
\]
we apply the differential operator $\ddx$ to both sides of the
equation to obtain
\[
\ddx \left(x^3+y^3\right) = \ddx 9xy.
\]
Applying the sum rule we see
\[
\ddx x^3+\ddx y^3 = \ddx 9xy.
\]
Let's examine each of these terms in turn. To start
\[
\ddx x^3 = 3x^2.
\]
On the other hand $\ddx y^3$ is somewhat different. Here you imagine that $y = y(x)$, and hence by the chain rule
\begin{align*}
\ddx y^3 &= \ddx (y(x))^3 \\ 
&= 3(y(x))^2 \cdot y'(x) \\
&= 3y^2\dydx.
\end{align*}
Considering the final term $\ddx 9xy$, we again imagine that $y=y(x)$. Hence 
\begin{align*}
\ddx 9xy &= 9\ddx x\cdot y(x) \\
&= 9 \left(x\cdot y'(x) + y(x)\right)\\
&= 9x \dydx + 9y.
\end{align*}
Putting this all together we are left with the equation
\[
3x^2 + 3y^2\dydx =9x \dydx + 9y.
\]
At this point, we solve for $\dydx$. Write
\begin{align*}
3x^2 + 3y^2\dydx &= 9x \dydx + 9y\\
3y^2\dydx -  9x \dydx &= 9y - 3x^2\\
\dydx\left(3y^2-9x\right)&= 9y - 3x^2\\
\dydx &=\frac{9y - 3x^2}{3y^2-9x} = \frac{3y - x^2}{y^2-3x}.
\end{align*}

For the second part of the problem, we simply plug $x=4$ and $y=2$
into the formula above, hence the slope of the tangent line at $(4,2)$
is $\frac{5}{4}$, see Figure~\ref{plot:x^3+y^3=9xy}.
\end{solution}
%\enlargethispage{1in}
\begin{marginfigure}[0in]
\begin{tikzpicture}
	\begin{axis}[
            xmin=-6,xmax=6,ymin=-6,ymax=6,
            axis lines=center,
            xlabel=$x$, ylabel=$y$,
            every axis y label/.style={at=(current axis.above origin),anchor=south},
            every axis x label/.style={at=(current axis.right of origin),anchor=west},
          ]        
          \addplot [very thick, penColor, smooth, samples=100, domain=(-.99:0)] ({9*x/(1+x^3)},{9*x^2/(1+x^3)});
          \addplot [very thick, penColor, smooth, samples=100, domain=(-.99:0)] ({9*x^2/(1+x^3)},{9*x/(1+x^3)});
          \addplot [very thick, penColor, smooth, samples=100, domain=(0:1)] ({9*x/(1+x^3)},{9*x^2/(1+x^3)});
          \addplot [very thick, penColor, smooth, samples=100, domain=(0:1)] ({9*x^2/(1+x^3)},{9*x/(1+x^3)});

          \addplot [very thick, penColor2, domain=(-6:6)] {(5/4)*(x-4)+2};

          \addplot[color=penColor,fill=penColor,only marks,mark=*] coordinates{(4,2)};  %% closed hole          
        \end{axis}
\end{tikzpicture}
\caption{A plot of $x^3+y^3 = 9xy$ along with the tangent line at
  $(4,2)$.}
\label{plot:x^3+y^3=9xy}
\end{marginfigure}

You might think that the step in which we solve for $\dydx$ could
sometimes be difficult---after all, we're using implicit
differentiation here instead of the more difficult task of solving the
equation $x^3+y^3=9xy$ for $y$, so maybe there are functions where
after taking the derivative we obtain something where it is hard to
solve for $\dydx$. In fact, \textit{this never happens}. All
occurrences $\dydx$ arise from applying the chain rule, and whenever
the chain rule is used it deposits a single $\dydx$ multiplied by some
other expression. Hence our expression is linear in $\dydx$, it will
always be possible to group the terms containing $\dydx$ together and
factor out the $\dydx$, just as in the previous example. 




\subsection*{The Derivative of Inverse Functions}


Geometrically, there is a close relationship between the plots of
$e^x$ and $\ln(x)$, they are reflections of each other over the line
$y=x$, see Figure~\ref{plot:e^x lnx}. One may suspect that we can use the fact
that $\ddx e^x = e^x$, to deduce the derivative of $\ln(x)$.  We will
use implicit differentiation to exploit this relationship
computationally.

\begin{marginfigure}
\begin{tikzpicture}
	\begin{axis}[
            xmin=-6,xmax=6,ymin=-6,ymax=6,
            axis lines=center,
            xlabel=$x$, ylabel=$y$,
            every axis y label/.style={at=(current axis.above origin),anchor=south},
            every axis x label/.style={at=(current axis.right of origin),anchor=west},
          ]        
          \addplot [very thick, penColor, smooth, domain=(-6:6)] {e^x};
          \addplot [very thick, penColor2, samples=100, smooth, domain=(.002:6)] {ln(x)};
          \addplot [dashed, textColor, domain=(-6:6)] {x};
          \node at (axis cs:-2,1) [penColor] {$e^x$};
          \node at (axis cs:1,-2) [penColor2] {$\ln(x)$};
        \end{axis}
\end{tikzpicture}
\caption{A plot of $e^x$ and $\ln(x)$. Since they are inverse
  functions, they are reflections of each other across the line $y=x$.}
\label{plot:e^x lnx}
\end{marginfigure}


\begin{mainTheorem}[The Derivative of the Natural Logrithm]\index{derivative!of the natural logarithm}
\[
\ddx \ln(x) = \frac{1}{x}.
\]
\end{mainTheorem}
\begin{proof}
Recall
\[
\ln(x) = y \qquad\Leftrightarrow\qquad e^y = x.
\]
Hence
\begin{align*}
e^y &= x\\
\ddx e^y &= \ddx x &\text{Differentiate both sides.}\\
e^y \dydx &= 1 &\text{Implicit differentiation.}\\
\dydx &= \frac{1}{e^y} = \frac{1}{x}.
\end{align*}
Since $y=\ln(x)$, $\ddx \ln(x) = \frac{1}{x}$.
\end{proof}

There is one catch to the proof given above. To write
$\ddx(e^y)=e^y\dydx$ we need to know that the function $y$
\textit{has} a derivative. All we have shown is that \textit{if} it
has a derivative then that derivative must be $1/x$. The \textit{Inverse
Function Theorem} guarantees this.

\begin{mainTheorem}[Inverse Function Theorem]\index{Inverse Function Theorem}\label{theorem:IFT}
If $f(x)$ is a differentiable function, and $f'(x)$ is continuous, and
$f'(a) \neq 0$, then
\begin{enumerate}
\item $f^{-1}(y)$ is defined for $y$ near $f(a)$,
\item $f^{-1}(y)$ is differentiable near $f(a)$, 
\item $\dd{y} f^{-1}(y)$ is continuous near $f(a)$, and
\item $\dd{y} f^{-1}(y)  = \displaystyle\frac{1}{f'(f^{-1}(y))}$.
\end{enumerate}
\end{mainTheorem}




\begin{exercises}

\noindent Compute $\dydx$:
\twocol

\begin{exercise} $x^2 + y^2 = 4$
\begin{answer} $-x/y$
\end{answer}\end{exercise}

\begin{exercise} $y^2=1+x^2$
\begin{answer} $x/y$
\end{answer}\end{exercise}

\begin{exercise} $x^2+xy+y^2=7$
\begin{answer} $-(2x+y)/(x+2y)$
\end{answer}\end{exercise}

\begin{exercise} $x^3+xy^2=y^3+yx^2$
\begin{answer} $(2xy-3x^2-y^2)/(2xy-3y^2-x^2)$
\end{answer}\end{exercise}

\begin{exercise} $x^2y-y^3 = 6$
\begin{answer} $\frac{-2xy}{x^2-3y^2}$
\end{answer}\end{exercise}

\begin{exercise} $\sqrt{x} + \sqrt{y} = 9$
\begin{answer} $-\sqrt{y}/\sqrt{x}$
\end{answer}\end{exercise}

\begin{exercise} $xy^{3/2}+4 = 2x+y$
\begin{answer} $\frac{y^{3/2}-2}{1-y^{1/2}3x/2}$
\end{answer}\end{exercise}

\begin{exercise} ${1\over x} + {1\over y} = 7$
\begin{answer} $-y^2/x^2$
\end{answer}\end{exercise}

\endtwocol

\begin{exercise}
A hyperbola passing through $(8,6)$ consists of all points whose distance
from the origin is a constant more than its distance from the point (5,2).
Find the slope of the tangent line to the hyperbola at $(8,6)$.
\begin{answer} $1$
\end{answer}\end{exercise}

\begin{exercise} The graph of the equation $x^2 - xy + y^2 = 9$ is an ellipse.
Find the lines tangent to this curve at the two
 points where it intersects the $x$-axis. Show that these lines are
 parallel.
\begin{answer} $y=2x\pm6$
\end{answer}\end{exercise}

\begin{exercise} Repeat the previous problem for the points at which the
 ellipse intersects the $y$-axis.
\begin{answer} $y=x/2\pm3$
\end{answer}\end{exercise}

\begin{exercise} Find the points on the ellipse from the previous two problems
 where the slope is horizontal and where it is vertical.
\begin{answer} $(\sqrt3,2\sqrt3)$, $(-\sqrt3,-2\sqrt3)$, $(2\sqrt3,\sqrt3)$,
$(-2\sqrt3,-\sqrt3)$ 
\end{answer}\end{exercise}

\begin{exercise} Find an equation for the tangent line to 
$x^4 = y^2 + x^2$ at $(2, \sqrt{12})$. 
This curve is the \textit{kampyle of Eudoxus}.
\begin{answer} $y=7x/\sqrt3-8/\sqrt3$
\end{answer}\end{exercise}

\begin{exercise} Find an equation for the tangent line to $x^{2/3} +
y^{2/3} = a^{2/3}$ at a point $(x_1 ,y_1)$ on the curve, 
with $x_1 \neq 0$ and $y_1 \neq 0$. This curve is an \textit{astroid}.
\begin{answer} $y=(-y_1^{1/3}x+y_1^{1/3}x_1+x_1^{1/3}y_1)/x_1^{1/3}$
\end{answer}\end{exercise}

\begin{exercise} Find an equation for the tangent line to $(x^2 +y^2 )^2 =x^2
-y^2$ at a point $(x_1 , y_1)$ on the curve, with $x_1 \neq 0, -1, 1$.
This curve is a \textit{lemniscate}.
\begin{answer} $(y-y_1)/(x-x_1)=(2x_1^3+2x_1y_1^2-x_1)/(2y_1^3+2y_1x_1^2+y_1)$
\end{answer}\end{exercise}


\end{exercises}







\section{Logarithmic Differentiation}\index{logarithmic differentiation}
\label{S:logdiff}
Logarithms were originally developed as a computational tool. The key
fact that made this possible is that:
\[
\log_b(xy) = \log_b(x)+\log_b(y).
\]
\begin{marginfigure}
\begin{tikzpicture}
	\begin{axis}[
            xmin=-1,xmax=7,ymin=-4,ymax=3.5,
            axis lines=center,
            xlabel=$x$, ylabel=$y$,
            every axis y label/.style={at=(current axis.above origin),anchor=south},
            every axis x label/.style={at=(current axis.right of origin),anchor=west},
            xtick={1,...,7},
          ]        
          \addplot [very thick, penColor, samples=100, smooth, domain=(.01:7)] {ln(x)};
          \addplot [very thick, penColor4] plot coordinates {(2,0) (2,.693)};
          \addplot [very thick, penColor5] plot coordinates {(3,0) (3,1.099)};
          \addplot [very thick, penColor4] plot coordinates {(6,0) (6,.693)};
          \addplot [very thick, penColor5] plot coordinates {(6,.693) (6,1.792)};
        \end{axis}
\end{tikzpicture}
\caption{A plot of $\ln(x)$. Here we see that \[\ln(2\cdot 3) = \ln(2) + \ln(3).\]}
\label{plot:ln additive property}
\end{marginfigure}

While this may seem quite abstract, before the days of calculators and
computers, this was critical knowledge for anyone in a computational
discipline. Suppose you wanted to compute
\[
138\cdot 23.4
\]
You would start by writing both in scientific notation
\[
\left(1.38\cdot 10^{2}\right)\cdot \left(2.34 \cdot 10^1\right).
\]
Next you would use a log-table, which gives $\log_{10}(N)$ for values
of $N$ ranging between $0$ and $9$. We've reproduced part of such a
table below.
\begin{figure*}
\[
\begin{tchart}{lllllllllll}
  N & 0 & 1 & 2 & 3 & 4 & 5 & 6 & 7 & 8 & 9 \\ \hline
  1.3 & 0.1139 & 0.1173 & 0.1206 & 0.1239 & 0.1271 & 0.1303 & 0.1335 & 0.1367 & 0.1399 & 0.1430 \\
  \hdotsfor{11} \\
  2.3 & 0.3617 & 0.3636 & 0.3655 & 0.3674 & 0.3692 & 0.3711 & 0.3729 & 0.3747 & 0.3766 & 0.3784 \\
   \hdotsfor{11} \\
  3.2 & 0.5052 & 0.5065 & 0.5079 & 0.5092 & 0.5105 & 0.5119 & 0.5132 & 0.5145 & 0.5159 & 0.5172\\
\end{tchart}
\]
\caption{Part of a base-10 logarithm table.}
\end{figure*}


From the table, we see that 
\[
\log_{10}(1.38) \approx 0.1399\qquad\text{and}\qquad \log_{10}(2.34)\approx 0.3692
\]
Add these numbers together to get $0.5091$. Essentially, we know the
following at this point:
\[
\begin{array}{ccccc}
\log_{10}(\text{?}) & = & \log_{10}(1.38) & + & \log_{10}(2.34)\\
  \rotatebox[origin=c]{90}{$\approx$}  & & \rotatebox[origin=c]{90}{$\approx$}   & & \rotatebox[origin=c]{90}{$\approx$}  \\
0.5091 & = & 0.1399 & + & 0.3692
\end{array}
\]



Using the table again, we see that $\log_{10}(3.23)\approx
0.5091$. Since we were working in scientific notation, we need to
multiply this by $10^3$. Our final answer is
\[
3230 \approx 138\cdot 23.4
\]
Since $138\cdot 23.4 = 3229.2$, this is a good approximation. The moral is:
\begin{center}
\textbf{Logarithms allow us to use addition in place of multiplication.}
\end{center}

When taking derivatives, both the product rule and the quotient rule
can be cumbersome to use. Logarithms will save the day. A key point is the following
\[
\ddx \ln(f(x)) = \frac{1}{f(x)}\cdot f'(x) = \frac{f'(x)}{f(x)}
\]
which follows from the chain rule. Let's look at an illustrative
example to see how this is actually used.

\begin{example} 
Compute:
\[
\ddx \frac{x^9e^{4x}}{\sqrt{x^2+4}}
\]
\end{example}
\marginnote{
Recall the properties of logarithms:
\begin{itemize}
\item $\log_b(xy) = \log_b(x) + \log_b(y)$
\item $\log_b(x/y) = \log_b(x) - \log_b(y)$
\item $\log_b(x^y) = y\log_b(x)$
\end{itemize}
}
\begin{solution}
While we could use the product and quotient rule to solve this
problem, it would be tedious. Start by taking the logarithm of the
function to be differentiated.
\begin{align*}
\ln\left(\frac{x^9e^{4x}}{\sqrt{x^2+4}} \right) &= \ln\left(x^9e^{4x}\right) - \ln\left(\sqrt{x^2+4}\right)\\
&= \ln\left(x^9\right)+\ln\left(e^{4x}\right) - \ln\left((x^2+4)^{1/2}\right)\\
&= 9\ln(x)+4x - \frac{1}{2}\ln(x^2+4).
\end{align*}
Setting $f(x) = \frac{x^9e^{4x}}{\sqrt{x^2+4}}$, we can write
\[
\ln(f(x)) = 9\ln(x)+4x - \frac{1}{2}\ln(x^2+4).
\]
Differentiating both sides, we find
\[
\frac{f'(x)}{f(x)} = \frac{9}{x}+4 - \frac{x}{x^2+4}.
\]
Finally we solve for $f'(x)$, write
\[
f'(x) = \left(\frac{9}{x}+4 - \frac{x}{x^2+4}\right)\left(\frac{x^9e^{4x}}{\sqrt{x^2+4}}\right).
\]
\end{solution}

The process above is called \textit{logarithmic
  differentiation}. Logarithmic differentiation allows us to compute
new derivatives too.

\begin{example}
Compute:
\[
\ddx x^x
\]
\end{example}


\begin{solution}
The function $x^x$ is tricky to differentiate. We cannot use the power
rule, as the exponent is not a constant. However, if we set $f(x) = x^x$ we can write
\begin{align*}
\ln(f(x)) &= \ln\left(x^x\right)\\
&=x\ln(x).
\end{align*}
Differentiating both sides, we find
\begin{align*}
\frac{f'(x)}{f(x)} &= x\cdot \frac{1}{x} + \ln(x)\\
&= 1 + \ln(x).
\end{align*}
Now we can solve for $f'(x)$, 
\[
f'(x) = x^x + x^x\ln(x).
\]
\end{solution}

Finally recall that previously we only proved the power rule,
Theorem~\ref{T:powerrule}, for positive exponents. Now we'll use
logarithmic differentiation to give a proof for all real-valued
exponents. We restate the power rule for convenience sake:

\begin{theorem}[Power Rule]
For any real number $n$, 
\[
\ddx x^n = n x^{n-1}.
\]
\end{theorem}

\begin{proof}
We will use logarithmic differentiation. Set $f(x) = x^n$. Write
\begin{align*}
\ln(f(x)) &= \ln\left(x^n\right) \\ 
&= n\ln(x).
\end{align*}
Now differentiate both sides, and solve for $f'(x)$
\begin{align*}
\frac{f'(x)}{f(x)} &= \frac{n}{x}\\
f'(x) &= \frac{nf(x)}{x}\\
&= nx^{n-1}.
\end{align*}
Thus we see that the power rule holds for all real-valued exponents.
\end{proof}

While logarithmic differentiation might seem strange and new at
first, with a little practice it will seem much more natural to you.


\begin{exercises}

\noindent Use logarithmic differentiation to compute the following:

\twocol

\begin{exercise}
$\ddx (x+1)^3\sqrt{x^4+5}$
\begin{answer}
$(x+1)^3\sqrt{x^4+5}(3/(x+1) + 2x^3/(x^4+5))$
\end{answer}
\end{exercise}


\begin{exercise}
$\ddx x^2 e^{5x}$
\begin{answer}
$(2/x + 5)x^2e^{5x}$
\end{answer}
\end{exercise}


\begin{exercise}
$\ddx x^{\ln(x)}$
\begin{answer}
$2\ln(x)x^{\ln(x)-1}$
\end{answer}
\end{exercise}


\begin{exercise}
$\ddx x^{100 x}$
\begin{answer}
$(100 + 100 \ln(x))x^{100x}$
\end{answer}
\end{exercise}


\begin{exercise}
$\ddx \left((3x)^{4x}\right)$
\begin{answer}
$(4+ 4\ln(3x)) (3x)^{4x}$
\end{answer}
\end{exercise}


\begin{exercise}
$\ddx x^{(e^x)}$
\begin{answer}
$((e^x)/x+ e^x\ln(x))x^{e^x}$
\end{answer}
\end{exercise}


\begin{exercise}
$\ddx x^\pi + \pi^x$
\begin{answer}
$\pi x^{\pi-1} + \pi^x\ln(\pi)$
\end{answer}
\end{exercise}


\begin{exercise}
$\ddx\left(1+\frac{1}{x}\right)^x$
\begin{answer}
$(\ln(1+1/x) - 1/(x+1))(1+1/x)^x$
\end{answer}
\end{exercise}


\begin{exercise}
$\ddx (\ln(x))^x$
\begin{answer}
$(1/\ln(x)+\ln(\ln(x)))(\ln(x))^x$
\end{answer}
\end{exercise}


\begin{exercise}
$\ddx \left(f(x) g(x) h(x)\right)$
\begin{answer}
$(f'(x)/f(x) + g'(x)/g(x) + h'(x)/h(x))f(x) g(x) h(x)$
\end{answer}
\end{exercise}

\endtwocol

\end{exercises}
