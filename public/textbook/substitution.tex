\section{Substitution}{}{}


Needless to say, most problems we encounter will not be so
simple. Here's a slightly more complicated example: find
$$\int 2x\cos(x^2)\,dx.$$
This is not a ``simple'' derivative, but a little thought reveals that
it must have come from an application of the chain rule. Multiplied
on the ``outside'' is $2x$, which is the derivative of the ``inside''
function $\ds x^2$. Checking:
$${d\over dx}\sin(x^2)  = \cos(x^2){d\over dx}x^2 = 2x\cos(x^2),$$
so 
$$\int 2x\cos(x^2)\,dx=\sin(x^2)+C .$$

Even when the chain rule has ``produced'' a certain derivative, it is
not always easy to see. Consider this problem:
$$\int x^3\sqrt{1-x^2}\,dx.$$
There are two factors in this expression, $\ds x^3$ and $\ds \sqrt{1-x^2}$,
but it is not apparent that the chain rule is involved. Some clever
rearrangement reveals that it is:
$$
  \int x^3\sqrt{1-x^2}\,dx=
  \int(-2x)\left(-{1\over2}\right)(1-(1-x^2))\sqrt{1-x^2}\,dx.
$$ 
This looks messy, but we do now have something that looks like the
result of the chain rule: the function $\ds 1-x^2$ has been substituted
into $\ds -(1/2)(1-x)\sqrt{x}$, and the derivative of $\ds 1-x^2$, $-2x$,
multiplied on the outside. If we can find a function $F(x)$ whose derivative
is $\ds -(1/2)(1-x)\sqrt{x}$ we'll be done, since then
$$\eqalign{
  {d\over dx}F(1-x^2) = -2xF'(1-x^2) &= 
  (-2x)\left(-{1\over2}\right)(1-(1-x^2))\sqrt{1-x^2} \\
  &=x^3\sqrt{1-x^2} \\
}$$
But this isn't hard:
$$\eqalignno{
  \int-{1\over2}(1-x)\sqrt{x}\,dx&=\int -{1\over2}(x^{1/2}-x^{3/2})\,dx&
  \eqrdef{eq:auxiliary integral}(\xrefn{eq:auxiliary integral}) \\
  &=-{1\over2}\left({2\over3}x^{3/2}-{2\over 5}x^{5/2}\right)+C \\
  &=\left({1\over5}x-{1\over3}\right)x^{3/2}+C. \\
  }$$
So finally we have
$$
  \int x^3\sqrt{1-x^2}\,dx=\left({1\over5}(1-x^2)-{1\over3}\right)
  (1-x^2)^{3/2}+C.
$$

So we succeeded, but it required a clever first step, rewriting the
original function so that it looked like the result of using the chain
rule. Fortunately, there is a technique that makes such problems
simpler, without requiring cleverness to rewrite a function in just
the right way. It does sometimes not work, or may require more than one
attempt, but the idea is simple: guess at the most likely candidate for
the ``inside function'', then do some algebra to see what this
requires the rest of the function to look like.

One frequently good guess is any complicated expression inside a
square root, so we start by trying $\ds u=1-x^2$, using a new variable,
$u$, for convenience in the manipulations that follow. Now we know
that the chain rule will multiply by the derivative of this inner
function:
$${du\over dx} = -2x,$$
so we need to rewrite the original function to include this:
$$
  \int x^3\sqrt{1-x^2}=\int x^3\sqrt{u}{-2x\over-2x}\,dx=
  \int {x^2\over-2}\sqrt{u}\,{du\over dx}\,dx.
$$ 
Recall that one benefit of the Leibniz notation is that it often turns out
that what looks like ordinary arithmetic gives the correct answer,
even if something more complicated is going on. For example, in
Leibniz notation the chain rule is
$${dy\over dx}={dy\over dt}{dt\over dx}.$$
The same is true of our current expression:
$$
  \int {x^2\over-2}\sqrt{u}\,{du\over dx}\,dx=
  \int {x^2\over-2}\sqrt{u}\,du.
$$
Now we're almost there: since $\ds u=1-x^2$, $\ds x^2=1-u$ and the integral is
$$\int -{1\over2}(1-u)\sqrt{u}\,du.$$
It's no coincidence that this is exactly the integral we computed in
(\xrefn{eq:auxiliary integral}), 
we have simply renamed the variable $u$ to make the
calculations less confusing. Just as before:
$$
  \int-{1\over2}(1-u)\sqrt{u}\,du=
  \left({1\over5}u-{1\over3}\right)u^{3/2}+C.
$$
Then since $\ds u=1-x^2$:
$$
  \int x^3\sqrt{1-x^2}\,dx=\left({1\over5}(1-x^2)-{1\over3}\right)
  (1-x^2)^{3/2}+C.
$$
To summarize: if we suspect that a given function is the derivative of
another via the chain rule, we let $u$ denote a likely candidate for
the inner function, then translate the given function so that it is
written entirely in terms of $u$, with no $x$ remaining in the
expression. If we can integrate this new function of $u$, then the
antiderivative of the original function is obtained by replacing $u$
by the equivalent expression in $x$.

Even in simple cases you may prefer to use this mechanical procedure,
since it often helps to avoid silly mistakes. For example, consider
again this simple problem:
$$\int 2x\cos(x^2)\,dx.$$
Let $\ds u=x^2$, then $du/dx = 2x$ or $du = 2x\,dx$. Since we have exactly 
$2x\,dx$ in the original integral, we can replace it by $du$:
$$\int 2x\cos(x^2)\,dx=\int \cos u\,du=\sin u +C = \sin(x^2)+C.$$
This is not the only way to do the algebra, and typically there are
many paths to the correct answer. Another possibility, for example,
is: Since $du/dx = 2x$, $dx=du/2x$, and then the integral becomes
$$\int 2x\cos(x^2)\,dx=\int 2x\cos u\,{du\over 2x}=\int \cos u\,du.$$
The important thing to remember is that you must eliminate all
instances of the original variable $x$.

\begin{example}
Evaluate $\ds\int(ax+b)^n\,dx$, assuming that $a$ and $b$ are
constants, $a\not=0$, and $n$ is a positive integer.
We let $u=ax+b$ so $du=a\,dx$ or $dx=du/a$. Then
$$
  \int(ax+b)^n\,dx=\int {1\over a} u^n\,du={1\over a(n+1)}u^{n+1}+C=
  {1\over a(n+1)}(ax+b)^{n+1}+C.
$$
\vskip-10pt\end{example}

\begin{example}
Evaluate $\ds\int \sin(ax+b)\,dx$, assuming that $a$ and $b$ are
constants and $a\not=0$.
Again we let $u=ax+b$ so $du=a\,dx$ or $dx=du/a$. Then
$$
  \int\sin(ax+b)\,dx=\int {1\over a} \sin u\,du={1\over a}(-\cos u)+C=
-{1\over a}\cos(ax+b)+C.
$$
\vskip-10pt\end{example}

\begin{example}
Evaluate $\ds\int_2^4 x\sin(x^2)\,dx$. First we compute the
antiderivative, then evaluate the definite integral.
Let $\ds u=x^2$ so $du=2x\,dx$ or $x\,dx=du/2$. Then
$$
  \int x\sin(x^2)\,dx=\int {1\over 2} \sin u\,du={1\over 2}(-\cos u)+C=
  -{1\over 2}\cos(x^2)+C.
$$
Now
$$
  \int_2^4 x\sin(x^2)\,dx=\left.-{1\over 2}\cos(x^2)\right|_2^4
  =-{1\over 2}\cos(16)+{1\over 2}\cos(4).
$$
A somewhat neater alternative to this method is to change the original
limits to match the variable $u$. Since $\ds u=x^2$, when $x=2$, $u=4$,
and when $x=4$, $u=16$. So we can do this:
$$
  \int_2^4 x\sin(x^2)\,dx=
  \int_4^{16} {1\over 2} \sin u\,du=\left.-{1\over 2}(\cos u)\right|_4^{16}
  =-{1\over 2}\cos(16)+{1\over 2}\cos(4).
$$
An incorrect, and dangerous, alternative is something like this:
$$
  \int_2^4 x\sin(x^2)\,dx=\int_2^4 {1\over 2} \sin u\,du=
  \left.-{1\over 2}\cos (u)\right|_2^4=
  \left.-{1\over 2}\cos(x^2)\right|_2^4=-{1\over 2}\cos(16)+{1\over
  2}\cos(4).
$$
This is incorrect because $\ds\int_2^4 {1\over 2} \sin u\,du$
means that $u$ takes on values between 2 and 4, which is wrong. It
is dangerous, because it is very easy to get to 
the point $\ds\left.-{1\over 2}\cos (u)\right|_2^4$ and forget to substitute
$\ds x^2$ back in for $u$, thus getting the incorrect answer
$\ds -{1\over 2}\cos(4)+{1\over 2}\cos(2)$. A somewhat clumsy, but
acceptable, alternative is something like this:
$$
  \int_2^4 x\sin(x^2)\,dx=\int_{x=2}^{x=4} {1\over 2} \sin u\,du=
  \left.-{1\over 2}\cos (u)\right|_{x=2}^{x=4}=
  \left.-{1\over 2}\cos(x^2)\right|_2^4=-{\cos(16)\over 2}+{\cos(4)\over2}.
$$
\vskip-10pt\end{example}

\begin{example}
Evaluate $\ds\int_{1/4}^{1/2}{\cos(\pi t)\over\sin^2(\pi t)}\,dt$.
Let $u=\sin(\pi t)$ so $du=\pi\cos(\pi t)\,dt$ or $du/\pi=\cos(\pi
t)\,dt$. We change the limits to $\ds \sin(\pi/4)=\sqrt2/2$ and 
$\sin(\pi/2)=1$.
Then
$$
  \int_{1/4}^{1/2}{\cos(\pi t)\over\sin^2(\pi t)}\,dt=
  \int_{\sqrt2/2}^{1}{1\over \pi}{1\over u^2}\,du=
  \int_{\sqrt2/2}^{1} {1\over \pi}u^{-2}\,du=
  \left.{1\over \pi}{u^{-1}\over -1}\right|_{\sqrt2/2}^{1}=
  -{1\over\pi}+{\sqrt2\over\pi}.
$$
\vskip-10pt\end{example}

\begin{exercises}

Find the antiderivatives.


\twocol

\begin{exercise} $\ds\int (1-t)^9\,dt$
\begin{answer} $\ds -(1-t)^{10}/10+C$
\end{answer}\end{exercise}

\begin{exercise} $\ds\int (x^2+1)^2\,dx$
\begin{answer} $\ds x^5/5+2x^3/3+x+C$
\end{answer}\end{exercise}

\begin{exercise} $\ds\int x(x^2+1)^{100}\,dx$
\begin{answer} $\ds (x^2+1)^{101}/202+C$
\end{answer}\end{exercise}

\begin{exercise} $\ds\int {1\over\root 3 \of {1-5t}}\,dt$ 
\begin{answer} $\ds -3(1-5t)^{2/3}/10+C$
\end{answer}\end{exercise}

\begin{exercise} $\ds\int \sin^3x\cos x\,dx$
\begin{answer} $\ds (\sin^4x)/4+C$
\end{answer}\end{exercise}

\begin{exercise} $\ds\int x\sqrt{100-x^2}\,dx$
\begin{answer} $\ds -(100-x^2)^{3/2}/3+C$
\end{answer}\end{exercise}

\begin{exercise} $\ds\int {x^2\over\sqrt{1-x^3}}\,dx$
\begin{answer} $\ds \ds -2\sqrt{1-x^3}/3+C$
\end{answer}\end{exercise}

\begin{exercise} $\ds\int \cos(\pi t)\cos\bigl(\sin(\pi t)\bigr)\,dt$
\begin{answer} $\ds \sin(\sin\pi t)/\pi+C$
\end{answer}\end{exercise}

\begin{exercise} $\ds\int {\sin x\over\cos^3 x}\,dx$
\begin{answer} $\ds \ds 1/(2\cos^2 x)=(1/2)\sec^2x+C$
\end{answer}\end{exercise}

\begin{exercise} $\ds\int\tan x\,dx$
\begin{answer} $-\ln|\cos x|+C$
\end{answer}\end{exercise}

\begin{exercise}  $\ds\int_0^\pi\sin^5(3x)\cos(3x)\,dx$
\begin{answer} $0$
\end{answer}\end{exercise}

\begin{exercise} $\ds\int\sec^2x\tan x\,dx$
\begin{answer} $\ds \tan^2(x)/2+C$
\end{answer}\end{exercise}

\begin{exercise} $\ds\int_0^{\sqrt{\pi}/2} x\sec^2(x^2)\tan(x^2)\,dx$
\begin{answer} $1/4$
\end{answer}\end{exercise}

\begin{exercise} $\ds\int {\sin(\tan x)\over\cos^2x}\,dx$
\begin{answer} $-\cos(\tan x)+C$
\end{answer}\end{exercise}

\begin{exercise} $\ds\int_3^4 {1\over(3x-7)^2}\,dx$
\begin{answer} $1/10$
\end{answer}\end{exercise}

\begin{exercise} $\ds\int_0^{\pi/6}(\cos^2x - \sin^2x)\,dx$
\begin{answer} $\ds \sqrt3/4$
\end{answer}\end{exercise}

\begin{exercise} $\ds\int {6x\over(x^2 - 7)^{1/9}}\,dx$
\begin{answer} $\ds (27/8)(x^2-7)^{8/9}$
\end{answer}\end{exercise}

\begin{exercise} $\ds\int_{-1}^1 (2x^3-1)(x^4-2x)^6\,dx$
\begin{answer} $\ds -(3^7+1)/14$
\end{answer}\end{exercise}

\begin{exercise} $\ds\int_{-1}^1 \sin^7 x\,dx$
\begin{answer} $0$
\end{answer}\end{exercise}

\begin{exercise} $\ds\int f(x) f'(x)\,dx$ 
\begin{answer} $\ds f(x)^2/2$
\end{answer}\end{exercise}

\endtwocol

\end{exercises}
