\section{The Chain Rule}{}{}
\label{sec:multivariable chain rules}
Consider the surface $z=x^2y+xy^2$, and suppose that 
$x=2+t^4$ and $y=1-t^3$. We can think of the latter two equations as
describing how $x$ and $y$ change relative to, say, time. Then
$$z=x^2y+xy^2=(2+t^4)^2(1-t^3)+(2+t^4)(1-t^3)^2$$ 
tells us explicitly how the $z$ coordinate of the corresponding point on the
surface depends on $t$. If we want to know $dz/dt$ we can compute it
more or less directly---it's actually a bit simpler to use the chain
rule:
$$\eqalign{
  {dz\over dt}&=x^2y'+2xx'y+x2yy'+x'y^2 \\
  &=(2xy+y^2)x'+(x^2+2xy)y' \\
  &=(2(2+t^4)(1-t^3)+(1-t^3)^2)(4t^3)+((2+t^4)^2+2(2+t^4)(1-t^3))(-3t^2) \\
}$$ 
If we look carefully at the middle step,
$dz/dt=(2xy+y^2)x'+(x^2+2xy)y'$, we notice that $2xy+y^2$ is $\partial
z/\partial x$, and $x^2+2xy$ is $\partial z/\partial y$.
This turns out to be true in general, and gives us a new chain rule:

\begin{theorem} Suppose that $z=f(x,y)$, $f$ is differentiable,
$x=g(t)$, and $y=h(t)$.
Assuming that the relevant derivatives exist, 
$${dz\over dt}={\partial z\over \partial x}{dx\over dt}+
{\partial z\over \partial y}{dy\over dt}.
$$
\begin{proof}
If $f$ is differentiable, then 
$$\Delta z=f_x(x_0,y_0)\Delta x+f_y(x_0,y_0)\Delta y+\epsilon_1\Delta
x + \epsilon_2\Delta y,$$
where $\epsilon_1$ and $\epsilon_2$ approach 0 as 
$(x,y)$ approaches $(x_0,y_0)$. Then
$${\Delta z\over\Delta t}=
f_x{\Delta x\over\Delta t}+f_y{\Delta y\over\Delta t}+\epsilon_1{\Delta
x\over\Delta t} + \epsilon_2{\Delta y\over\Delta t}.
\eqrdef{eq:dz over dt}
\eqno{(\xrefn{eq:dz over dt})}
$$
As $\Delta t$ approaches 0, $(x,y)$ approaches $(x_0,y_0)$ and so
$$\eqalign{
\lim_{\Delta t\to0}{\Delta z\over\Delta t} &=  {dz\over dt} \\
\lim_{\Delta t\to0}\epsilon_1{\Delta x\over\Delta t} &= 0\cdot{dx\over dt}  \\
\lim_{\Delta t\to0}\epsilon_2{\Delta y\over\Delta t} &= 0\cdot{dy\over dt}  \\
}$$
and so taking the limit of~(\xrefn{eq:dz over dt})
 as $\Delta t$ goes to 0 gives 
$$
{dz\over dt}=
f_x{dx\over dt}+f_y{dy\over dt},
$$
as desired.
\end{proof}

We can write the chain rule in way that is somewhat closer to the
single variable chain rule:
$${df\over dt}=\langle f_x,f_y\rangle\cdot\langle x',y'\rangle,$$
or (roughly) the derivatives of the outside function ``times'' the
derivatives of the inside functions.
Not surprisingly, essentially the same chain rule works for functions
of more than two variables, for example, given a function of three
variables $f(x,y,z)$, where each of $x$, $y$ and $z$ is a function of
$t$, 
$${df\over dt}=\langle f_x,f_y,f_z\rangle\cdot\langle x',y',z'\rangle.$$

We can even extend the idea further. Suppose that $f(x,y)$ is a
function and $x=g(s,t)$ and $y=h(s,t)$ are functions of two variables
$s$ and $t$. Then $f$ is ``really'' a function of $s$ and $t$ as well, and 
$${\partial f\over\partial s}=f_xg_s+f_yh_s\qquad
{\partial f\over\partial t}=f_xg_t+f_yh_t.$$
The natural extension of this to $f(x,y,z)$ works as well.

Recall that we used the ordinary chain rule to do implicit
differentiation\index{implicit differentiation}. We can do the same
with the new chain rule.

\exam $x^2+y^2+z^2 = 4$ defines a sphere, which is not a function of
$x$ and $y$, though it can be thought of as two functions, the top and
bottom hemispheres. We can think of $z$ as one of these two functions,
so really $z=z(x,y)$, and we can think of $x$ and $y$ as particularly
simple functions of $x$ and $y$, and let $f(x,y,z)=x^2+y^2+z^2$.
Since $f(x,y,z)=4$, $\partial f/\partial x=0$, but
using the chain rule:
$$\eqalign{
0={\partial f\over\partial x}&=f_x{\partial x\over\partial x}+
f_y{\partial y\over\partial x}+f_z{\partial z\over \partial x} \\
&=(2x)(1)+(2y)(0)+(2z){\partial z\over\partial x}, \\
}$$
noting that since $y$ is temporarily held constant its derivative
${\partial y/\partial x}=0$. Now we can solve for $\partial z/\partial
x$:
$${\partial z\over \partial x}=-{2x\over 2z}=-{x\over z}.
$$
In a similar manner we can compute $\partial z/\partial y$.
\endexam

\begin{exercises}

\begin{exercise} Use the chain rule to compute $dz/dt$ for
$z=\sin(x^2+y^2)$, $x=t^2+3$, $y=t^3$.
\begin{answer} $4xt\cos(x^2+y^2)+6yt^2\cos(x^2+y^2)$
\end{answer}\end{exercise}

\begin{exercise} Use the chain rule to compute $dz/dt$ for
$z=x^2y$, $x=\sin(t)$, $y=t^2+1$.
\begin{answer} $2xy\cos t+2x^2t$
\end{answer}\end{exercise}

\begin{exercise} Use the chain rule to compute $\partial z/\partial s$ and 
$\partial z/\partial t$ for
$z=x^2y$, $x=\sin(st)$, $y=t^2+s^2$.
\begin{answer} $2xyt\cos(st)+2x^2s$, $2xys\cos(st)+2x^2t$
\end{answer}\end{exercise}

\begin{exercise} Use the chain rule to compute $\partial z/\partial s$ and 
$\partial z/\partial t$ for
$z=x^2y^2$, $x=st$, $y=t^2-s^2$.
\begin{answer} $2xy^2t-4yx^2s$, $2xy^2s+4yx^2t$
\end{answer}\end{exercise}

\begin{exercise} Use the chain rule to compute $\partial z/\partial x$ and 
$\partial z/\partial y$ for $2x^2+3y^2-2z^2=9$.
\begin{answer} $x/z$, $3y/(2z)$
\end{answer}\end{exercise}

\begin{exercise} Use the chain rule to compute $\partial z/\partial x$ and 
$\partial z/\partial y$ for $2x^2+y^2+z^2=9$.
\begin{answer} $-2x/z$, $-y/z$
\end{answer}\end{exercise}

%% Balof
%% \begin{exercise} Find $\frac{dy}{dx}$ in each case both by Implicit
%%   Differentiation and by the Implicit Function Theorem.
%% 
%% \begin{enumerate}
%% \item $x^3+y^3+3xy=6$
%% \item $\sin(xy)=4$
%% \item $e^{x}+xy+\ln(y)=12$

%% \begin{exercise} Describe the points on the unit circle, $x^2+y^2=1$, where we
%%   {\em cannot\/} define one variable in terms of the other.
%% 
%% \begin{exercise} Describe the points on the unit sphere, $x^2+y^2+z^2=1$, where
%%   we {\em cannot} define one of the variables in terms of the other
%%   two.

\begin{exercise} Chemistry students will recognize the {\em ideal gas law\/}, given
  by $PV=nRT$ which relates the Pressure, Volume, and Temperature of
  $n$ moles of gas.  (R is the ideal gas constant).  Thus, we can view
  pressure, volume, and temperature as variables, each one dependent
  on the other two.

\begin{itemize} % BADBAD
\item{a.} If pressure of a gas is increasing at a rate of $0.2
  Pa/\hbox{min}$ and temperature is increasing at a rate of $1
  K/\hbox{min}$, how fast is the volume changing?
\item{b.} If the volume of a gas is decreasing at a rate of $0.3
  L/\hbox{min}$ and temperature is increasing at a rate of $.5
  K/\hbox{min}$, how fast is the pressure changing?
\item{c.} If the pressure of a gas is decreasing at a rate of $0.4
  Pa/\hbox{min}$ and the volume is increasing at a rate of $3
  L/\hbox{min}$, how fast is the temperature changing?
\end{itemize}
\begin{answer} (a) $\ds V'=(nR-0.2V)/P$\hfill\break
(b) $\ds P'=(nR+0.6P)/2V$\hfill\break
(c) $\ds T' = (3P-0.4V)/(nR)$
\end{answer}\end{exercise}

\begin{exercise} Verify the following identity in the case of the ideal gas law:
$${\partial P\over \partial V} {\partial V\over \partial T} 
{\partial T\over \partial P}=-1$$

\begin{exercise} The previous exercise was a special case of the following
fact, which you are to verify here: If $F(x,y,z)$ is a function of 3
variables, and the relation $F(x,y,z)=0$ defines each of the variables
in terms of the other two, namely $x=f(y,z)$, $y=g(x,z)$ and
$z=h(x,y)$, then
$${\partial x\over \partial y} {\partial y\over \partial z} 
{\partial z\over \partial x}=-1$$

\end{exercises}

