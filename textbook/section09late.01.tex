\section{Inverse functions}{}{}
\label{sec:inverse functions}

Informally, two functions $f$ and $g$ are {\dfont inverses\/} if
each reverses, or undoes, the other. More precisely:

\begin{definition} Two functions $f$ and $g$ are inverses if for all $x$ in the
domain of $g$,
$f(g(x))=x$, and for all $x$ in the domain of $f$, $g(f(x))=x$.
\end{definition}

\begin{example} $f=x^3$ and $g=x^{1/3}$ are inverses, since
$\ds (x^3)^{1/3}=x$ and $(x^{1/3})^3=x$.
\end{example}

\begin{example} $f=x^2$ and $g=x^{1/2}$ are not inverses. While
$(x^{1/2})^2=x$, it is not true that $(x^2)^{1/2}=x$. For example,
with $x=-2$, $((-2)^2)^{1/2}=4^{1/2}=2$.
\end{example}

The problem in the previous example can be traced to the fact that
there are two different numbers with square equal to 4. This turns out
to be precisely descriptive of functions without inverses.

\begin{definition}
Let $A$ and  $B$  be sets and let $f:A\to B$ be a
function. We say that $f$ is {\dfont injective\/} or
{\dfont one-to-one\/} if $f(x)=f(y)$ implies that $x=y$.

We say that $f$ is {\dfont surjective\/} or {\dfont onto\/} if for
 every $b\in B$ there is an $a\in A$ such that $f(a)=b$. 

 If $f$ is both injective and surjective, then $f$ is
 {\dfont bijective\/} or {\dfont one-to-one and onto}. 

\end{definition}



We are interested only in the case that $A$ and $B$ are sets of real
numbers, and in this case there is a nice geometric interpretation of
injectivity. It is often easy to use this interpretation to decide
whether a function is or is not 1-1.

\begin{theorem} (Horizontal line test) If $f$ is a function defined on some
subset of the real numbers, then $f$ is injective if and only if 
every horizontal line intersects the graph of $f$ at most once.
\end{theorem}

\begin{example} The function $f=x^2$ fails this test: horizontal lines $y=k$
for $k>0$ intersect the graph of $f$ twice. (The horizontal line $y=0$
does intersect it only once, and lines $y=k$, $k<0$, do not intersect the
graph at all.)
\end{example}


\begin{example} In each of these cases, we assume that $f\colon A\to \R$,
where
$A$ is the set of all real numbers for which $f$ makes sense.

The function $f(x)=x $ is bijective.

The function $f(x) = x^2 $ is neither injective nor
surjective. If we think of $f$ as a function from $\R$ to the
non-negative real numbers, then $f$ is surjective; in other words, if
a function is not surjective this is not a major stumbling block. 

The function $f(x) =1/x$ is injective but not
surjective since there is no value of $x$
such that $f(x)=0$.

The function $f(x) = x(x-1)(x+1) $ is surjective but
not injective; $f(x) =0 $ for three different values of
$x$. On the other hand $\ds\lim_{x\to \infty}
f(x)=\infty$ and $\ds\lim _{x\to -\infty}
f(x)=-\infty$. Since $f$ is continuous on $\R$, the
intermediate value theorem (\xrefn{thm:intermediate value theorem})
guarantees that $f$ takes all values
between $-\infty $ and $\infty$.
\end{example}


The derivative furnishes us with a convenient criterion
for injectivity without explicitly looking for points where
injectivity may fail.

\begin{theorem} If $f$ is differentiable and
$f'(x) >0 $ then $f$ is
injective.
\label{thm:differentiable criterion}
\begin{proof}
Suppose that $f(a) =f(b) $ for some $a< b$. By
Rolle's theorem (\xrefn{thm:rolle}) there exists $c\in (a,b) $ such that
$f'(c) =(f(b) -f(a))/(b-a)=0$,
which contradicts the hypothesis that $f'(x) >0 $.
Hence, if $f(a)=f(b)$ then $a=b$.
\end{proof}

In the same way, we can see that if $f'(x)<0$ then $f$ is injective.

\begin{example} Let $f(x) =x^5 + x $. Since $f'(x) = 5x^4 + 1 >0$,
$f$ is injective.
\end{example}

\begin{example} Let $f(x) =2x+\sin x $. Then $f'(x) = 2+\cos x \geq 1$
for every $x$. Hence, $f$ is injective. 
\end{example}

\begin{example} Let $f(x) =x^3 $. This $f$ is injective although the
above theorem does not apply, since $f'(0)=0$. 
Therefore, the conditions in the theorem
are sufficient but not necessary.
\end{example}

Our knowledge of derivatives can also lead us to conclude that a
function is not injective.

\begin{theorem} If $f$ is continuous and has a local maximum or minimum then $f$
is not injective.  
\label{thm:continuity criterion}
\begin{proof} Suppose that $f$ has a local maximum at
$x=c$. Then in some interval $(c-h, c+h)$, $f(x)\le f(c)$.  Let $a\in
(c-h, c)$.  If $f(a) =f(c)$ then $f$ is not injective; otherwise,
$f(a) < f(c)$.

Let $b\in (c,c+h)$. If $f(b) = f(c)$ or $f(b) =f(a) $ then $f$
is not injective. Otherwise, either $f(b) < f(a)<f(c) $ or $f(a)
<f(b)<f(c) $. If $f(b) < f(a) $ then by the intermediate value
theorem, there is a number $d\in (c, b) $ such that $f(d) =f(a)$ and
so $f$ is not injective.
 Likewise, if $f(a) < f(b) $ then there is a number $d$ in $(a,c) $
 such that $f(d) = f(b)$ and so $f$ is not injective. 

In every case, we see that $f$ is not injective.
\end{proof}

To return to our principal interest, inverse functions, we now connect
bijections and inverses.

\begin{theorem} Suppose $f\colon A\to B$ is a bijection. Then $f$ has an inverse
function $g\colon B\to A$. 
\begin{proof}
Suppose $b\in B$. Since $f$ is onto, there is an $a\in A$ such that 
$f(a)=b$. Since $f$ is 1--1, $a$ is the only element of $A$ with this
property. We let $g(b)=a$. Now it is easy to see that
for all $a\in A$, $g(f(a))=a$ and for all $b\in B$, $f(g(b))=b$. 
\end{proof}

We really don't have any choice about how to define $g$ in this proof;
if $f$ is a bijection, its inverse is completely determined. Thus,
instead of using a new symbol $g$, we normally refer to the inverse of
$f$ as $f^{-1}$.

Unfortunately, it is often difficult to find an explicit formula
for the inverse of a given function, $f$, even if it is known that $f$
is bijective. Generally, we attempt to find an inverse in this way:

\begin{itemize} % BADBAD
\item{1.} Write $y=f(x)$.
\item{2.} Interchange $x$ and $y$.
\item{3.} Solve for $y$.
\item{4.} Replace $y$ with $f^{-1}(x)$.
\end{itemize}

Step 3 is the hard part; indeed it is
sometimes   impossible to perform using algebraic operations.

\begin{example} Find the inverse of $f(x) =(2x-6)/(3x+7)$.
First we write $x=(2y-6)/(3y+7)$.
Now we solve for $y$:
$$\eqalign{
x&= {2y-6\over3y+7} \\
x(3y+7) &= 2y-6 \\
3xy+7x &= 2y - 6 \\
7x+6 &= 2y-3xy \\
7x+6 &= y(2-3x) \\
{7x+6\over 2-3x} &= y \\
}$$
Finally, we say $f^{-1} (x) =(7x+6)/(2-3x)$
\end{example}


\begin{example} Find the inverse function of $f(x) =x^2 - 4x + 8 $
where $x \geq 2 $. What are the domain and range of the inverse
function?

First, $y =x^2 - 4x + 8$ becomes $x=y^2-4y+8$.
Now we complete the square: $x=(y-2)^2 +4$ and
rearrange to get $x-4=(y-2)^2$. Since in the original function
$x-2\ge 0$, and we have switched $x$ and $y$, we know that
$y-2\ge 0$. Thus taking the square root, we know
$y-2=\sqrt{x-4}$, {\bf not} $y-2=-\sqrt{x-4}$.
Finally we write $y=f^{-1}(x)=2+\sqrt{x-4}$.
The domain of $f$ is $x\geq 4$ and the range is $y\geq 2$. 

\end{example}

While simple in principle, this method is sometimes difficult or
impossible to apply. For example, consider
$f(x)=x^3 + x $. Since $f'(x) = 3x^2 +1 >0 $ for every $x$,
$f$ is injective. (In fact it is bijective.) To find the inverse as
above, we would need to solve $x=y^3+y$ for $y$; while possible, this
is considerably more difficult than solving the quadratic of the
previous example. Some simple looking equations are impossible to
solve using algebraic manipulation. 

For example,
consider $f(x) =x^5 + x^3 + x +1$ a ``quintic'' polynomial (i.e., a
fifth degree polynomial). Since
$f'(x)= 5x^4 + 3x^2 + 1>0$, $f$ is injective
(and indeed $f$ is bijective). If there were a
quintic formula, analogous to the quadratic formula,
we could use that to compute $f^{-1}$. Unfortunately, no
such formula exists---fifth degree equations cannot in general be
solved. (There are exceptions; $x^5=1$ can be solved, for example.)

Fortunately, it is often more important to know that a function has
an inverse then to be able to come up with an explicit formula. Once
an inverse is known to exist, numerical techniques can often be
employed to obtain approximations of the inverse function. Thus,
theorem~\xrefn{thm:differentiable criterion}
and proposition~\xrefn{thm:continuity criterion} provide useful
criteria for deciding whether a function is invertible.

%\begin{exmp} The function $f(x) = \sqrt{x-4} + x^3 $ is increasing
%and hence has an inverse for $x\geq 4 $. \end{exmp}

We now turn to the calculus of inverse functions.

\begin{theorem} Let $A$ be an open interval and let $f:A\to \R$ be injective and
continuous. Then $f^{-1}$ is continuous on $f(A)$.  

\begin{proof} Since $A$ is an open interval and $f$ is injective and
continuous it follows by proposition~\xrefn{thm:continuity criterion}
that $f$ has no local maxima or minima. Hence, $f$ is either strictly
increasing or strictly decreasing. Without loss of generality, $f$ is
strictly increasing.

Fix $b\in f(A)$ . Then there exists a unique $a\in A$ such that
$f(a) = b $. Let $\epsilon > 0 $ and we may assume that $(a-\epsilon
, a+\epsilon ) \subseteq A $. Let $\delta =\min \{ b-
f(a-\epsilon ) , f(a+\epsilon ) -b \}$ and note that
$\delta
>0 $ since $f$ is increasing.  Then the interval $(b-\delta ,
b+\delta ) $ is mapped by $f^{-1} $ into $(a-\epsilon , a+\epsilon )
$. Since $\epsilon $ was arbitrary, it follows that $f^{-1 }$ is
continuous at $b$.
\end{proof}

Our principal interest in inverses is the simple relationship between
the derivative of a function and its inverse.

\begin{theorem} (Inverse function theorem) Let $A$ be an open interval and let $f:A\to
\label{thm:inverse function theorem}
\R$ be injective and differentiable.   If $f'(x) \neq 0 $
for every $x\in A $ then $f^{-1}$ is differentiable on $f(A)$ and
$(f^{-1})'(x) = 1/f'(f^{-1}(x))$.

\begin{proof} Fix $b\in f(A)$. Then there exists a unique $a\in A$ such
that $f(a)=b$. For $y\neq b $, let $x=f^{-1} (y) $. Since $f$ is
differentiable, it follows that $f$ and hence $f^{-1}$ are
continuous.

Then
$$\lim_{y\to b}{f^{-1} (b) - f^{-1} (y)\over b-y}
   = \lim_{x\to a}{a-x\over f(a) -f(x)} ={1\over f'(a)}.$$
\end{proof}

In Leibniz notation, this can be written as
$\ds{dx\over dy} ={1\over dy/dx}$, which is easy to remember since it looks
like ordinary fractional algebra.

\begin{example} Let $f(x) = 3x^3 + 5x - 7$. Since $f(0)=-7$, 
$f^{-1} (-7) =0$. Since $f'(x) = 9x^2 + 5$, $f'(0)=5$ and
so $(f^{-1})' (-7)=1/f'(0)=1/5$.
\end{example}

\begin{exercises}

\exercise Which of the following functions are injective? Which
are surjective? Which are bijective? Sketch the graph of each
function to illustrate your answers.

\begin{itemize} % BADBAD

\item{a.} $f\colon \R\to\R$, $f(x)=x^2$.
\item{b.} $f\colon [0, \infty )\to\R$, $f(x)=x^2$.
\item{c.} $f\colon \R\to[0, \infty )$, $f(x)=x^2$.
\item{d.} $f\colon (-\infty,0]\to[0, \infty )$, $f(x)=x^2$.

\end{itemize}

\exercise Which of the following functions are injective? Which
are surjective? Which are bijective? Sketch the graph of each
function to illustrate your answers.

\begin{itemize} % BADBAD

\item{a.} $f\colon \R\to\R$, $f(x)=x^3$.
\item{b.} $f\colon [0, \infty )\to\R$, $f(x)=\sqrt{x}$.
\item{c.} $f\colon \R\to[-1, 1]$, $f(x)=\sin x$.
\item{d.} $f\colon [0,\pi]\to[-1, 1]$, $f(x)=\cos x$.

\end{itemize}

\exercise Define $$f(x)=\cases{
\ds-{1\over x}&$x\neq 0$ \\
\ds 1&$x=0$ \\}$$

\item{}
Show that $f$ is not injective on $\R$. Show that $f'(x) >0
$ for $x\neq 0 $. Why does this not contradict 
theorem~\xrefn{thm:differentiable criterion}?


\exercise Define $$f(x)=\cases{
\ds 1-x&$-2\leq x<0$ \\
\ds 0&$x=0$ \\
\ds 10+x&$0< x \leq 2$ \\}$$

\item{} Show that
$f$ is injective and has a local minimum. Why does this not
contradict theorem~\xrefn{thm:continuity criterion}? 

\exercise If $A=\R$ sketch the graph of the identity
function on $A$.

\exercise Find the inverse function of $f(x) =(4x-6)/(7x+ 5)$. What
are the domain and range of $f^{-1}$?


\exercise Find the inverse function of $f(x) = 11x/(13x-6)$. What are
the domain and range of $f^{-1}$?


\exercise Find the inverse function of $f(x)=ax+b $ when $a\neq
0$. What are the domain and range of $f^{-1}$?

\exercise Find the inverse function of $f(x) = 1/(cx+d)$
when $c\neq 0$. What are the domain and range of $f^{-1}$?

\exercise Suppose that $ad-bc \neq 0$. Find the inverse function
of $f(x) = (ax+b)/(cx+d)$. What are the domain and range of
$f^{-1}$? (The domain and range will depend on which if any of
$a,b,c$, and $d$ are zero.)

\item{}
Note: The condition $ad-bc \neq 0 $ is a technical condition which
ensures that both the domain of $f$  will be all real numbers with
perhaps one exception and that that the range of $f$ will be all
real numbers with perhaps one exception.

\exercise Find the inverse function of $f(x) =|x-4|$ for
$x\leq4$. What are the domain and range of $f^{-1}$?

\exercise Find the inverse function of $f(x) = \sqrt{x-5}$. 
What are the domain and range of $f^{-1}$?

\exercise Find the inverse function of $f(x) = x^3 - 5$.  What are the
domain and range of $f^{-1}$?

\exercise Find the inverse function of $f(x) =x^7 - 2$. What are the
domain and range of $f^{-1}$?

\exercise Find the inverse function of $f(x) =2x^2 +8x - 4 $ for
$x\geq -2 $. What are the domain and range of $f^{-1 } $? 

\exercise Find the inverse function of $f(x) =x^2 -9x + 10 $ for
$x\leq 3 $.  What are the domain and range of $f^{-1 } $?  

\exercise Find the inverse function of $f(x)= x^2 +bx+ c$ for $x\geq
-b/2$. What are the domain and range of $f^{-1 } $?

\exercise Find the inverse function of $f(x)= x^2 +bx+ c$ for
$x\leq -b/2$. What are the domain and range of $f^{-1}$?

\exercise Find the inverse function of $f(x) =
(1+\sqrt{x})/(1-\sqrt{x})$. What are the domain and range of $f^{-1 }$?

\exercise Show that $f(x) =x^7 + 3x $ has an inverse
function on $\R$.

\exercise Show that $f(x) =x^{19/9} +x^5 $ has an inverse
function on $\R$. 

\exercise\label{exer:reflection} 
Let $A\subset \R$ and let $f: A\to\R$ be
injective.  Then $f^{-1}$ exists on $f(A)$.

\item{}
Note that the point $P(a, f(a)) $ is on the graph of $f$ and that
$Q(f(a), a) $ is the corresponding point on the graph of $f^{-1} $.
\label{exer:inverse is reflection}
\begin{itemize} % BADBAD

\item{a.} Show that if $a\neq f(a)$ then the slope of the line segment
$PQ$ is $-1 $.

\item{b.} Conclude that if $a\neq f(a)$ the line segment $PQ$ is
perpendicular to the graph, $L$,  of the identity function on
$\R$.

\item{c.} Show that the midpoint of $PQ$ is on $L$.

\item{d.} Conclude that the graph of $f^{-1}$ is the graph of $f$
reflected through $L$.
 
\end{itemize}

\exercise Let $f(x) =x^3 +x $. Sketch the graph of $f$ and
 $f^{-1} $ on the same diagram.

\exercise Let $f(x) =x^5 + x^3 +1 $. Sketch the graph of $f$ and
 $f^{-1} $ on the same diagram. 

\exercise 
\begin{itemize} % BADBAD

\item{a.} Suppose that $f$ is an increasing function on $\R$. What can
 you say about $f^{-1}$?

 \item{b.} Suppose that $f$ is a concave up function on $\R$.
 What can you say about $f^{-1} $?

 \item{}
In both parts, use exercise~\xrefn{exer:reflection} to illustrate your
 claim.

\end{itemize}

\exercise Let $f(x) = 3x^3 + 9x + 4$. Compute $(f^{-1})'(4)$.
 

\exercise Let $f(x) = 2x^2 + 11$. Show that $f$ is increasing at
$x=0$. Thus, there is an interval $I$ containing $0$ such that $f$ is
injective on $I$. Compute $(f^{-1})'(11)$.
 

\exercise Let $f(x) = ax+b$ with $a\neq 0$. Compute
$(f^{-1})'(b)$. Why do we need the condition $a\neq 0$?
 

\exercise Let $f(x) =ax^2 +bx + c $ with $b\neq 0$.  Compute
$(f^{-1})'(c)$. Why do we need the condition $b\neq 0$?

\exercise Let
 $f(x) =a_n x^n + a_{n-1} x^{n-1 } + \cdots + a_1 x
 +a_0 =\sum_{k=0}^n a_k x^k$ with $a_1 \neq 0$. Compute
 $(f^{-1} )'(a_0)$. Why do we need the condition $a_1 \neq 0$? 


\exercise Suppose that $f$ is injective on some interval containing $3$.
  If $f(3)=4$ and $f'(3)=6$ what is $(f^{-1})'(4)$?
  
\end{exercises}
