\section{Taylor's Theorem}{}{}
\nobreak
One of the most important uses of infinite series is the potential for
using an initial portion of the series for $f$ to approximate $f$. We
have seen, for example, that when we add up the first $n$ terms of an
alternating series with decreasing terms that the difference between
this and the true value is at most the size of the next term. A
similar result is true of many Taylor series.

\begin{theorem} Suppose that $f$ is defined on some open interval $I$ around $a$ and
suppose $\ds f^{(N+1)}(x)$ exists on this interval. Then
for each $x\not=a$ in $I$ there is a value $z$ between
$x$ and $a$ so that
$$ 
  f(x) = \sum_{n=0}^N {f^{(n)}(a)\over n!}\,(x-a)^n + 
  {f^{(N+1)}(z)\over (N+1)!}(x-a)^{N+1}. 
$$ 
\begin{proof}
The proof requires some cleverness to set up, but then the details are
quite elementary. We want to define a function $F(t)$. 
Start with the equation
$$F(t)=\sum_{n=0}^N{f^{(n)}(t)\over n!}\,(x-t)^n + B(x-t)^{N+1}.$$
Here we have replaced $a$ by $t$ in the first $N+1$ terms of the
Taylor series, and added a carefully chosen term on the end, with $B$
to be determined. Note that
we are temporarily keeping $x$ fixed, so the only variable in this
equation is $t$, and we will be interested
only in $t$ between $a$ and $x$. Now substitute $t=a$:
$$F(a)=\sum_{n=0}^N{f^{(n)}(a)\over n!}\,(x-a)^n + B(x-a)^{N+1}.$$
Set this equal to $f(x)$:
$$f(x)=\sum_{n=0}^N{f^{(n)}(a)\over n!}\,(x-a)^n + B(x-a)^{N+1}.$$
Since $x\not=a$, we can solve this for $B$, which is a
``constant''---it depends on $x$ and $a$ but those are temporarily 
fixed.  Now we
have defined a function $F(t)$ with the property that
$F(a)=f(x)$. Consider also $F(x)$: all terms with a positive power of
$(x-t)$ become zero when we substitute $x$ for $t$, so we are left
with $\ds F(x)=f^{(0)}(x)/0!=f(x)$. So $F(t)$ is a function with the same
value on the endpoints of the interval $[a,x]$. 
By Rolle's theorem (\xrefn{thm:rolle}), we
know that there is a value $z\in(a,x)$ such that $F'(z)=0$. Let's look
at $F'(t)$. Each term in $F(t)$, except the first term and the extra
term involving $B$, is a product, so to take the derivative we use the
product rule on each of these terms. It will help to write out the
first few terms of the definition:
$$\eqalign{
  F(t)=f(t)&+{f^{(1)}(t)\over 1!}(x-t)^1+{f^{(2)}(t)\over 2!}(x-t)^2+
  {f^{(3)}(t)\over 3!}(x-t)^3+\cdots \\
  &+{f^{(N)}(t)\over N!}(x-t)^N+
  B(x-t)^{N+1}. \\}
$$
Now take the derivative:
$$\eqalign{
  F'(t) = f'(t) &+ 
  \left({f^{(1)}(t)\over 1!}(x-t)^0(-1)+{f^{(2)}(t)\over
    1!}(x-t)^1\right) \\
  &+\left({f^{(2)}(t)\over 1!}(x-t)^1(-1)+{f^{(3)}(t)\over
    2!}(x-t)^2\right) \\
  &+\left({f^{(3)}(t)\over 2!}(x-t)^2(-1)+{f^{(4)}(t)\over
    3!}(x-t)^3\right)+\dots+ \\
  &+\left({f^{(N)}(t)\over (N-1)!}(x-t)^{N-1}(-1)+{f^{(N+1)}(t)\over
    N!}(x-t)^N\right) \\
  &+B(N+1)(x-t)^N(-1). \\}
$$
Now most of the terms in this expression cancel out,
leaving just
$$F'(t) = {f^{(N+1)}(t)\over N!}(x-t)^N+B(N+1)(x-t)^N(-1).$$
At some $z$, $F'(z)=0$ so
$$\eqalign{
  0&={f^{(N+1)}(z)\over N!}(x-z)^N+B(N+1)(x-z)^N(-1) \\
  B(N+1)(x-z)^N&={f^{(N+1)}(z)\over N!}(x-z)^N \\
  B&={f^{(N+1)}(z)\over (N+1)!}. \\
}$$
Now we can write 
$$
  F(t)=\sum_{n=0}^N{f^{(n)}(t)\over n!}\,(x-t)^n + 
  {f^{(N+1)}(z)\over (N+1)!}(x-t)^{N+1}.
$$
Recalling that $F(a)=f(x)$ we get
$$
  f(x)=\sum_{n=0}^N{f^{(n)}(a)\over n!}\,(x-a)^n + 
  {f^{(N+1)}(z)\over (N+1)!}(x-a)^{N+1},
$$
which is what we wanted to show.
\end{proof}

It may not be immediately obvious that this is particularly useful;
let's look at some examples.

\begin{example} Find a polynomial approximation for $\sin x$ accurate to $\pm
0.005$. 

From Taylor's theorem:
$$
  \sin x= \sum_{n=0}^N{f^{(n)}(a)\over n!}\,(x-a)^n + 
  {f^{(N+1)}(z)\over (N+1)!}(x-a)^{N+1}.
$$
What can we say about the size of the term
$${f^{(N+1)}(z)\over (N+1)!}(x-a)^{N+1}?$$
Every derivative of $\sin x$ is $\pm\sin x$ or $\pm\cos x$, so
$\ds |f^{(N+1)}(z)|\le 1$. The factor $\ds (x-a)^{N+1}$ is a bit more
difficult, since $x-a$ could be quite large. Let's pick $a=0$ and
$|x|\le\pi/2$; if we can compute $\sin x$ for $x\in[-\pi/2,\pi/2]$, we
can of course compute $\sin x$ for all $x$.

We need to pick $N$ so that 
$$\left|{x^{N+1}\over (N+1)!}\right|< 0.005.$$
Since we have limited $x$ to $[-\pi/2,\pi/2]$,
$$\left|{x^{N+1}\over (N+1)!}\right|<{2^{N+1}\over (N+1)!}.$$
The quantity on the right decreases with increasing $N$, so all we
need to do is find an $N$ so that 
$${2^{N+1}\over (N+1)!}<0.005.$$
A little trial and error shows that $N=8$ works, 
and in fact $\ds 2^{9}/9!<0.0015$, so 
$$\eqalign{
  \sin x &=\sum_{n=0}^8{f^{(n)}(0)\over n!}\,x^n \pm 0.0015 \\
  &=x-{x^3\over 6}+{x^5\over 120}-{x^7\over 5040}\pm 0.0015. \\
}$$
Figure~\xrefn{fig:sine approximation} shows the graphs of $\sin x$ and
and the approximation on $[0,3\pi/2]$. As $x$ gets larger, the
approximation heads to negative infinity very quickly, since it is
essentially acting like $\ds -x^7$.
\end{example}

\figure
\vbox{\beginpicture
\normalgraphs
\ninepoint
\setcoordinatesystem units <2truecm,0.7truecm>
\setplotarea x from 0 to 5, y from -5 to 1
\axis left ticks numbered from -5 to 1 by 1 /
\axis bottom shiftedto y=0 ticks numbered from 1 to 5 by 1 /
\setquadratic
\plot 0.000 0.000 0.047 0.047 0.094 0.094 0.141 0.141 0.188 0.187 
0.236 0.233 0.283 0.279 0.330 0.324 0.377 0.368 0.424 0.412 
0.471 0.454 0.518 0.495 0.565 0.536 0.613 0.575 0.660 0.613 
0.707 0.649 0.754 0.685 0.801 0.718 0.848 0.750 0.895 0.780 
0.942 0.809 0.990 0.836 1.037 0.861 1.084 0.884 1.131 0.905 
1.178 0.924 1.225 0.941 1.272 0.956 1.319 0.969 1.367 0.979 
1.414 0.988 1.461 0.994 1.508 0.998 1.555 1.000 1.602 0.999 
1.649 0.997 1.696 0.992 1.744 0.985 1.791 0.975 1.838 0.964 
1.885 0.950 1.932 0.934 1.979 0.917 2.026 0.896 2.073 0.874 
2.121 0.850 2.168 0.824 2.215 0.796 2.262 0.766 2.309 0.735 
2.356 0.701 2.403 0.666 2.450 0.629 2.498 0.591 2.545 0.550 
2.592 0.509 2.639 0.466 2.686 0.421 2.733 0.375 2.780 0.328 
2.827 0.279 2.875 0.230 2.922 0.179 2.969 0.126 3.016 0.073 
3.063 0.018 3.110 -0.037 3.157 -0.094 3.204 -0.152 3.252 -0.212 
3.299 -0.272 3.346 -0.334 3.393 -0.397 3.440 -0.461 3.487 -0.527 
3.534 -0.595 3.581 -0.664 3.629 -0.735 3.676 -0.808 3.723 -0.884 
3.770 -0.962 3.817 -1.042 3.864 -1.125 3.911 -1.212 3.958 -1.302 
4.006 -1.395 4.053 -1.493 4.100 -1.596 4.147 -1.703 4.194 -1.816 
4.241 -1.936 4.288 -2.061 4.335 -2.194 4.383 -2.335 4.430 -2.484 
4.477 -2.642 4.524 -2.811 4.571 -2.990 4.618 -3.181 4.665 -3.385 
4.712 -3.602 /
\plot 0.000 0.000 0.047 0.047 0.094 0.094 0.141 0.141 0.188 0.187 
0.236 0.233 0.283 0.279 0.330 0.324 0.377 0.368 0.424 0.412 
0.471 0.454 0.518 0.495 0.565 0.536 0.613 0.575 0.660 0.613 
0.707 0.649 0.754 0.685 0.801 0.718 0.848 0.750 0.895 0.780 
0.942 0.809 0.990 0.836 1.037 0.861 1.084 0.884 1.131 0.905 
1.178 0.924 1.225 0.941 1.272 0.956 1.319 0.969 1.367 0.979 
1.414 0.988 1.461 0.994 1.508 0.998 1.555 1.000 1.602 1.000 
1.649 0.997 1.696 0.992 1.744 0.985 1.791 0.976 1.838 0.965 
1.885 0.951 1.932 0.935 1.979 0.918 2.026 0.898 2.073 0.876 
2.121 0.853 2.168 0.827 2.215 0.800 2.262 0.771 2.309 0.740 
2.356 0.707 2.403 0.673 2.450 0.637 2.498 0.600 2.545 0.562 
2.592 0.522 2.639 0.482 2.686 0.440 2.733 0.397 2.780 0.353 
2.827 0.309 2.875 0.264 2.922 0.218 2.969 0.172 3.016 0.125 
3.063 0.078 3.110 0.031 3.157 -0.016 3.204 -0.063 3.252 -0.110 
3.299 -0.156 3.346 -0.203 3.393 -0.249 3.440 -0.294 3.487 -0.339 
3.534 -0.383 3.581 -0.426 3.629 -0.468 3.676 -0.509 3.723 -0.549 
3.770 -0.588 3.817 -0.625 3.864 -0.661 3.911 -0.696 3.958 -0.729 
4.006 -0.760 4.053 -0.790 4.100 -0.818 4.147 -0.844 4.194 -0.869 
4.241 -0.891 4.288 -0.911 4.335 -0.930 4.383 -0.946 4.430 -0.960 
4.477 -0.972 4.524 -0.982 4.571 -0.990 4.618 -0.996 4.665 -0.999 
4.712 -1.000 /
\endpicture}
\figrdef{fig:sine approximation}
\endfigure{$\sin x$ and a polynomial approximation.
(\expandafter\url\expandafter{\liveurl jsxgraph/taylor_series.html}%
AP\endurl)}

We can extract a bit more information from this example. If we do not
limit the value of $x$, we still have 
$$
  \left|{f^{(N+1)}(z)\over (N+1)!}x^{N+1}\right|\le 
  \left|{x^{N+1}\over (N+1)!}\right|
$$
so that $\sin x$ is represented by 
$$
  \sum_{n=0}^N{f^{(n)}(0)\over n!}\,x^n \pm 
  \left|{x^{N+1}\over (N+1)!}\right|.
$$
If we can show that 
$$
  \lim_{N\to\infty} \left|{x^{N+1}\over (N+1)!}\right|=0
$$
for each $x$ then 
$$
  \sin x=\sum_{n=0}^\infty{f^{(n)}(0)\over n!}\,x^n
  = \sum_{n=0}^\infty (-1)^n{x^{2n+1}\over (2n+1)!},
$$
that is, the sine function is actually equal to its
Maclaurin series for all $x$. How can we prove that the limit is zero?
Suppose that $N$ is larger than $|x|$, and let $M$ be the largest
integer less than $|x|$ (if $M=0$ the following is even easier). Then
$$
  \eqalign{
  {|x^{N+1}|\over (N+1)!} &= {|x|\over N+1}{|x|\over N}{|x|\over N-1}\cdots
    {|x|\over M+1}{|x|\over M}{|x|\over M-1}\cdots {|x|\over 2}{|x|\over 1} \\
  &\le {|x|\over N+1}\cdot 1\cdot 1\cdots 1\cdot
    {|x|\over M}{|x|\over M-1}\cdots {|x|\over 2}{|x|\over 1} \\
  &={|x|\over N+1}{|x|^M\over M!}.
  }
$$
The quantity $|x|^M/ M!$ is a constant, so 
$$
  \lim_{N\to\infty} {|x|\over N+1}{|x|^M\over M!} = 0
$$
and by the Squeeze Theorem (\xrefn{thm:squeeze theorem for sequences})
$$
  \lim_{N\to\infty} \left|{x^{N+1}\over (N+1)!}\right|=0
$$
as desired. Essentially the same argument works for $\cos x$ and $\ds
e^x$; unfortunately, it is more difficult to show that most functions
are equal to their Maclaurin series.

\begin{example} Find a polynomial approximation for $\ds e^x$ near $x=2$
accurate to $\pm
0.005$. 

From Taylor's theorem:
$$
  e^x= \sum_{n=0}^N{e^2\over n!}\,(x-2)^n + 
  {e^z\over (N+1)!}(x-2)^{N+1},
$$
since $\ds f^{(n)}(x)=e^x$ for all $n$. We are interested in $x$ near 2,
and we need to keep $\ds |(x-2)^{N+1}|$ in check, so we may as well
specify that $|x-2|\le 1$, so $x\in[1,3]$. Also
$$\left|{e^z\over (N+1)!}\right|\le {e^3\over (N+1)!},$$
so we need to find an $N$ that makes $\ds e^3/(N+1)!\le 0.005$. This time
$N=5$ makes $\ds e^3/(N+1)!< 0.0015$, so the approximating polynomial is
$$
  e^x=e^2+e^2(x-2)+{e^2\over2}(x-2)^2+{e^2\over6}(x-2)^3+
  {e^2\over24}(x-2)^4+{e^2\over120}(x-2)^5
  \pm 0.0015.
$$
This presents an additional problem for approximation, since we also
need to approximate $\ds e^2$, and any approximation we use will increase
the error, but we will not pursue this complication.
\end{example}

Note well that in these examples we found polynomials of a certain
accuracy only on a small interval, even though the series for $\sin x$
and $\ds e^x$ converge for all $x$; this is typical. To get the same
accuracy on a larger interval would require more terms. 

\begin{exercises}

\begin{exercise} Find a polynomial approximation for $\cos x$ on $[0,\pi]$,
accurate to $\ds \pm 10^{-3}$
\begin{answer} $\ds 1-{x^2\over2}+{x^4\over24}-{x^6\over 720}
+\cdots+{x^{12}\over 12!}$
\end{answer}\end{exercise}

\begin{exercise} How many terms of the series for $\ln x$ centered at 1 are
required so that the guaranteed error on $[1/2,3/2]$ is
at most $\ds 10^{-3}$? What if the interval is instead $[1,3/2]$? 
\begin{answer} $1000$; $8$ 
\end{answer}\end{exercise}

\begin{exercise} Find the first three nonzero terms in the Taylor series for
$\tan x$ on $[-\pi/4,\pi/4]$,
and compute the guaranteed error term as given by 
Taylor's theorem. (You may want to use Sage or a similar aid.)
\begin{answer} $\ds x+{x^3\over3}+{2x^5\over15}$, error $\pm 1.27$.
\end{answer}\end{exercise}

\begin{exercise} Show that $\cos x$ is equal to its Taylor series for all $x$
by showing that the limit of the error term is zero as $N$ approaches
infinity. 

\begin{exercise} Show that $\ds e^x$ is equal to its Taylor series for all $x$
by showing that the limit of the error term is zero as $N$ approaches
infinity. 

\end{exercises}
