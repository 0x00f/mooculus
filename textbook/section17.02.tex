\section{First Order Homogeneous Linear Equations}{}{}
%\label{sec:first order homogeneous linear}
\nobreak
A simple, but important and useful, type of separable equation is the
{\dfont first order homogeneous linear equation\/}:

\begin{definition} A first order homogeneous linear differential equation
is one of
the form $\ds \dot y + p(t)y=0$
or equivalently
$\ds \dot y = -p(t)y$.
\end{definition}

``Linear'' in this definition indicates that both $\dot y$ and $y$
occur to the first power; ``homogeneous'' refers to the zero on the
right hand side of the first form of the equation.

\begin{example} The equation $\ds \dot y = 2t(25-y)$ can be written
$\ds \dot y + 2ty= 50t$. This is linear, but not homogeneous. The
equation $\ds \dot y=ky$, or $\ds \dot y-ky=0$ is linear and
homogeneous, with a particularly simple $p(t)=-k$.
\end{example}

Because first order homogeneous linear equations are separable, we can
solve them in the usual way:
$$\eqalign{
\dot y &= -p(t)y \\
\int {1\over y}\,dy = \int -p(t)\,dt \\
\ln|y| &= P(t)+C \\
y&=\pm\,e^{P(t)} \\
y&=Ae^{P(t)}, \\}
$$
where $P(t)$ is an anti-derivative of $-p(t)$. As in previous
examples, if we allow $A=0$ we get the constant solution $y=0$.

\begin{example} Solve the initial value problems $\ds \dot y + y\cos t =0$,
$y(0)=1/2$ and $y(2)=1/2$. We start with
$$P(t)=\int -\cos t\,dt = -\sin t,$$
so the general solution to the differential equation is
$$y=Ae^{-\sin t}.$$
To compute $A$ we substitute:
$$ {1\over 2} = Ae^{-\sin 0} = A,$$
so the solutions is 
$$ y = {1\over 2} e^{-\sin t}.$$
For the second problem,
$$ \eqalign{
{1\over 2} &= Ae^{-\sin 2} \\
A &= {1\over 2}e^{\sin 2} \\}
$$
so the solution is 
$$ y = {1\over 2}e^{\sin 2}e^{-\sin t}.$$
\vskip-15pt\end{example}

\begin{example} Solve the initial value problem $y\dot y+3y=0$, $y(1)=2$,
assuming $t>0$. We
write the equation in standard form: $\dot y+3y/t=0$. Then
$$P(t)=\int -{3\over t}\,dt=-3\ln t$$
and 
$$ y=Ae^{-3\ln t}=At^{-3}.$$
Substituting to find $A$:
$\ds 2=A(1)^{-3}=A$, so the solution is $\ds y=2t^{-3}$.
\end{example}

\begin{exercises}

Find the general solution of each equation in 1--4.

\exercise $\ds\dot y+5y=0$
\begin{answer} $\ds y=Ae^{-5t}$
\end{answer}

\exercise $\ds\dot y-2y=0$
\begin{answer} $\ds y=Ae^{2t}$
\end{answer}

\exercise $\ds\dot y+{y\over 1+t^2}=0$
\begin{answer} $\ds y=Ae^{-\arctan t}$
\end{answer}

\exercise $\ds\dot y+t^2y=0$
\begin{answer} $\ds y=Ae^{-t^3/3}$
\end{answer}

In 5--14, solve the initial value problem.

\exercise $\ds\dot y + y=0$, $y(0)=4$
\begin{answer} $\ds y=4e^{-t}$
\end{answer}

\exercise $\ds\dot y -3y=0$, $y(1)=-2$
\begin{answer} $\ds y=-2e^{3t-3}$
\end{answer}

\exercise $\ds\dot y + y\sin t = 0$, $y(\pi)=1$
\begin{answer} $\ds y=e^{1+\cos t}$
\end{answer}

\exercise $\ds\dot y +ye^t=0$, $y(0)=e$
\begin{answer} $\ds y=e^2e^{-e^t}$
\end{answer}

\exercise $\ds\dot y +y\sqrt{1+t^4}=0$, $y(0)=0$
\begin{answer} $\ds y=0$
\end{answer}

\exercise $\ds\dot y + y\cos(e^t)=0$, $y(0)=0$
\begin{answer} $\ds y=0$
\end{answer}

\exercise $\ds t\dot y - 2y = 0$, $y(1)=4$
\begin{answer} $\ds y=4t^2$
\end{answer}

\exercise $\ds t^2\dot y + y = 0$, $y(1)=-2$, $t>0$
\begin{answer} $\ds y=-2e^{(1/t)-1}$
\end{answer}

\exercise $\ds t^3\dot y = 2y$, $y(1)=1$, $t>0$
\begin{answer} $\ds y=e^{1-t^{-2}}$
\end{answer}

\exercise $\ds t^3\dot y = 2y$, $y(1)=0$, $t>0$
\begin{answer} $\ds y=0$
\end{answer}


\exercise A function $y(t)$ is a solution of $\ds\dot y +
ky=0$. Suppose that $y(0)=100$ and $y(2)=4$. Find $k$ and find $y(t)$.
\begin{answer} $k=\ln 5$, $\ds y=100e^{-t\ln 5}$
\end{answer}

\exercise A function $y(t)$ is a solution of $\ds\dot y +
t^ky=0$. Suppose that $y(0)=1$ and $y(1)=e^{-13}$. Find $k$ and find
$y(t)$. 
\begin{answer} $k=-12/13$, $\ds y=\exp(-13 t^{1/13})$
\end{answer}

\exercise A bacterial culture grows at a rate proportional to its
population. If the population is one million at $t=0$ and 1.5
million at $t=1$ hour, find the population as a function of time.
\begin{answer} $\ds y=10^6e^{t\ln(3/2)}$
\end{answer}

\exercise A radioactive element decays with a half-life of 6 years. If
a mass of the element weighs ten pounds at $t=0$, find the amount of
the element at time $t$.
\begin{answer} $\ds y=10e^{-t\ln(2)/6}$
\end{answer}

\end{exercises}
