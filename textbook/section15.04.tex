\section{Surface Area}{}{}
\label{sec:surface area 3D}

\index{surface area}
We next seek to compute the area of a surface above (or below) a
region in the $x$-$y$ plane. How might we approximate this? We start,
as usual, by dividing the region into a grid of small rectangles. We
want to approximate the area of the surface above one of these small
rectangles. The area is very close to the area of the tangent plane
above the small rectangle. If the tangent plane just happened to be
horizontal, of course the area would simply be the area of the
rectangle. For a typical plane, however, the area is the area of a
parallelogram, as indicated in figure~\xrefn{fig:small parallelogram}.
Note that the area of the parallelogram is obviously larger the more
``tilted'' the tangent plane. In the Java applet you can see that
viewed from above the four parallelograms exactly cover a rectangular
region in the $x$-$y$ plane.

\figure
\vbox{\beginpicture
\normalgraphs
\ninepoint
\setcoordinatesystem units <1.5truecm,1.5truecm>
\setplotarea x from 0 to 2.1, y from -1.1 to 1.1
\put {\hbox{\epsfxsize6cm\epsfbox{small_parallelograms.eps}}} at 0 0
\endpicture}
\figrdef{fig:small parallelogram}
\endfigure{Small parallelograms at points of tangency.
(\expandafter\url\expandafter{\liveurl small_parallelograms.html}%
AP\endurl)}

Now recall a curious fact: the area of a
parallelogram\index{parallelogram!area of} can be computed as the cross
product of two vectors (page~\xrefn{page:parallelogram area}). We
simply need to acquire two vectors, parallel to the sides of the
parallelogram and with lengths to match. But this is easy: in the $x$
direction we use the tangent vector we already know, namely $\langle
1,0,f_x\rangle$ and multiply by $\Delta x$ to shrink it to the right
size: $\langle \Delta x,0,f_x\Delta x\rangle$. In the $y$ direction we
do the same thing and get $\langle 0,\Delta y,f_y\Delta y\rangle$. The
cross product of these vectors is $\langle f_x,f_y,-1\rangle\,\Delta
x\,\Delta y$ with length $\ds\sqrt{f_x^2+f_y^2+1}\,\Delta x\,\Delta
y$, the area of the parallelogram. Now we add these up and take the
limit, to produce the integral
$$\int_{x_0}^{x_1}\int_{y_0}^{y_1} \sqrt{f_x^2+f_y^2+1}\,dy\,dx.$$
As before, the limits need not be constant.

\begin{example} We find the area of the hemisphere $\ds z=\sqrt{1-x^2-y^2}$. We
compute the derivatives
$$f_x={-x\over\sqrt{1-x^2-y^2}} \qquad
f_x={-y\over\sqrt{1-x^2-y^2}},$$
and then the area is
$$\int_{-1}^{1}\int_{-\sqrt{1-x^2}}^{\sqrt{1-x^2}}
\sqrt{ {x^2\over1-x^2-y^2}+{y^2\over1-x^2-y^2}+1}\,dy\,dx.$$
This is a bit on the messy side, but we can use polar coordinates:
$$
\int_{0}^{2\pi}\int_{0}^{1}
\sqrt{ {1\over1-r^2}}\;r\,dr\,d\theta.$$
This integral is improper, since the function is undefined at the
limit $1$. We therefore compute
$$\lim_{a\to1^-}\int_{0}^{a}
\sqrt{ {1\over1-r^2}}\;r\,dr=\lim_{a\to1^-}-\sqrt{1-a^2}+1=1,$$
using the substitution $u=1-r^2$. Then the area is 
$$\int_{0}^{2\pi}1\;d\theta=2\pi.$$
You may
recall that the area of a sphere of radius $r$ is $4\pi r^2$, so half
the area of a unit sphere is $(1/2)4\pi=2\pi$, in agreement with our
answer. 
\end{example}

\begin{exercises}

\begin{exercise} Find the area of the surface of a right circular cone of
height $h$ and base radius $a$.
\begin{answer} $\pi a\sqrt{h^2+a^2}$
\end{answer}\end{exercise}

\begin{exercise} Find the area of the portion of the plane $z=mx$ inside the
cylinder $x^2+y^2=a^2$.
\begin{answer} $\pi a^2\sqrt{m^2+1}$
\end{answer}\end{exercise}

\begin{exercise} Find the area of the portion of the plane $x+y+z=1$ in the
first octant.
\begin{answer} $\sqrt3/2$
\end{answer}\end{exercise}

\begin{exercise} Find the area of the upper half of the cone
$x^2+y^2=z^2$ inside the cylinder $x^2+y^2-2x = 0$.
\begin{answer} $\pi\sqrt2$
\end{answer}\end{exercise}

\begin{exercise} Find the area of the upper half of the cone
$x^2+y^2=z^2$ above the interior of one loop of $r=\cos(2\theta)$.
\begin{answer} $\pi\sqrt2/8$
\end{answer}\end{exercise}

\begin{exercise} Find the area of the upper hemisphere of 
$x^2+y^2+z^2=1$ above the interior of one loop of $r=\cos(2\theta)$.
\begin{answer} $\pi/2-1$
\end{answer}\end{exercise}

\begin{exercise} The plane $ax+by+cz=d$ cuts a triangle in the first octant
provided that $a, b, c$ and $d$ are all positive.  Find the area of
this triangle.
\begin{answer} $\ds {d^2\sqrt{a^2+b^2+c^2}\over 2abc}$
\end{answer}\end{exercise}

% Albert

\begin{exercise} Find the area of the portion of the cone $x^2+y^2 = 3z^2$ lying
above the $xy$ plane and inside the cylinder $x^2+y^2 = 4y$.
\begin{answer} $8\sqrt{3}\pi/3$
\end{answer}\end{exercise}
%/Albert

\end{exercises}

