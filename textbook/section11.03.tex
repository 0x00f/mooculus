\section{The Integral Test}{}{}
\nobreak
It is generally quite difficult, often impossible, to determine
the value of a series exactly. In many cases it is possible at least
to determine whether or not the series converges, and so we will spend
most of our time on this problem.

If all of the terms $\ds a_n$ in a series are non-negative, then clearly 
the sequence of partial sums $\ds s_n$ is non-decreasing. This means that
if we can show that the sequence of partial sums is bounded, the
series must converge. We know that if the series converges, the terms
$\ds a_n$ approach zero, but this does not mean that $\ds a_n\ge a_{n+1}$ for
every $n$. Many useful and interesting series do have this property,
however, and they are among the easiest to understand. Let's look at
an example.

\begin{example} Show that $\ds\sum_{n=1}^\infty {1\over n^2}$ converges.
\par\nobreak\ssk\noindent
The terms $\ds 1/n^2$ are  positive and decreasing, and since 
$\ds\lim_{x\to\infty} 1/x^2=0$, the terms $\ds 1/n^2$ approach zero. We
seek an upper bound for all the partial sums, that is, we want to find
a number $N$ so that $s_n\le N$ for every $n$. The upper bound is
provided courtesy of integration, and is inherent in
figure~\xrefn{fig:integral test for one over n squared}.

\figure
\vbox{\beginpicture
\normalgraphs
\ninepoint
\setcoordinatesystem units <2truecm,2truecm>
\setplotarea x from 0 to 5, y from 0 to 2
\axis left ticks numbered from 0 to 2 by 1 /
\axis bottom  ticks numbered from 0 to 5 by 1 /
\setquadratic
\plot 0.750 1.778 0.892 1.258 1.033 0.937 1.175 0.724 1.317 0.577 
1.458 0.470 1.600 0.391 1.742 0.330 1.883 0.282 2.025 0.244 
2.167 0.213 2.308 0.188 2.450 0.167 2.592 0.149 2.733 0.134 
2.875 0.121 3.017 0.110 3.158 0.100 3.300 0.092 3.442 0.084 
3.583 0.078 3.725 0.072 3.867 0.067 4.008 0.062 4.150 0.058 
4.292 0.054 4.433 0.051 4.575 0.048 4.717 0.045 4.858 0.042 
5.000 0.040 /
\putrule from 1 0 to 1 1 
\putrule from 2 0 to 2 0.25
\putrule from 3 0 to 3 0.1111
\putrule from 0 1 to 1 1 
\putrule from 1 0.25 to 2 0.25
\putrule from 2 0.1111 to 3 0.1111
\sevenpoint
\put {$A=1$} at 0.5 0.5
\put {$A=1/4$} at 1.5 0.125
\endpicture}
\figrdef{fig:integral test for one over n squared}
\endfigure{Graph of $\ds y=1/x^2$ with rectangles.}

The figure shows the graph of $\ds y=1/x^2$ together with some rectangles
that lie completely below the curve and that all have base length
one. Because the heights of the rectangles are determined by the
height of the curve, the areas of the rectangles are $\ds 1/1^2$, $\ds 1/2^2$,
$\ds 1/3^2$, and so on---in other words, exactly the terms of the
series. The partial sum $\ds s_n$ is simply the sum of the areas of the
first $n$ rectangles. Because the rectangles all lie between the curve
and the $x$-axis, any sum of rectangle areas is less than the
corresponding area under the curve, and so of course any sum of
rectangle areas is less than the area under the entire curve, that is,
all the way to infinity.
There is a bit of trouble at the
left end, where there is an asymptote, but we can work around that
easily. Here it is:
$$
  s_n={1\over 1^2}+{1\over 2^2}+{1\over 3^2}+\cdots+{1\over n^2}
  < 1 + \int_1^n {1\over x^2}\,dx < 1+\int_1^\infty {1\over x^2}\,dx 
  =1+1=2,
$$
recalling that we computed this improper integral in 
section~\xrefn{sec:improper integrals}. Since the sequence of partial
sums $\ds s_n$ is increasing and bounded above by 2, we know that 
$\ds\lim_{n\to\infty}s_n=L<2$, and so the series converges to some
number less than 2. In fact, it is possible, though difficult, to show
that $\ds L=\pi^2/6\approx 1.6$.
\end{example}

We already know that $\sum 1/n$ diverges. What goes wrong if we try to
apply this technique to it? Here's the calculation:
$$
  s_n={1\over 1}+{1\over 2}+{1\over 3}+\cdots+{1\over n}
  < 1 + \int_1^n {1\over x}\,dx < 1+\int_1^\infty {1\over x}\,dx 
  =1+\infty.
$$
The problem is that the improper integral doesn't converge. Note well
that this does {\em not\/} prove that $\sum 1/n$ diverges, just that
this particular calculation fails to prove that it converges. A slight
modification, however, allows us to prove in a second way that $\sum
1/n$ diverges. 

\begin{example} Consider a slightly altered version of figure~\xrefn{fig:integral
test for one over n squared}, shown in figure~\xrefn{fig:integral test
for one over n}.
\figure
\vbox{\beginpicture
\normalgraphs
\ninepoint
\setcoordinatesystem units <2truecm,2truecm>
\setplotarea x from 0 to 5, y from 0 to 2
\axis left ticks numbered from 0 to 2 by 1 /
\axis bottom  ticks numbered from 0 to 5 by 1 /
\setquadratic
\plot 0.500 2.000 0.650 1.538 0.800 1.250 0.950 1.053 1.100 0.909 
1.250 0.800 1.400 0.714 1.550 0.645 1.700 0.588 1.850 0.541 
2.000 0.500 2.150 0.465 2.300 0.435 2.450 0.408 2.600 0.385 
2.750 0.364 2.900 0.345 3.050 0.328 3.200 0.312 3.350 0.298 
3.500 0.286 3.650 0.274 3.800 0.263 3.950 0.253 4.100 0.244 
4.250 0.235 4.400 0.227 4.550 0.220 4.700 0.213 4.850 0.206 
5.000 0.200 /
\putrule from 1 0 to 1 1 
\putrule from 2 0 to 2 1
\putrule from 3 0 to 3 0.5
\putrule from 4 0 to 4 0.3333
\putrule from 1 1 to 2 1
\putrule from 2 0.5 to 3 0.5
\putrule from 3 0.3333 to 4 0.3333
\sevenpoint
\put {$A=1$} at 1.5 0.5
\put {$A=1/2$} at 2.5 0.25
\put {$A=1/3$} at 3.5 0.1666
\endpicture}
\figrdef{fig:integral test for one over n}
\endfigure{Graph of $y=1/x$ with rectangles.}

The rectangles this time are above the curve, that is, each rectangle
completely contains the corresponding area under the curve. This means
that 
$$s_n = {1\over 1}+{1\over 2}+{1\over 3}+\cdots+{1\over n}
> \int_1^{n+1} {1\over x}\,dx = \ln x\Big|_1^{n+1}=\ln(n+1).$$
As $n$ gets bigger, $\ln(n+1)$ goes to infinity, so the sequence of
partial sums $\ds s_n$ must also go to infinity, so the harmonic series
diverges. 
\end{example}

The important fact that clinches this example is that
$$\lim_{n\to\infty} \int_1^{n+1} {1\over x}\,dx = \infty,$$
which we can rewrite as
$$\int_1^\infty {1\over x}\,dx = \infty.$$
So these two examples taken together indicate that we can prove that a
series converges or prove that it diverges with a single calculation
of an improper integral. This is known as the {\dfont integral
  test\index{integral test}\index{series!integral test}\/}, 
which we state as a theorem.

\begin{theorem} Suppose that $f(x)>0$ and is decreasing on the infinite interval
$[k,\infty)$ (for some $k\ge1$)
and that $\ds a_n=f(n)$. Then the series
$\ds\sum_{n=1}^\infty a_n$ converges if and only if the improper
integral $\ds\int_{1}^\infty f(x)\,dx$ converges.
\end{proof}

The two examples we have seen are called
$p$-series\index{p@$p$-series}\index{series!$p$-series}; a $p$-series is
any series of the form $\ds \sum 1/n^p$. If $p\le0$, $\ds\lim_{n\to\infty}
1/n^p\not=0$, so the series diverges. For positive values of $p$ we
can determine precisely which series converge.

\begin{theorem} A $p$-series with $p>0$ converges if and only if $p>1$.
\begin{proof}
We use the integral test; we have already done $p=1$, so assume that
$p\not=1$.
$$
  \int_1^{\infty} {1\over x^p}\,dx=\lim_{D\to\infty} \left.{x^{1-p}\over
  1-p}\right|_{1}^D=\lim_{D\to\infty} {D^{1-p}\over 1-p}-{1\over 1-p}.
$$
If $p>1$ then $1-p<0$ and $\ds\lim_{D\to\infty}D^{1-p}=0$, so the
  integral converges. If $0<p<1$ then $1-p>0$ and 
$\ds\lim_{D\to\infty}D^{1-p}=\infty$, so the integral diverges.
\end{proof}

\begin{example} Show that $\ds\sum_{n=1}^\infty {1\over {n^3}}$ converges. 
\par\nobreak\ssk\noindent
We could of course use
the integral test, but now that we have the theorem we may simply note
that this is a $p$-series with $p>1$.
\end{example}

\begin{example} Show that $\ds\sum_{n=1}^\infty {5\over n^4}$ converges. 
\par\nobreak\ssk\noindent
We know that if
$\ds \sum_{n=1}^\infty 1/n^4$ converges then $\ds \sum_{n=1}^\infty 5/n^4$
also converges, by theorem~\xrefn{thm:series are linear}. Since 
$\ds \sum_{n=1}^\infty 1/n^4$ is a convergent $p$-series, 
 $\ds \sum_{n=1}^\infty 5/n^4$ converges also.
\end{example}

\begin{example} Show that $\ds\sum_{n=1}^\infty {5\over \sqrt{n}}$ diverges.
\par\nobreak\ssk\noindent This also follows from
theorem~\xrefn{thm:series are linear}: Since $\ds\sum_{n=1}^\infty
{1\over \sqrt{n}}$ is a $p$-series with $p=1/2<1$, it diverges, and so
does $\ds\sum_{n=1}^\infty {5\over \sqrt{n}}$.  
\end{example}

Since it is typically difficult to compute the value of a series
exactly, a good approximation is frequently required. In a real sense,
a good approximation is only as good as we know it is, that is, while
an approximation may in fact be good, it is only valuable in practice
if we can guarantee its accuracy to some degree. This guarantee is
usually easy to come by for series with decreasing positive terms.

\begin{example} Approximate $\ds \sum 1/n^2$ to two decimal places.

Referring to figure~\xrefn{fig:integral test for one over n squared},
if we approximate the sum by $\ds \sum_{n=1}^N 1/n^2$, the error we make is the
total area of the remaining rectangles, all of which lie under the
curve $\ds 1/x^2$ from $x=N$ out to infinity. So we know the true value of
the series is larger than the approximation, and no bigger than the
approximation plus the area under the curve from $N$ to
infinity. Roughly, then, we need to find $N$ so that 
$$\int_N^\infty {1\over x^2}\,dx < 1/100.$$
We can compute the integral:
$$\int_N^\infty {1\over x^2}\,dx = {1\over N},$$ 
so $N=100$ is a good starting point.  Adding up the first 100 terms
gives approximately $1.634983900$, and that plus $1/100$ is
$1.644983900$, so approximating the series by the value halfway
between these will be at most $1/200=0.005$ in error.  The midpoint is
$1.639983900$, but while this is correct to $\pm0.005$, we can't tell
if the correct two-decimal approximation is $1.63$ or $1.64$. We need
to make $N$ big enough to reduce the guaranteed error, perhaps to
around $0.004$ to be safe, so we would need $1/N\approx 0.008$, or
$N=125$. Now the sum of the first 125 terms is approximately
$1.636965982$, and that plus $0.008$ is $1.644965982$ and the point
halfway between them is $1.640965982$. The true value is then
$1.640965982\pm 0.004$, and all numbers in this range round to $1.64$,
so $1.64$ is correct to two decimal places. We have mentioned that
the true value of this series can be shown to be $\ds
\pi^2/6\approx1.644934068$ which rounds down to $1.64$ (just barely)
and is indeed below the upper bound of $1.644965982$, again just
barely. Frequently approximations will be even better than the
``guaranteed'' accuracy, but not always, as this example demonstrates.
\end{example}

\begin{exercises}

Determine whether each series converges or diverges.

\twocol

\exercise $\ds\sum_{n=1}^\infty {1\over n^{\pi/4}}$
\begin{answer} diverges
\end{answer}

\exercise $\ds\sum_{n=1}^\infty {n\over n^2+1}$
\begin{answer} diverges
\end{answer}

\exercise $\ds\sum_{n=1}^\infty {\ln n\over n^2}$
\begin{answer} converges
\end{answer}

\exercise $\ds\sum_{n=1}^\infty {1\over n^2+1}$
\begin{answer} converges
\end{answer}

\exercise $\ds\sum_{n=1}^\infty {1\over e^n}$
\begin{answer} converges
\end{answer}

\exercise $\ds\sum_{n=1}^\infty {n\over e^n}$
\begin{answer} converges
\end{answer}

\exercise $\ds\sum_{n=2}^\infty {1\over n\ln n}$
\begin{answer} diverges
\end{answer}

\exercise $\ds\sum_{n=2}^\infty {1\over n(\ln n)^2}$
\begin{answer} converges
\end{answer}

\endtwocol

\msk
\exercise Find an $N$ so that
$\ds\sum_{n=1}^\infty {1\over n^4}$ is between
$\ds\sum_{n=1}^N {1\over n^4}$ and
$\ds\sum_{n=1}^N {1\over n^4} + 0.005$.
\begin{answer} $N=5$
\end{answer}

\exercise Find an $N$ so that
$\ds\sum_{n=0}^\infty {1\over e^n}$ is between
$\ds\sum_{n=0}^N {1\over e^n}$ and
$\ds\sum_{n=0}^N {1\over e^n} + 10^{-4}$.
\begin{answer} $N=10$
\end{answer}

\exercise Find an $N$ so that
$\ds\sum_{n=1}^\infty {\ln n\over n^2}$ is between
$\ds\sum_{n=1}^N {\ln n\over n^2}$ and
$\ds\sum_{n=1}^N {\ln n\over n^2} + 0.005$.
\begin{answer} $N=1687$
\end{answer}

\exercise Find an $N$ so that
$\ds\sum_{n=2}^\infty {1\over n(\ln n)^2}$ is between
$\ds\sum_{n=2}^N {1\over n(\ln n)^2}$ and
$\ds\sum_{n=2}^N {1\over n(\ln n)^2} + 0.005$.
\begin{answer} any integer greater than $\ds e^{200}$
\end{answer}

\end{exercises}

