\section{The slope of a function}{}{}
\label{sec:slope of a function}

Suppose that $y$ is a function of $x$, say $y = f(x)$.
It is often necessary to know how sensitive the value of $y$
is to small changes in $x$.

% Example 1
\begin{example}
Take, for example, $\ds y=f(x)=\sqrt{625-x^2}$ (the upper semicircle of radius
25 centered at the origin).  When $x=7$, we find that $\ds y=\sqrt{625-49}=24$.
Suppose we want to know how much $y$ changes when $x$ increases a little,
say to 7.1 or 7.01.

In the case of a straight line $y=mx+b$, the slope $m=\Delta y/\Delta
x$ measures the change in $y$ per unit change in $x$. This can be
interpreted as a measure of ``sensitivity''; for example, if
$y=100x+5$, a small change in $x$ corresponds to a change one hundred
times as large in $y$, so $y$ is quite sensitive to changes in $x$.

Let us look at the
same ratio $\Delta y/\Delta x$ for our function
$\ds y=f(x)=\sqrt{625-x^2}$ when $x$ changes from 7 to $7.1$.  Here $\Delta
x=7.1-7=0.1$ is the change in $x$, and
\begin{align*}
 \Delta y =f(x+\Delta x)-f(x)&=f(7.1)-f(7) \\
          &=\sqrt{625-7.1^2}-\sqrt{625-7^2}\approx 23.9706-24=-0.0294. \\
\end{align*}
Thus, $\Delta y/\Delta x\approx -0.0294/0.1=-0.294$. This means that $y$
changes by less than one third the change in $x$, so apparently $y$ is
not very sensitive to changes in $x$ at $x=7$. We say ``apparently''
here because we don't really know what happens between 7 and
$7.1$. Perhaps $y$ changes dramatically as $x$ runs through the values
from 7 to $7.1$, but at $7.1$ $y$ just happens to be close to its
value at $7$. This is not in fact the case for this particular
function, but we don't yet know why.
\end{example}

One way to interpret the above calculation is by reference to a line.
We have computed the slope of the line through $(7,24)$ and
$(7.1,23.9706)$, called a {\dfont chord\index{chord}} of the circle.
In general, if we draw the chord from the point $(7,24)$ to a nearby
point on the semicircle $(7+\Delta x,\,f(7+\Delta x))$, the slope of this
chord is the so-called {\dfont difference quotient\index{difference quotient}}
$$
\hbox{slope~of~chord}={f(7+\Delta x)-f(7)\over \Delta x}=
{\sqrt{625-(7+\Delta x)^2}-24\over \Delta x}.
$$ For example, if $x$ changes only from 7 to 7.01, then the
difference quotient (slope of the chord) is approximately equal to
$(23.997081-24)/0.01=-0.2919$.  This is slightly less steep than the
chord from $(7,24)$ to $(7.1,23.9706)$.

As the second value $7+\Delta x$ moves in towards 7, the chord joining
$(7,f(7))$ to $(7+\Delta x,f(7+\Delta x))$ shifts slightly.  As
indicated in figure~\xrefn{fig:chords}, as $\Delta x$ gets smaller and
smaller, the chord joining $(7,24)$ to $(7+\Delta x,f(7+\Delta x))$
gets closer and closer to the {\dfont tangent line\index{tangent
    line}} to the circle at the point $(7,24)$.  (Recall that the
tangent line is the line that just grazes the circle at that point,
i.e., it doesn't meet the circle at any second point.)  Thus, as
$\Delta x$ gets smaller and smaller, the slope $\Delta y/\Delta x$ of
the chord gets closer and closer to the slope of the tangent line.
This is actually quite difficult to see when $\Delta x$ is small,
because of the scale of the graph. The values of $\Delta x$ used for
the figure are $1$, $5$, $10$ and $15$, not really very small values.
The tangent line is the one that is uppermost at the right hand
endpoint.

% \figure
% \vbox{\beginpicture
% \normalgraphs
% \ninepoint
% \setcoordinatesystem units <2truemm,2truemm>
% \setplotarea x from 0 to 26, y from 0 to 26
% \circulararc 90 degrees from 25 0 center at 0 0
% \axis left ticks numbered from 5 to 25 by 5 /
% \axis bottom ticks numbered from 5 to 25 by 5 /
% \plot 0 26.04 25 18.75 /
% \plot 0 26.2 25 18.3379 /
% \plot 0 26.896 25 16.554 /
% \plot 0 27.97 25 13.8 /
% \plot 0 30.2 25 8.02 /
% \endpicture}
% \figrdef{fig:chords}
% \endfigure{Chords approximating the tangent line.
% (\expandafter\url\expandafter{\liveurl jsxgraph/secant_lines.html}%
% AP\endurl)}

So far we have found the slopes of two chords that should be close to
the slope of the tangent line, but what is the slope of the tangent
line exactly? Since the tangent line touches the circle at just one
point, we will never be able to calculate its slope directly, using
two ``known'' points on the line. What we need is a way to capture
what happens to the slopes of the chords as they get ``closer and
closer'' to the tangent line.

Instead of looking at more particular values of $\Delta x$, let's see
what happens if we do some algebra with the difference quotient using
just $\Delta x$. The slope of a chord from $(7,24)$ to a nearby point
is given by
\begin{align*}
{\sqrt{625-(7+\Delta x)^2} - 24\over \Delta x}&=
{\sqrt{625-(7+\Delta x)^2} - 24\over \Delta x}{\sqrt{625-(7+\Delta
    x)^2}+24\over \sqrt{625-(7+\Delta x)^2}+24} \\
&={625-(7+\Delta x)^2-24^2\over \Delta x(\sqrt{625-(7+\Delta x)^2}+24)} \\
&={49-49-14\Delta x-\Delta x^2\over  \Delta x(\sqrt{625-(7+\Delta
    x)^2}+24)} \\
&={\Delta x(-14-\Delta x)\over \Delta x(\sqrt{625-(7+\Delta
    x)^2}+24)} \\
&= {-14-\Delta x\over\sqrt{625-(7+\Delta
    x)^2}+24} \\
\end{align*}
Now, can we tell by looking at this last formula what happens when
$\Delta x$ gets very close to zero? The numerator clearly gets very
close to $-14$ while the denominator gets very close to
$\ds \sqrt{625-7^2}+24=48$. Is the fraction therefore very close to 
$-14/48 = -7/24 \cong -0.29167$? It certainly seems reasonable, and in
fact it is true: as $\Delta x$ gets closer and closer to zero, the
difference quotient does in fact get closer and closer to $-7/24$, and
so the slope of the tangent line is exactly $-7/24$.

What about the slope of the tangent line at $x=12$? Well, 12 can't be
all that different from 7; we just have to redo the calculation with
12 instead of 7. This won't be hard, but it will be a bit
tedious. What if we try to do all the algebra without using a specific
value for $x$? Let's copy from above, replacing 7 by  $x$. We'll have
to do a bit more than that---for example, 
the ``24'' in the calculation came from 
$\ds \sqrt{625-7^2}$, so we'll need to fix that too.
\begin{align*}
&{\sqrt{625-(x+\Delta x)^2} - \sqrt{625-x^2}\over \Delta x}= \\
\qquad&={\sqrt{625-(x+\Delta x)^2} - \sqrt{625-x^2}\over \Delta x}{\sqrt{625-(x+\Delta
    x)^2}+\sqrt{625-x^2}\over \sqrt{625-(x+\Delta x)^2}+\sqrt{625-x^2}} \\
&={625-(x+\Delta x)^2-625+x^2\over \Delta x(\sqrt{625-(x+\Delta x)^2}+\sqrt{625-x^2})} \\
&={625-x^2-2x\Delta x-\Delta x^2-625+x^2\over  \Delta x(\sqrt{625-(x+\Delta
    x)^2}+\sqrt{625-x^2})} \\
&={\Delta x(-2x-\Delta x)\over \Delta x(\sqrt{625-(x+\Delta
    x)^2}+\sqrt{625-x^2})} \\
&= {-2x-\Delta x\over\sqrt{625-(x+\Delta
    x)^2}+\sqrt{625-x^2}} \\
\end{align*}
Now what happens when $\Delta x$ is very close to zero? Again it seems
apparent that the quotient will be very close to
$${-2x\over \sqrt{625-x^2}+\sqrt{625-x^2}}
={-2x\over 2\sqrt{625-x^2}}={-x\over \sqrt{625-x^2}}.
$$
Replacing $x$ by 7 gives $-7/24$, as before, and now we can easily do
the computation for 12  or any other value of
$x$ between $-25$ and 25.

So now we have a single, simple formula, $\ds {-x/ \sqrt{625-x^2}}$,
that tells us the slope of the tangent line for any value of
$x$. This slope, in turn, tells us how sensitive the value of $y$ is
to changes in the value of $x$. 

What do we call such a formula? That is, a formula with one variable,
so that substituting an ``input'' value for the variable produces a
new ``output'' value? This is a function. Starting with one function,
$\ds \sqrt{625-x^2}$, we have derived, by means of some slightly nasty
algebra, a new function, $\ds {-x/ \sqrt{625-x^2}}$, that gives us
important information about the original function. This new function
in fact is called the {\dfont derivative\index{derivative}} of the
original function. If the original is referred to as $f$ or $y$ then
the derivative is often written $f'$ or $y'$ and pronounced ``f
prime'' or ``y prime'', so in this case we might write $\ds f'(x)=-x/
\sqrt{625-x^2}$. At a particular point, say $x=7$, we say that
$f'(7)=-7/24$ or ``$f$ prime of 7 is $-7/24$'' or ``the derivative of
$f$ at 7 is $-7/24$.''

To summarize, we compute the derivative of $f(x)$ by forming the
difference quotient
$$
{f(x+\Delta x)-f(x)\over \Delta x},
$$
which is the slope of a line, then we figure out what happens when
$\Delta x$ gets very close to 0. 


We should note that in 
the particular case of a circle, there's a simple way to find the
derivative.  Since the tangent to a circle at a point is perpendicular to
the radius drawn to the point of contact, its slope is the negative
reciprocal of the slope of the radius.  The radius joining $(0,0)$ to
$(7,24)$ has slope 24/7.  Hence, the tangent line has slope
$-7/24$. In general, a radius to the point $\ds (x,\sqrt{625-x^2})$ has
slope $\ds \sqrt{625-x^2}/x$, so the slope of the tangent line is
$\ds {-x/ \sqrt{625-x^2}}$, as before. It is {\bf NOT} always true that a
tangent line is perpendicular to a line from the origin---don't use
this shortcut in any other circumstance. 

As above, and as you might expect, for different values of $x$ we
generally get different values of the derivative $f'(x)$. Could it be
that the derivative always has the same value? This would mean that
the slope of $f$, or the slope of its tangent line, is the same
everywhere. One curve that always has the same slope is a line; it
seems odd to talk about the tangent line to a line, but if it makes
sense at all the tangent line must be the line itself. It is not hard
to see that the derivative of $f(x)=mx+b$ is $f'(x)=m$; see
exercise~\xrefn{ex:derivative of a line} .

\begin{exercises}

\exercise
Draw the graph of the function $\ds y=f(x)=\sqrt{169-x^2}$ between $x=0$
and $x=13$.  Find the slope $\Delta y/\Delta x$ of the chord between the
points of the circle lying over (a) $x=12$ and $x=13$, (b) $x=12$ and
$x=12.1$,  (c) $x=12$ and $x=12.01$, (d) $x=12$ and $x=12.001$.  Now use
the geometry of tangent lines on a circle to find (e) the exact value of the
derivative $f'(12)$.  Your answers to (a)--(d) should be getting closer and
closer to your answer to (e).
\begin{answer} $-5$, $-2.47106145$, $-2.4067927$, $-2.400676$, $-2.4$
\end{answer}

\exercise
Use geometry to find the derivative $f'(x)$ of the function
$\ds f(x)=\sqrt{625-x^2}$ in the text for each of the following $x$: (a) 20,
(b) 24, (c) $-7$, (d) $-15$.  Draw a graph of the upper semicircle, and
draw the tangent line at each of these four points.
\begin{answer} $-4/3$, $-24/7$, $7/24$, $3/4$
\end{answer}


\exercise
Draw the graph of the function $y=f(x)=1/x$ between $x=1/2$ and $x=4$.
Find the slope of the chord between (a) $x=3$ and $x=3.1$, (b) $x=3$ and
$x=3.01$, (c) $x=3$ and $x=3.001$.  Now use algebra to find a simple
formula for the slope of the chord between $(3,f(3))$ and $(3+\Delta
x,f(3+\Delta x))$.  Determine what happens when $\Delta x$ approaches 0.
In your graph of $y=1/x$, draw the straight line through the point
$(3,1/3)$ whose slope is this limiting value of the difference quotient as
$\Delta x$ approaches 0.
\begin{answer} $-0.107526881$, $-0.11074197$, $-0.1110741$, 
$\ds{-1\over3(3+\Delta x)}\rightarrow {-1\over9}$
\end{answer}

\exercise
Find an algebraic expression for the difference quotient $\ds \bigl(f(1+\Delta
x)-f(1)\bigr)/\Delta x$ when $\ds f(x)=x^2-(1/x)$.  Simplify the expression as
much as possible.  Then determine what happens as $\Delta x$ approaches 0.
That value is $f'(1)$.
\begin{answer} $\ds{3+3\Delta x+\Delta x^2\over1+\Delta x}\rightarrow3$ 
\end{answer}

\exercise
Draw the graph of $\ds y=f(x)=x^3$ between $x=0$ and $x=1.5$.  Find the slope
of the chord between (a) $x=1$ and $x=1.1$, (b) $x=1$ and $x=1.001$, (c)
$x=1$ and $x=1.00001$.  Then use algebra to find a simple formula for the
slope of the chord between $1$ and $1+\Delta x$.  (Use the expansion
$\ds (A+B)^3=A^3+3A^2B+3AB^2+B^3$.)  Determine what happens as $\Delta x$
approaches 0, and in your graph of $\ds y=x^3$ draw the straight line through
the point $(1,1)$ whose slope is equal to the value you just found.
\begin{answer} $3.31$, $3.003001$, $3.0000$,\hfill\break
 $3+3\Delta x+\Delta x^2\rightarrow3$
\end{answer}

\exercise
Find an algebraic expression for the difference quotient $(f(x+\Delta
x)-f(x))/\Delta x$ when $f(x)=mx+b$.  Simplify the expression as
much as possible.  Then determine what happens as $\Delta x$ approaches 0.
That value is $f'(x)$.
\begin{answer} $m$
\end{answer}
\label{ex:derivative of a line}

\exercise Sketch the unit circle.  Discuss the behavior of the slope
of the tangent line at various angles around the circle.  Which
trigonometric function gives the slope of the tangent line at an angle
$\theta$?  Why? Hint: think in terms of ratios of sides of
  triangles.

 \exercise Sketch the parabola $\ds y=x^2$.  For what values of $x$ on the parabola
 is the slope of the tangent line positive?  Negative?  What do you notice
 about the graph at the point(s) where the sign of the slope changes from
 positive to negative and vice versa?

\end{exercises}
