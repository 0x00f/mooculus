\section{Higher order derivatives}{}{}

In single variable calculus we saw that the second derivative is often
useful: in appropriate circumstances it measures acceleration; it can
be used to identify maximum and minimum points; it tells us something
about how sharply curved a graph is. Not surprisingly, second
derivatives are also useful in the multi-variable case, but again not
surprisingly, things are a bit more complicated.

It's easy to see where some complication is going to come from: with
two variables there are four possible second derivatives. To take a
``derivative,'' we must take a partial derivative with respect to $x$
or $y$, and there are four ways to do it: $x$ then $x$, $x$ then $y$, 
$y$ then $x$, $y$ then $y$.

\exam Compute all four second derivatives of $f(x,y)=x^2y^2$.

Using an obvious notation, we get:
$$f_{xx}=2y^2\qquad f_{xy}=4xy\qquad f_{yx}=4xy\qquad f_{yy}=2x^2.$$
\endexam

You will have noticed that two of these are the same, the ``mixed
partials'' computed by taking partial derivatives with respect to both
variables in the two possible orders. This is not an accident---as
long as the function is reasonably nice, this will always be true.

\begin{theorem} (Clairaut's Theorem) If the mixed partial derivatives are
continuous, they are equal.
\end{theorem}
\label{thm:clairaut}
\index{Clairaut's Theorem}

\exam Compute the mixed partials of $\ds f=xy/(x^2+y^2)$.
$$
f_x={y^3-x^2y\over(x^2+y^2)^2}\qquad
f_{xy}=-{x^4-6x^2y^2+y^4\over (x^2+y^2)^3}
$$
We leave $f_{yx}$ as an exercise.
\endexam

\begin{exercises}

\exercise Let $\ds f=xy/(x^2+y^2)$; compute $f_{xx}$, $f_{yx}$, and $f_{yy}$.
\begin{answer} $f_{xx}=(2x^3y-6xy^3)/(x^2+y^2)^3$,
$f_{yy}=(2xy^3-6x^3y)/(x^2+y^2)^3$
\end{answer}

\exercise Find all first and second partial derivatives of
$x^3y^2+y^5$.
\begin{answer} $f_x=3x^2y^2$, $f_y=2x^3y+5y^4$, 
$f_{xx}=6xy^2$, $f_{yy}=2x^3+20y^3$, $f_{xy}=6x^2y$
\end{answer}

\exercise Find all first and second partial derivatives of
$4x^3+xy^2+10$.
\begin{answer} $f_x=12x^2+y^2$, $f_y=2xy$, \hfill\break 
$f_{xx}=24x$, $f_{yy}=2x$, $f_{xy}=2y$
\end{answer}

\exercise Find all first and second partial derivatives of
$x\sin y$.
\begin{answer} $f_x=\sin y$, $f_y=x\cos y$, $f_{xx}=0$, $f_{yy}=-x\sin y$,
$f_{xy}=\cos y$
\end{answer}

\exercise Find all first and second partial derivatives of
$\sin(3x)\cos(2y)$.
\begin{answer} $\ds f_x=3\cos(3x)\cos(2y)$,\hfill\break 
$\ds f_y=-2\sin(3x)\sin(2y)$,\hfill\break 
$\ds f_{xy}=-6\cos(3x)\sin(2y)$,\hfill\break 
$\ds f_{yy}=-4\sin(3x)\cos(2y)$,\hfill\break 
$\ds f_{xx}=-9\sin(3x)\cos(2y)$
\end{answer}

\exercise Find all first and second partial derivatives of
$e^{x+y^2}$.
\begin{answer} $\ds f_x=e^{x+y^2}$, $\ds f_y=2ye^{x+y^2}$,\hfill\break 
$\ds f_{xx}=e^{x+y^2}$,\hfill\break 
$\ds f_{yy}=4y^2e^{x+y^2}+2e^{x+y^2}$,\hfill\break 
$\ds f_{xy}=2ye^{x+y^2}$
\end{answer}

\exercise Find all first and second partial derivatives of
$\ln\sqrt{x^3+y^4}$.
\begin{answer} $\ds f_x={3x^2\over2(x^3+y^4)}$, 
$\ds f_y={2y^3\over x^3+y^4}$,
$\ds f_{xx}={3x\over x^3+y^4}-{9x^4\over 2(x^3+y^4)^2}$, 
$\ds f_{yy}={6y^2\over x^3+y^4}-{8y^6\over (x^3+y^4)^2}$,\hfill\break 
$\ds f_{xy}={-6x^2y^3\over (x^3+y^4)^2}$
\end{answer}

\exercise Find all first and second partial derivatives of
$z$ with respect to $x$ and $y$ if 
$x^2+4y^2+16z^2-64=0$.
\begin{answer} $\ds z_x={-x\over16z}$, 
$\ds z_y={-y\over4z}$,\hfill\break 
$\ds z_{xx}=-{16z^2+x^2\over16^2z^3}$,\hfill\break 
$\ds z_{yy}=-{4z^2+y^2\over16z^3}$,\hfill\break 
$\ds z_{xy}={-xy\over64z^3}$
\end{answer}

\exercise Find all first and second partial derivatives of
$z$ with respect to $x$ and $y$ if 
$xy+yz+xz=1$.
\begin{answer} $\ds z_x=-{y+z\over x+y}$, 
$\ds z_y=-{x+z\over x+y}$,\hfill\break 
$\ds z_{xx}=2{y+z\over(x+y)^2}$, 
$\ds z_{yy}=2{x+z\over(x+y)^2}$,\hfill\break 
$\ds z_{xy}={2z\over(x+y)^2}$
\end{answer}

\exercise Let $\alpha$ and $k$ be constants.  Prove that the function 
$u(x,t)=e^{-\alpha^2k^2t}\sin(kx)$
is a solution to the heat equation $u_t=\alpha^2u_{xx}$

\exercise Let $a$ be a constant.  Prove that $u=\sin(x-at)+\ln(x+at)$ is
  a solution to the wave equation $u_{tt}=a^2u_{xx}$.

%% \exercise Let $f(x,y)$ be a continuous differentiable function.  Analyze
%%   the level curves near a critical value if that critical value is a
%%   max or a min.  What if the level curve is a saddle point?

\exercise How many third-order derivatives does a function of 2 variables
  have?  How many of these are distinct?

\exercise How many $n$th order derivatives does a function of 2 variables
  have?  How many of these are distinct?

\end{exercises}
