\chapter{Limits}

\section{Functions}{}{}

\fbox{This should be one of the last sections filled-in.}


\section{The Basic Ideas of Limits}{}{}


\begin{marginfigure}[-5in]
\begin{tikzpicture}
	\begin{axis}[
            domain=-2:4,
            axis lines =middle, xlabel=$x$, ylabel=$f(x)$,
            every axis y label/.style={at=(current axis.above origin),anchor=south},
            every axis x label/.style={at=(current axis.right of origin),anchor=west},
            grid=both,
            grid style={dashed, gray!30},
            xtick={-2,...,4},
            ytick={-3,...,3},
          ]
	  \addplot [very thick, pencolor, smooth] {x-1};
          \addplot[color=pencolor,fill=background,only marks,mark=*] coordinates{(2,1)};  %% open hole
        \end{axis}
\end{tikzpicture}
\caption{A plot of $f(x)=\protect\frac{x^2 - 3x + 2}{x-2}$.}
\label{plot:(x^2 - 3x + 2)/(x-2)}
\end{marginfigure}

\begin{margintable}[-1in]
\[
\begin{array}{c|c}
 x & f(x) \\ \hline
 1.7 &  0.7 \\
 1.9 &  0.9 \\
 1.99 &  0.99 \\
 1.999 &  0.999 \\
  2 &  \text{undefined}
\end{array}\qquad
\begin{array}{c|c}
 x & f(x) \\ \hline
  2 & \text{undefined}\\
 2.001&  1.001\\
 2.01&  1.01\\
 2.1 &  1.1 \\
 2.3 &  1.3 \\
\end{array}
\]
\caption{Values of $f(x)=\protect\frac{x^2 - 3x + 2}{x-2}$.}
\label{table:(x^2 - 3x + 2)/(x-2)}
\end{margintable}
Consider the function:
\[
f(x) = \frac{x^2 - 3x + 2}{x-2}
\]
While $f(x)$ is undefined at $x=2$, we can still plot $f(x)$ at other
values, see Figure~\ref{plot:(x^2 - 3x + 2)/(x-2)}. Examining
Table~\ref{table:(x^2 - 3x + 2)/(x-2)}, we see that as $x$ approaches
$2$, $f(x)$ approaches $1$. We write this: As $x \to 2$, $f(x) \to 1$.
This leads us to the definition of a \textit{limit}.



\begin{definition}\label{def:limit} 
The \textbf{limit} of $f(x)$ as $x$ goes to $a$ is $L$,
\[
\lim_{x\to a}f(x)=L,
\] 
if for every $\epsilon>0$ there is a $\delta > 0$ so that whenever
$x\ne a$ and
\[
a- \delta < x < a+ \delta, \qquad\text{we have} \qquad L-\epsilon< f(x)<L+\epsilon.
\] 
If no such value of $L$ can be
found, then we say that $f(x)$ \textbf{does not exist} at $x=a$.
\end{definition}
\marginnote[-1in]{
Equivalently, $\ds \lim_{x\to a}f(x)=L$,
if for every $\epsilon>0$ there is a $\delta > 0$ so that whenever 
$0< |x-a |< \delta$ we have $|f(x)-L|<\epsilon$.}



In Figure~\ref{figure:epsilon-delta}, we see a geometric
interpretation of this definition.

\begin{figure}
\begin{tikzpicture}
	\begin{axis}[
            domain=0:2, 
            axis lines =left, xlabel=$x$, ylabel=$f(x)$,
            every axis y label/.style={at=(current axis.above origin),anchor=south},
            every axis x label/.style={at=(current axis.right of origin),anchor=west},
            xtick={0.7,1,1.3}, ytick={3,4,5},
            xticklabels={$a-\delta$,$a$,$a+\delta$}, yticklabels={$L-\epsilon$,$L$,$L+\epsilon$},
            axis on top,
          ]          
          \addplot [color=textcolor, fill=fill2, smooth, domain=(0:1.570)] {5} \closedcycle;
          \addplot [color=textcolor, dashed, fill=fill1, smooth, domain=(0:1.3)] {4.537} \closedcycle;
          \addplot [color=textcolor, dashed, fill=fill2, domain=(0:.7)] {3.283} \closedcycle;       
          \addplot [textcolor, very thick, smooth, domain=(0:1)] {4};
          \addplot [color=textcolor, fill=background, smooth, domain=(0:0.607)] {3} \closedcycle;
	  \addplot [draw=none, fill=background, smooth] {x*(x-2)^2+3*x} \closedcycle;
          \addplot [fill=fill1, draw=none, domain=.7:1.3] {x*(x-2)^2+3*x} \closedcycle;
          \addplot [textcolor, very thick] plot coordinates {(1,0) (1,4)};
          \addplot [textcolor] plot coordinates {(.7,0) (.7,3.283)};
          \addplot [textcolor] plot coordinates {(1.3,0) (1.3,4.537)};
	  \addplot [very thick,pencolor, smooth] {x*(x-2)^2+3*x};
        \end{axis}
\end{tikzpicture}
\caption{A geometric interpretation of the
  $(\epsilon,\delta)$-criterion for limits.  Here we see that for
  every $x$ such that $a -\delta < x < a+\delta$, we are sure to have
  $L - \epsilon< f(x) < L+\epsilon$.}
\label{figure:epsilon-delta}
\end{figure}


From our plot and table above, it seems that 
\[
\lim_{x\to 2}\frac{x^2 - 3x + 2}{x-2} = 1.
\]
Let's see if we can find more evidence for this claim.

\begin{example}
Let $f(x) = \dfrac{x^2 - 3x + 2}{x-2}$. Suppose that 
\[
\lim_{x\to 2} f(x) = 1.
\]
Find $\delta$ such that $2 -\delta < x < 2+\delta$ implies that $1 -
\epsilon< f(x) < 1+\epsilon$ where $\epsilon = .1$.

\begin{solution}
This requires that
\[
1 - .1< \frac{x^2 - 3x + 2}{x-2}  < 1+.1.
\]
Since we are taking the limit as $x$ goes to $2$, we can safely
assume that $x\ne 2$.  Hence we may divide our fraction above giving
us
\[
.9 < x-1 < 1.1.
\]
Adding $1$ to each inequality, we find
\[
1.9 < x < 2.1.
\]
Comparing this to $2 -\delta < x < 2+\delta$, we see that setting
$\delta = .1$ will suffice.
\end{solution}
\end{example}\marginnote[-2in]{Limits allow us to examine functions near points where they are not defined, in this example $x=2$. Thus we may assume that $x\ne 2$.} 


Limits need not exist, let's examine two cases of this.

\begin{marginfigure}[0in]
\begin{tikzpicture}
	\begin{axis}[
            domain=-2:4,
            samples=100,
            axis lines =middle, xlabel=$x$, ylabel=$f(x)$,
            every axis y label/.style={at=(current axis.above origin),anchor=south},
            every axis x label/.style={at=(current axis.right of origin),anchor=west},
            clip=false,
          ]
          \addplot [draw=none, fill=fill1, smooth, domain=(1.8:2)] {1} \closedcycle;
          \addplot [draw=none, fill=fill2, smooth, domain=(2:2.2)] {2} \closedcycle;
          \addplot [textcolor, very thick] plot coordinates {(2,0) (2,2)};
          \addplot [textcolor] plot coordinates {(1.8,0) (1.8,1)};
          \addplot [textcolor] plot coordinates {(2.2,0) (2.2,2)};
          \addplot [textcolor, very thin, domain=(1.7:2.3)] {0}; % puts the axis back
	  \addplot [very thick, pencolor, smooth, domain=(-2:-1)] {-2};
          \addplot [very thick, pencolor, smooth, domain=(-1:0)] {-1};
          \addplot [very thick, pencolor, smooth, domain=(0:1)] {0};
          \addplot [very thick, pencolor, smooth, domain=(1:2)] {1};
          \addplot [very thick, pencolor, smooth, domain=(2:3)] {2};
          \addplot [very thick, pencolor, smooth, domain=(3:4)] {3};
          \addplot[color=pencolor,fill=pencolor,only marks,mark=*] coordinates{(-2,-2)};  %% closed hole          
          \addplot[color=pencolor,fill=pencolor,only marks,mark=*] coordinates{(-1,-1)};  %% closed hole          
          \addplot[color=pencolor,fill=pencolor,only marks,mark=*] coordinates{(0,0)};  %% closed hole          
          \addplot[color=pencolor,fill=pencolor,only marks,mark=*] coordinates{(1,1)};  %% closed hole          
          \addplot[color=pencolor,fill=pencolor,only marks,mark=*] coordinates{(2,2)};  %% closed hole  
          \addplot[color=pencolor,fill=pencolor,only marks,mark=*] coordinates{(3,3)};  %% closed hole                  
          \addplot[color=pencolor,fill=background,only marks,mark=*] coordinates{(-1,-2)};  %% open hole
          \addplot[color=pencolor,fill=background,only marks,mark=*] coordinates{(0,-1)};  %% open hole
          \addplot[color=pencolor,fill=background,only marks,mark=*] coordinates{(1,0)};  %% open hole
          \addplot[color=pencolor,fill=background,only marks,mark=*] coordinates{(2,1)};  %% open hole
          \addplot[color=pencolor,fill=background,only marks,mark=*] coordinates{(3,2)};  %% open hole
          \addplot[color=pencolor,fill=background,only marks,mark=*] coordinates{(4,3)};  %% open hole
        \end{axis}
\end{tikzpicture}
\caption{A plot of $f(x)=\lfloor x\rfloor$.}
\label{plot:greatist-integer}
\end{marginfigure}

\begin{example}
Let $f(x) = \lfloor x\rfloor$. Explain why the limit
\[
\lim_{x\to 2} f(x)
\]
does not exist.

\begin{solution}
This is the function that returns the greatest integer less than or
equal to $x$. Since $f(x)$ is defined for all real numbers, one might
be tempted to think that the limit above is simply $f(2) =
2$. However, this is not the case.  If $x<2$, then $f(x) =1$. Hence if
$\epsilon = .5$, we can \textbf{always} find a value for $x$ such that
\[
2- \delta < x < 2+ \delta, \qquad\text{where} \qquad f(x)< 2-\epsilon.
\]
On the other hand, $\lim_{x\to 2} f(x)\ne 1$, as in this case if $\epsilon=.5$, we can \textbf{always} find a value for $x$ such that
\[
2- \delta < x < 2+ \delta, \qquad\text{where} \qquad  1+\epsilon<f(x).
\]
We've illustrated this in
Figure~\ref{plot:greatist-integer}. Moreover, no matter what value one
chooses for $\ds \lim_{x\to 2} f(x)$, we will always have a similar
issue.
\end{solution}
\end{example}
\marginnote[-2in]{With the example of $f(x) = \lfloor x \rfloor$, we
  see that taking limits is truly different from evaluating
  functions.}


Limits may not exist even if the function looks innocent. 

\begin{example}
Let  $f(x) = \sin\left(\frac{1}{x}\right)$. Explain why the limit
\[
\lim_{x\to 0} f(x)
\]
does not exist.

\begin{solution}
In this case $f(x)$ oscillates ``wildly'' as $x$ approaches $0$, see
Figure~\ref{plot:sin 1/x}. In fact, one can show that for any given
  $\delta$, There is a value for $x$ in the interval
\[
0-\delta < x < 0+\delta
\]
such that $f(x)$ is any value in the interval $[-1,1]$. Hence the
limit does not exist.
\end{solution}
\end{example}
\begin{marginfigure}[-1in]
\begin{tikzpicture}
	\begin{axis}[
            domain=-.2:.2,    
            samples=500,
            axis lines =middle, xlabel=$x$, ylabel=$f(x)$,
            yticklabels = {}, 
            every axis y label/.style={at=(current axis.above origin),anchor=south},
            every axis x label/.style={at=(current axis.right of origin),anchor=west},
            clip=false,
          ]
	  \addplot [very thick, pencolor, smooth, domain=(-.2:.2)] {sin(deg(1/x))};
	  \addplot [color=pencolor, fill=pencolor, very thick, smooth,domain=(-.02:.02)] {1} \closedcycle;
          \addplot [color=pencolor, fill=pencolor, very thick, smooth,domain=(-.02:.02)] {-1} \closedcycle;
        \end{axis}
\end{tikzpicture}
\caption{A plot of $f(x)=\protect\sin\left(\frac{1}{x}\right)$.}
\label{plot:sin 1/x}
\end{marginfigure}

Sometimes the limit of a function exists from one side or the other
(or both) even though the limit does not exist. Since it is useful to
be able to talk about this situation, we introduce the concept of a
\textit{one-sided limit}\index{one-sided limit}:

\begin{definition} We say that the \textbf{limit} of $f(x)$ as $x$ goes to $a$ from the \textbf{left} is $L$,
\[
\lim_{x\to a-}f(x)=L
\]
if for every $\epsilon>0$ there is a $\delta > 0$ so that whenever $x\ne a$ and 
\[
a-\delta < x \qquad\text{we have}\qquad L-\epsilon< f(x)<L+\epsilon.
\]

We say that the \textbf{limit} of $f(x)$ as $x$ goes to $a$ from the \textbf{right} is $L$,
\[
\lim_{x\to a+}f(x)=L
\] 
if for every $\epsilon>0$ there is a $\delta > 0$ so that whenever $x \ne a$ and 
\[
x<a+\delta \qquad\text{we have}\qquad L-\epsilon< f(x)<L+\epsilon.
\]
\end{definition}
\marginnote[-2in]{Limits from the left, or from the right, are collectively called \textbf{one-sided limits}.}


\begin{example}
Let $f(x) = \lfloor x\rfloor$. We now have one sided limits 
\[
\lim_{x\to 2-} f(x) = 1 \qquad\text{and}\qquad \lim_{x\to 2+} f(x) = 2.
\]
\end{example}




\begin{exercises}
\begin{exercise} Evaluate the expressions by reference to the plot in Figure~\ref{plot:piecewise-exercise}.
\begin{marginfigure}
\begin{tikzpicture}
	\begin{axis}[
            domain=-4:6, xmin=-4, xmax=6, ymin=-3,ymax=10,    
            unit vector ratio*=1 1 1,
            axis lines =middle, xlabel=$x$, ylabel=$f(x)$,
            every axis y label/.style={at=(current axis.above origin),anchor=south},
            every axis x label/.style={at=(current axis.right of origin),anchor=west},
            xtick={-4,...,6}, ytick={-3,...,10},
            xticklabels={-4,,-2,,0,,2,,4,,6}, yticklabels={-2,,0,,2,,4,,6,,8,,10},
            grid=major,
            grid style={dashed, gray!30},
          ]
	  \addplot [very thick, pencolor, smooth, domain=(-4:-2)] {6};
	  \addplot [very thick, pencolor, smooth, domain=(-2:0)] {x^2-2};
          \addplot [very thick, pencolor, smooth, domain=(0:2)] {(x-1)^3+3*(x-1)+3};
          \addplot [very thick, pencolor, smooth, domain=(2:6)] {(x-4)^3+8};
          \addplot[color=pencolor,fill=background,only marks,mark=*] coordinates{(-2,6)};  %% open hole
          \addplot[color=pencolor,fill=background,only marks,mark=*] coordinates{(-2,2)};  %% open hole
          \addplot[color=pencolor,fill=background,only marks,mark=*] coordinates{(0,-2)};  %% open hole
          \addplot[color=pencolor,fill=background,only marks,mark=*] coordinates{(0,-1)};  %% open hole
          \addplot[color=pencolor,fill=background,only marks,mark=*] coordinates{(2,0)};  %% open hole
          \addplot[color=pencolor,fill=pencolor,only marks,mark=*] coordinates{(-2,8)};  %% closed hole
          \addplot[color=pencolor,fill=pencolor,only marks,mark=*] coordinates{(0,-1.5)};  %% closed hole
          \addplot[color=pencolor,fill=pencolor,only marks,mark=*] coordinates{(2,7)};  %% closed hole
        \end{axis}
\end{tikzpicture}
\caption{A piecewise defined function.}
\label{plot:piecewise-exercise}
\end{marginfigure}
\begin{enumerate}
\begin{multicols}{3}
\item $\ds \lim_{x\to 4} f(x)$  
\item $\ds \lim_{x\to -3} f(x)$  
\item $\ds \lim_{x\to 0} f(x)$ 
\item $\ds \lim_{x\to 0-} f(x)$  
\item $\ds \lim_{x\to 0+} f(x)$  
\item $\ds f(-2)$  
\item $\ds \lim_{x\to 2-} f(x)$  
\item $\ds \lim_{x\to -2-} f(x)$  
\item $\ds \lim_{x\to 0} f(x+1)$  
\item $\ds f(0)$ 
\item $\ds \lim_{x\to 1-} f(x-4)$  
\item $\ds \lim_{x\to 0+} f(x-2)$
\end{multicols}  
\end{enumerate}
\begin{answer} (a) $8$, (b) $6$, (c) dne, (d) $-2$, (e) $-1$, (f) $8$,
 (g) $7$, (h) $6$, (i) $3$, (j) $-3/2$, (k) $6$, (l) $2$
\end{answer}
\end{exercise}


\begin{exercise} 
Use a calculator to estimate $\ds\lim_{x\to 0} {\sin x\over x}$.
\end{exercise}

\begin{exercise} 
Use a calculator to estimate $\ds\lim_{x\to 0}\frac{\tan(3x)}{\tan(5x)}$.
\end{exercise}


\begin{exercise} 
Sketch a plot of $f(x) = \dfrac{x}{|x|}$ and explain why $\ds
\lim_{x\to 0} \frac{x}{|x|}$ does not exist.
\end{exercise}


\begin{exercise} 
For each of the following limits, $\ds \lim_{x\to a} f(x) =L$, find
$\delta$ such that $a -\delta < x < a+\delta$ implies that $L -
\epsilon< f(x) < L+\epsilon$ where $\epsilon = .1$.
\begin{enumerate}
\begin{multicols}{3}
\item $\ds \lim_{x\to 2}(3x+1) = 7$  
\item $\ds \lim_{x\to 1} (x^2+2) = 3$  
\item $\ds \lim_{x\to \pi} \sin(x) = 0$  
\item $\ds \lim_{x\to 0} \tan(x) = 0$
\item $\ds \lim_{x\to 1} \sqrt{3x+1} = 2$
\item $\ds \lim_{x\to -2} \sqrt[3]{1-4x} = 3$
\end{multicols}  
\end{enumerate}
\end{exercise}

\begin{exercise} 
Let $f(x) = \sin\left(\dfrac{\pi}{x}\right)$. Construct three tables
of the following form
\[
\begin{array}{c|c}
 x & f(x) \\ \hline
 0.d &   \\
 0.0d &  \\
 0.00d &   \\
 0.000d &  
\end{array}
\]
where $d = 1,3,7$. What do you notice? How do you reconcile the
entries in your tables with the value of $\ds\lim_{x\to 0} f(x)$?
\end{exercise}
\end{exercises}






\section{Limits by the Definition}


\begin{example} Show that $\ds \lim_{x\to 2} x^2=4$.
 
We want to show that for any given $\epsilon>0$, we can find a
$\delta>0$ such that $\ds |x^2-4|<\epsilon$ whenever $0<|x-2|<\delta$.

Write $\ds |x^2-4|=|(x+2)(x-2)|$. Now when $|x-2|$ is small, part of
$|(x+2)(x-2)|$ is small, namely $(x-2)$. What about $(x+2)$? If $x$ is
close to 2, $(x+2)$ certainly can't be too big, but we need to somehow
be precise about it. Let's recall the ``game'' version of what is
going on here. You get to pick an $\epsilon$ and I have to pick a
$\delta$ that makes things work out. Presumably it is the really tiny
values of $\epsilon$ I need to worry about, but I have to be prepared
for anything, even an apparently ``bad'' move like $\epsilon=1000$.  I
expect that $\epsilon$ is going to be small, and that the
corresponding $\delta$ will be small, certainly less than 1.  If
$\delta\le 1$ then $|x+2|<5$ when $|x-2|<\delta$ (because if $x$ is
within 1 of 2, then $x$ is between 1 and 3 and $x+2$ is between 3 and
5). So then I'd be trying to show that
$|(x+2)(x-2)|<5|x-2|<\epsilon$. So now how can I pick $\delta$ so that
$|x-2|<\delta$ implies $5|x-2|<\epsilon$? This is easy: use
$\delta=\epsilon/5$, so $5|x-2|<5(\epsilon/5) = \epsilon$. But what if
the $\epsilon$ you choose is not small? If you choose $\epsilon=1000$,
should I pick $\delta=200$? No, to keep things ``sane'' I will never
pick a $\delta$ bigger than 1. Here's the final ``game strategy:''
When you pick a value for $\epsilon$ I will pick $\delta=\epsilon/5$
or $\delta=1$, whichever is smaller. Now when $|x-2|<\delta$, I know
both that $|x+2|<5$ and that $|x-2|<\epsilon/5$. Thus
$|(x+2)(x-2)|<5(\epsilon/5) = \epsilon$.

This has been a long discussion, but most of it was explanation and
scratch work. If this were written down as a proof, it would be quite
short, like this:

Proof that $\ds \lim_{x\to 2}x^2=4$. Given any $\epsilon$, pick
$\delta=\epsilon/5$ or $\delta=1$, whichever is smaller. Then when
$|x-2|<\delta$, $|x+2|<5$ and
$|x-2|<\epsilon/5$. Hence $\ds |x^2-4|=|(x+2)(x-2)|<5(\epsilon/5) =
\epsilon$. 
\end{example}




\begin{theorem} 
Suppose $\ds \lim_{x\to a} f(x)=L$ and $\ds \lim_{x\to a}g(x)=M$. Then
$\lim_{x\to a} f(x)g(x) = LM$.
\end{theorem}

\begin{proof} 
Given any $\epsilon$ we need to find a $\delta$ so that
$0<|x-a|<\delta$ implies $|f(x)g(x)-LM|<\epsilon$. What do we have to
work with? We know that we can make $f(x)$ close to $L$ and $g(x)$
close to $M$, and we have to somehow connect these facts to make
$f(x)g(x)$ close to $LM$.

We use, as is so often the case, a little algebraic
trick: 
\begin{align*}
|f(x)g(x)-LM| &= |f(x)g(x)\boldsymbol{-f(x)M+f(x)M}-LM| \\
&=|f(x)(g(x)-M)+(f(x)-L)M| \\
&\le |f(x)(g(x)-M)|+|(f(x)-L)M| \\
&=|f(x)||g(x)-M|+|f(x)-L||M|.
\end{align*}

This is all straightforward except perhaps for the ``$\le$''. That is
an example of the {\dfont triangle inequality%
\index{triangle inequality}\pagerdef{page:triangle inequality}}, 
which says that if $a$ and $b$ are any real
numbers then $|a+b|\le |a|+|b|$. If you look at a few examples, using
positive and negative numbers in various combinations for $a$ and $b$,
you should quickly understand why this is true; we will not prove it
formally. 

Since $\ds \lim_{x\to a}f(x) =L$, there is a value $\ds \delta_1$ so that
$0<|x-a|<\delta_1$ implies $|f(x)-L|<|\epsilon/(2M)|$, 
This means that $0<|x-a|<\delta_1$ implies
$|f(x)-L||M|< \epsilon/2$. You can see where this is going: if we can
make $|f(x)||g(x)-M|<\epsilon/2$ also, then we'll be done.

We can make $|g(x)-M|$ smaller than any fixed number by making $x$
close enough to $a$; unfortunately, $\epsilon/(2f(x))$ is not a fixed
number, since $x$ is a variable. Here we need another little trick,
just like the one we used in analyzing $x^2$. We can find a $\delta_2$
so that $|x-a|<\delta_2$ implies that $|f(x)-L|<1$, meaning that $L-1
< f(x) < L+1$. This means that $|f(x)|<N$, where $N$ is either $|L-1|$
or $|L+1|$, depending on whether $L$ is negative or positive. The
important point is that $N$ doesn't depend on $x$. Finally, we know that
there is a $\delta_3$ so that $0<|x-a|<\delta_3$ implies
$|g(x)-M|<\epsilon/(2N)$. Now we're ready to put everything
together. Let $\delta$ be the smallest of $\delta_1$, $\delta_2$, and
$\delta_3$. Then $|x-a|<\delta$ implies that
$|f(x)-L|<|\epsilon/(2M)|$, $|f(x)|<N$, and
$|g(x)-M|<\epsilon/(2N)$. Then 
\begin{align*}
|f(x)g(x)-LM|&\le|f(x)||g(x)-M|+|f(x)-L||M| \\
&<N{\epsilon\over 2N}+\left|{\epsilon\over 2M}\right||M| \\
&={\epsilon\over 2}+{\epsilon\over 2}=\epsilon.
\end{align*}
This is just what we needed, so by the official definition,
$\ds \lim_{x\to a}f(x)g(x)=LM$.
\end{proof}




While theorem~\xrefn{thm:properties of limits} is very helpful, we
need a bit more to work easily with limits. Since the theorem applies
when some limits are already known, we need to know the behavior of
some functions that cannot themselves be constructed from the simple
arithmetic operations of the theorem, such as $\ds\sqrt{x}$. Also,
there is one other extraordinarily useful way to put functions
together: composition\index{composition of functions}. If $f(x)$ and
$g(x)$ are functions, we can form two functions by composition:
$f(g(x))$ and $g(f(x))$. For example, if $\ds f(x)=\sqrt{x}$ and $\ds
\ds g(x)=x^2+5$, then $\ds f(g(x))=\sqrt{x^2+5}$ and $\ds
g(f(x))=(\sqrt{x})^2+5=x+5$.  Here is a companion to
theorem~\xrefn{thm:properties of limits} for composition:

\begin{theorem} Suppose that $\ds \lim_{x\to a}g(x)=L$ and $\ds \lim_{x\to L}f(x)=f(L)$. Then
$$\lim_{x\to a} f(g(x)) = f(L).$$
\label{thm:limit of composition}
\end{theorem}

Note the special form of the condition on $f$: it is not enough to
know that $\ds\lim_{x\to L}f(x) = M$, though it is a bit tricky to see
why. Many of the most familiar functions do have this property, and
this theorem can therefore be applied. For example:

\begin{theorem} Suppose that $n$ is a positive integer. Then
$$\lim_{x\to a}\root n\of{x} = \root n\of{a},$$
provided that $a$ is positive if $n$ is even.
\label{thm:continuity of roots}
\end{theorem}

This theorem is not too difficult to prove from the definition of limit.








\begin{exercises}


\begin{exercise} Use the definition of limits to explain why $\ds\lim _{x\to 0 } x\sin \left( {1\over x}\right) = 0$.
Hint: Use the fact that $|\sin a |< 1 $ for any real number $a$.
\begin{answer} $0$
\end{answer}\end{exercise}

\begin{exercise} Use the definition of limits to explain why
$\ds \lim_{x\to 4} (2x-5) = 3$.
\end{exercise}
\end{exercises}


\section{Limit Laws}

limit
A handful of such theorems give us the tools to compute many limits
without explicitly working with the definition of limit.

\begin{mainTheorem}[Limit Laws] Suppose that $\ds \lim_{x\to a}f(x)=L$ and $\ds \lim_{x\to a}g(x)=M$ and
$k$ is some constant. Then
\begin{itemize}
\item $\ds\lim_{x\to a} kf(x) = k\lim_{x\to a}f(x)=kL$ 
\item $\ds\lim_{x\to a} (f(x)+g(x)) = \lim_{x\to a}f(x)+\lim_{x\to a}g(x)=L+M$  
\item $\ds\lim_{x\to a} (f(x)g(x)) = \lim_{x\to a}f(x)\cdot\lim_{x\to a}g(x)=LM$ 
\item $\ds\lim_{x\to a} \frac{f(x)}{g(x)} = \frac{\ds\lim_{x\to a}f(x)}{\ds\lim_{x\to
    a}g(x)}=\frac{L}{M},\text{ if $M$ is not 0}$
\end{itemize}
\label{thm:limit laws}
\end{mainTheorem}

Roughly speaking, these rules say that to compute the limit of an
algebraic expression, it is enough to compute the limits of the
``innermost bits'' and then combine these limits. This often means
that it is possible to simply plug in a value for the variable, since
$\ds \lim_{x\to a} x =a$.


\begin{example}
Compute $\ds
\lim_{x\to 1}{x^2-3x+5\over x-2}$. If we apply the theorem
in all its gory detail, we get
\begin{align*}
\lim_{x\to 1}{x^2-3x+5\over x-2}&=
{\lim_{x\to 1}(x^2-3x+5)\over \lim_{x\to1}(x-2)} \\
&={(\lim_{x\to 1}x^2)-(\lim_{x\to1}3x)+(\lim_{x\to1}5)\over 
  (\lim_{x\to1}x)-(\lim_{x\to1}2)} \\
&={(\lim_{x\to 1}x)^2-3(\lim_{x\to1}x)+5\over (\lim_{x\to1}x)-2} \\
&={1^2-3\cdot1+5\over 1-2} \\
&={1-3+5\over -1} = -3 \\
\end{align*}
\end{example}

It is worth commenting on the trivial limit $\ds \lim_{x\to1}5$. From one
point of view this might seem meaningless, as the number 5 can't
``approach'' any value, since it is simply a fixed number. But 5 can,
and should, be interpreted here as the function that has value 5
everywhere, $f(x)=5$, with graph a horizontal line. From this point of
view it makes sense to ask what happens to the height of the function
as $x$ approaches 1.

Of course, as we've already seen, we're primarily interested in limits
that aren't so easy, namely, limits in which a denominator approaches
zero. There are a handful of algebraic tricks that work on many of
these limits.

\begin{example}
Compute $\ds\lim_{x\to1}{x^2+2x-3\over x-1}$. We can't
simply plug in $x=1$ because that makes the denominator zero. 
However:
\begin{align*}
\lim_{x\to1}{x^2+2x-3\over x-1}&=\lim_{x\to1}{(x-1)(x+3)\over x-1} \\
&=\lim_{x\to1}(x+3)=4 \\
\end{align*}
\vskip-10pt
\end{example}

Another of the most common algebraic tricks was used in
section~\xrefn{sec:slope of a function}. Here's another example:

\begin{example}
Compute $\ds\lim_{x\to-1} {\sqrt{x+5}-2\over x+1}$.
\begin{align*}
\lim_{x\to-1} {\sqrt{x+5}-2\over x+1}&=
\lim_{x\to-1} {\sqrt{x+5}-2\over x+1}{\sqrt{x+5}+2\over \sqrt{x+5}+2} \\
&=\lim_{x\to-1} {x+5-4\over (x+1)(\sqrt{x+5}+2)} \\
&=\lim_{x\to-1} {x+1\over (x+1)(\sqrt{x+5}+2)} \\
&=\lim_{x\to-1} {1\over \sqrt{x+5}+2}={1\over4} \\
\end{align*}
At the very last step we have used theorems~\xrefn{thm:limit of
composition} and \xrefn{thm:continuity of roots}.
\end{example}




\begin{exercises}

\noindent Compute the limits. If a limit does not exist, explain why.

\twocol

\begin{exercise} $\ds \lim_{x\to 3}{x^2+x-12\over x-3}$
\begin{answer} 7
\end{answer}\end{exercise}

\begin{exercise} $\ds \lim_{x\to 1}{x^2+x-12\over x-3}$
\begin{answer} 5
\end{answer}\end{exercise}

\begin{exercise} $\ds \lim_{x\to -4}{x^2+x-12\over x-3}$
\begin{answer} 0
\end{answer}\end{exercise}

\begin{exercise} $\ds \lim_{x\to 2} {x^2+x-12\over x-2}$
\begin{answer} undefined
\end{answer}\end{exercise}

\begin{exercise} $\ds \lim_{x\to 1} {\sqrt{x+8}-3\over x-1}$
\begin{answer} $1/6$
\end{answer}\end{exercise}

\begin{exercise} $\ds \lim_{x\to 0+} \sqrt{{1\over x}+2} - \sqrt{1\over x}$.
\begin{answer} 0
\end{answer}\end{exercise}

\begin{exercise} $\ds\lim _{x\to 2} 3$
\begin{answer} 3
\end{answer}\end{exercise}

\begin{exercise} $\ds\lim _{x\to 4 } 3x^3 - 5x $
\begin{answer} 172
\end{answer}\end{exercise}

\begin{exercise} $\ds \lim _{x\to 0 } {4x - 5x^2\over x-1}$
\begin{answer} 0
\end{answer}\end{exercise}

\begin{exercise} $\ds\lim _{x\to 1 } {x^2 -1 \over x-1 }$
\begin{answer} 2
\end{answer}\end{exercise}

\begin{exercise} $\ds\lim _{x\to 0 + } {\sqrt{2-x^2 }\over x}$
\begin{answer} does not exist
\end{answer}\end{exercise}

\begin{exercise} $\ds\lim _{x\to 0 + } {\sqrt{2-x^2}\over x+1}$
\begin{answer} $\ds \sqrt2$
\end{answer}\end{exercise}

\begin{exercise} $\ds\lim _{x\to a } {x^3 -a^3\over x-a}$
\begin{answer} $\ds 3a^2$
\end{answer}\end{exercise}

\begin{exercise} $\ds\lim _{x\to 2 } (x^2 +4)^3$
\begin{answer} 512
\end{answer}\end{exercise}

\begin{exercise} $\ds\lim _{x\to 1 } \begin{cases}
x-5 & x\neq 1, \\
7 & x=1. \end{cases}$
\begin{answer} $-4$
\end{answer}\end{exercise}

\endtwocol

\end{exercises}


\section{Limits to and from Infinity}


\begin{figure}
\begin{tikzpicture}
	\begin{axis}[
            domain=-2:2,
            samples=100,
            axis lines =middle, xlabel=$x$, ylabel=$f(x)$,
            every axis y label/.style={at=(current axis.above origin),anchor=south},
            every axis x label/.style={at=(current axis.right of origin),anchor=west}
          ]
	  \addplot [very thick, pencolor, smooth, domain=(.1:2)] {1/x};
          \addplot [very thick, pencolor, smooth, domain=(-2:-.1)] {1/x};
        \end{axis}
\end{tikzpicture}
\caption{A plot of $f(x)=\protect\frac{1}{x}$.}
\label{plot:1/x}
\end{figure}
