\section{Lagrange Multipliers}{}{}

Many applied max/min problems take the form of the last two examples:
we want to find an extreme value of a function, like $V=xyz$, subject
to a constraint, like $\ds1=\sqrt{x^2+y^2+z^2}$. Often this can be
done, as we have, by explicitly combining the equations and then
finding critical points. There is another approach that is often
convenient, the method of {\dfont Lagrange multipliers}\index{Lagrange
  multipliers}.

It is somewhat easier to understand two variable problems, so we begin
with one as an example. Suppose the perimeter of a rectangle is to be
100 units. Find the rectangle with largest area. This is a fairly
straightforward problem from single variable calculus. We write down
the two equations: $A=xy$, $P=100=2x+2y$, solve the second of these
for $y$ (or $x$), substitute into the first, and end up with a
one-variable maximization problem. Let's now think of it differently:
the equation $A=xy$ defines a surface, and the equation $100=2x+2y$
defines a curve (a line, in this case) in the $x$-$y$ plane. If we
graph both of these in the three-dimensional coordinate system, we can
phrase the problem like this: what is the highest point on the surface
above the line? The solution we already understand effectively
produces the equation of the cross-section of the surface above the
line and then treats it as a single variable problem. Instead, imagine
that we draw the level curves (the contour lines) for the surface in
the $x$-$y$ plane, along with the line.

\figure
\vbox{\beginpicture
\normalgraphs
\ninepoint
\setcoordinatesystem units <3truecm,3truecm>
\setplotarea x from -1 to 1, y from 0 to 1
\put {\hbox{\epsfxsize6cm\epsfbox{lagrange.eps}}} at 0 0
\endpicture}
\figrdef{fig:lagrange}
\endfigure{Constraint line with contour plot of the surface $xy$.}

Imagine that the line represents a hiking trail and the contour lines
are, as on a topographic map, the lines of constant altitude. How
could you estimate, based on the graph, the high (or low) points on
the path? As the path crosses contour lines, you know the path must be
increasing or decreasing in elevation. At some point you will see the
path just touch a contour line (tangent to it), and then begin to
cross contours in the opposite order---that point of tangency must be
a maximum or minimum point. If we can identify all such points, we can
then check them to see which gives the maximum and which the minimum
value. As usual, we also need to check boundary points; in this
problem, we know that $x$ and $y$ are positive, so we are interested
in just the portion of the line in the first quadrant, as shown. The
endpoints of the path, the two points on the axes, are not points of
tangency, but they are the two places that the function $xy$ is a
minimum in the first quadrant.

How can we actually make use of this? At the points of tangency that
we seek, the constraint curve (in this case the line) and the level
curve have the same slope---their tangent lines are parallel. This also
means that the constraint curve is perpendicular to the gradient
vector of the function; going a bit further, if we can express the
constraint curve itself as a level curve, then we seek the points at
which the two level curves have parallel gradients.
The curve $100=2x+2y$ can be thought of as a level curve of the
function $2x+2y$; figure~\xrefn{fig:lagrange two} shows both sets of
level curves on a single graph. We are interested in those points
where two level curves are tangent---but there are many such points,
in fact an infinite number, as we've only shown a few of the level
curves. All along the line $y=x$ are points at which two level curves
are tangent. While this might seem to be a show-stopper, it is
not. 

\figure
\vbox{\beginpicture
\normalgraphs
\ninepoint
\setcoordinatesystem units <3truecm,3truecm>
\setplotarea x from -1 to 1, y from 0 to 1
\put {\hbox{\epsfxsize6cm\epsfbox{lagrange2.eps}}} at 0 0
\endpicture}
\figrdef{fig:lagrange two}
\endfigure{Contour plots for $2x+2y$ and $xy$.}

The gradient of $2x+2y$ is $\langle 2,2\rangle$, and the gradient of
$xy$ is $\langle y,x\rangle$. They are parallel when
$\langle 2,2\rangle=\lambda\langle y,x\rangle$, that is, when
$2=\lambda y$ and $2=\lambda x$. We have two equations in three
unknowns, which typically results in many solutions (as we
expected). A third equation will reduce the number of solutions; the
third equation is the original constraint, $100=2x+2y$. So we have the
following system to solve:
$$2=\lambda y \qquad 2=\lambda x\qquad 100=2x+2y.$$
In the first two equations, $\lambda$ can't be 0, so we may divide by
it to get $x=y=2/\lambda$. Substituting into the third equation we get 
$$\eqalign{
2{2\over \lambda}+2{2\over \lambda}&=100 \\
{8\over100}&=\lambda \\
}$$
so $x=y=25$. Note that we are not really interested in the value of
$\lambda$---it is a clever tool, the Lagrange multiplier, introduced
to solve the problem. In many cases, as here, it is easier to find
$\lambda$ than to find everything else without using $\lambda$.

The same method works for functions of three variables, except of
course everything is one dimension higher: the function to be
optimized is a function of three variables and the constraint represents
a surface---for example, the function may represent temperature, and
we may be interested in the maximum temperature on some surface, like
a sphere.
The points we seek are those at which the constraint
surface is tangent to a level surface of the function. Once again, we
consider the constraint surface to be a level surface of some
function, and we look for points at which the two gradients are
parallel, giving us three equations in four unknowns. The constraint
provides a fourth equation.

\begin{example} 
Recall example~\xrefn{exam:box diagonal}: the diagonal of a box is 1,
we seek to maximize the volume. The constraint is $\ds
1=\sqrt{x^2+y^2+z^2}$, which is the same as
$1=x^2+y^2+z^2$. The function to maximize is $xyz$. The two gradient
vectors are $\langle 2x,2y,2z\rangle$ and $\langle yz,xz,xy\rangle$,
so the equations to be solved are
$$\eqalign{
yz&=2x\lambda \\
xz&=2y\lambda \\
xy&=2z\lambda \\
1&=x^2+y^2+z^2 \\
}$$
If $\lambda=0$ then at least two of $x$, $y$, $z$ must be 0, giving a
volume of 0, which will not be the maximum. If we multiply the first
two equations by $x$ and $y$ respectively, we get
$$\eqalign{
xyz&=2x^2\lambda \\
xyz&=2y^2\lambda \\
}$$
so $2x^2\lambda=2y^2\lambda$ or $x^2=y^2$; in the same way we can show
$x^2=z^2$. Hence the fourth equation becomes
$1=x^2+x^2+x^2$ or $x=1/\sqrt3$, and so $x=y=z=1/\sqrt3$ gives the
maximum volume. This is of course the same answer we obtained
previously.
\end{example}

Another possibility is that we have a function of three variables, and
we want to find a maximum or minimum value not on a surface but on a
curve; often the curve is the intersection of two surfaces, so that we
really have two constraint equations, say $g(x,y,z)=c_1$ and
$h(x,y,z)=c_2$. It turns out that at points on the intersection of the
surfaces where $f$ has a maximum or minimum value,
$$\nabla f=\lambda\nabla g+\mu \nabla h.$$
As before, this gives us three equations, one for each component of
the vectors, but now in five unknowns, $x$, $y$, $z$, $\lambda$, and
$\mu$. Since there are two constraint functions, we have a total of
five equations in five unknowns, and so can usually find the solutions
we need.

\begin{example} The plane $x+y-z=1$ intersects the cylinder $x^2+y^2=1$ in an
ellipse. Find the points on the ellipse closest to and farthest from
the origin.

We want the extreme values of $f=\sqrt{x^2+y^2+z^2}$ subject to the
constraints  $g=x^2+y^2=1$ and $h=x+y-z=1$. To simplify the algebra,
we may use instead $f=x^2+y^2+z^2$, since this has a maximum or
minimum value at exactly the points at which $\sqrt{x^2+y^2+z^2}$ does.
The gradients are
$$\nabla f =\langle 2x,2y,2z\rangle\qquad
\nabla g = \langle 2x,2y,0\rangle\qquad
\nabla h = \langle 1,1,-1\rangle,$$
so the equations we need to solve are
$$\eqalign{
2x&=\lambda 2x+\mu \\
2y&=\lambda 2y+\mu \\
2z&=0-\mu \\
1&=x^2+y^2 \\
1&=x+y-z. \\
}$$
Subtracting the first two we get
$2y-2x=\lambda(2y-2x)$, so either $\lambda=1$ or $x=y$. If $\lambda=1$
then $\mu=0$, so $z=0$ and the last two equations are
$$1=x^2+y^2\qquad\hbox{and}\qquad 1=x+y.$$
Solving these gives $x=1$, $y=0$, or $x=0$, $y=1$, so the points of
interest are $(1,0,0)$ and $(0,1,0)$, which are both distance 1 from
the origin. If $x=y$, the fourth equation is $2x^2=1$, giving 
$x=y=\pm1/\sqrt2$, and from the fifth equation we get
$z=-1\pm\sqrt2$. The distance from the origin to 
$(1/\sqrt2,1/\sqrt2,-1+\sqrt2)$ is $\sqrt{4-2\sqrt2}\approx 1.08$ and
the distance from the origin to 
$(-1/\sqrt2,-1/\sqrt2,-1-\sqrt2)$ is $\sqrt{4+2\sqrt2}\approx 2.6$.
Thus, the points $(1,0,0)$ and $(0,1,0)$ are closest to the origin and 
$(-1/\sqrt2,-1/\sqrt2,-1-\sqrt2)$ is farthest from the origin.
The Java 
\expandafter\url\expandafter{\liveurl lagrange_two_constraints.html}%
applet \endurl shows the cylinder, the plane, the four points of
interest, and the origin.
\end{example}

\begin{exercises}

\begin{exercise} A six-sided rectangular box is to hold $1/2$ cubic meter;
what shape should the box be to minimize surface area?
\begin{answer} a cube
\end{answer}\end{exercise}

\begin{exercise} The post office will accept packages whose combined length
and girth are at most 130 inches (girth is the maximum distance around
the package perpendicular to the length). What is the largest volume
that can be sent in a rectangular box?
\begin{answer} $65/3\times 65/3\times 130/3$
\end{answer}\end{exercise}

\begin{exercise} The bottom of a rectangular box costs twice as much per unit
area as the sides and top. Find the shape for a given volume that will
minimize cost.
\begin{answer} It has a square base, and is one and one half times as tall as wide.
If the volume is $V$ the dimensions are $\root 3 \of {2V/3}\times
\root 3 \of {2V/3}\times \root 3\of {9V/4}$.
\end{answer}\end{exercise}

\begin{exercise} Using Lagrange multipliers, find the shortest
distance from the point $(x_0,y_0,z_0)$ to the plane $ax+by+cz=d$.
\begin{answer} $|ax_0+by_0+cz_0-d|/\sqrt{a^2+b^2+c^2}$
\end{answer}\end{exercise}

\begin{exercise} Find all points on the surface $xy-z^2+1=0$ that are closest
to the origin.
\begin{answer} $(0,0,1)$, $(0,0,-1)$
\end{answer}\end{exercise}

\begin{exercise} The material for the bottom of an aquarium costs half as
much as the high strength glass for the four sides. Find the shape of
the cheapest aquarium that holds a given volume $V$.
\begin{answer} $\root 3\of{4V}\times\root 3\of{4V}\times\root 3\of{V/16}$
\end{answer}\end{exercise}

\begin{exercise} The plane $x-y+z=2$ intersects the cylinder $x^2+y^2=4$ in an
ellipse. Find the points on the ellipse closest to and farthest from
the origin.
\begin{answer} Farthest: $(-\sqrt2,\sqrt2,2+2\sqrt2)$; closest:
$(2,0,0)$, $(0,-2,0)$
\end{answer}\end{exercise}

%% Albert

\begin{exercise} Find three positive numbers whose sum is 48 and whose
product is as large as possible.
\begin{answer} $x=y=z=16$
\end{answer}\end{exercise}

\begin{exercise} Find all points on the plane $x+y+z = 5$ in the first octant at
which $\ds f(x,y,z) = xy^2z^2$ has a maximum value.
\begin{answer} $(1,2,2)$
\end{answer}\end{exercise}

\begin{exercise} Find the points on the surface $x^2 -yz = 5$ that are closest to the
origin.
\begin{answer} $\ds (0,\sqrt{5},-\sqrt{5})$, $\ds (0,-\sqrt{5},\sqrt{5})$
\end{answer}\end{exercise}

\begin{exercise} A manufacturer makes two models of an item, standard and deluxe.  It
costs \$40 to manufacture the standard model and \$60 for the deluxe.  A
market research firm estimates that if the standard model is priced at $x$
dollars and the deluxe at $y$ dollars, then the manufacturer will sell
$500(y-x)$ of the standard items and $45,000+500(x-2y)$ of the deluxe each
year.  How should the items be priced to maximize profit?
\begin{answer} standard \$65, deluxe \$75
\end{answer}\end{exercise}

\begin{exercise} A length of sheet metal 27 in.\ wide is to be made into a
water trough by bending up two sides as shown in
figure~\xrefn{fig:trough two}.  Find $x$ and $\phi$ so that the
trapezoid--shaped cross section has maximum area.
\begin{answer} $x=9$, $\phi=\pi/3$
\end{answer}\end{exercise}

\figure
\vbox{\beginpicture
\normalgraphs
\ninepoint
\setcoordinatesystem units <1truecm,1truecm>
\setplotarea x from 0 to 5, y from -0.5 to 1
\plot 0 1 1 0 4 0 5 1 /
\put {$x$} [bl] <2pt,2pt> at 0.5 0.5
\put {$x$} [br] <-2pt,2pt> at 4.5 0.5
\put {$27-2x$} [t] <0pt,-3pt> at 2.5 0
\put {$\phi$} at 4.7 0.35
\put {$\phi$} at 0.3 0.35
\circulararc 45 degrees from 4.5 0 center at 4 0
\circulararc -45 degrees from 0.5 0 center at 1 0
\setdashes
\putrule from 4 0 to 5.3 0
\putrule from -0.3 0 to 1 0
\endpicture}
\figrdef{fig:trough two}
\endfigure{Cross-section of a trough.}

\begin{exercise} Find the maximum and minimum values of $f(x,y,z)=6x+3y+2z$ subject
to the constraint $\ds g(x,y,z) = 4x^2+2y^2 + z^2 - 70 = 0$.
\begin{answer} $35$, $-35$
\end{answer}\end{exercise}

\begin{exercise} Find the maximum and minimum values of $f(x,y)=e^{xy}$ subject
to the constraint $g(x,y) = x^3+y^3 - 16 = 0$.  
\begin{answer} maximum $e^4$, no minimum
\end{answer}\end{exercise}

\begin{exercise} Find the maximum and minimum values of $\ds f(x,y) = xy +
\sqrt{9-x^2-y^2}$ when $\ds x^2+y^2 \leq 9$.
\begin{answer} $5$, $-9/2$
\end{answer}\end{exercise}

\begin{exercise} Find three real numbers whose sum is 9 and the sum of whose squares
is a small as possible.  
\begin{answer} $3$, $3$, $3$
\end{answer}\end{exercise}

\begin{exercise} Find the dimensions of the closed rectangular box with maximum volume
that can be inscribed in the unit sphere.
\begin{answer} a cube of side length $\ds 2/\sqrt{3}$
\end{answer}\end{exercise}

\vfill

\begin{exercise} The contour map here shows wind speed in knots during Hurricane
Andrew on August 24, 1992.  Use it to estimate the value of the directional
derivative of the wind speed at Homestead, FL, in the direction of the eye
of the hurricane.  Explain the meaning of your answer to a lay person.
\begin{answer} $\approx 5/8$ knots/mi
\end{answer}\end{exercise}

\centerline{\epsfxsize8truecm\epsfbox{hurricane_wind.eps}}

%% /Albert

\end{exercises}
