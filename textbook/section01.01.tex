\section{Lines}{}{}



If we have two points $A=(x_1,y_1)$ and $B=(x_2,y_2)$, then we can draw one
and only one line through both points.  By the {\it slope} of this line
we mean the ratio of $\Delta y$ to $\Delta x$.  The slope is often denoted
$m$: $m=\Delta y/\Delta x=(y_2-y_1)/(x_2-x_1)$.  For example, the line
joining the points $(1,-2)$ and $(3,5)$ has slope
$(5+2)/(3-1)=7/2$.


% example 1:

\begin{example} According to the 1990 U.S. federal income tax schedules, a head
of household paid 15\% on taxable income up to \$26050. If taxable
income was between \$26050 and \$134930, then, in addition, 28\% was to
be paid on the amount between \$26050 and \$67200, and 33\% paid on
the amount over \$67200 (if any).  Interpret the tax bracket
information (15\%, 28\%, or 33\%) using mathematical terminology, and
graph the tax on the $y$-axis against the taxable income on the
$x$-axis.

The percentages, when converted to decimal values 0.15, 0.28, and 0.33,
are the {\it slopes} of the straight lines which form the graph of the
tax for
the corresponding tax brackets.  The tax graph is what's called a
{\it polygonal line}, i.e., it's made up of several straight line segments
of different slopes.  The first line starts at the point (0,0) and heads
upward with slope 0.15 (i.e., it goes upward 15 for every increase of
100 in the $x$-direction), until it reaches the point above $x=26050$.
Then the graph ``bends upward,'' i.e., the slope changes to 0.28.  As
the horizontal coordinate goes from $x=26050$ to $x=67200$, the line
goes upward 28 for each 100 in the $x$-direction.  At $x=67200$ the
line turns upward again and continues with slope 0.33.  See
Figure~\xrefn{fig:irs}.
\end{example}

% BADBAD
% \figure
% \vbox{
% %\centerline{\epsfysize100pt\epsfxsize300pt\epsfbox{124.figAB.ps}}
% %\bsk
% \centerline{\vbox{\beginpicture
% \normalgraphs
% \ninepoint
% \setcoordinatesystem units <1truemm,1.5truemm>
% \setplotarea x from 0 to 100, y from 0 to 31
% \axis left shiftedto x=0 ticks withvalues {$10000$} {$20000$}
%       {$30000$} / at 10 20 30 / /
% \axis bottom shiftedto y=0 ticks withvalues {$50000$}
%       {$100000$} / at 50 100 / /
% \setlinear
% \plot 0 0 26.05 3.907 67.2 15.429 100 26.253 /
% \multiput {$\bullet$} at 26.05 3.907 67.2 15.429 /
% \endpicture}}}
% \figrdef{fig:irs}
% \endfigure{Tax vs.~income.}
%%---------------------------------
%\begin{centering}
%\begin{psfigure}{\textwidth}{1.75in}{124.figAB.ps}
%\end{psfigure} \\
%\end{centering}
%
%%---------------------------------
%\end{example}

The most familiar form of the equation of a straight line is:
$y=mx+b$.  Here $m$ is the slope of the line: if you increase $x$ by
1, the equation tells you that you have to increase $y$ by $m$.  If
you increase $x$ by $\Delta x$, then $y$ increases by $\Delta
y=m\Delta x$.  The number $b$ is called the {\it y-intercept}, because
it is where the line crosses the $y$-axis.  If you know two points on
a line, the formula $m=(y_2-y_1)/ (x_2-x_1)$ gives you the slope.
Once you know a point and the slope, then the $y$-intercept can be
found by substituting the coordinates of either point in the equation:
$y_1=mx_1+b$, i.e., $b=y_1-mx_1$.  Alternatively, one can use the
``point-slope'' form of the equation of a straight line: start with
$(y-y_1)/(x-x_1)=m$ and then multiply to get $(y-y_1)=m(x-x_1)$, the
point-slope form. Of course, this may be further manipulated to get
$y=mx-mx_1+y_1$, which is essentially the ``$mx+b$'' form.

It is possible to find the equation of a line between two points directly
from the relation $(y-y_1)/(x-x_1)=(y_2-y_1)/(x_2-x_1)$, which says ``the
slope measured between the point $(x_1,y_1)$ and the point $(x_2,y_2)$ is
the same as the slope measured between the point $(x_1,y_1)$ and any other
point $(x,y)$ on the line.''  For example, if we want to find the equation
of the line joining our earlier points $A=(2,1)$ and $B=(3,3)$, we can use
this formula:
$$
{y-1\over x-2}={3-1\over 3-2}=2,\qquad\hbox{so~that}\qquad
y-1=2(x-2),\qquad\hbox{i.e.,}\qquad y=2x-3.
$$
Of course, this is really just the point-slope formula, except that we
are not computing $m$ in a separate step.

The slope $m$ of a line in the form $y=mx+b$ tells us the
direction in which the line is pointing.  If $m$ is positive, the line goes
into the 1st quadrant as you go from left to right.   If $m$ is large and
positive, it has a steep incline, while if $m$ is small and positive, 
then the line has a small angle of inclination.  If $m$ is negative, the
line goes into the 4th quadrant as you go from left to right.  If $m$ is
a large negative number (large in absolute value), then the line points
steeply downward; while if $m$ is negative but near zero, then it points
only a little downward.  These four possibilities are illustrated in 
figure~\xrefn{fig:graphs of lines}.

% BADBAD
% \figure
% \vbox{
% %\centerline{\epsfbox{124.figAC.ps}}
% \centerline{\vbox{\beginpicture
% \normalgraphs
% \sevenpoint
% \setcoordinatesystem units <3truemm,3truemm>
% \setplotarea x from -5 to 5, y from -5 to 5
% \axis left shiftedto x=0 /
% \axis bottom shiftedto y=0 /
% \setlinear
% \plot -1 -5 2.33 5 /
% \linethickness 0.1pt
% \axis left ticks in andacross from -5 to 5 by 1 /
% \axis left ticks in andacross numbered from -4 to 4 by 2 /
% \axis bottom ticks in andacross from -5 to 5 by 1 /
% \axis bottom ticks in andacross numbered from -4 to 4 by 2 /
% \endpicture}\qquad\vbox{\beginpicture
% \normalgraphs
% \sevenpoint
% \setcoordinatesystem units <3truemm,3truemm>
% \setplotarea x from -5 to 5, y from -5 to 5
% \axis left shiftedto x=0 /
% \axis bottom shiftedto y=0 /
% \setlinear
% \plot -5 1 5 2 /
% \linethickness 0.1pt
% \axis left ticks in andacross from -5 to 5 by 1 /
% \axis left ticks in andacross numbered from -4 to 4 by 2 /
% \axis bottom ticks in andacross from -5 to 5 by 1 /
% \axis bottom ticks in andacross numbered from -4 to 4 by 2 /
% \endpicture}\qquad\vbox{\beginpicture
% \normalgraphs
% \sevenpoint
% \setcoordinatesystem units <3truemm,3truemm>
% \setplotarea x from -5 to 5, y from -5 to 5
% \axis left shiftedto x=0 /
% \axis bottom shiftedto y=0 /
% \setlinear
% \plot -0.5 5 2 -5 /
% \linethickness 0.1pt
% \axis left ticks in andacross from -5 to 5 by 1 /
% \axis left ticks in andacross numbered from -4 to 4 by 2 /
% \axis bottom ticks in andacross from -5 to 5 by 1 /
% \axis bottom ticks in andacross numbered from -4 to 4 by 2 /
% \endpicture}\qquad\vbox{\beginpicture
% \normalgraphs
% \sevenpoint
% \setcoordinatesystem units <3truemm,3truemm>
% \setplotarea x from -5 to 5, y from -5 to 5
% \axis left shiftedto x=0 /
% \axis bottom shiftedto y=0 /
% \setlinear
% \plot -5 2 5 1 /
% \linethickness 0.1pt
% \axis left ticks in andacross from -5 to 5 by 1 /
% \axis left ticks in andacross numbered from -4 to 4 by 2 /
% \axis bottom ticks in andacross from -5 to 5 by 1 /
% \axis bottom ticks in andacross numbered from -4 to 4 by 2 /
% \endpicture}}}
% \figrdef{fig:graphs of lines}
% \endfigure{Lines with slopes 3, $0.1$, $-4$, and $-0.1$.}
%---------------------------------

%\medskip
%
%\begin{centering}
%\begin{psfigure}{\hboxwidth}{1.75in}{124.figAC.ps}
%\puttext{-120,-48}{$m=3$}
%\puttext{-30,-48}{$m=0.1$}
%\puttext{50,-48}{$m=-4$} 
%\puttext{130,-48}{$m=-0.1$}
%\end{psfigure}\\
%\end{centering}
%---------------------------------
%\smallskip

If $m=0$, then the line is horizontal: its equation is simply $y=b$.

%\smallskip

There is one type of line that cannot be written in the form $y=mx+b$,
namely, vertical lines.  A vertical line has an equation of the form $x=a$.
Sometimes one says that a vertical line has an ``infinite'' slope.

%\smallskip

Sometimes it is useful to find the $x$-intercept of a line $y=mx+b$.  This is
the $x$-value when $y=0$.  Setting $mx+b$ equal to 0 and solving for
$x$ gives: $x=-b/m$.  For example, the line $y=2x-3$ through the points
$A=(2,1)$ and $B=(3,3)$ has $x$-intercept $3/2$.

%\bigskip

\begin{example} Suppose that you are driving to Seattle at constant speed, and
notice that after you have been traveling for 1 hour (i.e., $t=1$),
you pass a sign saying it is 110 miles to Seattle, and after driving
another half-hour you pass a sign saying it is 85 miles to Seattle.  Using
the horizontal axis for the time $t$ and the vertical axis for the
distance $y$ from Seattle, graph and find the equation $y=mt+b$ for
your distance from Seattle.  Find the slope, $y$-intercept, and
$t$-intercept, and describe the practical meaning of each.
%\smallskip

The graph of $y$ versus $t$ is a straight line because you are traveling
at constant speed.  The line passes through the two points $(1,110)$ and
$(1.5,85)$, so its slope is $m=(85-110)/(1.5-1)=-50$.  The meaning of the
slope is that you are traveling at 50 mph; $m$ is negative because you are
traveling {\it toward\/} Seattle, i.e., your distance $y$ is {\it
decreasing}.  The word ``velocity'' is often used for $m=-50$, when we want
to indicate direction, while the word ``speed'' refers to the magnitude
(absolute value) of velocity, which is 50 mph.  To find the equation of the
line, we use the point-slope formula:
\begin{align*}
    {y-110\over t-1}&=-50,\qquad\hbox{so~that} \\
    y&=-50(t-1)+110=-50t+160.
\end{align*}
The meaning of the $y$-intercept 160 is that when $t=0$ (when you
started the trip) you were 160 miles from Seattle.  To find the
$t$-intercept, set $0=-50t+160$, so that $t=160/50=3.2$.  The meaning
of the $t$-intercept is the duration of your trip, from the start
until you arrive in Seattle.
After traveling 3 hours and 12 minutes, your distance $y$ from Seattle will be
0.  
\label{example:drive to Seattle}
\end{example} 

\begin{exercises}

\begin{exercise} Find the equation of the line through $(1,1)$ and $(-5, -3)$ in
the form $y=mx+b$.
\begin{answer} $(2/3)x+(1/3)$
\end{answer}\end{exercise}

\begin{exercise} Find the equation of the line through $(-1,2)$ with slope $-2$ in
the form $y=mx+b$.
\begin{answer} $y=-2x$
\end{answer}\end{exercise}

\begin{exercise} Find the equation of the line through $(-1,1)$ and $(5, -3)$ in
the form $y=mx+b$.
\begin{answer} $(-2/3)x+(1/3)$
\end{answer}\end{exercise}


\begin{exercise} Change the equation $y-2x=2$ to the form $y=mx+b$, graph the
line, and find the $y$-intercept and $x$-intercept.
\begin{answer} $y=2x+2$, 2, $-1$
\end{answer}\end{exercise}

\begin{exercise} Change the equation $x+y=6$ to the form $y=mx+b$, graph the
line, and find the $y$-intercept and $x$-intercept.
\begin{answer} $y=-x+6$, 6, 6
\end{answer}\end{exercise}


\begin{exercise} Change the equation $x=2y-1$ to the form $y=mx+b$, graph the
line, and find the $y$-intercept and $x$-intercept.
\begin{answer} $y=x/2+1/2$, $1/2$, $-1$
\end{answer}\end{exercise}

\begin{exercise} Change the equation $3=2y$ to the form $y=mx+b$, graph the
line, and find the $y$-intercept and $x$-intercept.
\begin{answer} $y=3/2$, $y$-intercept: $3/2$, no $x$-intercept
\end{answer}\end{exercise}


\begin{exercise} Change the equation $2x+3y+6=0$ to the form $y=mx+b$, graph the
line, and find the $y$-intercept and $x$-intercept.
\begin{answer} $y=(-2/3)x-2$, $-2$, $-3$
\end{answer}\end{exercise}

\begin{exercise} Determine whether the lines $3x+6y=7$ and $2x+4y=5$ are parallel.
\begin{answer} yes
\end{answer}\end{exercise}

\begin{exercise} Suppose a triangle in the $x,y$--plane has vertices $(-1,0)$,
$(1,0)$ and $(0,2)$.  Find the equations of the three lines that lie along
 the sides of the triangle in $y=mx+b$ form.
\begin{answer} $y=0$, $y=-2x+2$, $y=2x+2$
\end{answer}\end{exercise}

\begin{exercise} Suppose that you are driving to Seattle at constant speed.
After you have been traveling for an hour you pass a sign saying it is
130 miles to Seattle, and after driving another 20 minutes you pass a
sign saying it is 105 miles to Seattle.  Using the horizontal axis for
the time $t$ and the vertical axis for the distance $y$ from your
starting point, graph and find the equation $y=mt+b$ for your distance
from your starting point. How long does the trip to Seattle take?
\begin{answer} $y=75t$, 164 minutes
\end{answer}\end{exercise}


\begin{exercise}
Let $x$ stand for temperature in degrees Celsius (centigrade), and let
$y$ stand for temperature in degrees Fahrenheit.  A temperature of $0^\circ$C
corresponds to $32^\circ$F, and a temperature of
$100^\circ$C corresponds to $212^\circ$F.  Find the
equation of the line that relates temperature Fahrenheit $y$ to
temperature Celsius $x$ in the form $y=mx+b$.  
Graph the line, and find the point at which this line intersects $y=x$.
What is the practical meaning of this point?
\begin{answer} $y=(9/5)x+32$, $(-40,-40)$
\end{answer}\end{exercise}

\begin{exercise}
A car rental firm has the following charges for a certain type of car:
\$25 per day with 100 free miles included, \$0.15 per mile for more than
100 miles.  Suppose you want to rent a car for one day, and you know you'll
use it for more than 100 miles.  What is the equation relating the cost
$y$ to the number of miles $x$ that you drive the car?
\begin{answer} $y=0.15x+10$
\end{answer}\end{exercise}

\begin{exercise} A photocopy store advertises the following prices: 5\cents~per
copy for the first 20 copies, 4\cents~per copy for the 21st through
100th copy, and 3\cents~per copy after the 100th copy.  Let $x$ be the
number of copies, and let $y$ be the total cost of photocopying.  (a)
Graph the cost as $x$ goes from 0 to 200 copies.  (b) Find the
equation in the form $y=mx+b$ that tells you the cost of making $x$
copies when $x$ is more than 100.
\begin{answer} $0.03x+1.2$
\end{answer}\end{exercise}

\begin{exercise}
In the Kingdom of Xyg the tax system works as follows.  Someone who
earns less than 100 gold coins per month pays no tax.  Someone who earns
between 100 and 1000 gold coins pays tax equal to 10\% of the amount over
100 gold coins that he or she earns.  Someone who earns over 1000 gold coins
must hand over to the King all of the money earned over 1000 in addition to
the tax on the first 1000.  (a) Draw a
graph of the tax paid $y$ versus the money earned $x$, and give
formulas for $y$ in terms of $x$ in each of the regions $0\le x\le 100$,
$100\le x\le 1000$, and $x\ge 1000$.  (b) Suppose that the King of Xyg
decides to use the second of these line segments (for $100\le x\le 1000$)
for $x\le 100$ as well.  Explain in practical terms what the King is doing,
and what the meaning is of the $y$-intercept. 
\begin{answer} (a) $\ds y=\begin{cases} 0 & 0\le x<100 \\
(x/10)-10& 100\le x\le 1000 \\
x-910 & 1000<x 
\end{cases}$
\end{answer}\end{exercise}

\begin{exercise}
The tax for a single taxpayer is described in the
figure~\xrefn{fig:tax schedule}. Use this
information to graph tax versus taxable income (i.e., $x$ is the
amount on Form 1040, line 37, and $y$ is the amount on Form 1040, line 38).
Find the slope and $y$-intercept of each line that makes up the polygonal
graph, up to $x=97620$.
\begin{answer} $\ds y=\begin{cases} 0.15x & 0\le x\le 19450 \\
0.28x-2528.50&  19450<x\le 47050 \\
0.33x-4881 & 47050<x\le 97620 \\
\end{cases}$
\end{answer}\end{exercise}

% \font\bigbf cmbx10 scaled \magstep1
% \font\Bigbf cmbx10 scaled \magstep2
% \newbox\temp\setbox\temp\hbox{{\bf \bigbf Schedule X}---}
% \figure
% \vbox{\leftline{\bf \Bigbf 1990 Tax Rate Schedules}
%\noindent{\bf Caution:} 
%{\it use ONLY if your taxable income (form 1040, line 37) is {\rm\$}50,000 or
%more. If less, use the {\bf Tax Table}. (Even though you cannot use the
%tax rate schedules below if your taxable income is less than {\rm\$}50,000, we
%show all levels of taxable income so that taxpayers can see the tax rate
%that applies to each level.)}

% BADBAD
% \msk
% \hrule
% \hbox to \hsize{\vbox{\divide\hsize by 2\msk
% \leftline{{\bf \bigbf Schedule X}---Use if your filing status  is}
% \leftline{\hskip\wd\temp{\bf   Single}}\msk
% \sevenpoint\advance\hsize by -5pt
% \halign to \hsize{#\tabskip0pt plus10pt&#&#&#\tabskip0pt \\
% If the amount\hfill&&Enter on\hfill&of the\hfill \\
% on Form 1040\hfill&But not\hfill&Form 1040\hfill&amount\hfill \\
% line 37 is over:\hfill&over:\hfill&line 38\hfill&over:\hfill \\
% \noalign{\ssk\hrule\ssk}
% \hfill\$0&\hfill\$19,450&\hfill {\bf 15\%}&\hfill {\bf\$0} \\
% \hfill 19,450&\hfill 47,050&\hfill {\bf\$2,917.50+28\%}&\hfill {\bf 19,450} \\
% \hfill 47,050&\hfill 97,620&\hfill {\bf\$10,645.50+33\%}&\hfill {\bf 47,050} \\
% &&\leaders\hrule\hfill\span  \\
% &&Use {\bf Worksheet}\hfill \\
% \hfill 97,620&\leaders\hbox{.}\hfill&below to figure\hfill \\
% &&your tax\hfill \\
% }\ssk
% }\vrule\hfill\vbox{\divide\hsize by 2\advance\hsize by -5pt\msk
% \setbox\temp\hbox{{\bf \bigbf Schedule Z}---}
% \leftline{{\bf \bigbf Schedule Z}---Use if your filing status is} 
% \leftline{\hskip\wd\temp{\bf  Head of household}}
% \msk
% \sevenpoint\tabskip0pt
% \halign to \hsize{#\tabskip0pt plus10pt&#&#&#\tabskip0pt \\
% If the amount\hfill&&Enter on\hfill&of the\hfill \\
% on Form 1040\hfill&But not\hfill&Form 1040\hfill&amount\hfill \\
% line 37 is over:\hfill&over:\hfill&line 38\hfill&over:\hfill \\
% \noalign{\ssk\hrule\ssk}
% \hfill\$0&\hfill\$26,050&\hfill {\bf 15\%}&\hfill {\bf\$0} \\
% \hfill 26,050&\hfill 67,200&\hfill {\bf\$3,907.50+28\%}&\hfill {\bf 26,050} \\
% \hfill 67,200&\hfill 134,930&\hfill {\bf\$15,429.50+33\%}&\hfill {\bf 67,200} \\
% &&\leaders\hrule\hfill\span  \\
% &&Use {\bf Worksheet}\hfill \\
% \hfill 134,930&\leaders\hbox{.}\hfill&below to figure\hfill \\
% &&your tax\hfill \\
% }\ssk}}\hrule}
% \figrdef{fig:tax schedule}
% \endfigure{Tax Schedule.}

\begin{exercise}
Market research tells you that if you set the price of an item at
\$1.50, you will be able to sell 5000 items; and for every 10 cents you
lower the price below \$1.50 you will be able to sell another 1000 items.
Let $x$ be the number of items you can sell, and let $P$ be the price of
an item.  (a)~Express $P$ linearly in terms of $x$, in other words,
express $P$ in the form $P=mx+b$.  (b)~Express $x$
linearly in terms of $P$.
\begin{answer} (a) $P=-0.0001x+2$
(b) $x=-10000P+20000$
\end{answer}\end{exercise}

\begin{exercise}
An instructor gives a 100-point final exam, and decides that a score
90 or above will be a grade of 4.0, a score of 40 or below will be a grade
of 0.0, and between 40 and 90 the grading will be linear.  Let $x$ be
the exam score, and let $y$ be the corresponding grade.  Find a formula
of the form $y=mx+b$ which applies to scores $x$ between 40 and 90.
\begin{answer} $(2/25)x-(16/5)$
\end{answer}\end{exercise}

\end{exercises}


