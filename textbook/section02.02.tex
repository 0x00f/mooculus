\section{An example}{}{}

We started the last section by saying, ``It is often necessary to know
how sensitive the value of $y$ is to small changes in $x$.'' We have
seen one purely mathematical example of this: finding the
``steepness'' of a curve at a point is precisely this problem. Here is
a more applied example.

With careful measurement it might be possible to discover that a
dropped ball has height $\ds h(t)=h_0-kt^2$, $t$ seconds after it is
released. (Here $h_0$ is the initial height of the ball, when $t=0$,
and $k$ is some number determined by the experiment.)  A natural
question is then, ``How fast is the ball going at time $t$?'' We can
certainly get a pretty good idea with a little simple arithmetic. To
make the calculation more concrete, let's say $\ds h_0=100$ meters and $k=4.9$
and suppose we're interested in the speed at $t=2$. We know that when
$t=2$ the height is $100-4\cdot 4.9=80.4$. A second later, at $t=3$,
the height is $100-9\cdot 4.9=55.9$, so in that second the ball has
traveled $80.4-55.9=24.5$ meters. This means that the {\it average\/}
speed during that time was 24.5 meters per second. So we might guess
that 24.5 meters per second is not a terrible estimate of the speed at
$t=2$. But certainly we can do better. At $t=2.5$ the height is
$\ds 100-4.9(2.5)^2=69.375$. During the half second from $t=2$ to $t=2.5$ the
ball dropped $80.4-69.375=11.025$ meters, at an average speed of 
$11.025/(1/2)=22.05$ meters per second; this should be a better estimate
of the speed at $t=2$. So it's clear now how to get better and better
approximations: compute average speeds over shorter and shorter time
intervals. Between $t=2$ and $t=2.01$, for example, the ball drops
0.19649 meters in one hundredth of a second, at an average speed of
19.649 meters per second.

We can't do this forever, and we still might reasonably ask what the
actual speed precisely at $t=2$ is. If $\Delta t$ is some tiny amount
of time, what we want to know is what happens to the average speed
$(h(2)-h(2+\Delta t))/\Delta t$ as $\Delta t$ gets smaller and
smaller. Doing a bit of algebra:
\begin{align*}
{h(2)-h(2+\Delta t)\over \Delta t}&={80.4-(100-4.9(2+\Delta
    t)^2)\over \Delta t} \\
&={80.4 - 100 + 19.6+19.6\Delta t+4.9\Delta t^2\over \Delta t} \\
&={19.6\Delta t+4.9\Delta t^2\over \Delta t} \\
&=19.6 + 4.9\Delta t \\
\end{align*}

When $\Delta t$ is very small, this is very close to 19.6, and indeed
it seems clear that as $\Delta t$ goes to zero, the average speed goes
to 19.6, so the exact speed at $t=2$ is 19.6 meters per second. This
calculation should look very familiar. In the language of the previous
section, we might have started with $\ds f(x)=100-4.9x^2$ and asked for
the slope of the tangent line at $x=2$. We would have answered that
question by computing
$$
{f(2+\Delta x) - f(2)\over \Delta x}
={-19.6\Delta x-4.9\Delta x^2\over \Delta x}
=-19.6-4.9\Delta x
$$ The algebra is the same, except that following the pattern of the
previous section the subtraction would be reversed, and we would say
that the slope of the tangent line is $-19.6$. Indeed, in hindsight,
perhaps we should have subtracted the other way even for the dropping
ball. At $t=2$ the height is 80.4; one second later the height is
55.9. The usual way to compute a ``distance traveled'' is to subtract
the earlier position from the later one, or $55.9-80.4=-24.5$. This
tells us that the distance traveled is 24.5 meters, and the negative
sign tells us that the height went down during the second. If we
continue the original calculation we then get $-19.6$ meters per
second as the exact speed at $t=2$. If we interpret the negative sign
as meaning that the motion is downward, which seems reasonable, then
in fact this is the same answer as before, but with even more
information, since the numerical answer contains the direction of
motion as well as the speed. Thus, the speed of the ball is the value
of the derivative of a certain function, namely, of the function that
gives the position of the ball. (More properly, this is the {\em
  velocity\index{velocity}\/} of the ball; velocity is signed speed,
that is, speed with a direction indicated by the sign.)

The upshot is that this problem, finding the speed of the ball, is
{\it exactly\/} the same problem mathematically as finding the slope
of a curve. This may already be enough evidence to convince you that
whenever some quantity is changing (the height of a curve or the
height of a ball or the size of the economy or the distance of a space
probe from earth or the population of the world) the rate at which the
quantity is changing can, in principle, be computed in exactly the
same way, by finding a derivative.

\begin{exercises}

\begin{exercise}
An object is traveling in a straight line so that its position (that
is, distance from some fixed point) is
given by this table:

\begin{table}[h]
  \begin{center}
    \begin{tabular}{lrrrr}
      \toprule
      time (seconds) & 0 & 1 & 2 & 3 \\
      distance (meters) & 0 & 10 & 25 & 60 \\
      \bottomrule
    \end{tabular}
  \end{center}
\end{table}

{}Find the average speed of the object during the following time
intervals: $[0,1]$, $[0,2]$, $[0,3]$,
$[1,2]$, $[1,3]$, $[2,3]$. If you had to guess the speed at
$t=2$ just on the basis of these, what would you guess?
\begin{answer} $10$, $25/2$, $20$, $15$, $25$, $35$.
\end{answer}\end{exercise}

\begin{exercise}
Let $\ds y=f(t)=t^2$, where $t$ is the time in seconds and $y$ is the distance
in meters that an object falls on a certain airless planet.  Draw a graph
of this function between $t=0$ and $t=3$.  Make a table of the average
speed of the falling object between (a) 2 sec and 3 sec, (b) 2 sec and
2.1 sec, (c) 2 sec and 2.01 sec, (d) 2 sec and 2.001 sec.  Then use algebra
to find a simple formula for the average speed between time $2$ and time
$2+
\Delta t$.  (If you substitute $\Delta t=1,\>0.1,\>0.01,\>0.001$ in this
formula you should again get the answers to parts (a)--(d).)  Next, in your
formula for average speed (which should be in simplified form) determine
what happens as $\Delta t$ approaches zero.  This is the instantaneous
speed.  Finally, in your graph of $\ds y=t^2$ draw the straight line
through the point $(2,4)$ whose slope is the instantaneous velocity you
just computed; it should of course be the tangent line.
\begin{answer} $5$, $4.1$, $4.01$, $4.001$, $4+\Delta t\rightarrow 4$
\end{answer}\end{exercise}

\begin{exercise}
If an object is dropped from an 80-meter high window, its height $y$ above
the ground at time $t$ seconds is given by the formula $\ds y=f(t)=80-4.9t^2$.
(Here we are neglecting air resistance; the graph of this function was
shown in figure~\xrefn{fig:data plot}.)  Find the average velocity of
the falling object between (a) 1 sec and 1.1 sec, (b) 1 sec and 1.01 sec,
(c) 1 sec and 1.001 sec.  Now use algebra to find a simple formula for the
average velocity of the falling object between 1 sec and $1+\Delta t$ sec.
Determine what happens to this average velocity as $\Delta t$ approaches 0.
That is the instantaneous velocity at time $t=1$ second (it will be negative,
because the object is falling).
\begin{answer} $-10.29$, $-9.849$, $-9.8049$, \hfill\break
$-9.8-4.9\Delta t\rightarrow -9.8$
\end{answer}\end{exercise}

\end{exercises}
