\section{Stokes's Theorem}{}{}

Recall that one version of Green's Theorem (see
equation~\xrefn{eq:greens theorem second form}) is
$$\int_{\partial D} {\bf F}\cdot d{\bf r}
=\dint{D}(\nabla\times {\bf F})\cdot{\bf k}\,dA.
$$
Here $D$ is a region in the $x$-$y$ plane and $\bf k$ is a unit normal
to $D$ at every point. If $D$ is instead an orientable surface in
space, there is an obvious way to alter this equation, and it turns
out still to be true:

\begin{theorem} (Stokes's Theorem) Provided that the quantities involved are
sufficiently nice, and in particular if $D$ is orientable, 
$$\int_{\partial D} {\bf F}\cdot d{\bf r}
=\dint{D}(\nabla\times {\bf F})\cdot{\bf N}\,dS,
$$
if $\partial D$ is oriented counter-clockwise relative to $\bf N$.
\end{theorem}

Note how little has changed: $\bf k$ becomes $\bf N$, a unit normal to
the surface, and $dA$ becomes $dS$, since this is now a general
surface integral. The phrase ``counter-clockwise relative to $\bf N$''
means that if we take the direction of $\bf N$ to be ``up'', then we
go around the boundary counter-clockwise when viewed from ``above''.

\begin{example} Let 
${\bf F}=\langle e^{xy}\cos z,x^2z,xy\rangle$ 
and the surface $D$ be $x=\sqrt{1-y^2-z^2}$, oriented in
the positive $x$ direction.
It quickly becomes apparent that the surface integral in Stokes's
Theorem is intractable, so we try the line integral. The boundary of
$D$ is the unit circle in the $y$-$z$ plane, ${\bf r}=\langle 0,\cos
u,\sin u\rangle$, $0\le u\le 2\pi$. The integral is
$$\int_0^{2\pi} \langle e^{xy}\cos z,x^2z,xy\rangle\cdot
\langle 0,-\sin u,\cos u\rangle\,du=
\int_0^{2\pi} 0\,du = 0,$$
because $x=0$.
\end{example}

An interesting consequence of Stokes's Theorem is that if $D$ and $E$
are two orientable surfaces with the same boundary, then
$$
\dint{D}(\nabla\times {\bf F})\cdot{\bf N}\,dS
=\int_{\partial D} {\bf F}\cdot d{\bf r}
=\int_{\partial E} {\bf F}\cdot d{\bf r}
=\dint{E}(\nabla\times {\bf F})\cdot{\bf N}\,dS.
$$
Sometimes both of the integrals 
$$\dint{D}(\nabla\times {\bf F})\cdot{\bf N}\,dS
\qquad\hbox{and}\qquad\int_{\partial D} {\bf F}\cdot d{\bf r}$$
are difficult, but you may be able to find a second surface $E$ so
that
$$\dint{E}(\nabla\times {\bf F})\cdot{\bf N}\,dS
$$
has the same value but is easier to compute.

\begin{example} In the previous example, the line integral was easy to
compute. But we might also notice that another surface $E$ with the
same boundary is the flat disk $y^2+z^2\le 1$. The unit normal $\bf N$
for this surface is simply ${\bf i}=\langle 1,0,0\rangle$. We compute
the curl:
$$\nabla\times{\bf F}=\langle x-x^2,-e^{xy}\sin z-y,2xz-xe^{xy}\cos
z\rangle.$$ 
Since $x=0$ everywhere on the surface,
$$(\nabla\times{\bf F})\cdot {\bf N}=
\langle 0,-e^{xy}\sin z-y,2xz-xe^{xy}\cos
z\rangle\cdot\langle 1,0,0\rangle=0,$$
so the surface integral is
$$\dint{E}0\,dS=0,$$
as before. In this case, of course, it is still somewhat easier to
compute the line integral, avoiding $\nabla\times{\bf F}$ entirely.
\end{example}

\begin{example} Let ${\bf F}=\langle -y^2,x,z^2\rangle$, and let the curve $C$
be the intersection of the cylinder $x^2+y^2=1$ with the plane
$y+z=2$, oriented counter-clockwise when viewed from above.
We compute $\ds\int_C {\bf F}\cdot d{\bf r}$ in two ways.

First we do it directly: a vector function for $C$ is
${\bf r}=\langle \cos u,\sin u, 2-\sin u\rangle$, so
${\bf r}'=\langle -\sin u,\cos u,-\cos u\rangle$, and the integral is then
$$\int_0^{2\pi} y^2\sin u+x\cos u-z^2\cos u\,du
=\int_0^{2\pi} \sin^3 u+\cos^2 u-(2-\sin u)^2\cos u\,du
=\pi.$$

To use Stokes's Theorem, we pick a surface with $C$ as the boundary;
the simplest such surface is that portion of the plane $y+z=2$ inside
the cylinder. This has vector equation ${\bf r}=\langle
v\cos u,v\sin u,2-v\sin u\rangle$. We compute
${\bf r}_u= \langle -v\sin u,v\cos u,-v\cos u\rangle$,
${\bf r}_v= \langle \cos u,\sin u, -\sin u\rangle$, and 
${\bf r}_u\times{\bf r}_v=\langle 0,-v,-v\rangle$. To match the
orientation of $C$ we need to use the normal $\langle
0,v,v\rangle$. The curl of $\bf F$ is $\langle 0,0,1+2y\rangle=
\langle 0,0,1+2v\sin u\rangle$, and
the surface integral from Stokes's Theorem is
$$\int_0^{2\pi}\int_0^1 (1+2v\sin u)v\,dv\,du=\pi.$$
In this case the surface integral was more work to set up, but the
resulting integral is somewhat easier.
\end{example}

\vskip6pt\noindent{\begin{proof}font Proof of 
Stokes's Theorem.}\kern1pc\bgroup

We can prove here a special case of Stokes's Theorem, which perhaps
not too surprisingly uses Green's Theorem.

Suppose the surface $D$ of interest can be expressed in the form
$z=g(x,y)$, and let ${\bf F}=\langle P,Q,R\rangle$. Using the vector
function ${\bf r}=\langle x,y,g(x,y)\rangle$ for the surface we get the
surface integral
$$\eqalign{
\dint{D} \nabla\times{\bf F}\cdot d{\bf S}&=
\dint{E} \langle R_y-Q_z,P_z-R_x,Q_x-P_y\rangle\cdot
\langle -g_x,-g_y,1\rangle\,dA \\
&=\dint{E} -R_yg_x+Q_zg_x-P_zg_y+R_xg_y+Q_x-P_y\,dA. \\}$$
Here $E$ is the region in the $x$-$y$ plane directly below the surface
$D$. 

For the line integral, we need a vector function for $\partial D$. If 
$\langle x(t),y(t)\rangle$ is a vector function for 
$\partial E$ then we may use ${\bf r}(t)=\langle x(t),y(t),g(x(t),y(t))\rangle$
to represent $\partial D$. Then
$$\int_{\partial D}{\bf F}\cdot d{\bf r}
=\int_a^b P{dx\over dt}+Q{dy\over dt}+R{dz\over dt}\,dt
=\int_a^b P{dx\over dt}+Q{dy\over dt}+R\left({\partial z\over\partial
    x}{dx\over dt}+{\partial z\over\partial y}{dy\over dt}\right)\,dt.$$
using the chain rule for $dz/dt$. Now we continue to manipulate this:
$$
\eqalign{
\int_a^b P{dx\over dt}+Q{dy\over dt}+&R\left({\partial z\over\partial
    x}{dx\over dt}+{\partial z\over\partial y}{dy\over dt}\right)\,dt \\
&=\int_a^b \left[\left(P+R{\partial z\over\partial x}\right){dx\over dt}+
\left(Q+R{\partial z\over\partial y}\right){dy\over dt}\right]\,dt \\
&=\int_{\partial E} \left(P+R{\partial z\over\partial x}\right)\,dx+
\left(Q+R{\partial z\over\partial y}\right)\,dy, \\}$$
which now looks just like the line integral of Green's Theorem, except
that the functions $P$ and $Q$ of Green's Theorem have been replaced
by the more complicated $P+R(\partial z/\partial x)$
and $Q+R(\partial z/\partial y)$. We can apply Green's Theorem to get
$$\int_{\partial E} \left(P+R{\partial z\over\partial x}\right)\,dx+
\left(Q+R{\partial z\over\partial y}\right)\,dy=
\dint{E} {\partial\over \partial x}\left(Q+R{\partial z\over\partial y}\right)
-{\partial\over \partial y}\left(P+R{\partial z\over\partial x}\right)\,dA.$$
Now we can use the chain rule again to evaluate the derivatives
inside this integral, and it becomes
$$\eqalign{
\dint{E} &Q_x+Q_zg_x+R_xg_y+R_zg_xg_y+Rg_{yx}-
\left(P_y+P_zg_y+R_yg_x+R_zg_yg_x+Rg_{xy}\right)\,dA \\
&=\dint{E} Q_x+Q_zg_x+R_xg_y-P_y-P_zg_y-R_yg_x\,dA, \\
}$$
which is the same as the expression we obtained for the surface
integral.
\end{proof}

\begin{exercises}

\exercise
Let ${\bf F}=\langle z,x,y\rangle$.
The plane $z=2x+2y-1$ and the paraboloid $z=x^2+y^2$ intersect in a
closed curve. Stokes's Theorem implies that
$$\dint{D_1} (\nabla\times{\bf F})\cdot {\bf N}\,dS=
\oint_C {\bf F}\cdot d{\bf r}=
\dint{D_2} (\nabla\times{\bf F})\cdot {\bf N}\,dS,
$$
where the line integral is computed over the intersection $C$ of the plane
and the paraboloid, and the two surface integrals are computed over
the portions of the two surfaces that have boundary $C$ (provided, of
course, that the orientations all match). Compute all three integrals.
\begin{answer} $-3\pi$
\end{answer}

\exercise Let $D$ be the portion of $z=1-x^2-y^2$ above the $x$-$y$
plane, oriented up, and let ${\bf F}=\langle
xy^2,-x^2y,xyz\rangle$. Compute $\ds\dint{D} (\nabla\times{\bf
F})\cdot {\bf N}\,dS$.
\begin{answer} $0$
\end{answer}

\exercise Let $D$ be the portion of $z=2x+5y$ inside $x^2+y^2=1$,
oriented up, and
let ${\bf F}=\langle y,z,-x\rangle$. Compute
$\ds\int_{\partial D} {\bf F}\cdot d{\bf r}$.
\begin{answer} $-4\pi$
\end{answer}

%Albert
\exercise Compute $\ds\oint_C x^2z\,dx + 3x\,dy x^3 - y^3\,dz$, where $C$
is the unit circle $\ds x^2+y^2=1$ oriented counter-clockwise.
\begin{answer} $3\pi$
\end{answer}
%/Albert

\exercise Let $D$ be the portion of $z=px+qy+r$ over a region in the
$x$-$y$ plane that has area $A$, oriented up, and 
let ${\bf F}=\langle ax+by+cz,ax+by+cz,ax+by+cz\rangle$. Compute
$\ds\int_{\partial D} {\bf F}\cdot d{\bf r}$.
\begin{answer} $A(p(c-b)+q(a-c)+a-b)$
\end{answer}

\exercise Let $D$ be any surface and 
let ${\bf F}=\langle P(x),Q(y),R(z)\rangle$ ($P$ depends only on $x$,
$Q$ only on $y$, and $R$ only on $z$). Show that
$\ds\int_{\partial D} {\bf F}\cdot d{\bf r}=0$.

\exercise Show that $\ds\int_C f\nabla g+g\nabla f\cdot d{\bf r}=0$, where
$\bf r$ describes a closed curve $C$ to which Stokes's Theorem
applies.

\end{exercises}

