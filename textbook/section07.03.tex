\section{Some Properties of Integrals}{}{}
\nobreak
Suppose an object moves so that its speed, or more properly velocity,
is given by $\ds v(t)=-t^2+5t$, as shown in figure~\xrefn{fig:speed can be
negative}. Let's examine the motion of this object carefully. We know
that the velocity is the derivative of position, so position is 
given by $\ds s(t)=-t^3/3+5t^2/2+C$. Let's suppose that at time $t=0$ the
object is at position 0, so $\ds s(t)=-t^3/3+5t^2/2$; this function is
also pictured in figure~\xrefn{fig:speed can be
negative}.

\figure
\vbox{\beginpicture
\normalgraphs
\ninepoint
\setcoordinatesystem units <1truecm,0.4truecm>
\setplotarea x from 0 to 6, y from -6 to 6.4
\axis left shiftedto x=0 ticks length <2pt> numbered from -6 to 6 by 2 /
\axis bottom shiftedto y=0 ticks length <2pt> numbered from 1 to 6 by 1 /
\setquadratic
\plot 0.000 0.000 0.150 0.728 0.300 1.410 0.450 2.048 0.600 2.640 
0.750 3.188 0.900 3.690 1.050 4.148 1.200 4.560 1.350 4.928 
1.500 5.250 1.650 5.528 1.800 5.760 1.950 5.948 2.100 6.090 
2.250 6.188 2.400 6.240 2.550 6.248 2.700 6.210 2.850 6.128 
3.000 6.000 3.150 5.828 3.300 5.610 3.450 5.348 3.600 5.040 
3.750 4.688 3.900 4.290 4.050 3.848 4.200 3.360 4.350 2.828 
4.500 2.250 4.650 1.628 4.800 0.960 4.950 0.248 5.100 -0.510 
5.250 -1.312 5.400 -2.160 5.550 -3.052 5.700 -3.990 5.850 -4.972 
6.000 -6.000  /
\setcoordinatesystem units <1truecm,0.2truecm> point at -8 12
\setplotarea x from 0 to 6, y from 0 to 22
\axis left shiftedto x=0 ticks length <2pt> numbered from 0 to 22 by 5 /
\axis bottom shiftedto y=0 ticks length <2pt> numbered from 1 to 6 by 1 /
\setquadratic
\plot 0.000 0.000 0.150 0.055 0.300 0.216 0.450 0.476 0.600 0.828 
0.750 1.266 0.900 1.782 1.050 2.370 1.200 3.024 1.350 3.736 
1.500 4.500 1.650 5.309 1.800 6.156 1.950 7.035 2.100 7.938 
2.250 8.859 2.400 9.792 2.550 10.729 2.700 11.664 2.850 12.590 
3.000 13.500 3.150 14.388 3.300 15.246 3.450 16.068 3.600 16.848 
3.750 17.578 3.900 18.252 4.050 18.863 4.200 19.404 4.350 19.869 
4.500 20.250 4.650 20.541 4.800 20.736 4.950 20.827 5.100 20.808 
5.250 20.672 5.400 20.412 5.550 20.022 5.700 19.494 5.850 18.822 
6.000 18.000  /
\endpicture}
\figrdef{fig:speed can be negative}
\endfigure{The velocity of an object and its position.}

Between $t=0$ and $t=5$ the velocity is positive, so the object moves
away from the starting point, until it is a bit past position 20. Then
the velocity becomes negative and the object moves back toward its
starting point. The position of the object at $t=5$ is exactly 
$s(5)=125/6$, and at $t=6$ it is $s(6)=18$. The total distance
traveled by the object is therefore $125/6 + (125/6 - 18) =
71/3\approx 23.7$. 

As we have seen, we can also compute distance traveled with an
integral; let's try it.
$$
  \int_0^6 v(t)\,dt = \int_0^6 -t^2+5t\,dt = 
\left.{-t^3\over 3}+{5\over2}t^2\right|_0^6  = 
18.
$$
What went wrong? Well, nothing really, except that it's not really
true after all that ``we can also compute distance traveled with an
integral''. Instead, as you might guess from this example, the
integral actually computes the {\it net\/} distance traveled, that is,
the difference between the starting and ending point.

As we have already seen,
$$
  \int_0^6 v(t)\,dt=\int_0^5 v(t)\,dt+\int_5^6 v(t)\,dt.
$$
Computing the two integrals on the right (do it!) gives
$125/6$ and $-17/6$, and the sum of these is indeed 18. But what does
that negative sign mean? It means precisely what you might think: it
means that the object moves backwards. To get the total distance
traveled we can add $125/6+17/6=71/3$, the same answer we got before.

Remember that we can also interpret an integral as measuring an area,
but now we see that this too is a little more complicated that we have
suspected. The area under the curve $v(t)$ from 0 to 5 is given by 
$$
  \int_0^5 v(t)\,dt={125\over6},
$$
and the ``area'' from 5 to 6 is 
$$
  \int_5^6 v(t)\,dt=-{17\over 6}.
$$
In other words, the area between the $x$-axis and the curve, but under
the $x$-axis, ``counts as negative area''. So the integral
$$
  \int_0^6 v(t)\,dt=18
$$ 
measures ``net area'', the area above the axis minus the (positive)
area below the axis.

If we recall that the integral is the limit of a certain kind of sum,
this behavior is not surprising. Recall the sort of sum involved:
$$
  \sum_{i=0}^{n-1} v(t_i)\Delta t.
$$
In each term $v(t)\Delta t$ the $\Delta t$ is positive, but if
$\ds v(t_i)$ is negative then the term is negative. If over an entire
interval, like 5 to 6, the function is always negative, then the
entire sum is negative. In terms of area, $v(t)\Delta t$ is then a
negative height times a positive width, giving a negative rectangle
``area''. 

So now we see that when evaluating $$\ds\int_5^6 v(t)\,dt=-{17\over 6}$$
by finding an antiderivative, substituting, and subtracting, we get a
surprising answer, but one that turns out to make sense.

Let's now try something a bit different:
$$
  \int_6^5 v(t)\,dt=\left.{-t^3\over 3}+{5\over2}t^2\right|_6^5 =
  {-5^3\over 3}+{5\over2}5^2-{-6^3\over 3}-{5\over2}6^2 ={17\over 6}.
$$
Here we simply interchanged the limits 5 and 6, so of course when we
substitute and subtract we're subtracting in the opposite order and we
end up multiplying the answer by $-1$. This too makes sense in terms
of the underlying sum, though it takes a bit more thought. Recall that
in the sum
$$
  \sum_{i=0}^{n-1} v(t_i)\Delta t,
$$
the $\Delta t$ is the ``length'' of each little subinterval, but more
precisely we could say that $\ds \Delta t = t_{i+1}-t_i$, the difference
between two endpoints of a subinterval. We have until now assumed that
we were working left to right, but could as well number the
subintervals from right to left, so that $\ds t_0=b$ and $\ds t_n=a$.
Then $\ds \Delta t=t_{i+1}-t_i$ is negative and
in 
$$
  \int_6^5 v(t)\,dt=\sum_{i=0}^{n-1} v(t_i)\Delta t,
$$
the values $\ds v(t_i)$ are negative but also $\Delta t$ is negative, so all
terms are positive again. On the other hand, in
$$
  \int_5^0 v(t)\,dt=\sum_{i=0}^{n-1} v(t_i)\Delta t,
$$
the values $\ds v(t_i)$ are positive but $\Delta t$ is negative,and we get
a negative result:
$$
  \int_5^0 v(t)\,dt=\left.{-t^3\over 3}+{5\over2}t^2\right|_5^0 =
  0-{-5^3\over 3}-{5\over2}5^2 = -{125\over6}.
$$

Finally we note one simple property of integrals:
$$
  \int_a^b f(x)+g(x)\,dx=\int_a^b f(x)\,dx+\int_a^b g(x)\,dx.
$$
This is easy to understand once you recall that
$(F(x)+G(x))'=F'(x)+G'(x)$. Hence, if $F'(x)=f(x)$ and $G'(x)=g(x)$,
then
$$
  \eqalign{
  \int_a^b f(x)+g(x)\,dx&=\left.(F(x)+G(x))\right|_a^b \\
  &=F(b)+G(b)-F(a)-G(a) \\
  &=F(b)-F(a)+G(b)-G(a) \\
  &=\left.F(x)\right|_a^b+\left.G(x)\right|_a^b \\
  &=\int_a^b f(x)\,dx+\int_a^b g(x)\,dx. \\
  }
$$

In summary, we will frequently use these properties of 
\index{properties of integrals}\index{integral!properties of}integrals:
$$\displaylines{
  \int_a^b f(x)\,dx = \int_a^c f(x)\,dx + \int_c^b f(x)\,dx \\
  \int_a^b f(x)+g(x)\,dx=\int_a^b f(x)\,dx+\int_a^b g(x)\,dx \\
  \int_a^b f(x)\,dx=-\int_b^a f(x)\,dx \\
}$$
and if $a<b$ and $f(x)\le 0$ on $[a,b]$ then
$$
  \int_a^b f(x)\,dx\le 0$$
and in fact
$$
  \int_a^b f(x)\,dx=-\int_a^b |f(x)|\,dx.
$$

\begin{exercises}

\begin{exercise} An object moves so that its velocity at time $t$ is
$v(t)=-9.8t+20$ m/s. Describe the motion of the object between $t=0$ and
$t=5$, find the total distance traveled by the object during that
time, and find the net distance traveled.
\begin{answer} It rises until $t=100/49$, then falls. The position of the
object at time $t$ is $\ds s(t)=-4.9t^2+20t+k$. The net distance traveled
is $-45/2$, that is, it ends up $45/2$ meters below where it started.
The total distance traveled is $6205/98$ meters. 
\end{answer}\end{exercise}

\begin{exercise} An object moves so that its velocity at time $t$ is $v(t)=\sin t$.
Set up and evaluate a single definite integral to compute the
net distance traveled between $t=0$ and $t=2\pi$.
\begin{answer} $\ds\int_0^{2\pi}\sin t\,dt=0$
\end{answer}\end{exercise}

\begin{exercise} An object moves so that its velocity at time $t$ is
$v(t)=1+2\sin t$ m/s. Find the net distance traveled by the object
between $t=0$ and $t=2\pi$, and find the total distance traveled
during the same period.
\begin{answer} net: $2\pi$, total: $\ds 2\pi/3+4\sqrt3$ 
\end{answer}\end{exercise}

\begin{exercise} Consider the function $f(x)=(x+2)(x+1)(x-1)(x-2)$ on
$[-2,2]$. Find the total area between the curve and the $x$-axis
(measuring all area as positive).
\begin{answer} $8$
\end{answer}\end{exercise}

\begin{exercise} Consider the function $\ds f(x)=x^2-3x+2$ on
$[0,4]$. Find the total area between the curve and the $x$-axis
(measuring all area as positive).
\begin{answer} $17/3$
\end{answer}\end{exercise}

\begin{exercise} Evaluate the three integrals:
$$
  A=\int_0^3 (-x^2+9)\,dx\qquad B=\int_0^{4} (-x^2+9)\,dx\qquad 
  C=\int_{4}^3 (-x^2+9)\,dx,
$$
and verify that $A=B+C$.
\begin{answer} $A=18$, $B=44/3$, $C=10/3$
\end{answer}\end{exercise}

\end{exercises}

