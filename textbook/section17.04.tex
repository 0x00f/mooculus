\section{Approximation}{}{}
%\label{sec:differential equations}
\nobreak
We have seen how to solve a restricted collection of differential
equations, or more  accurately, how to attempt to solve them---we may
not be able to find the required anti-derivatives. Not surprisingly,
non-linear equations can be even more difficult to solve. Yet much is
known about solutions to some more general equations.

Suppose $\phi(t,y)$ is a function of two variables. A more general
class of first order differential equations has the form $\ds\dot
y=\phi(t,y)$. This is not necessarily a linear first order equation,
since $\phi$ may depend on $y$ in some complicated way; note however
that $\ds\dot y$ appears in a very simple form. Under suitable
conditions on the function $\phi$, it can be shown that every such
differential equation has a solution, and moreover that for each
initial condition the associated initial value problem has exactly one
solution. In practical applications this is obviously a very desirable
property. 

\begin{example} The equation $\ds\dot y=t-y^2$ is a first order non-linear
equation, because $y$ appears to the second power. We will not be able
to solve this equation.
\end{example}

\begin{example} The equation $\ds\dot y=y^2$ is also non-linear, but it is
separable and can be solved by separation of variables.
\end{example}

Not all differential equations that are important in practice can be
solved exactly, so techniques have been developed to approximate
solutions. We describe one such technique, {\dfont Euler's
  Method\index{Euler's Method}}, which is simple though not particularly
useful compared to some more sophisticated techniques.

Suppose we wish to approximate a solution to the initial value problem
$\ds\dot y=\phi(t,y)$, $\ds y(t_0)=y_0$, for $t\ge t_0$. Under
reasonable conditions on $\phi$, we know the solution exists,
represented by a curve in the $t$-$y$ plane; call this solution
$f(t)$. The point $\ds (t_0,y_0)$ is of course on this curve. We also
know the slope of the curve at this point, namely
$\ds\phi(t_0,y_0)$. If we follow the tangent line for a brief
distance, we arrive at a point that should be almost on the graph of
$f(t)$, namely $\ds(t_0+\Delta t, y_0+\phi(t_0,y_0)\Delta t)$; call
this point $\ds(t_1,y_1)$. Now we pretend, in effect, that this point
really is on the graph of $f(t)$, in which case we again know the
slope of the curve through $\ds(t_1,y_1)$, namely
$\ds\phi(t_1,y_1)$. So we can compute a new point,
$\ds(t_2,y_2)=(t_1+\Delta t, y_1+\phi(t_1,y_1)\Delta t)$ that 
is a little farther along, still close to the graph of $f(t)$ but
probably not quite so close as $\ds(t_1,y_1)$. We can continue in this
way, doing a sequence of straightforward calculations, until we have
an approximation $\ds(t_n,y_n)$ for whatever time $\ds t_n$ we need. 
At each step we do essentially the same calculation, namely
$$(t_{i+1},y_{i+1})=(t_i+\Delta t, y_i+\phi(t_i,y_i)\Delta t).$$
We expect that smaller time steps $\Delta t$ will give better
approximations, but of course it will require more work to compute to
a specified time. It is possible to compute a guaranteed upper bound
on how far off the approximation might be, that is, how far $\ds y_n$
is from $f(t_n)$. Suffice it to say that the bound is not particularly
good and that there are other more complicated approximation
techniques that do better.

\begin{example} Let us compute an approximation to the solution for $\ds\dot
y=t-y^2$, $y(0)=0$, when $t=1$. We will use $\Delta t=0.2$, which is
easy to do even by hand, though we should not expect the resulting
approximation to be very good. We get
$$\eqalign{
(t_1,y_1)&=(0+0.2,0+(0-0^2)0.2) = (0.2,0) \\
(t_2,y_2)&=(0.2+0.2,0+(0.2-0^2)0.2) = (0.4,0.04) \\
(t_3,y_3)&=(0.6,0.04+(0.4-0.04^2)0.2) = (0.6,0.11968) \\
(t_4,y_4)&=(0.8,0.11968+(0.6-0.11968^2)0.2) = (0.8,0.23681533952) \\
(t_5,y_5)&=(1.0,0.23681533952+(0.6-0.23681533952^2)0.2) = (1.0,0.385599038513605) \\}
$$
So $y(1)\approx 0.3856$. As it turns out, this is not accurate to
even one decimal place. Figure~\xrefn{fig:euler method vs rk} shows
these points connected by line segments (the lower curve) compared to
a solution obtained by a much better approximation technique. Note
that the shape is approximately correct even though the end points are
quite far apart.

It is easy to write a short function in Sage to do Euler's method; see 
\expandafter\url\expandafter{\sageurl 2381}this Sage worksheet.\endurl
\end{example}

\figure
\vbox{\beginpicture
\normalgraphs
\sevenpoint
\setcoordinatesystem units <8truecm,8truecm>
\setplotarea x from 0 to 1.05, y from 0 to 0.55
\axis left shiftedto x=0 ticks length <2pt> numbered from 0 to 0.5 by 0.5 /
\axis bottom shiftedto y=0 ticks length <2pt> numbered from 0 to 1 by 1 /
\put {$t$} [tl] <0pt,-3pt> at 1.06 0
\put {$y$} [b] <0pt,3pt> at 0 0.56
\setlinear
\plot 
0 0
0.2 0.0000
0.4 0.0400
0.6 0.1196
0.8 0.2368
1.0 0.3855 /
\setquadratic
\plot 0  0  
0.025000000000000001  0.00031249951171971195 
0.050000000000000003  0.0012499843752447799  
0.075000000000000011 0.0028123813539146945  
0.10000000000000001 0.0049995000624952881  
0.125  0.0078109744935353819 
0.14999999999999999  0.0112462047261264  
0.17499999999999999 0.015304298951036615  
0.19999999999999998  0.019984015983738772 
0.22499999999999998  0.025283708473608849  
0.24999999999999997 0.031201267053176683  
0.27499999999999997  0.037734065705321729 
0.29999999999999999  0.044878908658235558  
0.32500000000000001 0.052631979147269851  
0.35000000000000003  0.060988790408866761 
0.37500000000000006  0.069944139294007487  
0.40000000000000008 0.079492062906380986  
0.4250000000000001  0.089625798683132307 
0.45000000000000012  0.10033774834298456  
0.47500000000000014 0.11161944612716239  
0.50000000000000011  0.12346153175236174 
0.52500000000000013  0.1358537284815772  
0.55000000000000016 0.14878482669758133  
0.57500000000000018  0.16224267333504916 
0.6000000000000002  0.17621416749066984  
0.62500000000000022 0.1906852624861928  
0.65000000000000024  0.20564097460749087 
0.67500000000000027  0.22106539868385133  
0.70000000000000029 0.23694173060647841  
0.72500000000000031  0.25325229681444411 
0.75000000000000033  0.26997859070107927  
0.77500000000000036 0.28710131581523379  
0.80000000000000038  0.30460043565128503 
0.8250000000000004  0.32245522974067869  
0.85000000000000042 0.34064435567768192  
0.87500000000000044  0.35914591663448664 
0.90000000000000047  0.37793753384740725  
0.92500000000000049 0.39699642348821867  
0.95000000000000051  0.41629947727413563 
0.97500000000000053  0.4358233461178857  
1.0000000000000004 0.45554452607694396 /
\setplotsymbol ({\teeny.})
\plotsymbolspacing=.2pt
\arrow <4pt> [0.35, 1] from 1.05 0 to 1.06 0
\arrow <4pt> [0.35, 1] from 0 0.55 to 0 0.56
\endpicture}
\figrdef{fig:euler method vs rk}
\endfigure{Approximating a solution to $\ds\dot y=t-y^2$, $y(0)=0$.}

Euler's method is related to another technique that can help in
understanding a differential equation in a qualitative way. Euler's
method is based on the ability to compute the slope of a solution
curve at any point in the plane, simply by computing $\phi(t,y)$. If
we compute $\phi(t,y)$ at many points, say in a grid, and plot a small
line segment with that slope at the point, we can get an idea of how
solution curves must look. Such a plot is called a 
{\dfont slope field\index{slope field}}. A slope field for 
$\ds\phi=t-y^2$ is shown in figure~\xrefn{fig:slope field}; compare
this to figure~\xrefn{fig:euler method vs rk}.
With a little practice, one can sketch reasonably accurate solution
curves based on the slope field, in essence doing Euler's method
visually.

\figure
\hbox to \hsize{\hss
\epsfxsize6.5cm\epsfbox{slope_field.eps}
\hss}
\figrdef{fig:slope field}
\endfigure{A slope field for $\ds\dot y=t-y^2$.}

Even when a differential equation can be solved explicitly, the slope
field can help in understanding what the solutions look like with
various initial conditions. Recall the logistic equation from
exercise~\xrefn{exer:logistic equation} in 
section~\xrefn{sec:first order differential equations}, $\ds\dot y =
ky(M-y)$: $y$ is a population at time $t$, $M$ is a measure of how
large a population the environment can support, and $k$ measures the
reproduction rate of the population.
Figure~\xrefn{fig:logistic slope field} shows a slope field
for this equation that is quite informative. 
It is apparent that if the initial population is smaller than $M$ it
rises to $M$ over the long term, while if the initial population is
greater than $M$ it decreases to $M$.
It is quite easy to
generate slope fields with Sage; follow the AP link in the figure caption.

\figure
\hbox to \hsize{\hss
\epsfxsize12cm\epsfbox{logistic_slope_field.eps}
\hss}
\figrdef{fig:logistic slope field}
\endfigure{A slope field for $\ds\dot y=0.2y(10-y)$.
(\expandafter\url\expandafter{\sageurl 2380}%
AP\endurl)}

\begin{exercises}

In problems 1--4, compute the Euler approximations for the initial
value problem for $0\le t\le 1$ and $\Delta t=0.2$. If you have access
to Sage, generate the slope field first and attempt to sketch the
solution curve. Then use Sage to compute better approximations with
smaller values of $\Delta t$.

\exercise $\ds\dot y=t/y$, $y(0)=1$
\begin{answer} $y(1)\approx 1.355$
\end{answer}

\exercise $\ds\dot y=t+y^3$, $y(0)=1$
\begin{answer} $y(1)\approx 40.31$
\end{answer}

\exercise $\ds\dot y=\cos(t+y)$, $y(0)=1$
\begin{answer} $y(1)\approx 1.05$
\end{answer}

\exercise $\ds\dot y=t\ln y$, $y(0)=2$
\begin{answer} $y(1)\approx 2.30$
\end{answer}

\end{exercises}
