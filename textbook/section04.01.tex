\section{Trigonometric Functions}{}{}
\nobreak
When you first encountered the trigonometric functions it was probably
in the context of ``triangle trigonometry,'' defining, for example,
the sine of an angle as the ``side opposite over the hypotenuse.''
While this will still be useful in an informal way, we need to use a
more expansive definition of the trigonometric functions. First an
important note: while degree measure of angles is sometimes convenient
because it is so familiar, it turns out to be ill-suited to
mathematical calculation, so (almost) everything we do will be in
terms of {\dfont radian\index{radian measure} measure\/} of angles.

To define the radian measurement system,
we consider the unit circle in the $xy$-plane:
% BADBAD
% $$\vbox{
% \beginpicture
% \normalgraphs
% \sevenpoint
% \setcoordinatesystem units <3truecm,3truecm>
% \setplotarea x from -1.1 to 1.1, y from -1.1 to 1.1
% \axis left shiftedto x=0 /
% \axis bottom shiftedto y=0 /
% \circulararc 360 degrees from 1 0 center at 0 0
% %\putrule from  0.8660254040 0 to 0.8660254040 0.5
% %\putrule from  1 0 to 1 .577
% \plot 0 0 0.8660254040 0.5 /
% \plot 0 0 0.8660254040 -0.5 /
% \put {$x$} [bl] <5pt,2.5pt> at 0.1 0
% \put {$(\cos x, \sin x)$} [bl] <3pt,3pt> at  0.8660254040 0.5
% \put {$y$} [tl] <5pt,-2.5pt> at 0.1 0
% \put {$A$} [r] <-4pt,2pt> at 0.8660254040 0.5
% \put {$B$} [r] <-4pt,-2pt> at 0.8660254040 -0.5
% \put {$(1,0)$} [tl] <2pt,-3pt> at 1 0
% \endpicture}$$

An angle, $x$, at the center of the circle is
associated with an arc of the circle which is said to {\dfont
  subtend\index{subtend}} the angle. In the figure, this arc is the portion of
the circle from point $(1,0)$ to point $A$. The length of this arc is
the radian measure of the angle $x$; the fact that the radian measure
is an actual geometric length is largely responsible for the
usefulness of radian measure. The circumference of the unit circle is
$2\pi r=2\pi(1)=2\pi$, so the radian measure of the full circular
angle (that is, of the 360 degree angle) is $2\pi$.

While an angle with a particular measure can appear anywhere around
the circle, we need a fixed, conventional location so that we can use
the coordinate system to define properties of the angle. 
The standard convention is to place the starting radius for the angle
on the positive $x$-axis, and to measure positive angles
counterclockwise around the circle. In the figure, $x$ is the standard
location of the angle $\pi/6$, that is, the length of the arc from
$(1,0)$ to $A$ is $\pi/6$. The angle $y$ in the picture is $-\pi/6$,
because the distance from $(1,0)$ to $B$ along the circle is also 
$\pi/6$, but in a clockwise direction.

Now the fundamental trigonometric definitions are:
the cosine of $x$ and the sine of $x$ are the first and second
coordinates of the point $A$, as indicated in the figure. The angle
$x$ shown can be viewed as an angle of a right triangle, meaning the
usual triangle definitions of the sine and cosine also make
sense. Since the hypotenuse of the triangle is 1, the ``side opposite
over hypotenuse'' definition of the sine is the second coordinate of
point $A$ over 1, which is just the second coordinate; in other words,
both methods give the same value for the sine.

The simple triangle definitions work only for angles that can ``fit''
in a right triangle, namely, angles between 0 and $\pi/2$. The
coordinate definitions, on the other hand, apply to any angles, as
indicated in this figure:
% BADBAD
% $$\vbox{
% \beginpicture
% \normalgraphs
% \sevenpoint
% \setcoordinatesystem units <3truecm,3truecm>
% \setplotarea x from -1.1 to 1.1, y from -1.1 to 1.1
% \axis left shiftedto x=0 /
% \axis bottom shiftedto y=0 /
% \circulararc 360 degrees from 1 0 center at 0 0
% \plot 0 0 -0.8660254040 -0.5 /
% \setplotsymbol ({\tenrm.})
% \circulararc 210 degrees from 1 0 center at 0 0
% \put {$x$} [br] <-5pt,5pt> at 0 0
% \put {$A$} [b] <2pt,5pt> at -0.8660254040 -0.5
% \put {$(\cos x, \sin x)$} [tr] <-3pt,-3pt> at  -0.8660254040 -0.5
% \endpicture}$$

The angle $x$ is subtended by the heavy arc in the figure, that is, 
$x=7\pi/6$. Both coordinates of point $A$ in this figure are negative,
so the sine and cosine of $7\pi/6$ are both negative.

The remaining trigonometric functions can be most easily defined in
terms of the sine and cosine, as usual:
\begin{align*}
\tan x &= {\sin x\over \cos x} \\
\cot x &= {\cos x \over \sin x} \\
\sec x &= {1\over \cos x} \\
\csc x &= {1\over \sin x} \\
\end{align*}
and they can also be defined as the corresponding ratios of
coordinates. 

Although the trigonometric functions are defined in terms of the unit
circle, the unit circle diagram is not what we normally consider the
graph of a trigonometric function. (The unit circle is the graph of,
well, the circle.) We can easily get a qualitatively correct idea of
the graphs of the trigonometric functions from the unit circle
diagram. Consider the sine function, $y=\sin x$. As $x$ increases from
0 in the unit circle diagram, the second coordinate of the point $A$
goes from 0 to a maximum of 1, then back to 0, then to a minimum of
$-1$, then back to 0, and then it obviously repeats itself. So the
graph of $y=\sin x$ must look something like this:

% BADBAD
% $$\vbox{
% \beginpicture
% \normalgraphs
% \sevenpoint
% \setcoordinatesystem units <1truecm,1truecm>
% \setplotarea x from -6.3 to 6.3, y from -1.1 to 1.1
% \axis left shiftedto x=0 ticks length <2pt> withvalues {$-1$} {$1$} / at -1 1 / /
% \axis bottom shiftedto y=0 ticks length <2pt> withvalues 
%   {$\pi/2$} {$\pi$} {$3\pi/2$} {$2\pi$} {$-\pi/2$} {$-\pi$} {$-3\pi/2$} {$-2\pi$} /
%   at 1.57 3.14 4.71 6.28 -1.57 -3.14 -4.71 -6.28 / /
% \setquadratic
% \plot -6.283 0.000 -6.158 0.125 -6.032 0.249 -5.906 0.368 -5.781 0.482 
% -5.655 0.588 -5.529 0.685 -5.404 0.771 -5.278 0.844 -5.152 0.905 
% -5.027 0.951 -4.901 0.982 -4.775 0.998 -4.650 0.998 -4.524 0.982 
% -4.398 0.951 -4.273 0.905 -4.147 0.844 -4.021 0.771 -3.896 0.685 
% -3.770 0.588 -3.644 0.482 -3.519 0.368 -3.393 0.249 -3.267 0.125 
% -3.142 0.000 -3.016 -0.125 -2.890 -0.249 -2.765 -0.368 -2.639 -0.482 
% -2.513 -0.588 -2.388 -0.685 -2.262 -0.771 -2.136 -0.844 -2.011 -0.905 
% -1.885 -0.951 -1.759 -0.982 -1.634 -0.998 -1.508 -0.998 -1.382 -0.982 
% -1.257 -0.951 -1.131 -0.905 -1.005 -0.844 -0.880 -0.771 -0.754 -0.685 
% -0.628 -0.588 -0.503 -0.482 -0.377 -0.368 -0.251 -0.249 -0.126 -0.125 
% 0.000 0.000 0.126 0.125 0.251 0.249 0.377 0.368 0.503 0.482 
% 0.628 0.588 0.754 0.685 0.880 0.771 1.005 0.844 1.131 0.905 
% 1.257 0.951 1.382 0.982 1.508 0.998 1.634 0.998 1.759 0.982 
% 1.885 0.951 2.011 0.905 2.136 0.844 2.262 0.771 2.388 0.685 
% 2.513 0.588 2.639 0.482 2.765 0.368 2.890 0.249 3.016 0.125 
% 3.142 0.000 3.267 -0.125 3.393 -0.249 3.519 -0.368 3.644 -0.482 
% 3.770 -0.588 3.896 -0.685 4.021 -0.771 4.147 -0.844 4.273 -0.905 
% 4.398 -0.951 4.524 -0.982 4.650 -0.998 4.775 -0.998 4.901 -0.982 
% 5.027 -0.951 5.152 -0.905 5.278 -0.844 5.404 -0.771 5.529 -0.685 
% 5.655 -0.588 5.781 -0.482 5.906 -0.368 6.032 -0.249 6.158 -0.125 
% 6.283 0.000 /
% \endpicture}$$

Similarly, as angle $x$ increases from 0 in the unit circle diagram,
the first coordinate of the point $A$ goes from 1 to 0 then to $-1$,
back to 0 and back to 1, so the graph of $y=\cos x$ must look
something like this:

% BADBAD
% $$\vbox{
% \beginpicture
% \normalgraphs
% \sevenpoint
% \setcoordinatesystem units <1truecm,1truecm>
% \setplotarea x from -6.3 to 6.3, y from -1.1 to 1.1
% \axis left shiftedto x=0 ticks length <2pt> withvalues {$-1$} {$1$} / at -1 1 / /
% \axis bottom shiftedto y=0 ticks length <2pt> withvalues 
%   {$\pi/2$} {$\pi$} {$3\pi/2$} {$2\pi$} {$-\pi/2$} {$-\pi$} {$-3\pi/2$} {$-2\pi$} /
%   at 1.57 3.14 4.71 6.28 -1.57 -3.14 -4.71 -6.28 / /
% \setquadratic
% \plot
% -6.283 1.000 -6.178 0.995 -6.074 0.978 -5.969 0.951 -5.864 0.914 
% -5.760 0.866 -5.655 0.809 -5.550 0.743 -5.445 0.669 -5.341 0.588 
% -5.236 0.500 -5.131 0.407 -5.027 0.309 -4.922 0.208 -4.817 0.105 
% -4.712 0.000 -4.608 -0.105 -4.503 -0.208 -4.398 -0.309 -4.294 -0.407 
% -4.189 -0.500 -4.084 -0.588 -3.979 -0.669 -3.875 -0.743 -3.770 -0.809 
% -3.665 -0.866 -3.560 -0.914 -3.456 -0.951 -3.351 -0.978 -3.246 -0.995 
% -3.142 -1.000 -3.037 -0.995 -2.932 -0.978 -2.827 -0.951 -2.723 -0.914 
% -2.618 -0.866 -2.513 -0.809 -2.409 -0.743 -2.304 -0.669 -2.199 -0.588 
% -2.094 -0.500 -1.990 -0.407 -1.885 -0.309 -1.780 -0.208 -1.676 -0.105 
% -1.571 0.000 -1.466 0.105 -1.361 0.208 -1.257 0.309 -1.152 0.407 
% -1.047 0.500 -0.942 0.588 -0.838 0.669 -0.733 0.743 -0.628 0.809 
% -0.524 0.866 -0.419 0.914 -0.314 0.951 -0.209 0.978 -0.105 0.995 
% 0.000 1.000 0.105 0.995 0.209 0.978 0.314 0.951 0.419 0.914 
% 0.524 0.866 0.628 0.809 0.733 0.743 0.838 0.669 0.942 0.588 
% 1.047 0.500 1.152 0.407 1.257 0.309 1.361 0.208 1.466 0.105 
% 1.571 0.000 1.676 -0.105 1.780 -0.208 1.885 -0.309 1.990 -0.407 
% 2.094 -0.500 2.199 -0.588 2.304 -0.669 2.409 -0.743 2.513 -0.809 
% 2.618 -0.866 2.723 -0.914 2.827 -0.951 2.932 -0.978 3.037 -0.995 
% 3.142 -1.000 3.246 -0.995 3.351 -0.978 3.456 -0.951 3.560 -0.914 
% 3.665 -0.866 3.770 -0.809 3.875 -0.743 3.979 -0.669 4.084 -0.588 
% 4.189 -0.500 4.294 -0.407 4.398 -0.309 4.503 -0.208 4.608 -0.105 
% 4.712 0.000 4.817 0.105 4.922 0.208 5.027 0.309 5.131 0.407 
% 5.236 0.500 5.341 0.588 5.445 0.669 5.550 0.743 5.655 0.809 
% 5.760 0.866 5.864 0.914 5.969 0.951 6.074 0.978 6.178 0.995 
% 6.283 1.000 /
% \endpicture}$$

\begin{exercises} Some useful trigonometric identities are in
appendix~\xrefn{chap:formulas}.

\exercise Find all values of $\theta$ such that
$\sin(\theta) = -1$; give your answer in radians.
\begin{answer} $2n\pi-\pi/2$, any integer $n$
\end{answer}

\exercise Find all values of $\theta$ such that
$\cos(2\theta) = 1/2$; give your answer in radians.
\begin{answer} $n\pi\pm\pi/6$, any integer $n$
\end{answer}

\exercise Use an angle sum identity to compute
  $\cos(\pi/12)$.
\begin{answer} $(\sqrt2+\sqrt6)/4$
\end{answer}

\exercise Use an angle sum identity to compute
  $\tan(5\pi/12)$.
\begin{answer} $-(1+\sqrt3)/(1-\sqrt3)=2+\sqrt3$
\end{answer}

\exercise Verify the identity $\ds \cos^2(t)/(1-\sin(t)) =
  1+\sin(t)$.

\exercise Verify the identity $2\csc(2\theta)=\sec(\theta)\csc(\theta)$.

\exercise Verify the identity $\sin(3\theta) - \sin(\theta) = 2\cos(2\theta)
  \sin(\theta)$.

\exercise Sketch $y=2\sin(x)$.

\exercise Sketch $y=\sin(3x)$.

\exercise Sketch $y=\sin(-x)$.

\exercise Find all of the solutions of $\ds 2\sin(t) -1 -\sin^2(t) =0$ in the
 interval $[0,2\pi]$.
\begin{answer} $t=\pi/2$
\end{answer}

\end{exercises}
