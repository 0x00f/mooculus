\section{Numerical Integration}{}{}
\nobreak 
We have now seen some of the most generally useful methods
for discovering antiderivatives, and there are others. 
Unfortunately, some functions
have no simple antiderivatives; in such cases if the value of a
definite integral is needed it will have to be approximated. We will
see two methods that work reasonably well and yet are fairly simple;
in some cases more sophisticated techniques will be needed.

Of course, we already know one way to approximate an integral: if we
think of the integral as computing an area, we can add up the areas of
some rectangles. While this is quite simple, it is usually the case
that a large number of rectangles is needed to get acceptable
accuracy. A similar approach is much better: we approximate the area
under a curve over a small interval as the area of a trapezoid. In
figure~\xrefn{fig:trapezoid approximation} we see an area under 
a curve approximated by rectangles and by trapezoids; it is apparent
that the trapezoids give a substantially better approximation on each
subinterval. 

\figure
\vbox{\beginpicture
\normalgraphs
\ninepoint
\setcoordinatesystem units <1.25truecm,0.5truecm>
\setplotarea x from 0 to 4.1, y from 0 to 7.1
\axis left shiftedto x=0 /
\axis bottom shiftedto y=0 /
\setquadratic
\plot 0.000 4.000 0.200 5.368 0.400 6.304 0.600 6.856 0.800 7.072
1.000 7.000 1.200 6.688 1.400 6.184 1.600 5.536 1.800 4.792
2.000 4.000 2.200 3.208 2.400 2.464 2.600 1.816 2.800 1.312
3.000 1.000 3.200 0.928 3.400 1.144 3.600 1.696 3.800 2.632
4.000 4.000 /
\setdashes <2pt>
\putrule from 0 4 to 1 4
\putrule from 1 7 to 1 0
\putrule from 1 7 to 2 7
\putrule from 2 7 to 2 0
\putrule from 2 4 to 3 4
\putrule from 3 4 to 3 0
\putrule from 3 1 to 4 1
\putrule from 4 1 to 4 0
\setsolid
\setcoordinatesystem units <1.25truecm,0.5truecm> point at -6 0
\setplotarea x from 0 to 4.1, y from 0 to 7.1
\axis left shiftedto x=0 /
\axis bottom shiftedto y=0 /
\setquadratic
\plot 0.000 4.000 0.200 5.368 0.400 6.304 0.600 6.856 0.800 7.072
1.000 7.000 1.200 6.688 1.400 6.184 1.600 5.536 1.800 4.792
2.000 4.000 2.200 3.208 2.400 2.464 2.600 1.816 2.800 1.312
3.000 1.000 3.200 0.928 3.400 1.144 3.600 1.696 3.800 2.632
4.000 4.000 /
\setlinear
\setdashes <2pt>
\plot 0 4 1 7 2 4 3 1 4 4 /
\putrule from 1 0 to 1 7
\putrule from 2 0 to 2 4
\putrule from 3 0 to 3 1
\putrule from 4 0 to 4 4
\endpicture}
\figrdef{fig:trapezoid approximation}
\endfigure{Approximating an area with rectangles and with trapezoids.
(\expandafter\url\expandafter{\liveurl jsxgraph/numerical_int.html}%
AP\endurl)}

As with rectangles, we divide the interval into $n$ equal subintervals
of length $\Delta x$.
A typical trapezoid is pictured in figure~\xrefn{fig:one trapezoid};
it has area $\ds{f(x_i)+f(x_{i+1})\over2}\Delta x$. If we add up the
  areas of all trapezoids we get
$$
  \eqalign{
  {f(x_0)+f(x_1)\over2}\Delta x&+{f(x_1)+f(x_2)\over2}\Delta x+\cdots+
  {f(x_{n-1})+f(x_n)\over2}\Delta x= \\
  &\left({f(x_0)\over2}+f(x_1)+f(x_2)+\cdots+f(x_{n-1})+{f(x_n)\over2}\right)
  \Delta x. \\}
$$ 
For a modest number of subintervals this is not too difficult to do
with a calculator; a computer can easily do many subintervals.

\figure
\vbox{\beginpicture
\normalgraphs
\ninepoint
\setcoordinatesystem units <1.25truecm,0.5truecm>
\setplotarea x from 0 to 3.0, y from 0 to 7.1
\axis bottom shiftedto y=0 ticks length <2pt> 
  withvalues {$x_i$} {$x_{i+1}$} / at 1 2 / /
\put {$(x_i,f(x_i))$} [r] <-3pt,0pt> at 1 7
\put {$(x_{i+1},f(x_{i+1}))$} [l] <3pt,0pt> at 2 4
\setquadratic
\plot 1.000 7.000 1.167 6.755 1.333 6.370 1.500 5.875 1.667 5.296
1.833 4.662 2.000 4.000 /
\setlinear
\setdashes <2pt>
\plot 1 7 2 4  /
\putrule from 2 0 to 2 4
\putrule from 1 0 to 1 7
\endpicture}
\figrdef{fig:one trapezoid}
\endfigure{A single trapezoid.}

In practice, an approximation is useful only if we know how accurate
it is; for example, we might need a particular value accurate to three
decimal places. When we compute a particular approximation to an
integral, the error is the difference between the approximation and
the true value of the integral. For any approximation technique, we
need an {\dfont error estimate\index{error estimate}\/}, a value that
is guaranteed to be larger than the actual error. If $A$ is an
approximation and $E$ is the associated error estimate, then we know
that the true value of the integral is between $A-E$ and
$A+E$. In the case of our approximation of the integral, we want
$E=E(\Delta x)$ to be a function of $\Delta x$ that gets small rapidly
as $\Delta x$ gets small. Fortunately, for many functions, there is
such an error estimate associated with the trapezoid approximation.

\begin{theorem} Suppose $f$ has a second derivative $f''$ everywhere on the
interval $[a,b]$, and $|f''(x)|\le M$ for all $x$ in the
interval. With $\Delta x= (b-a)/n$, an error estimate for the
trapezoid approximation is
$$
  E(\Delta x) = {b-a\over12}M(\Delta x)^2={(b-a)^3\over 12n^2}M.
$$
\end{theorem}

Let's see how we can use this. 

\begin{example} Approximate $\ds\int_0^1 e^{-x^2}\,dx$ to two decimal places.
The second derivative of $\ds f=e^{-x^2}$ is $\ds(4x^2-2)e^{-x^2}$, and
it is not hard to see that on $[0,1]$, $\ds|(4x^2-2)e^{-x^2}|\le 2$.
We begin by estimating the number of subintervals we are likely to
need. To get two decimal places of accuracy, we will certainly need
$E(\Delta x)<0.005$ or
$$
  \eqalign{
  {1\over12}(2){1\over n^2} &< 0.005 \\
  {1\over6}(200)&<n^2 \\
  5.77\approx\sqrt{100\over3}&<n \\}
$$
With $n=6$, the error estimate is thus $\ds1/6^3<0.0047$.
We compute the trapezoid approximation for six intervals:
$$
  \left({f(0)\over2}+f(1/6)+f(2/6)+\cdots+f(5/6)+{f(1)\over2}\right){1\over6}
  \approx 0.74512.
$$
So the true value of the integral is between $0.74512-0.0047=0.74042$ and
$0.74512+0.0047=0.74982$. Unfortunately, the first rounds to $0.74$
and the second 
rounds to $0.75$, so we can't be sure of the correct value in
the second decimal place; we need to pick a larger $n$.
As it turns out, we need to go to $n=12$ to get two bounds that both
round to the same value, which turns out to be $0.75$. For comparison,
using $12$ rectangles to approximate the area gives $0.7727$, which is
considerably less accurate than the approximation using six trapezoids.

In practice it
generally pays to start by requiring better than the maximum possible
error; for example, we might have initially required $E(\Delta
x)<0.001$, or 
$$
  \eqalign{
  {1\over12}(2){1\over n^2} &< 0.001 \\
  {1\over6}(1000)&<n^2 \\
  12.91\approx\sqrt{500\over3}&<n \\}
$$
Had we immediately tried $n=13$ this would have given us the desired
answer. 
\end{example}

The trapezoid approximation works well, especially compared to
rectangles, because the tops of the trapezoids form a reasonably good
approximation to the curve when $\Delta x$ is fairly small. We can
extend this idea: what if we try to approximate the curve more
closely, by using something other than a straight line? The obvious
candidate is a parabola: if we can approximate a short piece of the
curve with a parabola with equation $\ds y=ax^2+bx+c$, we can easily
compute the area under the parabola.

There are an infinite number of parabolas through any two given
points, but only one through three given points. If we find a parabola
through three consecutive points $(x_i,f(x_i))$,
$(x_{i+1},f(x_{i+1}))$, $(x_{i+2},f(x_{i+2}))$ on the curve, it should
be quite close to the curve over the whole interval $[x_i,x_{i+2}]$,
as in figure~\xrefn{fig:one parabola}. If we divide the interval
$[a,b]$ into an even number of subintervals, we can then approximate
the curve by a sequence of parabolas, each covering two of the
subintervals. For this to be practical, we would like a simple formula
for the area under one parabola, namely, the parabola through
$(x_i,f(x_i))$, $(x_{i+1},f(x_{i+1}))$, and
$(x_{i+2},f(x_{i+2}))$. That is, we should attempt to write down the
parabola $y=ax^2+bx+c$ through these points and then integrate it, and
hope that the result is fairly simple. Although the algebra involved
is messy, this turns out to be possible. The algebra is well within
the capability of a good computer algebra system like Sage, so we will
present the result without all of the algebra; you can see how to do
it in this \expandafter\url\expandafter{\sageurl 3956}% Sage
worksheet\endurl. To find the parabola, we solve these three equations
for $a$, $b$, and $c$:
$$
  \eqalign{
  f(x_i)&=a(x_{i+1}-\Delta x)^2+b(x_{i+1}-\Delta x)+c \\
  f(x_{i+1})&=a(x_{i+1})^2+b(x_{i+1})+c \\
  f(x_{i+2})&=a(x_{i+1}+\Delta x)^2+b(x_{i+1}+\Delta x)+c \\}
$$
Not surprisingly, the solutions turn out to be quite
messy. Nevertheless, Sage can easily compute and simplify the integral
to get
$$
  \int_{x_{i+1}-\Delta x}^{x_{i+1}+\Delta x} ax^2+bx+c\,dx=
  {\Delta x\over3}(f(x_i)+4f(x_{i+1})+f(x_{i+2})).
$$
Now the sum of the areas under all parabolas is
$$
  \displaylines{
  {\Delta x\over3}(f(x_0)+4f(x_{1})+f(x_{2})+f(x_2)+4f(x_{3})+f(x_{4})+\cdots
  +f(x_{n-2})+4f(x_{n-1})+f(x_{n}))= \\
  {\Delta x\over3}(f(x_0)+4f(x_{1})+2f(x_{2})+4f(x_{3})+2f(x_{4})+\cdots
  +2f(x_{n-2})+4f(x_{n-1})+f(x_{n})). \\}
$$
This is just slightly more complicated than the formula for
trapezoids; we need to remember the alternating 2 and 4 coefficients.
This approximation technique is referred to as 
{\dfont Simpson's Rule\index{Simpson's Rule}}.

\figure
\vbox{\beginpicture
\normalgraphs
\ninepoint
\setcoordinatesystem units <3truecm,8truecm>
\setplotarea x from 0.8 to 2.2, y from 0 to 0.4
\axis bottom shiftedto y=0 ticks length <2pt> 
  withvalues {$x_i$} {$x_{i+1}$} {$x_{i+2}$} / at 1 1.5 2 / /
\put {$(x_i,f(x_i))$} [r] <-3pt,0pt> at 1 0.163
\put {$(x_{i+2},f(x_{i+2}))$} [l] <3pt,0pt> at 2 0.381
\setquadratic
\plot 1.000 0.163 1.050 0.138 1.100 0.120 1.150 0.109 1.200 0.102
1.250 0.100 1.300 0.102 1.350 0.107 1.400 0.116 1.450 0.128
1.500 0.142 1.550 0.158 1.600 0.177 1.650 0.197 1.700 0.219
1.750 0.243 1.800 0.268 1.850 0.295 1.900 0.322 1.950 0.351
2.000 0.381 /
\setdashes <2pt>
\plot 
1.000 0.162 1.050 0.149 1.100 0.137 1.150 0.129 1.200 0.123
1.250 0.120 1.300 0.119 1.350 0.121 1.400 0.125 1.450 0.132
1.500 0.142 1.550 0.154 1.600 0.169 1.650 0.186 1.700 0.206
1.750 0.229 1.800 0.254 1.850 0.282 1.900 0.312 1.950 0.346
2.000 0.381 /
\endpicture}
\figrdef{fig:one parabola}
\endfigure{A parabola (dashed) approximating a curve (solid).
(\expandafter\url\expandafter{\liveurl jsxgraph/numerical_int.html}%
AP\endurl)}

As with the trapezoid method, this is useful only with an error
estimate:

\begin{theorem} Suppose $f$ has a fourth derivative $f^{(4)}$ everywhere on the
interval $[a,b]$, and $|f^{(4)}(x)|\le M$ for all $x$ in the
interval. With $\Delta x= (b-a)/n$, an error estimate for Simpson's
approximation is
$$
  E(\Delta x) = {b-a\over180}M(\Delta x)^4={(b-a)^5\over 180n^4}M.
$$
\end{theorem}

\begin{example} Let us again approximate $\ds\int_0^1 e^{-x^2}\,dx$ to two
decimal places.  The fourth derivative of $\ds f=e^{-x^2}$ is
$\ds(16x^2-48x^2+12)e^{-x^2}$; on $[0,1]$ this is at most
$12$ in absolute value.  We begin by estimating the number of
subintervals we are likely to need. To get two decimal places of
accuracy, we will certainly need $E(\Delta x)<0.005$, but taking a cue
from our earlier example, let's require $E(\Delta x)<0.001$:
$$
  \eqalign{
  {1\over180}(12){1\over n^4} &< 0.001 \\
  {200\over3}&<n^4 \\
  2.86\approx\root[4] \of {200\over3}&<n \\}
$$
So we try $n=4$, since we need an even number of subintervals. Then
the error estimate is $\ds12/180/4^4<0.0003$ and the approximation is
$$
  (f(0)+4f(1/4)+2f(1/2)+4f(3/4)+f(1)){1\over3\cdot4}
  \approx 0.746855.
$$
So the true value of the integral is between $0.746855-0.0003=0.746555$ and
$0.746855+0.0003=0.7471555$, both of which round to $0.75$.
\end{example}

\begin{exercises}

In the following problems, compute the trapezoid and Simpson
approximations using 4 subintervals, and compute the error estimate
for each. (Finding the maximum values of the second and fourth
derivatives can be challenging for some of these; you may use a
graphing calculator or computer software to estimate the maximum
values.)  If you have access to Sage or similar software, approximate
each integral to two decimal places.  You can use this
\expandafter\url\expandafter{\sageurl 4005}% 
Sage worksheet\endurl\ to
get started.

\twocol

\begin{exercise} $\ds\int_1^3 x\,dx$
\begin{answer} T,S: $4\pm0$
\end{answer}\end{exercise}

\begin{exercise} $\ds\int_0^3 x^2\,dx$
\begin{answer} T: $9.28125\pm0.281125 $; S: $9\pm0$
\end{answer}\end{exercise}

\begin{exercise} $\ds\int_2^4 x^3\,dx$
\begin{answer} T: $60.75\pm1$; S: $60\pm0$
\end{answer}\end{exercise}

\begin{exercise} $\ds\int_1^3 {1\over x}\,dx$
\begin{answer} T: $1.1167\pm 0.0833$; S: $1.1000\pm 0.0167$
\end{answer}\end{exercise}

\begin{exercise} $\ds\int_1^2 {1\over 1+x^2}\,dx$
\begin{answer} T: $0.3235\pm 0.0026$; S: $0.3217\pm 0.000065$
\end{answer}\end{exercise}

\begin{exercise} $\ds\int_0^1 x\sqrt{1+x}\,dx$
\begin{answer} T: $0.6478\pm 0.0052$; S: $0.6438\pm 0.000033$
\end{answer}\end{exercise}

\begin{exercise} $\ds\int_1^5 {x\over 1+x}\,dx$
\begin{answer} T: $2.8833\pm 0.0834$; S: $2.9000\pm 0.0167$
\end{answer}\end{exercise}

\begin{exercise} $\ds\int_0^1 \sqrt{x^3+1}\,dx$
\begin{answer} T: $1.1170\pm 0.0077$; S: $1.1114\pm 0.0002$
\end{answer}\end{exercise}

\begin{exercise} $\ds\int_0^1 \sqrt{x^4+1}\,dx$
\begin{answer} T: $1.097\pm 0.0147$; S: $1.089\pm 0.0003$
\end{answer}\end{exercise}

\begin{exercise} $\ds\int_1^4 \sqrt{1+1/x}\,dx$
\begin{answer} T: $3.63\pm 0.087$; S: $3.62\pm 0.032$
\end{answer}\end{exercise}

\endtwocol

\msk
\begin{exercise} Using Simpson's rule on a parabola $f(x)$, even with just
two subintervals, gives the exact value of the integral, because the
parabolas used to approximate $f$ will be $f$ itself. Remarkably,
Simpson's rule also computes the integral of a cubic function
$f(x)=ax^3+bx^2+cx+d$ exactly. Show this is true by showing that
$$
  \int_{x_0}^{x_2}
  f(x)\,dx={x_2-x_0\over3\cdot2}(f(x_0)+4f((x_0+x_2)/2)+f(x_2)).
$$
This does require a bit of messy algebra, so you may prefer to use Sage.

\end{exercises}
