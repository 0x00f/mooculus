% From Albert Schueller 8/29/2007{}
\section{Adjectives For Functions}{}{}
\label{sec:adjectives for functions}

As we have defined it in Section~\xrefn{sec:functions}, a function is
a very general object.  At this point, it is useful to introduce a
collection of adjectives to describe certain kinds of functions; these
adjectives name useful properties that functions may have.  Consider
the graphs of the functions in Figure~\xrefn{fig:function types}.  It
would clearly be useful to have words to help us describe the distinct
features of each of them.  We will point out and define a few
adjectives (there are many more) for the functions pictured here.  For
the sake of the discussion, we will assume that the graphs do not
exhibit any unusual behavior off-stage (i.e., outside the view of the
graphs).

%Figure, (a) rough function, (b) continuous but not differentiable
%function, (c) differentiable function bounded by $\pm 1$, (d) unbounded
%parabola function.
% BADBAD
% \figure
% \vbox{\beginpicture
% \normalgraphs
% \sevenpoint
% \longticklength3pt
% % exp(-x^2)
% \setcoordinatesystem units <1truecm,1truecm>
% \setplotarea x from -3 to 3, y from -3 to 3
% \axis left shiftedto x=0 ticks length <2pt> numbered from -3 to -1 by 1 /
% \axis left shiftedto x=0 ticks length <2pt> numbered from 1 to 3 by 1 /
% \axis bottom shiftedto y=0 ticks length <2pt> numbered from -3 to -1 by 1 /
% \axis bottom shiftedto y=0 ticks length <2pt> numbered from 1 to 3 by 1 /
% \setquadratic
% \plot -3.000 0.000 -2.850 0.000 -2.700 0.001 -2.550 0.001 -2.400 0.003 
% -2.250 0.006 -2.100 0.012 -1.950 0.022 -1.800 0.039 -1.650 0.066 
% -1.500 0.105 -1.350 0.162 -1.200 0.237 -1.050 0.332 -0.900 0.445 
% -0.750 0.570 -0.600 0.698 -0.450 0.817 -0.300 0.914 -0.150 0.978 
% 0.000 1.000 0.150 0.978 0.300 0.914 0.450 0.817 0.600 0.698 
% 0.750 0.570 0.900 0.445 1.050 0.332 1.200 0.237 1.350 0.162 
% 1.500 0.105 1.650 0.066 1.800 0.039 1.950 0.022 2.100 0.012 
% 2.250 0.006 2.400 0.003 2.550 0.001 2.700 0.001 2.850 0.000 
% 3.000 0.000 /
% \put {(c)} [b] <0pt,3pt> at 0 3
% \setcoordinatesystem units <1truecm,1truecm> point at 0 -7
% \setplotarea x from -3 to 3, y from -3 to 3
% \axis left shiftedto x=0 ticks length <2pt> numbered from -3 to -1 by 1 /
% \axis left shiftedto x=0 ticks length <2pt> numbered from 1 to 3 by 1 /
% \axis bottom shiftedto y=0 ticks length <2pt> numbered from -3 to -1 by 1 /
% \axis bottom shiftedto y=0 ticks length <2pt> numbered from 1 to 3 by 1 /
% \setlinear
% \putrule from -3 1 to -1 1
% \putrule from 2 -0.5 to 3 -0.5
% \plot 1 1 2 0 /
% \setquadratic
% \plot -1.000 -1.000 -0.967 -1.034 -0.933 -1.071 -0.900 -1.111 -0.867 -1.154 
% -0.833 -1.200 -0.800 -1.250 -0.767 -1.304 -0.733 -1.364 -0.700 -1.429 
% -0.667 -1.500 -0.633 -1.579 -0.600 -1.667 -0.567 -1.765 -0.533 -1.875 
% -0.500 -2.000 -0.467 -2.143 -0.433 -2.308 -0.400 -2.500 -0.367 -2.727 
% -0.333 -3.000 /
% \plot 0.333 3.000 0.367 2.727 0.400 2.500 0.433 2.308 0.467 2.143 
% 0.500 2.000 0.533 1.875 0.567 1.765 0.600 1.667 0.633 1.579 
% 0.667 1.500 0.700 1.429 0.733 1.364 0.767 1.304 0.800 1.250 
% 0.833 1.200 0.867 1.154 0.900 1.111 0.933 1.071 0.967 1.034 
% 1.000 1.000 /
% \put {(a)} [b] <0pt,3pt> at 0 3
% \setcoordinatesystem units <1truecm,1truecm> point at -8 -7
% \setplotarea x from -3 to 3, y from -3 to 3
% \axis left shiftedto x=0 ticks length <2pt> numbered from -3 to -1 by 1 /
% \axis left shiftedto x=0 ticks length <2pt> numbered from 1 to 3 by 1 /
% \axis bottom shiftedto y=0 ticks length <2pt> numbered from -3 to -1 by 1 /
% \axis bottom shiftedto y=0 ticks length <2pt> numbered from 1 to 3 by 1 /
% \setlinear
% \putrule from 1 1 to 3 1
% \plot -3 1 -1 3 0 0 /
% \setquadratic
% \plot
% 0.000 0.000 0.012 0.112 0.025 0.158 0.038 0.194 0.050 0.224 
% 0.062 0.250 0.075 0.274 0.088 0.296 0.100 0.316 0.112 0.335 
% 0.125 0.354 0.138 0.371 0.150 0.387 0.162 0.403 0.175 0.418 
% 0.188 0.433 0.200 0.447 0.212 0.461 0.225 0.474 0.238 0.487 
% 0.250 0.500 / 
% \plot
% 0.250 0.500 0.288 0.536 0.325 0.570 0.362 0.602 0.400 0.632 
% 0.438 0.661 0.475 0.689 0.512 0.716 0.550 0.742 0.588 0.766 
% 0.625 0.791 0.662 0.814 0.700 0.837 0.738 0.859 0.775 0.880 
% 0.812 0.901 0.850 0.922 0.888 0.942 0.925 0.962 0.962 0.981 
% 1.000 1.000 /
% \put {(b)} [b] <0pt,3pt> at 0 3
% \setcoordinatesystem units <1truecm,1truecm> point at -8 0
% \setplotarea x from -3 to 3, y from -3 to 3
% \axis left shiftedto x=0 ticks length <2pt> length <2pt> numbered from -3 to -1 by 1 /
% \axis left shiftedto x=0 ticks length <2pt> length <2pt> numbered from 1 to 3 by 1 /
% \axis bottom shiftedto y=0 ticks length <2pt> numbered from -3 to -1 by 1 /
% \axis bottom shiftedto y=0 ticks length <2pt> numbered from 1 to 3 by 1 /
% \setquadratic
% \plot 0.333 3.000 0.467 2.143 0.600 1.667 0.733 1.364 0.867 1.154 
% 1.000 1.000 1.133 0.882 1.267 0.789 1.400 0.714 1.533 0.652 
% 1.667 0.600 1.800 0.556 1.933 0.517 2.067 0.484 2.200 0.455 
% 2.333 0.429 2.467 0.405 2.600 0.385 2.733 0.366 2.867 0.349 
% 3.000 0.333 /
% \plot -3.000 -0.333 -2.867 -0.349 -2.733 -0.366 -2.600 -0.385 -2.467 -0.405 
% -2.333 -0.429 -2.200 -0.455 -2.067 -0.484 -1.933 -0.517 -1.800 -0.556 
% -1.667 -0.600 -1.533 -0.652 -1.400 -0.714 -1.267 -0.789 -1.133 -0.882 
% -1.000 -1.000 -0.867 -1.154 -0.733 -1.364 -0.600 -1.667 -0.467 -2.143 
% -0.333 -3.000 /
% \put {(d)} [b] <0pt,3pt> at 0 3
% \endpicture}
% \figrdef{fig:function types}
% \endfigure{Function Types: (a) a discontinuous function, (b) a
%   continuous function, (c) a bounded, differentiable function, (d) an
%   unbounded, differentiable function}

\noindent{\bf Functions.}  Each graph in
Figure~\xrefn{fig:function types}
certainly represents a function---since each passes the {\em vertical
line test}.  In other words, as you sweep a vertical line across the graph
of each function, the line never intersects the graph more than once.  If
it did, then the graph would not represent a function.

\noindent{\bf Bounded.} The graph in (c) appears to approach zero as $x$ goes
to both positive and negative infinity.  It also never exceeds the value
$1$ or drops below the value $0$.  Because the graph never increases or
decreases without bound, we say that the function represented by the graph
in (c) is a {\dfont bounded} 
function\index{function!bounded}\index{bounded function}.  

\begin{definition} (Bounded) A function $f$ is bounded if there is a
number $M$ such that $|f(x)| < M$ for every $x$ in the domain of $f$.
\end{definition}

For the function in (c), one such choice for $M$ would be $10$.  However,
the smallest (optimal) choice would be $M=1$.  In either case, simply
finding an $M$ is enough to establish boundedness.  No such $M$ exists for
the hyperbola in (d) and hence we can say that it is 
{\dfont unbounded}\index{function!unbounded}\index{unbounded function}.  

\noindent{\bf Continuity.}  The graphs shown in (b) and (c) both represent
{\bf continuous} functions.  Geometrically, this is because there are no
jumps in the graphs.  That is, if you pick a point on the graph and
approach it from the left and right, the values of the function approach
the value of the function at that point.  For example, we can see that this
is not true for function values near $x=-1$ on the graph in (a) which is
not continuous at that location.  

\begin{definition} (Continuous at a Point) A function $f$ is continuous
at a point $a$ if $\ds \lim_{x\to a} f(x) = f(a)$.  
\end{definition}
\index{continuous}

\begin{definition} (Continuous) A function $f$ is continuous if it is
continuous at every point in its domain.
\end{definition}

Strangely, we can also say that (d) is continuous even though there is a
vertical asymptote.  A careful reading of the definition of continuous
reveals the phrase ``{\em at every point in its domain.}''  Because the
location of the asymptote, $x=0$, is not in the domain of the function, and
because the rest of the function is {\em well-behaved}, we can say that (d)
is continuous.

\noindent{\bf Differentiability.} Now that we have introduced the derivative
of a function at a point, we can begin to use the adjective {\dfont
differentiable}\index{differentiable}\index{function!differentiable}.  
We can see that the tangent line is well-defined at every
point on the graph in (c).  Therefore, we can say that (c) is a
differentiable function.

\begin{definition} (Differentiable at a Point) A function $f$ is
differentiable at point $a$ if $f'(a)$ exists.
\end{definition}

\begin{definition} (Differentiable) A function $f$ is differentiable if is
differentiable at every point (excluding endpoints and isolated points in the
domain of $f$) in the domain of $f$.
\end{definition}

Take note that, for technical reasons not discussed here, both of these
definitions exclude endpoints and isolated points in the domain from
consideration.

We now have a collection of adjectives to describe the very rich and
complex set of objects known as functions.

We close with a useful theorem about continuous functions:

\begin{theorem} (Intermediate Value Theorem) If $f$ is continuous on the interval
$[a,b]$ and $d$ is between $f(a)$ and $f(b)$, then there is a number
$c$ in $[a,b]$ such that $f(c)=d$. 
\end{theorem}
\label{thm:intermediate value theorem}
\index{Intermediate Value Theorem}

This is most frequently used when $d=0$.

\begin{example} Explain why the function $\ds f=x^3 + 3x^2+x-2$ has a root between 0
and 1.

By theorem~\xrefn{thm:properties of limits}, $f$ is continuous.
Since $f(0)=-2$ and $f(1)=3$, and $0$ is between $-2$ and $3$, there
is a $c\in[0,1]$ such that $f(c)=0$.
\end{example}

This example also points the way to a simple method for approximating
roots. 

\begin{example} Approximate the root of the previous example  to one decimal
place.

If we compute $f(0.1)$, $f(0.2)$, and so on, we find that 
$f(0.6)<0$ and $f(0.7)>0$, so by the Intermediate Value Theorem, $f$
has a root between $0.6$ and $0.7$. Repeating the process with
$f(0.61)$, $f(0.62)$, and so on, we find that
$f(0.61)<0$ and $f(0.62)>0$, so $f$ has a root between
$0.61$ and $0.62$, and the root is $0.6$ rounded to one decimal place.
\end{example}

\begin{exercises}

\begin{exercise} Along the lines of Figure~\xrefn{fig:function types},
for each part below sketch the
 graph of a function that is:

\begin{itemize} % BADBAD

  \item{a.} bounded, but not continuous.

  \item{b.} differentiable and unbounded.

  \item{c.} continuous at $x=0$, not continuous at $x=1$, and bounded.
  
  \item{d.} differentiable everywhere except at $x=-1$, continuous, and unbounded.

 \end{itemize}
\end{exercise}

\begin{exercise} Is $f(x)=\sin(x)$ a bounded function?  If so, find the smallest $M$.
\end{exercise}

\begin{exercise} Is $\ds s(t) = 1/(1+t^2)$ a bounded function?  If so, find the
smallest $M$.
\end{exercise}

 \begin{exercise} Is $v(u) = 2\ln|u|$ a bounded function?  If so, find the smallest $M$.
\end{exercise}

 \begin{exercise} Consider the function 
$$h(x) = \begin{cases}
2x - 3, & \mbox{if $x<1$,} \\
0, & \mbox{if $x\geq 1$.}
\end{cases}$$
Show that it is continuous at the point $x=0$.  Is $h$ a continuous function?
\end{exercise}

\begin{exercise}
Approximate a root of $\ds f=x^3-4x^2+2x+2$ to one decimal place.
\end{exercise}

\begin{exercise}
Approximate a root of $\ds f=x^4+x^3-5x+1$ to one decimal place.
\end{exercise}

\end{exercises}
