\documentclass[justified,marginals=raggedouter]{tufte-book}
\usepackage{mooculus}
\usepackage{pgfplots}


\usepackage{amsmath}
\usepackage{amssymb}
\newcommand{\R}{\mathbb{R}}
% BADBAD should include cents symbol?
\newcommand\cents{cents}

\setcounter{secnumdepth}{2}


\DeclareMathOperator{\arccot}{arccot}
\DeclareMathOperator{\sech}{sech}
\DeclareMathOperator{\csch}{csch}
\DeclareMathOperator{\arcsinh}{arcsinh}
\DeclareMathOperator{\arcsech}{arcsech}
\DeclareMathOperator{\arccosh}{arccosh}


% For nicely typeset tabular material
\usepackage{booktabs}

% Prints the month name (e.g., January) and the year (e.g., 2008)
\newcommand{\monthyear}{%
  \ifcase\month\or January\or February\or March\or April\or May\or June\or
  July\or August\or September\or October\or November\or
  December\fi\space\number\year
}

% Generates the index
\usepackage{makeidx}
\makeindex

\newcommand{\xrefn}[1]{\ref{#1}}

%%%%%%%%%%%%%%%%%%%%%%%%%%%%%%%%%%%%%%%%%%%%%%%%%%%%%%%%%%%%%%%%
\makeatletter
\newwrite\answer@stream
\immediate\openout\answer@stream=\jobname.ans

\def\dumpanswer[#1]{\immediate\write\answer@stream{#1}}

\newtoks{\answercontent}
\usepackage{environ}
 \NewEnviron{answer}{%
   \answercontent=\expandafter{\BODY}
   \dumpanswer[\arabic{exercise}. \the\answercontent]
 }
\makeatother
%%%%%%%%%%%%%%%%%%%%%%%%%%%%%%%%%%%%%%%%%%%%%%%%%%%%%%%%%%%%%%%%

\usepackage{multicol} % Use letters for lists
\renewcommand{\theenumi}{$(\mathrm{\alph{enumi}})$}
\renewcommand{\labelenumi}{\theenumi}

\usepackage{enumerate}

\newcounter{exercise}
\newenvironment{exercises}{%
\subsection*{Exercises for Section \arabic{chapter}.\arabic{section}\hrule}
\setcounter{exercise}{0}
\dumpanswer[\noexpand\subsection*{Answers for \arabic{chapter}.\arabic{section}}]
}%
{}
\newenvironment{exercise}{
\begin{enumerate}[(1)]
  \setcounter{enumi}{\value{exercise}}
  \item \addtocounter{exercise}{1}
}
{\end{enumerate}}

%%%%%%%%%%%%%%%%%%%%%%%%%%%%%%%%%%%%%%%%%%%%%%%%%%%%%%%%%%%%%%%%

%\newenvironment{example}{\subsection*{Example}}{}
\newenvironment{lemma}{\subsection*{Lemma}}{}
\newenvironment{remark}[1]{\subsection*{Remark: #1}}{}

\def\beginpicture{\null}

%\usepackage{amsthm}
%\newtheorem{theorem}{Theorem}[section]
%\theoremstyle{definition}
%\newtheorem{definition}[theorem]{Definition}


\def\exam{\null}
\def\pagerdef{\null}

\def\dfont{\bf}
\def\em{\it}           % for emphasis

\let\ds\displaystyle

\let\ssk\smallskip \let\msk\medskip \let\bsk\bigskip

\usepackage{multicol}
\def\twocol{\begin{multicols}{2}}
\def\endtwocol{\end{multicols}}

\title{Calculus}
\author{Mooculus}

\newcommand{\blankpage}{\newpage\hbox{}\thispagestyle{empty}\newpage}

% Prints an epigraph and speaker in sans serif, all-caps type.
\newcommand{\openepigraph}[2]{%
  %\sffamily\fontsize{14}{16}\selectfont
  \begin{fullwidth}
  \sffamily\large
  \begin{doublespace}
  \noindent\allcaps{#1}\\% epigraph
  \noindent\allcaps{#2}% author
  \end{doublespace}
  \end{fullwidth}
}

\begin{document}

% Front matter
%\frontmatter

% r.1 blank page
%\blankpage

% v.2 epigraphs
%\newpage\thispagestyle{empty}
% \openepigraph{%
% The public is more familiar with bad design than good design.
% It is, in effect, conditioned to prefer bad design, 
% because that is what it lives with. 
% The new becomes threatening, the old reassuring.
% }{Paul Rand%, {\itshape Design, Form, and Chaos}
% }
% \vfill
% \openepigraph{%
% A designer knows that he has achieved perfection 
% not when there is nothing left to add, 
% but when there is nothing left to take away.
% }{Antoine de Saint-Exup\'{e}ry}
% \vfill
% \openepigraph{%
% \ldots the designer of a new system must not only be the implementor and the first 
% large-scale user; the designer should also write the first user manual\ldots 
% If I had not participated fully in all these activities, 
% literally hundreds of improvements would never have been made, 
% because I would never have thought of them or perceived 
% why they were important.
% }{Donald E. Knuth}

% r.3 full title page
\maketitle

% v.4 copyright page

% BADBAD
%\leftline{\epsfxsize3truecm\epsfbox{cc-by-nc-sa.eps}}

\begin{fullwidth}
~\vfill
\thispagestyle{empty}
\setlength{\parindent}{0pt}
\setlength{\parskip}{\baselineskip}
Copyright \copyright\ \the\year\ David Guichard

This work is licensed under the Creative Commons
Attribution-NonCommercial-ShareAlike License. To view a copy of this
license, visit 
\url{http://creativecommons.org/licenses/by-nc-sa/3.0/}~or
send a letter to Creative Commons, 543 Howard Street, 5th Floor, San
Francisco, California, 94105, USA. If you distribute this work or a
derivative, include the history of the document. 

\msk\noindent
This text was initially written by David Guichard. The single variable
material in chapters 1--9 is a modification and expansion of notes
written by Neal Koblitz at the University of Washington, who
generously gave permission to use, modify, and distribute his
work. New material has been added, and old material has been modified,
so some portions now bear little resemblance to the original.

\msk\noindent The book includes some exercises and examples from {\it
  Elementary Calculus: An Approach Using Infinitesimals}, by H.~Jerome
Keisler, available at
\url{http://www.math.wisc.edu/~keisler/calc.html}~under a Creative
Commons license. In addition, the chapter on differential equations
and the section on numerical integration are largely derived from the
corresponding portions of Keisler's book.  Albert Schueller, Barry
Balof, and Mike Wills have contributed additional material.

% BADBAD
%\msk\noindent
%This copy of the text was compiled from source at 
%\the\bighand:\ifnum\littlehand<10{0}\fi
%        \the\littlehand\ on \the\month/\the\day/\the\year.

\msk\noindent
I will be glad to receive corrections and
suggestions for improvement at {\tt guichard@whitman.edu}. 

\end{fullwidth}

% r.5 contents
\tableofcontents

\listoffigures

\listoftables

% r.7 dedication
%\cleardoublepage
%~\vfill
%\begin{doublespace}
%\thispagestyle{empty}
%\noindent\fontsize{18}{22}\selectfont\itshape
%\nohyphenation
%\centerline{\it For Kathleen,\/}
%\centerline{\it without whose encouragement\/}
%\centerline{\it this book would not have\/}
%\centerline{\it been written.\/}
%\end{doublespace}
%\vfill
%\vfill

% r.9 introduction
%\cleardoublepage

\chapter*{Introduction}

BADBAD

What is calculus?

%%
% Start the main matter (normal chapters)
\mainmatter

\input limits
\input chapter01
\input chapter02
\input chapter03
\input chapter04
\input chapter05
%\input chapter06

%\input chapter07
%\input chapter08
%\input chapter09
%\input chapter10
%\input chapter11


\backmatter

%\bibliography{sample-handout}
%\bibliographystyle{plainnat}

\chapter*{Answers to selected exercises}
\input{\jobname.ans}

\printindex


\end{document}


