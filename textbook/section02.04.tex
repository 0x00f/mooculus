\section{The Derivative Function}{}{}

We have seen how to create, or derive, a new function $f'(x)$ from a
function $f(x)$, and that this new function carries important
information. In one example we saw that $f'(x)$ tells us how steep the
graph of $f(x)$ is; in another we saw that $f'(x)$ tells us the velocity
of an object if $f(x)$ tells us the  position of the object at time
$x$. As we said earlier, this same mathematical idea is useful
whenever $f(x)$ represents some changing quantity and we want to know
something about how it changes, or roughly, the ``rate'' at which it
changes. Most functions encountered in practice are built up from a
small collection of ``primitive'' functions in a few simple ways, for
example, by adding or multiplying functions together to get new, more
complicated functions. To make good use of the information provided by
$f'(x)$ we need to be able to compute it for a variety of such functions.

We will begin to use different notations for the derivative of a
function. While initially confusing, each is often useful so it is
worth maintaining multiple versions of the same thing.

Consider again the function $\ds f(x)=\sqrt{625-x^2}$.
We have computed the derivative $\ds f'(x)=-x/\sqrt{625-x^2}$, and have
already noted that if we use the alternate notation
$\ds y=\sqrt{625-x^2}$ then we might write $\ds y'=-x/\sqrt{625-x^2}$.
Another notation is quite different, and in time it will become clear
why it is often a useful one. Recall that to compute the the
derivative of $f$ we computed 
$$
\lim_{\Delta x\to0} {\sqrt{625-(7+\Delta x)^2} - 24\over \Delta x}.
$$
The denominator here measures a distance in the $x$ direction,
sometimes called the ``run'', and the numerator measures a distance in
the $y$ direction, sometimes called the ``rise,'' and ``rise over
run'' is the slope of a line. Recall that sometimes such a numerator is
abbreviated $\Delta y$, exchanging brevity for a more detailed
expression. So in general, a derivative is given by
$$
y'=\lim_{\Delta x\to0} {\Delta y\over \Delta x}.
$$
To recall the form of the limit, we sometimes say instead that
$$
{dy\over dx}=\lim_{\Delta x\to0} {\Delta y\over \Delta x}.
$$ In other words, $dy/dx$ is another notation for the derivative, and
it reminds us that it is related to an actual slope between two
points. This notation is called {\it Leibniz\index{Leibniz notation}
\index{derivative!Leibniz notation}
notation\/}, 
after Gottfried Leibniz, who developed the fundamentals
of calculus independently, at about the same time that Isaac Newton
did.  Again, since we often use $f$ and $f(x)$ to mean the original
function, we sometimes use $df/dx$ and $df(x)/dx$ to refer to the
derivative. If the function $f(x)$ is written out in full we often
write the last of these something like this
$$f'(x)={d\over dx}\sqrt{625-x^2}
$$
with the function written to the side, instead of trying to fit it into
the numerator.

\begin{example}
Find the derivative of $\ds y=f(t)=t^2$.

We compute 
\begin{align*}
y' = \lim_{\Delta t\to0}{\Delta y\over\Delta t}&=
\lim_{\Delta t\to0}{(t+\Delta t)^2-t^2\over\Delta t} \\
&=\lim_{\Delta t\to0}{t^2+2t\Delta t+\Delta t^2-t^2\over\Delta t} \\
&=\lim_{\Delta t\to0}{2t\Delta t+\Delta t^2\over\Delta t} \\
&=\lim_{\Delta t\to0} 2t+\Delta t=2t.
\end{align*}
Remember that $\Delta t$ is a single quantity, not a ``$\Delta$''
times a ``$t$'', and so $\ds \Delta t^2$ is $\ds (\Delta t)^2$ not 
$\ds \Delta (t^2)$.
\end{example}

\begin{example}
Find the derivative of $y=f(x)=1/x$.

The computation:
\begin{align*}
y' = \lim_{\Delta x\to0}{\Delta y\over\Delta x}&=
\lim_{\Delta x\to0}{ {1\over x+\Delta x} - {1\over x}\over \Delta
  x} \\
&=\lim_{\Delta x\to0}{ {x\over x(x+\Delta x)} - 
{x+\Delta x\over x(x+\Delta x)}\over \Delta x} \\
&=\lim_{\Delta x\to0}{ {x-(x+\Delta x)\over x(x+\Delta x)}\over \Delta x} \\
&=\lim_{\Delta x\to0} {x-x-\Delta x\over x(x+\Delta x)\Delta x} \\
&=\lim_{\Delta x\to0} {-\Delta x\over x(x+\Delta x)\Delta x} \\
&=\lim_{\Delta x\to0} {-1\over x(x+\Delta x)}={-1\over x^2} \\
\end{align*}
\vskip-10pt
\end{example}

{\bf Note.}  If you happen to know some ``derivative formulas'' from
an earlier course, for the time being you should pretend that you do
not know them.
In examples like the ones above and the exercises below, you are required
to know how to find the derivative formula starting from basic principles.
We will later develop some formulas so that we do not always need to
do such computations, but we will continue to need to know how to do
the more involved computations.


Sometimes one encounters a point in the domain of a function $y=f(x)$ where
there is {\bf no derivative}, because there is no tangent line.  In order
for the notion of the tangent line at a point to make sense, the curve must
be ``smooth'' at that point.  This means that if you imagine a particle
traveling at some steady speed along the curve, then the particle does not
experience an abrupt change of direction.  There are two types of
situations you should be aware of---corners and cusps---where there's a
sudden change of direction and hence no derivative.

\begin{example}
  Discuss the derivative of the absolute value function $y=f(x)=|x|$.

If $x$ is positive, then this is the function $y=x$, whose derivative is
the constant 1.  (Recall that when $y=f(x)=mx+b$, the derivative is the
slope $m$.)  If $x$ is negative, then we're dealing with the function $y=-x$,
whose derivative is the constant $-1$.  If $x=0$, then the function has
a corner, i.e., there is no tangent line.  A tangent line 
would have to point in the direction of the curve---but there are {\it
two} directions of the curve that come together at the origin.  We can
summarize this as
$$ 
y'= \begin{cases}
1&\mbox{if $x>0$;} \\
-1&\mbox{if $x<0$;} \\
\hbox{undefined}&\mbox{if $x=0$.}
\end{cases}
$$

\end{example}

\begin{example}

Discuss the derivative of the function $\ds y=x^{2/3}$, shown in
figure~\xrefn{fig:cusp}. We will later see how to compute this
derivative; for now we use the fact that $\ds
y'=(2/3)x^{-1/3}$. Visually this looks much like the absolute value
function, but it technically has a cusp, not a corner. The absolute
value function has no tangent line at 0 because there are (at least)
two obvious contenders---the tangent line of the left side of the
curve and the tangent line of the right side.
The function $\ds y=x^{2/3}$ does not have a tangent line at 0, but
unlike the absolute value function it can be said to have a single
direction: as we approach 0 from either side the tangent line becomes
closer and closer to a vertical line; the curve is vertical at 0. But
as before, if you imagine traveling along the curve, an abrupt change
in direction is required at 0: a full 180 degree turn.  \end{example}

% BADBAD
% \figure
% \vbox{\beginpicture
% \normalgraphs
% \ninepoint
% \setcoordinatesystem units <2truecm,2truecm>
% \setplotarea x from -2 to 2, y from 0 to 1.6
% \axis left shiftedto x=0 ticks numbered from 0 to 1 by 1 /
% \axis bottom ticks numbered from -2 to 2 by 1 /
% \setquadratic
% \plot -2.000 1.587 -1.933 1.552 -1.867 1.516 -1.800 1.480 -1.733 1.443 
% -1.667 1.406 -1.600 1.368 -1.533 1.330 -1.467 1.291 -1.400 1.251 
% -1.333 1.211 -1.267 1.171 -1.200 1.129 -1.133 1.087 -1.067 1.044 
% -1.000 1.000 -0.933 0.955 -0.867 0.909 -0.800 0.862 -0.733 0.813 
% -0.667 0.763 -0.600 0.711 -0.533 0.658 -0.467 0.602 -0.400 0.543 
% -0.333 0.481 -0.267 0.414 -0.200 0.342 -0.133 0.261 -0.067 0.164 
% 0.000 0.000 0.067 0.164 0.133 0.261 0.200 0.342 0.267 0.414 
% 0.333 0.481 0.400 0.543 0.467 0.602 0.533 0.658 0.600 0.711 
% 0.667 0.763 0.733 0.813 0.800 0.862 0.867 0.909 0.933 0.955 
% 1.000 1.000 1.067 1.044 1.133 1.087 1.200 1.129 1.267 1.171 
% 1.333 1.211 1.400 1.251 1.467 1.291 1.533 1.330 1.600 1.368 
% 1.667 1.406 1.733 1.443 1.800 1.480 1.867 1.516 1.933 1.552 
% 2.000 1.587   /
% \endpicture}
% \figrdef{fig:cusp}
% \endfigure{A cusp on $\ds x^{2/3}$.}

In practice we won't worry much about the distinction between these
examples; in both cases the function has a ``sharp point'' where there
is no tangent line and no derivative.

\begin{exercises}

\begin{exercise}
Find the derivative of $\ds y=f(x)=\sqrt{169-x^2}$.
\begin{answer} $\ds -x/\sqrt{169-x^2}$
\end{answer}\end{exercise}

\begin{exercise}
Find the derivative of $\ds y=f(t)=80-4.9t^2$.
\begin{answer} $-9.8t$
\end{answer}\end{exercise}

\begin{exercise}
Find the derivative of $\ds y=f(x)=x^2-(1/x)$.
\begin{answer} $\ds 2x+1/x^2$
\end{answer}\end{exercise}

\begin{exercise}
Find the derivative of $\ds y=f(x)=
ax^2+bx+c$ (where $a$, $b$, and $c$ are constants).
\begin{answer} $2ax+b$
\end{answer}\end{exercise}

\begin{exercise}
Find the derivative of $\ds y=f(x)=x^3$.
\begin{answer} $\ds 3x^2$
\end{answer}\end{exercise}

\begin{exercise}
Shown is the graph of a function $f(x)$. Sketch the graph of $f'(x)$
by estimating the derivative at a number of points in the interval:
estimate the derivative at regular intervals from one end of the
interval to the other, and also at ``special'' points, as when the
derivative is zero. Make sure you indicate any places where the
derivative does not exist.
% BADBAD
% $$\vbox{\beginpicture
% \normalgraphs
% \sevenpoint
% \setcoordinatesystem units <4.5truecm,4.5truecm>
% \setplotarea x from -1 to 1, y from 0 to 1.6
% \axis left ticks numbered from 0.2 to 1.6 by 0.2 /
% \axis left shiftedto x=0 /
% \axis bottom ticks numbered from -1 to 1 by 0.2 /
% \setquadratic
% \plot -0.900 0.800 -0.810 1.003 -0.720 1.148 -0.630 1.242 -0.540 1.291 
% -0.450 1.301 -0.360 1.280 -0.270 1.232 -0.180 1.166 -0.090 1.086 
% 0.000 1.000 0.090 0.914 0.180 0.834 0.270 0.768 0.360 0.720 
% 0.450 0.699 0.540 0.709 0.630 0.758 0.720 0.852 0.810 0.997 
% 0.900 1.200  /
% \linethickness 0.1truept
% \axis left ticks in andacross from 0.1 to 1.6 by 0.1 /
% \axis bottom ticks in andacross from -1 to 1 by 0.1 /
% \endpicture}$$
\end{exercise}

\begin{exercise}
Shown is the graph of a function $f(x)$. Sketch the graph of $f'(x)$
by estimating the derivative at a number of points in the interval:
estimate the derivative at regular intervals from one end of the
interval to the other, and also at ``special'' points, as when the
derivative is zero. Make sure you indicate any places where the
derivative does not exist.
% BADBAD
% $$\vbox{\beginpicture
% \normalgraphs
% \sevenpoint
% \setcoordinatesystem units <1.8truecm,1.8truecm>
% \setplotarea x from 0 to 5, y from 0 to 4
% \axis left ticks numbered from 0 to 4 by 1 /
% \axis bottom ticks numbered from 0 to 5 by 1 /
% \plot 0 0 2 2 /
% \setquadratic
% \plot 2.000 2.000 2.135 1.747 2.270 1.531 2.405 1.351 2.540 1.209 
% 2.675 1.103 2.810 1.034 2.945 1.002 3.080 1.007 3.215 1.049 
% 3.350 1.128 3.485 1.244 3.620 1.396 3.755 1.586 3.890 1.812 
% 4.025 2.075 4.160 2.375 4.295 2.712 4.430 3.086 4.565 3.497 
% 4.700 3.945  /
% \linethickness 0.1truept
% \axis left ticks in andacross from 0.2 to 4 by 0.2 /
% \axis bottom ticks in andacross from 0.2 to 5 by 0.2 /
% \endpicture}$$
\end{exercise}

\begin{exercise}
Find the derivative of $\ds y=f(x)=2/\sqrt{2x+1}$ 
\begin{answer} $\ds -2/(2x+1)^{3/2}$
\end{answer}\end{exercise}

\begin{exercise}
Find the derivative of $y=g(t)=(2t-1)/(t+2)$
\begin{answer} $\ds 5/(t+2)^2$
\end{answer}\end{exercise}

\begin{exercise}
Find an equation for the tangent line to the graph of $\ds f(x)=5-x-3x^2$ at the point $x=2$
\begin{answer} $y=-13x+17$
\end{answer}\end{exercise}

\begin{exercise}
Find a value for $a$ so that the graph of $\ds f(x)=x^2+ax-3$ has a horizontal tangent line at $x=4$.
\begin{answer} $-8$
\end{answer}\end{exercise}

\end{exercises}
