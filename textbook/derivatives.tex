\chapter{Derivatives}


Suppose that $f(x)$ is a function.  It is often useful to know how
sensitive the value of $f(x)$ is to small changes in $x$. To give you
a feeling why this is true, consider the following:
\begin{itemize}
\item If the change is zero, then $x$ gives a local maximal or minimal
  values for $f(x)$.
\item If $p(t)$ determines the position of an object with resepct to
  time, the change gives the velocity of the object.
\item If $v(t)$ determines the velocity of an object with resepct to
  time, the change gives the accelleration of the object.
\item The change can help us approximate a complicated function with a
  simple function.
\item The change can be used to help us solve equations that we would
  not be able to solve via other methods.
\end{itemize}

The rate of change of a function is the slope of the tangent line.
Informally, a tangent line to a curve is a line moving in the
direction of the curve that ``just touches'' the curve at that point.


\begin{definition}\index{tangent line}
Given a function $f(x)$ a line $\l(x)$ is \textbf{tangent} FILL IN
\end{definition}

\begin{figure*}
\begin{tikzpicture}
  \begin{groupplot}[group style={group size=3 by 1}]
    \nextgroupplot
          \addplot [very thick,penColor, smooth,domain=(0:1.833)] {-1/(x-2)};
          \addplot[color=penColor,fill=penColor,only marks,mark=*] coordinates{(1,1)};  %% closed hole          
    \nextgroupplot
          \addplot [very thick,penColor, smooth,domain=(0:1.833)] {-1/(x-2)};
          \addplot[color=penColor,fill=penColor,only marks,mark=*] coordinates{(1,1)};  %% closed hole          
    \nextgroupplot
          \addplot [very thick,penColor, smooth,domain=(0:1.833)] {-1/(x-2)};
          \addplot[color=penColor,fill=penColor,only marks,mark=*] coordinates{(1,1)};  %% closed hole          
\end{groupplot}
\end{tikzpicture}
\caption{Tangent lines as zooming}
\end{figure*}






\begin{figure}
\begin{tikzpicture}
	\begin{axis}[
            domain=0:2, range=0:6,ymax=6,ymin=0,
            axis lines =left, xlabel=$x$, ylabel=$y$,
            every axis y label/.style={at=(current axis.above origin),anchor=south},
            every axis x label/.style={at=(current axis.right of origin),anchor=west},
            xtick={1,1.666}, ytick={1,3},
            xticklabels={$x$,$x+h$}, yticklabels={$f(x)$,$f(x+h)$},
            axis on top,
          ]         
          \addplot [penColor2!15!background, smooth,domain=(0:2)] {-3.348+4.348*x};
          \addplot [penColor2!32!background, smooth,domain=(0:2)] {-2.704+3.704*x};
          \addplot [penColor2!49!background, smooth,domain=(0:2)] {-1.994+2.994*x};         
          \addplot [penColor2!66!background, smooth,domain=(0:2)] {-1.326+2.326*x}; 
          \addplot [penColor2!83!background, smooth,domain=(0:2)] {-0.666+1.666*x};
	  \addplot [textColor,dashed] plot coordinates {(1,0) (1,1)};
          \addplot [textColor,dashed] plot coordinates {(0,1) (1,1)};
          \addplot [textColor,dashed] plot coordinates {(0,3) (1.666,3)};
          \addplot [textColor,dashed] plot coordinates {(1.666,0) (1.666,3)};
          \addplot [very thick,penColor, smooth,domain=(0:1.833)] {-1/(x-2)};
          \addplot[color=penColor,fill=penColor,only marks,mark=*] coordinates{(1.666,3)};  %% closed hole          
          \addplot[color=penColor,fill=penColor,only marks,mark=*] coordinates{(1,1)};  %% closed hole          
          \addplot [very thick,penColor2, smooth,domain=(0:2)] {x};
        \end{axis}
\end{tikzpicture}
\caption{Tangent lines can be found as the limit of secant lines. The slope of the tangent line is given by
$\ds \lim_{h\to 0} \frac{f(x+h) - f(x)}{h}.$}
\label{figure:epsilon-delta}
\end{figure}
