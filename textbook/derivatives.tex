\chapter{Derivatives}


Suppose that $f(x)$ is a function.  It is often useful to know how
sensitive the value of $f(x)$ is to small changes in $x$. To give you
a feeling why this is true, consider the following:
\begin{itemize}
\item If the change is zero, then $x$ gives a local maximal or minimal
  values for $f(x)$.
\item If $p(t)$ determines the position of an object with respect to
  time, the change gives the velocity of the object.
\item If $v(t)$ determines the velocity of an object with respect to
  time, the change gives the acceleration of the object.
\item The change can help us approximate a complicated function with a
  simple function.
\item The change can be used to help us solve equations that we would
  not be able to solve via other methods.
\end{itemize}

The rate of change of a function is the slope of the tangent
line. Part of our goal will be to give a formal definition of a
tangent line. For now, consider the following informal definition:
\begin{quote}\index{tangent line!informal definition}
Given a function $f(x)$, if one can ``zoom in''
on $f(x)$ sufficiently so that $f(x)$ seems to be a straight line,
then that line is the \textbf{tangent line} to $f(x)$ at the point
determined by $x$.
\end{quote}
While this is merely an informal definition of a tangent line, it
contains the essence of how the formal definition will be eventually
constructed. We illustrate this informal definition with
Figure~\ref{figure:informal-tangent}.
\begin{figure*}
\begin{tikzpicture}
	\begin{axis}[
            domain=0:6, range=0:7,
            ymin=-.2,ymax=7,
            width=\textwidth,
            height=7cm, %% Hard coded height! Moreover this effects the aspect ratio of the zoom--sort of BAD
            axis lines=none,
          ]   
          \addplot [draw=none, fill=textColor!7!background] plot coordinates {(1.2,1.6) (2.834,5)} \closedcycle; %% zoom fill
          \addplot [draw=none, fill=background] plot coordinates {(1.2,.4) (2.834,1)} \closedcycle; %% zoom fill

          \addplot [draw=none, fill=textColor!7!background] plot coordinates {(3.7,3.6) (5.334,5)} \closedcycle; %% zoom fill
          \addplot [draw=none, fill=background] plot coordinates {(3.7,2.4) (5.334,1)} \closedcycle; %% zoom fill

          \addplot[very thick,penColor, smooth,domain=(0:1.833)] {-1/(x-2)};
          \addplot[very thick,penColor, smooth,domain=(2.834:4.166)] {3.333/(2.050-.3*x)-0.333}; %% 2.5 to 4.333
          \addplot[very thick,penColor, smooth,domain=(5.334:6.666)] {11.11/(1.540-.09*x)-8.109}; %% 5 to 6.833
          
          \addplot[color=penColor,fill=penColor,only marks,mark=*] coordinates{(1,1)};  %% point to be zoomed
          \addplot[color=penColor,fill=penColor,only marks,mark=*] coordinates{(3.5,3)};  %% zoomed pt 1
          \addplot[color=penColor,fill=penColor,only marks,mark=*] coordinates{(6,3)};  %% zoomed pt 2

          \addplot [->,textColor] plot coordinates {(0,0) (0,6)}; %% axis
          \addplot [->,textColor] plot coordinates {(0,0) (2,0)}; %% axis
          
          \addplot [textColor] plot coordinates {(.8,.4) (.8,1.6)}; %% box around pt
          \addplot [textColor] plot coordinates {(1.2,.4) (1.2,1.6)}; %% box around pt
          \addplot [textColor] plot coordinates {(.8,1.6) (1.2,1.6)}; %% box around pt
          \addplot [textColor] plot coordinates {(.8,.4) (1.2,.4)}; %% box around pt
          
          \addplot [textColor] plot coordinates {(2.834,1) (2.834,5)}; %% zoomed box 1
          \addplot [textColor] plot coordinates {(4.166,1) (4.166,5)}; %% zoomed box 1
          \addplot [textColor] plot coordinates {(2.834,1) (4.166,1)}; %% zoomed box 1
          \addplot [textColor] plot coordinates {(2.834,5) (4.166,5)}; %% zoomed box 1

          \addplot [textColor] plot coordinates {(3.3,2.4) (3.3,3.6)}; %% box around zoomed pt
          \addplot [textColor] plot coordinates {(3.7,2.4) (3.7,3.6)}; %% box around zoomed pt
          \addplot [textColor] plot coordinates {(3.3,3.6) (3.7,3.6)}; %% box around zoomed pt
          \addplot [textColor] plot coordinates {(3.3,2.4) (3.7,2.4)}; %% box around zoomed pt

          \addplot [textColor] plot coordinates {(5.334,1) (5.334,5)}; %% zoomed box 2
          \addplot [textColor] plot coordinates {(6.666,1) (6.666,5)}; %% zoomed box 2
          \addplot [textColor] plot coordinates {(5.334,1) (6.666,1)}; %% zoomed box 2
          \addplot [textColor] plot coordinates {(5.334,5) (6.666,5)}; %% zoomed box 2

          \node at (axis cs:2.2,0) [anchor=east] {$x$};
          \node at (axis cs:0,6.6) [anchor=north] {$y$};
        \end{axis}
\end{tikzpicture}
\caption{Given a function $f(x)$, if one can ``zoom in''
on $f(x)$ sufficiently so that $f(x)$ seems to be a straight line,
then that line is the \textbf{tangent line} to $f(x)$ at the point
determined by $x$.}
\label{figure:informal-tangent}
\end{figure*}






\begin{figure}
\begin{tikzpicture}
	\begin{axis}[
            domain=0:2, range=0:6,ymax=6,ymin=0,
            axis lines =left, xlabel=$x$, ylabel=$y$,
            every axis y label/.style={at=(current axis.above origin),anchor=south},
            every axis x label/.style={at=(current axis.right of origin),anchor=west},
            xtick={1,1.666}, ytick={1,3},
            xticklabels={$x$,$x+h$}, yticklabels={$f(x)$,$f(x+h)$},
            axis on top,
          ]         
          \addplot [penColor2!15!background, smooth,domain=(0:2)] {-3.348+4.348*x};
          \addplot [penColor2!32!background, smooth,domain=(0:2)] {-2.704+3.704*x};
          \addplot [penColor2!49!background, smooth,domain=(0:2)] {-1.994+2.994*x};         
          \addplot [penColor2!66!background, smooth,domain=(0:2)] {-1.326+2.326*x}; 
          \addplot [penColor2!83!background, smooth,domain=(0:2)] {-0.666+1.666*x};
	  \addplot [textColor,dashed] plot coordinates {(1,0) (1,1)};
          \addplot [textColor,dashed] plot coordinates {(0,1) (1,1)};
          \addplot [textColor,dashed] plot coordinates {(0,3) (1.666,3)};
          \addplot [textColor,dashed] plot coordinates {(1.666,0) (1.666,3)};
          \addplot [very thick,penColor, smooth,domain=(0:1.833)] {-1/(x-2)};
          \addplot[color=penColor,fill=penColor,only marks,mark=*] coordinates{(1.666,3)};  %% closed hole          
          \addplot[color=penColor,fill=penColor,only marks,mark=*] coordinates{(1,1)};  %% closed hole          
          \addplot [very thick,penColor2, smooth,domain=(0:2)] {x};
        \end{axis}
\end{tikzpicture}
\caption{Tangent lines can be found as the limit of secant lines. The slope of the tangent line is given by
$\ds \lim_{h\to 0} \frac{f(x+h) - f(x)}{h}.$}
\label{figure:epsilon-delta}
\end{figure}
