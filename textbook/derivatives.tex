\chapter{Differentiation}


\section{Slopes of Tangent Lines via Limits}

Suppose that $f(x)$ is a function.  It is often useful to know how
sensitive the value of $f(x)$ is to small changes in $x$. To give you
a feeling why this is true, consider the following:
\begin{itemize}
\item If the change is zero, then $x$ gives a local maximal or minimal
  values for $f(x)$.
\item If $p(t)$ determines the position of an object with respect to
  time, the change gives the velocity of the object.
\item If $v(t)$ determines the velocity of an object with respect to
  time, the change gives the acceleration of the object.
\item The change can help us approximate a complicated function with a
  simple function.
\item The change can be used to help us solve equations that we would
  not be able to solve via other methods.
\end{itemize}

The rate of change of a function is the slope of the tangent
line. Part of our goal will be to give a formal definition of a
tangent line. For now, consider the following informal definition:
\begin{quote}\index{tangent line!informal definition}
Given a function $f(x)$, if one can ``zoom in''
on $f(x)$ sufficiently so that $f(x)$ seems to be a straight line,
then that line is the \textbf{tangent line} to $f(x)$ at the point
determined by $x$.
\end{quote}
While this is merely an informal definition of a tangent line, it
contains the essence of how the formal definition will be
constructed. We illustrate this informal definition with
Figure~\ref{figure:informal-tangent}.
\begin{figure*}
\begin{tikzpicture}
	\begin{axis}[
            domain=0:6, range=0:7,
            ymin=-.2,ymax=7,
            width=\textwidth,
            height=7cm, %% Hard coded height! Moreover this effects the aspect ratio of the zoom--sort of BAD
            axis lines=none,
          ]   
          \addplot [draw=none, fill=textColor!10!background] plot coordinates {(.8,1.6) (2.834,5)} \closedcycle; %% zoom fill
          \addplot [draw=none, fill=textColor!10!background] plot coordinates {(2.834,5) (4.166,5)} \closedcycle; %% zoom fill
          \addplot [draw=none, fill=background] plot coordinates {(1.2,1.6) (4.166,5)} \closedcycle; %% zoom fill
          \addplot [draw=none, fill=background] plot coordinates {(.8,1.6) (1.2,1.6)} \closedcycle; %% zoom fill

          \addplot [draw=none, fill=textColor!10!background] plot coordinates {(3.3,3.6) (5.334,5)} \closedcycle; %% zoom fill
          \addplot [draw=none, fill=textColor!10!background] plot coordinates {(5.334,5) (6.666,5)} \closedcycle; %% zoom fill
          \addplot [draw=none, fill=background] plot coordinates {(3.7,3.6) (6.666,5)} \closedcycle; %% zoom fill
          \addplot [draw=none, fill=background] plot coordinates {(3.3,3.6) (3.7,3.6)} \closedcycle; %% zoom fill
          
          \addplot [draw=none, fill=textColor!10!background] plot coordinates {(3.7,2.4) (6.666,1)} \closedcycle; %% zoom fill
          \addplot [draw=none, fill=textColor!10!background] plot coordinates {(3.3,2.4) (3.7,2.4)} \closedcycle; %% zoom fill
          \addplot [draw=none, fill=background] plot coordinates {(3.3,2.4) (5.334,1)} \closedcycle; %% zoom fill          
          \addplot [draw=none, fill=background] plot coordinates {(5.334,1) (6.666,1)} \closedcycle; %% zoom fill
          

          \addplot [draw=none, fill=textColor!10!background] plot coordinates {(.8,.4) (2.834,1)} \closedcycle; %% zoom fill
          \addplot [draw=none, fill=textColor!10!background] plot coordinates {(2.834,1) (4.166,1)} \closedcycle; %% zoom fill
          \addplot [draw=none, fill=background] plot coordinates {(1.2,.4) (4.166,1)} \closedcycle; %% zoom fill
          \addplot [draw=none, fill=background] plot coordinates {(.8,.4) (1.2,.4)} \closedcycle; %% zoom fill

          \addplot[very thick,penColor, smooth,domain=(0:1.833)] {-1/(x-2)};
          \addplot[very thick,penColor, smooth,domain=(2.834:4.166)] {3.333/(2.050-.3*x)-0.333}; %% 2.5 to 4.333
          %\addplot[very thick,penColor, smooth,domain=(5.334:6.666)] {11.11/(1.540-.09*x)-8.109}; %% 5 to 6.833
          \addplot[very thick,penColor, smooth,domain=(5.334:6.666)] {x-3}; %% 5 to 6.833
          
          \addplot[color=penColor,fill=penColor,only marks,mark=*] coordinates{(1,1)};  %% point to be zoomed
          \addplot[color=penColor,fill=penColor,only marks,mark=*] coordinates{(3.5,3)};  %% zoomed pt 1
          \addplot[color=penColor,fill=penColor,only marks,mark=*] coordinates{(6,3)};  %% zoomed pt 2

          \addplot [->,textColor] plot coordinates {(0,0) (0,6)}; %% axis
          \addplot [->,textColor] plot coordinates {(0,0) (2,0)}; %% axis
          
          \addplot [textColor!50!background] plot coordinates {(.8,.4) (.8,1.6)}; %% box around pt
          \addplot [textColor!50!background] plot coordinates {(1.2,.4) (1.2,1.6)}; %% box around pt
          \addplot [textColor!50!background] plot coordinates {(.8,1.6) (1.2,1.6)}; %% box around pt
          \addplot [textColor!50!background] plot coordinates {(.8,.4) (1.2,.4)}; %% box around pt
          
          \addplot [textColor!50!background] plot coordinates {(2.834,1) (2.834,5)}; %% zoomed box 1
          \addplot [textColor!50!background] plot coordinates {(4.166,1) (4.166,5)}; %% zoomed box 1
          \addplot [textColor!50!background] plot coordinates {(2.834,1) (4.166,1)}; %% zoomed box 1
          \addplot [textColor!50!background] plot coordinates {(2.834,5) (4.166,5)}; %% zoomed box 1

          \addplot [textColor] plot coordinates {(3.3,2.4) (3.3,3.6)}; %% box around zoomed pt
          \addplot [textColor] plot coordinates {(3.7,2.4) (3.7,3.6)}; %% box around zoomed pt
          \addplot [textColor] plot coordinates {(3.3,3.6) (3.7,3.6)}; %% box around zoomed pt
          \addplot [textColor] plot coordinates {(3.3,2.4) (3.7,2.4)}; %% box around zoomed pt

          \addplot [textColor] plot coordinates {(5.334,1) (5.334,5)}; %% zoomed box 2
          \addplot [textColor] plot coordinates {(6.666,1) (6.666,5)}; %% zoomed box 2
          \addplot [textColor] plot coordinates {(5.334,1) (6.666,1)}; %% zoomed box 2
          \addplot [textColor] plot coordinates {(5.334,5) (6.666,5)}; %% zoomed box 2

          \node at (axis cs:2.2,0) [anchor=east] {$x$};
          \node at (axis cs:0,6.6) [anchor=north] {$y$};
        \end{axis}
\end{tikzpicture}
\caption{Given a function $f(x)$, if one can ``zoom in''
on $f(x)$ sufficiently so that $f(x)$ seems to be a straight line,
then that line is the \textbf{tangent line} to $f(x)$ at the point
determined by $x$.}
\label{figure:informal-tangent}
\end{figure*}


The \textit{derivative} of a function $f(x)$ at $x$, is the slope of
the tangent line at $x$. To find the slope of this line, we consider
\textit{secant} lines, lines that locally intersect the curve at two
points.  The slope of any secant line that passes through the points
$(x,f(x))$ and $(x+h, f(x+h))$ is given by
\[
\frac{\Delta y}{\Delta x}=\frac{f(x+h) -f(x)}{(x+h)-x} = \frac{f(x+h)-f(x)}{h},
\]
see Figure~\ref{figure:limit-dfn}. This leads to the \textit{Limit Definition of the Derivative}:

\begin{mainDefinition}\index{limit!definition of the derivative}\index{derivative!limit definition}
The \textbf{derivative} of $f(x)$ is the function
\[
\ddx f(x) = \lim_{h\to 0} \frac{f(x+h) - f(x)}{h}.
\]
If this limit does not exist for a given value of $x$, then $f(x)$ is
not \textbf{differentiable} at $x$.
\end{mainDefinition}
\begin{marginfigure}[-1.75in]
\begin{tikzpicture}
	\begin{axis}[
            domain=0:2, range=0:6,ymax=6,ymin=0,
            axis lines =left, xlabel=$x$, ylabel=$y$,
            every axis y label/.style={at=(current axis.above origin),anchor=south},
            every axis x label/.style={at=(current axis.right of origin),anchor=west},
            xtick={1,1.666}, ytick={1,3},
            xticklabels={$x$,$x+h$}, yticklabels={$f(x)$,$f(x+h)$},
            axis on top,
          ]         
          \addplot [penColor2!15!background, smooth,domain=(0:2)] {-3.348+4.348*x};
          \addplot [penColor2!32!background, smooth,domain=(0:2)] {-2.704+3.704*x};
          \addplot [penColor2!49!background, smooth,domain=(0:2)] {-1.994+2.994*x};         
          \addplot [penColor2!66!background, smooth,domain=(0:2)] {-1.326+2.326*x}; 
          \addplot [penColor2!83!background, smooth,domain=(0:2)] {-0.666+1.666*x};
	  \addplot [textColor,dashed] plot coordinates {(1,0) (1,1)};
          \addplot [textColor,dashed] plot coordinates {(0,1) (1,1)};
          \addplot [textColor,dashed] plot coordinates {(0,3) (1.666,3)};
          \addplot [textColor,dashed] plot coordinates {(1.666,0) (1.666,3)};
          \addplot [very thick,penColor, smooth,domain=(0:1.833)] {-1/(x-2)};
          \addplot[color=penColor,fill=penColor,only marks,mark=*] coordinates{(1.666,3)};  %% closed hole          
          \addplot[color=penColor,fill=penColor,only marks,mark=*] coordinates{(1,1)};  %% closed hole          
          \addplot [very thick,penColor2, smooth,domain=(0:2)] {x};
        \end{axis}
\end{tikzpicture}
\caption{Tangent lines can be found as the limit of secant lines. The slope of the tangent line is given by
$\lim_{h\to 0} \frac{f(x+h) - f(x)}{h}.$}
\label{figure:limit-dfn}
\end{marginfigure}

\begin{definition}\index{derivative!notation}
There are several different notations for the derivative, we'll mainly
use
\[
\ddx f(x) = f'(x).
\]
If one is working with a function of a variable other than $x$, say $t$ we write
\[
\dd{t} f(t) = f'(t).
\]
However, if $y = f(x)$, $\frac{dy}{dx}$, $\dot{y}$, and $D_x f(x)$ are
also used.
\end{definition}

\begin{example}
Compute 
\[
\ddx (x^3 + 1).
\] 
\end{example}

\begin{solution}
Using the definition of the derivative,
\begin{align*}
\ddx f(x) &= \lim_{h\to 0}\frac{(x+h)^3 + 1 - (x^3 +1)}{h}\\
&= \lim_{h\to 0}\frac{x^3+3x^2h+3xh^2 + h^3 + 1 - x^3 -1}{h}\\
&= \lim_{h\to 0}\frac{3x^2h+3xh^2 + h^3}{h}\\
&= \lim_{h\to 0}(3x^2+3xh + h^2)\\
&= 3x^2.
\end{align*}
\end{solution}



\begin{example}
Compute
\[
\dd t \frac{1}{t}.
\]
\end{example}
\begin{marginfigure}
\begin{tikzpicture}
	\begin{axis}[
            domain=-3:3,
            ymax=4,
            ymin=-4,
            samples=100,
            axis lines =middle, xlabel=$t$, ylabel=$f(t)$,
            every axis y label/.style={at=(current axis.above origin),anchor=south},
            every axis x label/.style={at=(current axis.right of origin),anchor=west}
          ]
	  \addplot [very thick, penColor, smooth,domain=(-3:-.1)] {1/x};
          \addplot [very thick, penColor, smooth,domain=(.1:3)] {1/x};
        \end{axis}
\end{tikzpicture}
\caption{A plot of $f(t) = \frac{1}{t}$.}
\label{figure:plot1/x}
\end{marginfigure}
\begin{solution}
Using the definition of the derivative,
\begin{align*}
\dd{t}\frac{1}{t}&=\lim_{ h\to0}\frac{\frac{1}{t+ h} - \frac{1}{t}}{h} \\
&=\lim_{h\to0}\frac{\frac{t}{t(t+ h)} - \frac{t+ h}{t(t+ h)}}{h} \\
&=\lim_{h\to0}\frac{\frac{t-(t+ h)}{t(t+ h)}}{h} \\
&=\lim_{h\to0}\frac{t-t- h}{t(t+ h) h} \\
&=\lim_{h\to0}\frac{- h}{t(t+ h) h} \\
&=\lim_{h\to0}\frac{-1}{t(t+ h)}\\
&=\frac{-1}{t^2}.
\end{align*}
This function is differentiable at all real numbers except for $t=0$, see Figure~\ref{figure:plot1/x}.
\end{solution}


\begin{example}
Compute 
\[
\ddx |x|.
\]
\end{example}
\begin{marginfigure}
\begin{tikzpicture}
	\begin{axis}[
            domain=-3:3,
            ymax=4,
            ymin=-1,
            samples=100,
            axis lines =middle, xlabel=$x$, ylabel=$f(x)$,
            every axis y label/.style={at=(current axis.above origin),anchor=south},
            every axis x label/.style={at=(current axis.right of origin),anchor=west}
          ]
	  \addplot [very thick, penColor, smooth] {abs(x)};
        \end{axis}
\end{tikzpicture}
\caption{A plot of $f(x) = |x|$.}
\label{figure:plot-abs}
\end{marginfigure}
\begin{solution}
Using the definition of the derivative,
\[
\ddx |x| = \lim_{h\to0}\frac{|x+h| -|x|}{h}.
\]
If $x$ is positive we may assume that $x$ is larger than $h$, as we are
taking the limit as $h$ goes to $0$,
\begin{align*}
\lim_{h\to0}\frac{|x+h| -|x|}{h} &= \lim_{h\to0}\frac{x+h -x}{h}\\
&= \lim_{h\to0}\frac{h}{h}\\
&= 1.
\end{align*}
If $x$ is negative we may assume that $|x|$ is larger than $h$, as we are taking
the limit as $h$ goes to $0$,
\begin{align*}
\lim_{h\to0}\frac{|x+h| -|x|}{h} &= \lim_{h\to0}\frac{x-h -x}{h}\\
&= \lim_{h\to0+}\frac{-h}{h}\\
&= -1.
\end{align*}
However we still have one case left, when $x=0$. In this situation, we
must consider the one-sided limits:
\[
\lim_{h\to0+}\frac{|x+h| -|x|}{h}\qquad\text{and}\qquad \lim_{h\to0-}\frac{|x+h| -|x|}{h}.
\]
In the first case, 
\begin{align*}
\lim_{h\to0+}\frac{|x+h| -|x|}{h} &= \lim_{h\to 0+}\frac{0+h - 0}{h}\\
&= \lim_{h\to 0+}\frac{h}{h}\\
&=1.
\end{align*}
On the other hand
\begin{align*}
\lim_{h\to0-}\frac{|x+h| -|x|}{h} &= \lim_{h\to 0-}\frac{|0+h| - 0}{h}\\
&= \lim_{h\to 0-}\frac{|h|}{h}\\
&=-1.
\end{align*}
Hence we see that the derivative does not exist at $0$, see Figure~\ref{figure:plot-abs}. 
\end{solution}





\begin{example}
Discuss the derivative of the function $y=x^{2/3}$, shown in
figure~\ref{fig:cusp}. We will later see how to compute this
derivative; for now we use the fact that $
y'=(2/3)x^{-1/3}$. Visually this looks much like the absolute value
function, but it technically has a cusp, not a corner. The absolute
value function has no tangent line at 0 because there are (at least)
two obvious contenders---the tangent line of the left side of the
curve and the tangent line of the right side.  The function $
y=x^{2/3}$ does not have a tangent line at 0, but unlike the absolute
value function it can be said to have a single direction: as we
approach 0 from either side the tangent line becomes closer and closer
to a vertical line; the curve is vertical at 0. But as before, if you
imagine traveling along the curve, an abrupt change in direction is
required at 0: a full 180 degree turn.
\end{example}







differentiable => continuous

not continuous => not diff


Now some examples of of computing derivatives via limits. 














Take, for example, $y=f(x)=\sqrt{625-x^2}$ (the upper semicircle of radius
25 centered at the origin).  When $x=7$, we find that $y=\sqrt{625-49}=24$.
Suppose we want to know how much $y$ changes when $x$ increases a little,
say to 7.1 or 7.01.


What if we try to do all the algebra without using a specific
value for $x$? Let's copy from above, replacing 7 by  $x$. We'll have
to do a bit more than that---for example, 
the ``24'' in the calculation came from 
$\sqrt{625-7^2}$, so we'll need to fix that too.
\begin{align*}
&{\sqrt{625-(x+ h)^2} - \sqrt{625-x^2}\over  h}= \\
\qquad&={\sqrt{625-(x+ h)^2} - \sqrt{625-x^2}\over  h}{\sqrt{625-(x+\Delta
    x)^2}+\sqrt{625-x^2}\over \sqrt{625-(x+ h)^2}+\sqrt{625-x^2}} \\
&={625-(x+ h)^2-625+x^2\over  h(\sqrt{625-(x+ h)^2}+\sqrt{625-x^2})} \\
&={625-x^2-2x h- h^2-625+x^2\over   h(\sqrt{625-(x+\Delta
    x)^2}+\sqrt{625-x^2})} \\
&={ h(-2x- h)\over  h(\sqrt{625-(x+\Delta
    x)^2}+\sqrt{625-x^2})} \\
&= {-2x- h\over\sqrt{625-(x+\Delta
    x)^2}+\sqrt{625-x^2}} \\
\end{align*}
Now what happens when $h$ is very close to zero? Again it seems
apparent that the quotient will be very close to
$${-2x\over \sqrt{625-x^2}+\sqrt{625-x^2}}
={-2x\over 2\sqrt{625-x^2}}={-x\over \sqrt{625-x^2}}.
$$
Replacing $x$ by 7 gives $-7/24$, as before, and now we can easily do
the computation for 12  or any other value of
$x$ between $-25$ and 25.

So now we have a single, simple formula, ${-x/ \sqrt{625-x^2}}$,
that tells us the slope of the tangent line for any value of
$x$. This slope, in turn, tells us how sensitive the value of $y$ is
to changes in the value of $x$. 

What do we call such a formula? That is, a formula with one variable,
so that substituting an ``input'' value for the variable produces a
new ``output'' value? This is a function. Starting with one function,
$\sqrt{625-x^2}$, we have derived, by means of some slightly nasty
algebra, a new function, ${-x/ \sqrt{625-x^2}}$, that gives us
important information about the original function. This new function
in fact is called the {\dfont derivative\index{derivative}} of the
original function. If the original is referred to as $f$ or $y$ then
the derivative is often written $f'$ or $y'$ and pronounced ``f
prime'' or ``y prime'', so in this case we might write $f'(x)=-x/
\sqrt{625-x^2}$. At a particular point, say $x=7$, we say that
$f'(7)=-7/24$ or ``$f$ prime of 7 is $-7/24$'' or ``the derivative of
$f$ at 7 is $-7/24$.''

To summarize, we compute the derivative of $f(x)$ by forming the
difference quotient
$$
{f(x+ h)-f(x)\over  h},
$$
which is the slope of a line, then we figure out what happens when
$h$ gets very close to 0. 


We should note that in 
the particular case of a circle, there's a simple way to find the
derivative.  Since the tangent to a circle at a point is perpendicular to
the radius drawn to the point of contact, its slope is the negative
reciprocal of the slope of the radius.  The radius joining $(0,0)$ to
$(7,24)$ has slope 24/7.  Hence, the tangent line has slope
$-7/24$. In general, a radius to the point $(x,\sqrt{625-x^2})$ has
slope $\sqrt{625-x^2}/x$, so the slope of the tangent line is
${-x/ \sqrt{625-x^2}}$, as before. It is {\bf NOT} always true that a
tangent line is perpendicular to a line from the origin---don't use
this shortcut in any other circumstance. 

As above, and as you might expect, for different values of $x$ we
generally get different values of the derivative $f'(x)$. Could it be
that the derivative always has the same value? This would mean that
the slope of $f$, or the slope of its tangent line, is the same
everywhere. One curve that always has the same slope is a line; it
seems odd to talk about the tangent line to a line, but if it makes
sense at all the tangent line must be the line itself. It is not hard
to see that the derivative of $f(x)=mx+b$ is $f'(x)=m$; see
exercise~\xrefn{ex:derivative of a line} .








\section{Basic Derivative Rules}

constant rule


\subsection{The Sum Rule}

We don't want to rely on some more-or-less obvious physical
interpretation to determine what is true mathematically, so let's see
how to verify these rules by computation. We'll do one and leave the
other for the exercises.
\begin{align*}
\ddx(f(x)+g(x)) &= \lim_{ h\to 0} {f(x+\Delta
  x)+g(x+ h) - (f(x)+g(x))\over  h}  \\
&= \lim_{ h\to 0} {f(x+\Delta
  x)+g(x+ h) - f(x)-g(x)\over  h}  \\
&= \lim_{ h\to 0} {f(x+\Delta
  x)-f(x) +g(x+ h) -g(x)\over  h}  \\
&= \lim_{ h\to 0} \left({f(x+\Delta
  x)-f(x)\over  h}  +{g(x+ h) -g(x)\over  h}\right)  \\
&= \lim_{ h\to 0} {f(x+\Delta
  x)-f(x)\over  h}  +
\lim_{ h\to 0} {g(x+ h) -g(x)\over  h}  \\
&=f'(x)+g'(x) \\
\end{align*}
This is sometimes called the {\dfont sum rule\index{sum rule}} for derivatives.
\begin{example}
Find the derivative of $f(x)=x^5+5x^2$. We have to invoke linearity
twice here: 
\[
f'(x) = \ddx(x^5+5x^2) = \ddx x^5 + \ddx(5x^2) =
5x^4+5\ddx(x^2) = 5x^4+5\cdot 2x^1 = 5x^4+10x.
\]
\vskip-10pt\end{example}

\begin{example}
Find the derivative of $f(x)=3/x^4-2x^2+6x-7$.
$$f'(x) = \ddx\left( {3\over x^4}-2x^2+6x-7\right)
= \ddx(3x^{-4}-2x^2+6x-7) 
= -12x^{-5}-4x+6.$$
\vskip-10pt\end{example}




\subsection{The Power Rule}

 Now
let's look at the limit:
\begin{align*}
\ddx x^n &= \lim_{ h\to0} {(x+ h)^n-x^n\over \Delta
  x} \\
&=\lim_{ h\to0} {x^n + nx^{n-1} h + a_2x^{n-2} h^2+\cdots+
a_{n-1}x h^{n-1} +  h^n-x^n\over  h} \\
&=\lim_{ h\to0} {nx^{n-1} h + a_2x^{n-2} h^2+\cdots+
a_{n-1}x h^{n-1} +  h^n\over  h} \\
&=\lim_{ h\to0} nx^{n-1} + a_2x^{n-2} h+\cdots+
a_{n-1}x h^{n-2} +  h^{n-1} = nx^{n-1}. \\
\end{align*}

Now without much trouble we can verify the formula for negative
integers. First let's look at an example:

\begin{example} Find the derivative of $y=x^{-3}$. Using the formula,
$y'=-3x^{-3-1}=-3x^{-4}$. 
\end{example}

Here is the general computation. Suppose $n$ is a negative integer;
the algebra is easier to follow if we use $n=-m$ in the computation,
where $m$ is a positive integer.
\begin{align*}
\ddx x^n &= \ddx x^{-m} =
\lim_{ h\to0} {(x+ h)^{-m}-x^{-m}\over \Delta
  x} \\
&=\lim_{ h\to0} { {1\over (x+ h)^m} - {1\over x^m} \over
   h}  \\
&=\lim_{ h\to0} { x^m - (x+ h)^m \over
(x+ h)^m x^m  h}  \\
&=\lim_{ h\to0} { x^m - (x^m + mx^{m-1} h + a_2x^{m-2} h^2+\cdots+
a_{m-1}x h^{m-1} +  h^m)\over
(x+ h)^m x^m  h}  \\
&=\lim_{ h\to0} { -mx^{m-1} - a_2x^{m-2} h-\cdots-
a_{m-1}x h^{m-2} -  h^{m-1})\over
(x+ h)^m x^m}  \\
&={ -mx^{m-1} \over x^mx^m}=
{ -mx^{m-1} \over x^{2m}}=
-mx^{m-1-2m}= nx^{-m-1} = nx^{n-1}. \\
\end{align*}

We will later see why the other cases of the power rule work, but from
now on we will use the power rule whenever $n$ is any real number.
Let's note here a simple case in which the power rule applies, or
almost applies, but is not really needed. Suppose that $f(x)=1$;
remember that this ``1'' is a function, not ``merely'' a number, and
that $f(x)=1$ has a graph that is a horizontal line, with slope zero
everywhere. So we know that $f'(x)=0$. We might also write $f(x)=x^0$,
though there is some question about just what this means at $x=0$. If
we apply the power rule, we get $f'(x)=0x^{-1}=0/x=0$, again noting
that there is a problem at $x=0$. So the power rule ``works'' in this
case, but it's really best to just remember that the derivative of any
constant function is zero.



exponent rule

dfn $e^x$




\begin{exercises}

Find the derivatives of the given functions.

\twocol

\begin{exercise} $x^{100}$
\begin{answer} $100x^{99}$
\end{answer}\end{exercise}

\begin{exercise} $x^{-100}$
\begin{answer} $-100x^{-101}$
\end{answer}\end{exercise}

\begin{exercise} $\displaystyle {1\over x^5}$
\begin{answer} $-5x^{-6}$
\end{answer}\end{exercise}

\begin{exercise} $x^\pi$
\begin{answer} $\pi x^{\pi-1}$
\end{answer}\end{exercise}

\begin{exercise} $x^{3/4}$
\begin{answer} $(3/4)x^{-1/4}$
\end{answer}\end{exercise}

\begin{exercise} $x^{-9/7}$
\begin{answer} $-(9/7)x^{-16/7}$
\end{answer}\end{exercise}

\endtwocol

Find the derivatives of the functions in 1--6.

\begin{exercise} $5x^3+12x^2-15$
\begin{answer} $15x^2+24x$
\end{answer}\end{exercise}

\begin{exercise} $-4x^5 + 3x^2 - 5/x^2$
\begin{answer} $-20x^4+6x+10/x^3$
\end{answer}\end{exercise}

\begin{exercise} $5(-3x^2 + 5x + 1)$
\begin{answer} $-30x+25$
\end{answer}\end{exercise}

\begin{exercise} $f(x)+g(x)$, where $f(x)=x^2-3x+2$ and $g(x)=2x^3-5x$
\begin{answer} $6x^2+2x-8$
\end{answer}\end{exercise}

\begin{exercise} $(x+1)(x^2+2x-3)$
\begin{answer} $3x^2+6x-1$
\end{answer}\end{exercise}

\begin{exercise} $\sqrt{625-x^2}+3x^3+12$ (See section \xrefn{sec:slope of a function}.)
\begin{answer} $9x^2-x/\sqrt{625-x^2}$
\end{answer}\end{exercise}

\begin{exercise}
 Find an equation for the tangent line to $f(x) = x^3/4 - 1/x$ at $x=-2$.
\begin{answer} $y=13x/4+5$
\end{answer}\end{exercise}

\begin{exercise} Find an equation for 
the tangent line to $f(x)= 3x^2 - \pi ^3$ at $x= 4$.
\begin{answer} $y=24x-48-\pi^3$
\end{answer}\end{exercise}

\begin{exercise} Suppose the position of an object at time $t$ is  given by
$f(t)=-49 t^2/10+5t+10$. Find a function giving the speed of the object
at time $t$. The acceleration of an object is the rate at which its
speed is changing, which means it is given by the derivative of the
speed function. Find the acceleration of the object at time $t$.
\begin{answer} $-49t/5+5$, $-49/5$
\end{answer}\end{exercise}

\begin{exercise} Let $f(x) =x^3$ and $c= 3$. Sketch the graphs of $f$,
$cf$, $f'$, and $(cf)'$ on the same diagram.
\end{exercise}

\begin{exercise} The general polynomial $P$ of degree $n$ in the variable $x$
has the form $P(x)= \sum _{k=0 } ^n a_k x^k = a_0 + a_1 x + \ldots
+ a_n x^n$. What is the derivative (with respect to $x$)
of $P$?
\begin{answer} $\sum_{k=1}^n ka_kx^{k-1}$
\end{answer}\end{exercise}

\begin{exercise} Find a cubic polynomial whose graph has horizontal tangents at
$(-2 , 5)$ and $(2, 3)$.
\begin{answer} $x^3/16-3x/4+4$
\end{answer}\end{exercise}
 
\begin{exercise} Prove that $\ddx(cf(x))= cf'(x)$ using the
definition of the derivative.
\end{exercise}

\begin{exercise} Suppose that $f$ and $g$ are differentiable at $x$. Show
that $f-g$ is differentiable at $x$ using the two linearity
properties from this section.
\end{exercise}

\end{exercises}



















\section{The Product Rule and Quotient Rule}

\section{The Derivative of Trigonometric Functions}

\section{The Chain Rule}


\section{Rates of Change}

\section{Implicit Differentiation}

\section{Applications}

\subsection{Related Rates}
