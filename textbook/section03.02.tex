\section{Linearity of the Derivative}{}{}
\index{linearity of the derivative}

An operation is linear if it behaves ``nicely'' with respect to
multiplication by a constant and addition. The name comes from the
equation of a line through the origin, $f(x)=mx$, and the following two
properties of this equation. First, $f(cx)=m(cx)=c(mx)=cf(x)$, so the
constant $c$ can be ``moved outside'' or ``moved through'' the
function $f$. Second, $f(x+y)=m(x+y)=mx+my= f(x)+f(y)$, so the
addition symbol likewise can be moved through the function.

The corresponding properties for the derivative are:

$$(cf(x))' = {d\over dx}cf(x) = c {d\over dx} f(x) = cf'(x),$$
and
$$(f(x)+g(x))' = {d\over dx}(f(x)+g(x)) = {d\over dx} f(x)+{d\over dx} g(x)
=f'(x)+g'(x).$$

It is easy to see, or at least to believe,  that these are true by
thinking of the distance/speed interpretation of derivatives. If one
object is at position $f(t)$ at time $t$, we know its speed is given
by $f'(t)$. Suppose another object is at position $5f(t)$ at time $t$,
namely, that it is always 5 times as far along the route as the first
object. Then it ``must'' be going 5 times as fast at all times.

The second rule is somewhat more complicated, but here is one way to
picture it. Suppose a flat bed railroad car is at position $f(t)$ at
time $t$, so the car is traveling at a speed of $f'(t)$ (to be
specific, let's say that $f(t)$ gives the position on the track of the
rear end of the car). Suppose that an ant is crawling from the back of
the car to the front so that its position {\it on the car\/} is $g(t)$
and its speed {\it relative to the car\/} is $g'(t)$. Then in reality,
at time $t$, the ant is at position $f(t)+g(t)$ along the track, and
its speed is ``obviously'' $f'(t)+g'(t)$.

We don't want to rely on some more-or-less obvious physical
interpretation to determine what is true mathematically, so let's see
how to verify these rules by computation. We'll do one and leave the
other for the exercises.
\begin{align*}
{d\over dx}(f(x)+g(x)) &= \lim_{\Delta x\to 0} {f(x+\Delta
  x)+g(x+\Delta x) - (f(x)+g(x))\over \Delta x}  \\
&= \lim_{\Delta x\to 0} {f(x+\Delta
  x)+g(x+\Delta x) - f(x)-g(x)\over \Delta x}  \\
&= \lim_{\Delta x\to 0} {f(x+\Delta
  x)-f(x) +g(x+\Delta x) -g(x)\over \Delta x}  \\
&= \lim_{\Delta x\to 0} \left({f(x+\Delta
  x)-f(x)\over \Delta x}  +{g(x+\Delta x) -g(x)\over \Delta x}\right)  \\
&= \lim_{\Delta x\to 0} {f(x+\Delta
  x)-f(x)\over \Delta x}  +
\lim_{\Delta x\to 0} {g(x+\Delta x) -g(x)\over \Delta x}  \\
&=f'(x)+g'(x) \\
\end{align*}
This is sometimes called the {\dfont sum rule\index{sum rule}} for derivatives.
\begin{example}
Find the derivative of $\ds f(x)=x^5+5x^2$. We have to invoke linearity
twice here: 
$$f'(x) = {d\over dx}(x^5+5x^2) = {d\over dx}x^5 + {d\over dx}(5x^2) =
5x^4+5{d\over dx}(x^2) = 5x^4+5\cdot 2x^1 = 5x^4+10x.$$
\vskip-10pt\end{example}

Because it is so easy with a little practice, we can usually combine
all uses of linearity into a single step. The following example shows
an acceptably detailed computation.

\begin{example}
Find the derivative of $\ds f(x)=3/x^4-2x^2+6x-7$.
$$f'(x) = {d\over dx}\left( {3\over x^4}-2x^2+6x-7\right)
= {d\over dx}(3x^{-4}-2x^2+6x-7) 
= -12x^{-5}-4x+6.$$
\vskip-10pt\end{example}

\begin{exercises}

Find the derivatives of the functions in 1--6.

\begin{exercise} $\ds 5x^3+12x^2-15$
\begin{answer} $\ds 15x^2+24x$
\end{answer}\end{exercise}

\begin{exercise} $\ds -4x^5 + 3x^2 - 5/x^2$
\begin{answer} $\ds -20x^4+6x+10/x^3$
\end{answer}\end{exercise}

\begin{exercise} $\ds 5(-3x^2 + 5x + 1)$
\begin{answer} $\ds -30x+25$
\end{answer}\end{exercise}

\begin{exercise} $f(x)+g(x)$, where $\ds f(x)=x^2-3x+2$ and $\ds g(x)=2x^3-5x$
\begin{answer} $\ds 6x^2+2x-8$
\end{answer}\end{exercise}

\begin{exercise} $\ds (x+1)(x^2+2x-3)$
\begin{answer} $\ds 3x^2+6x-1$
\end{answer}\end{exercise}

\begin{exercise} $\ds \sqrt{625-x^2}+3x^3+12$ (See section \xrefn{sec:slope of a function}.)
\begin{answer} $\ds 9x^2-x/\sqrt{625-x^2}$
\end{answer}\end{exercise}

\begin{exercise}
 Find an equation for the tangent line to $\ds f(x) = x^3/4 - 1/x$ at $x=-2$.
\begin{answer} $y=13x/4+5$
\end{answer}\end{exercise}

\begin{exercise} Find an equation for 
the tangent line to $\ds f(x)= 3x^2 - \pi ^3$ at $x= 4$.
\begin{answer} $\ds y=24x-48-\pi^3$
\end{answer}\end{exercise}

\begin{exercise} Suppose the position of an object at time $t$ is  given by
$\ds f(t)=-49 t^2/10+5t+10$. Find a function giving the speed of the object
at time $t$. The acceleration of an object is the rate at which its
speed is changing, which means it is given by the derivative of the
speed function. Find the acceleration of the object at time $t$.
\begin{answer} $-49t/5+5$, $-49/5$
\end{answer}\end{exercise}

\begin{exercise} Let $\ds f(x) =x^3$ and $c= 3$. Sketch the graphs of $f$,
$cf$, $f'$, and $(cf)'$ on the same diagram.
\end{exercise}

\begin{exercise} The general polynomial $P$ of degree $n$ in the variable $x$
has the form $\ds P(x)= \sum _{k=0 } ^n a_k x^k = a_0 + a_1 x + \ldots
+ a_n x^n$. What is the derivative (with respect to $x$)
of $P$?
\begin{answer} $\ds\sum_{k=1}^n ka_kx^{k-1}$
\end{answer}\end{exercise}

\begin{exercise} Find a cubic polynomial whose graph has horizontal tangents at
$(-2 , 5)$ and $(2, 3)$.
\begin{answer} $\ds x^3/16-3x/4+4$
\end{answer}\end{exercise}
 
\begin{exercise} Prove that $\ds{d\over dx}(cf(x))= cf'(x)$ using the
definition of the derivative.
\end{exercise}

\begin{exercise} Suppose that $f$ and $g$ are differentiable at $x$. Show
that $f-g$ is differentiable at $x$ using the two linearity
properties from this section.
\end{exercise}

\end{exercises}

