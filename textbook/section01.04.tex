\section{Shifts and Dilations}{}{}

Many functions in applications are built up from simple functions by
inserting constants in various places.  It is important to understand
the effect such constants have on the appearance of the graph.

\ssk\noindent
{\bf Horizontal shifts.} {\it If we replace $x$ by $x-C$ everywhere it
occurs in the formula for $f(x)$, then the graph shifts over $C$ to the
right.} (If $C$ is negative, then this means that the graph shifts over
$|C|$ to the left.)  For example, the graph of $y=(x-2)^2$ is the
$x^2$-parabola shifted over to have its vertex at the point 2 on the
$x$-axis.  The graph of $y=(x+1)^2$ is the same parabola shifted over to
the left so as to have its vertex at $-1$ on the $x$-axis. Note well:
when replacing $x$ by $x-C$ we must pay attention to meaning, not
merely appearance. Starting with $y=x^2$ and literally replacing $x$
by $x-2$ gives $y=x-2^2$. This is $y=x-4$, a line with slope 1, not a
shifted parabola.

\ssk\noindent
{\bf Vertical shifts.} {\it If we replace $y$ by $y-D$, then the graph
moves up $D$ units.}  (If $D$ is negative, then this means that the graph
moves down $|D|$ units.)  If the formula is written in the form
$y=f(x)$ and if $y$ is replaced by $y-D$ to get $y-D=f(x)$, we can
equivalently move $D$ to the other side of the equation and write
$y=f(x)+D$.  Thus, this principle can be stated: {\it to get the
graph of $y=f(x)+D$, take the graph of $y=f(x)$ and move it $D$ units up.}
For example, the function $y=x^2-4x=(x-2)^2-4$ can be obtained from
$y=(x-2)^2$ (see the last paragraph) by moving the graph 4 units down.
The result is the $x^2$-parabola shifted 2 units to the right and 4 units
down so as to have its vertex at the point $(2,-4)$.

\ssk\noindent
{\bf Warning.}  Do not confuse $f(x)+D$ and $f(x+D)$.  For example,
if $f(x)$ is the function $x^2$, then $f(x)+2$ is the function $x^2+2$,
while $f(x+2)$ is the function $(x+2)^2=x^2+4x+4$.

\begin{example} (Circles) An important example of the above two principles
starts with the circle $x^2+y^2=r^2$.  This is the circle of radius
$r$ centered at the origin.  (As we saw, this is not a single function
$y=f(x)$, but rather two functions $y=\pm\sqrt{r^2-x^2}$ put together;
in any case, the two shifting principles apply to equations like this
one that are not in the form $y=f(x)$.)  If we replace $x$
by $x-C$ and replace $y$ by $y-D$---getting the equation
$(x-C)^2+(y-D)^2=r^2$---the effect on the circle is to move it $C$ to
the right and $D$ up, thereby obtaining the circle of radius $r$
centered at the point $(C,D)$.  This tells us how to write the
equation of any circle, not necessarily centered at the origin.
\end{example}

We will later want to use two more principles concerning the effects of
constants on the appearance of the graph of a function.

\ssk\noindent {\bf Horizontal dilation.}  {\it If $x$ is replaced by
$x/A$ in a formula and $A>1$, then the effect on the graph is to
expand it by a factor of $A$ in the $x$-direction (away from the
$y$-axis).} If $A$ is between 0 and
1 then the effect on the graph is to contract by a factor of $1/A$
(towards the $y$-axis). 
We use the word ``dilate'' to mean expand or contract.

For example, replacing $x$ by
$x/0.5=x/(1/2)=2x$ has the effect of contracting toward the $y$-axis by a factor
of 2.  If $A$ is negative, we dilate by a factor of $|A|$ and then
flip about the $y$-axis.  Thus, replacing $x$ by $-x$ has the effect of
taking the mirror image of the graph with respect to the $y$-axis.  For
example, the function $y=\sqrt{-x}$, which has domain 
$\{x\in\R\mid x\le 0\}$, is obtained
by taking the graph of $\sqrt{x}$ and flipping it around the $y$-axis into
the second quadrant.

\smallskip
\noindent
{\bf Vertical dilation.}  {\it If $y$ is replaced by $y/B$ in a formula and
$B>0$, then the effect on the graph is to dilate it by a factor of $B$ in
the vertical direction.} As before, this is an expansion or
contraction depending on whether $B$ is larger or smaller than one.
Note that if we have a function $y=f(x)$,
replacing $y$ by $y/B$ is equivalent to multiplying the function on the
right by $B$: $y=Bf(x)$.  The effect on the graph is to expand the picture
away from the $x$-axis by a factor of $B$ if $B>1$, to contract it toward
the $x$-axis by a factor of $1/B$ if $0<B<1$, and to dilate by $|B|$ and
then flip about the $x$-axis if $B$ is negative.

\begin{example} (Ellipses)
A basic example of the two expansion principles is given by an {\dfont ellipse
of semimajor axis $a$ and semiminor axis $b$}.  We get such an ellipse by
starting with the unit circle---the circle of radius 1 centered at the
origin, the equation of which is $x^2+y^2=1$---and dilating  by a factor
of $a$ horizontally and by a factor of $b$ vertically.  To get the equation
of the resulting
ellipse, which crosses the $x$-axis at $\pm a$ and crosses the $y$-axis
at $\pm b$, we replace $x$ by $x/a$ and $y$ by $y/b$ in the equation
for the unit circle.  This gives 
$$
\left({x\over a}\right)^2+\left({y\over b}\right)^2=1
\qquad\hbox{or}\qquad {x^2\over a^2}+{y^2\over b^2}=1.
$$
\end{example}

Finally, if we want to analyze a function that involves both
shifts and dilations, it is usually simplest to work with the
dilations first, and then the shifts.  For instance, if we want to
dilate a function by a factor of $A$ in the $x$-direction and then
shift $C$ to the right, we do this by replacing $x$ first by $x/A$
and then by $(x-C)$ in the formula.  As an example, suppose that,
after dilating our unit circle by $a$ in the $x$-direction and by $b$
in the $y$-direction to get the ellipse in the last paragraph, we then
wanted to shift it a distance $h$ to the right and a distance $k$
upward, so as to be centered at the point $(h,k)$.  The new ellipse
would have equation
$$
\left({x-h\over a}\right)^2+\left({y-k\over b}\right)^2=1.
$$
Note well that this is different than first doing shifts by $h$ and $k$ and
then dilations by $a$ and $b$:
$$
\left({x\over a}-h\right)^2+\left({y\over b}-k\right)^2=1.
$$
See figure~\xrefn{fig:ellipses}.

% BADBAD
% \figure
% \vbox{\beginpicture
% \normalgraphs
% \ninepoint
% \setcoordinatesystem units <1truecm,1truecm> point at 3.5 0
% \setplotarea x from -1.5 to 3.5, y from -2.5 to 4.5
% \axis bottom shiftedto y=0 ticks numbered from 1 to 3 by 1 /
% \axis bottom shiftedto y=0 ticks numbered from -1 to -1 by 1 /
% \axis left shiftedto x=0 ticks numbered from 1 to 4 by 1 /
% \axis left shiftedto x=0 ticks numbered from -2 to -1 by 1 /
% \setquadratic
% \ellipticalarc axes ratio 4:6 360 degrees from 3 1 center at 1 1
% \setcoordinatesystem units <1truecm,1truecm> point at -3.5 2
% \setplotarea x from 0 to 4.5, y from 0 to 6.5
% \axis bottom shiftedto y=0 ticks numbered from 0 to 4 by 1 /
% \axis left shiftedto x=0 ticks numbered from 0 to 6 by 1 /
% \setquadratic
% \ellipticalarc axes ratio 4:6 360 degrees from 4 3 center at 2 3
% \endpicture}
% \figrdef{fig:ellipses}
% \endfigure{Ellipses: $\left({x-1\over 2}\right)^2+\left({y-1\over 3}\right)^2=1$ on the left, $\left({x\over 2}-1\right)^2+\left({y\over 3}-1\right)^2=1$ on the right.}

\begin{exercises}

Starting with the graph of $\ds y=\sqrt{x}$, the graph of $\ds y=1/x$, and the
graph of $\ds y=\sqrt{1-x^2}$ (the upper unit semicircle), sketch the
graph of each of the following functions:

\twocol
\begin{exercise} $\ds f(x)=\sqrt{x-2}$ \end{exercise}
\begin{exercise} $\ds f(x)=-1-1/(x+2)$ \end{exercise}
\begin{exercise} $\ds f(x)=4+\sqrt{x+2}$ \end{exercise}
\begin{exercise} $\ds y=f(x)=x/(1-x)$ \end{exercise}
\begin{exercise} $\ds y=f(x)=-\sqrt{-x}$ \end{exercise}
\begin{exercise} $\ds f(x)=2+\sqrt{1-(x-1)^2}$ \end{exercise}
\begin{exercise} $\ds f(x)=-4+\sqrt{-(x-2)}$ \end{exercise}
\begin{exercise} $\ds f(x)=2\sqrt{1-(x/3)^2}$ \end{exercise}
\begin{exercise} $\ds f(x)=1/(x+1)$\end{exercise}
\begin{exercise} $\ds f(x)=4+2\sqrt{1-(x-5)^2/9}$\end{exercise}
\begin{exercise} $\ds f(x)=1+1/(x-1)$\end{exercise}
\begin{exercise} $\ds f(x)=\sqrt{100-25(x-1)^2}+2$\end{exercise}
\endtwocol

\msk
\noindent
The graph of $f(x)$ is shown below.
Sketch the graphs of the following functions.

% BADBAD
% \begin{exercise} $\ds y=f(x-1)$
% \vadjust{\rightline{%
% \vbox to 0pt{\vskip-5pt\beginpicture
% \normalgraphs
% \ninepoint
% \setcoordinatesystem units <1truecm,1truecm>
% \setplotarea x from 0 to 3.25, y from -1.5 to 2
% \axis bottom shiftedto y=0 ticks numbered from 1 to 3 by 1 /
% \axis left ticks numbered from -1 to 2 by 1 /
% \setquadratic
% \plot 
% 0.000 -1.285 0.054 -0.643 0.108 -0.088 0.162 0.387 0.217 0.788 
% 0.271 1.120 0.325 1.391 0.379 1.604 0.433 1.766 0.488 1.882 
% 0.542 1.956 0.596 1.993 0.650 1.998 0.704 1.976 0.758 1.929 
% 0.812 1.862 0.867 1.779 0.921 1.684 0.975 1.578 1.029 1.467 
% 1.083 1.351 1.138 1.235 1.192 1.120 1.246 1.008 1.300 0.903 
% 1.354 0.805 1.408 0.716 1.462 0.637 1.517 0.571 1.571 0.517 
% 1.625 0.476 1.679 0.450 1.733 0.438 1.788 0.441 1.842 0.458 
% 1.896 0.490 1.950 0.536 2.004 0.595 2.058 0.666 2.112 0.749 
% 2.167 0.841 2.221 0.943 2.275 1.051 2.329 1.164 2.383 1.279 
% 2.438 1.396 2.492 1.510 2.546 1.620 2.600 1.722 2.654 1.813 
% 2.708 1.890 2.762 1.949 2.817 1.987 2.871 2.000 2.925 1.983 
% 2.979 1.932 3.033 1.842 3.088 1.710 3.142 1.528 3.196 1.294 
% 3.250 1.000 /
% \endpicture\vskip0pt\vss}\hskip3cm}}

\begin{exercise} $\ds y=1+f(x+2)$\end{exercise}
\begin{exercise} $\ds y=1+2f(x)$\end{exercise}
\begin{exercise} $\ds y=2f(3x)$\end{exercise}
\begin{exercise} $\ds y=2f(3(x-2))+1$\end{exercise}
\begin{exercise} $\ds y=(1/2)f(3x-3)$\end{exercise}
\begin{exercise} $\ds y=f(1+x/3)+2$\end{exercise}

\end{exercises}
