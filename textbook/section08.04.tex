\section{Integration by Parts}{}{}
\nobreak
We have already seen that recognizing the product rule can be useful,
when we noticed that
$$\int \sec^3u+\sec u \tan^2u\,du=\sec u \tan u.$$
As with substitution, we do not have to rely on insight or cleverness
to discover such antiderivatives; there is a technique that will often
help to uncover the product rule.

Start with the product rule:
$${d\over dx}f(x)g(x)=f'(x)g(x)+f(x)g'(x).$$
We can rewrite this as
$$f(x)g(x)=\int f'(x)g(x)\,dx +\int f(x)g'(x)\,dx,$$
and then
$$\int f(x)g'(x)\,dx=f(x)g(x)-\int f'(x)g(x)\,dx.$$
This may not seem particularly useful at first glance, but it turns
out that in many cases we have an integral of the form
$$\int f(x)g'(x)\,dx$$
but that 
$$\int f'(x)g(x)\,dx$$
is easier. This technique for turning one integral into another is
called {\dfont integration by parts\index{integration!by parts}\/},
and is usually written in more compact form. If we let $u=f(x)$ and
$v=g(x)$ then $du=f'(x)\,dx$ and $dv=g'(x)\,dx$ and 
$$\int u\,dv = uv-\int v\,du.$$
To use this technique we need to identify likely candidates for
$u=f(x)$ and $dv=g'(x)\,dx$.

\begin{example}
Evaluate $\ds\int x\ln x\,dx$. Let $u=\ln x$ so $du=1/x\,dx$. Then
we must let $dv=x\,dx$ so $\ds v=x^2/2$ and
$$
 \int x\ln x\,dx={x^2\ln x\over 2}-\int {x^2\over2}{1\over x}\,dx=
 {x^2\ln x\over 2}-\int {x\over2}\,dx={x^2\ln x\over 2}-{x^2\over4}+C.
$$
\vskip-10pt
\end{example}

\begin{example}
Evaluate $\ds\int x\sin x\,dx$. Let $u=x$ so $du=dx$. Then
we must let $dv=\sin x\,dx$ so $v=-\cos x$ and
$$\int x\sin x\,dx=-x\cos x-\int -\cos x\,dx=
-x\cos x+\int \cos x\,dx=-x\cos x+\sin x+C.$$
\vskip-10pt
\end{example}

\begin{example}
Evaluate $\ds\int\sec^3 x\,dx$. Of course we already know the answer
to this, but we needed to be clever to discover it. Here we'll use the
new technique to discover the antiderivative.
Let $u=\sec x$ and $\ds dv=\sec^2 x\,dx$. Then $du=\sec x\tan x$ and
$v=\tan x$ and
$$\eqalign{
  \int\sec^3 x\,dx&=\sec x\tan x-\int \tan^2x\sec x\,dx \\
  &=\sec x\tan x-\int (\sec^2x-1)\sec x\,dx \\
  &=\sec x\tan x-\int \sec^3x\,dx +\int\sec x\,dx. \\
}$$
At first this looks useless---we're right back to
$\ds \int\sec^3x\,dx$. But looking more closely:
$$\eqalign{
  \int\sec^3x\,dx&=\sec x\tan x-\int \sec^3x\,dx +\int\sec x\,dx \\
  \int\sec^3x\,dx+\int \sec^3x\,dx&=\sec x\tan x +\int\sec x\,dx \\
  2\int\sec^3x\,dx&=\sec x\tan x +\int\sec x\,dx \\
  \int\sec^3x\,dx&={\sec x\tan x\over2} +{1\over2}\int\sec x\,dx \\
  &={\sec x\tan x\over2} +{\ln|\sec x+\tan x|\over2}+C. \\
}$$
\vskip-10pt
\end{example}

\begin{example}
Evaluate $\ds\int x^2\sin x\,dx$. 
Let $u=x^2$, $dv=\sin x\,dx$; then $du=2x\,dx$ and $v=-\cos x$. 
Now $\ds \int x^2\sin x\,dx=-x^2\cos x+\int 2x\cos x\,dx$. This is
better than the original integral, but we need to do integration by
parts again. Let $u=2x$, $dv=\cos x\,dx$; then
$du=2$ and $v=\sin x$, and
$$\eqalign{
  \int x^2\sin x\,dx&=-x^2\cos x+\int 2x\cos x\,dx \\
  &=-x^2\cos x+ 2x\sin x - \int 2\sin x\,dx \\
  &=-x^2\cos x+ 2x\sin x + 2\cos x + C. \\
}$$
\vskip-10pt
\end{example}

Such repeated use of integration by parts is fairly common, but it can
be a bit tedious to accomplish, and it is easy to make
errors, especially sign errors involving the subtraction in the
formula. There is a nice tabular method to accomplish the calculation
that minimizes the chance for error and speeds up the whole
process. We illustrate with the previous example. Here is the
table:

\stablestyle3
\null
\hbox to\hsize{\hfill\begintable
sign\vt $u$\vt $dv$\eltt
\vt $x^2$\vt $\sin x$\elt
$-$\vt $2x$\vt $-\cos x$\elt
\vt $2$\vt $-\sin x$\elt
$-$\vt $0$\vt $\cos x$\endtable
\hfill \raise 40pt\hbox{or}%
\hfill\begintable
$u$\vt $dv$\eltt
$x^2$\vt $\sin x$\elt
$-2x$\vt $-\cos x$\elt
$2$\vt $-\sin x$\elt
$0$\vt $\cos x$\endtable
\hfill}

To form the first table, we start with $u$ at the top of the second
column and repeatedly compute the derivative; starting with $dv$ at
the top of the third column, we repeatedly compute the
antiderivative. In the first column, we place a ``$-$'' in every
second row. To form the 
second table we combine the first and second columns by
ignoring the boundary; if you do this by hand, you may simply start
with two columns and add a ``$-$'' to every second row.

To compute with this second table we begin at the top. Multiply the
first entry in column $u$ by the second entry in column $dv$ to get
$\ds -x^2\cos x$, and add this to the integral of the product of the
second entry in column $u$ and second entry in column $dv$.  This
gives:
$$-x^2\cos x+\int 2x\cos x\,dx,$$
or exactly the result of the first application of integration by
parts.  Since this integral is not yet easy, we return to the table.
Now we multiply twice on the diagonal, $\ds (x^2)(-\cos x)$ and
$(-2x)(-\sin x)$ and then once straight across, $(2)(-\sin x)$, and
combine these as
$$-x^2\cos x+2x\sin x-\int 2\sin x\,dx,$$
giving the same result as the second application of integration by
parts. While this integral is easy, we may return yet once more to the
table. Now multiply three times on the diagonal to get $\ds
(x^2)(-\cos x)$, $(-2x)(-\sin x)$, and $(2)(\cos x)$, and once
straight across, $(0)(\cos x)$. We combine these as before to get
$$
  -x^2\cos x+2x\sin x +2\cos x+\int 0\,dx=
  -x^2\cos x+2x\sin x +2\cos x+C.
$$
Typically we would fill in the table one line at a time, until the
``straight across'' multiplication gives an easy integral. If we can
see that the $u$ column will eventually become zero, we can instead
fill in the whole table; computing the products as indicated will then
give the entire integral, including the ``$+C\,$'', as above.

\begin{exercises}

Find the antiderivatives.

\twocol

\exercise $\ds\int x\cos x\,dx$
\begin{answer} $\cos x+x\sin x+C$
\end{answer}

\exercise $\ds\int x^2\cos x\,dx$
\begin{answer} $\ds x^2\sin x-2 \sin x+2x\cos x +C$
\end{answer}

\exercise $\ds\int xe^x\,dx$
\begin{answer} $\ds (x-1)e^x +C$
\end{answer}

\exercise $\ds\int xe^{x^2}\,dx$
\begin{answer} $\ds (1/2)e^{x^2} +C$
\end{answer}

\exercise $\ds\int \sin^2 x\,dx$
\begin{answer} $(x/2)-\sin(2x)/4 +C=$\hfill\break$(x/2)-(\sin x\cos x)/2+C$
\end{answer}

\exercise $\ds\int \ln x\,dx$
\begin{answer} $x\ln x-x +C$
\end{answer}

\exercise $\ds\int x\arctan x\,dx$
\begin{answer} $\ds (x^2\arctan x +\arctan x -x)/2+C$
\end{answer}

\exercise $\ds\int x^3\sin x\,dx$
\begin{answer} $\ds -x^3\cos x+3x^2\sin x+6x\cos x-6\sin x+C$
\end{answer}

\exercise $\ds\int x^3\cos x\,dx$
\begin{answer} $\ds x^3\sin x+3x^2\cos x-6x\sin x-6\cos x+C$
\end{answer}

\exercise $\ds\int x\sin^2 x\,dx$
\begin{answer} $\ds x^2/4-(\cos^2 x)/4-(x\sin x\cos x)/2+C$
\end{answer}

\exercise $\ds\int x\sin x\cos x\,dx$
\begin{answer} $\ds x/4-(x\cos^2 x)/2+(\cos x\sin x)/4+C$
\end{answer}

\exercise $\ds\int \arctan(\sqrt x)\,dx$
\begin{answer} $x\arctan(\sqrt x)+\arctan(\sqrt x)-\sqrt{x}+C$
\end{answer}

\exercise $\ds\int \sin(\sqrt x)\,dx$
\begin{answer} $2\sin(\sqrt x)-2\sqrt x\cos(\sqrt x)+C$
\end{answer}

\exercise $\ds\int\sec^2 x\csc^2 x\,dx$
\begin{answer} $\sec x\csc x-2\cot x+C$
\end{answer}

\endtwocol

\end{exercises}
