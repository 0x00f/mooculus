\section{Triple Integrals}{}{}

It will come as no surprise that we can also do triple
integrals---integrals over a three-dimensional region. The simplest
application allows us to compute volumes in an alternate way.

To approximate a volume in three dimensions, we can divide the
three-dimensional region into small rectangular boxes, each 
$\Delta x\times\Delta y\times\Delta z$ with volume
$\Delta x\Delta y\Delta z$. Then we add them all up and take the
limit, to get an integral:
$$\int_{x_0}^{x_1}\int_{y_0}^{y_1}\int_{z_0}^{z_1} dz\,dy\,dx.$$
Of course, if the limits are constant, we are simply computing the
volume of a rectangular box.

\begin{example} We use an integral to compute the volume of the box
with opposite corners at $(0,0,0)$ and $(1,2,3)$.
$$\int_0^1\int_0^2\int_0^3
dz\,dy\,dx=\int_0^1\int_0^2\left.z\right|_0^3 \,dy\,dx
=\int_0^1\int_0^2 3\,dy\,dx
=\int_0^1 \left.3y\right|_0^2 \,dx
=\int_0^1 6\,dx = 6.
$$
\end{example}

Of course, this is more interesting and useful when the limits are not
constant. 

\begin{example} Find the volume of the tetrahedron with corners at $(0,0,0)$,
$(0,3,0)$, $(2,3,0)$, and $(2,3,5)$.

The whole problem comes down to correctly describing the region by
inequalities:
$0\le x\le 2$, $3x/2\le y\le 3$, $0\le z\le 5x/2$.
The lower $y$ limit comes from the equation of the line
$y=3x/2$ that forms one edge of the tetrahedron in the $x$-$y$ plane;
the upper $z$ limit comes from the equation of the plane $z=5x/2$ that
forms the ``upper'' side of the tetrahedron; see 
figure~\xrefn{fig:tetrahedron}. 
Now the volume is 
$$\eqalign{
\int_0^2\int_{3x/2}^3\int_0^{5x/2}
dz\,dy\,dx
&=\int_0^2\int_{3x/2}^3\left.z\right|_0^{5x/2} \,dy\,dx \\
&=\int_0^2\int_{3x/2}^3 {5x\over2}\,dy\,dx \\
&=\int_0^2 \left.{5x\over2}y\right|_{3x/2}^3 \,dx \\
&=\int_0^2 {15x\over2}-{15x^2\over4}\,dx \\
&=\left. {15x^2\over4}-{15x^3\over12}\right|_0^2 \\
&=15-10=5. \\
}$$
\end{example}

\figure
\vbox{\beginpicture
\normalgraphs
\ninepoint
\setcoordinatesystem units <1.5truecm,1.5truecm>
\setplotarea x from 0 to 2.1, y from -1.1 to 1.1
\put {\hbox{\epsfxsize6cm\epsfbox{tetrahedron.eps}}} at 0 0
\endpicture}
\figrdef{fig:tetrahedron}
\endfigure{A tetrahedron.
(\expandafter\url\expandafter{\liveurl jmol_tetrahedron}%
AP\endurl)}
%(\expandafter\url\expandafter{\sageurl 2277}%
%AP\endurl)}

Pretty much just the way we did for two dimensions we can use triple
integration to compute mass\index{mass}, center of mass\index{center
  of mass}, and various average\index{average} quantities.

\begin{example} Suppose the temperature at a point is given by
$T=xyz$. Find the average temperature in the cube with opposite
corners at $(0,0,0)$ and $(2,2,2)$.

In two dimensions we add up the temperature at ``each'' point and
divide by the area; here we add up the temperatures and divide by the
volume, $8$:
$$\eqalign{
{1\over8}\int_{0}^2\int_{0}^2\int_{0}^2 xyz\,dz\,dy\,dx
&={1\over8}\int_{0}^2\int_{0}^2\left.{xyz^2\over2}\right|_0^2\,dy\,dx
={1\over16}\int_{0}^2\int_{0}^2 xy\,dy\,dx \\
&={1\over4}\int_{0}^2\left.{xy^2\over2}\right|_0^2\,dx
={1\over8}\int_{0}^2 4x\,dx
={1\over2}\left.{x^2\over2}\right|_0^2
=1. \\
}$$
\end{example}

\begin{example} Suppose the density of an object is given by $xz$, and the
object occupies the tetrahedron with corners
$(0,0,0)$, $(0,1,0)$, $(1,1,0)$, and $(0,1,1)$. Find the mass and
center of mass of the object.

As usual, the mass is the integral of density over the region:
$$\eqalign{
M&=\int_{0}^1\int_{x}^1\int_{0}^{y-x} xz\,dz\,dy\,dx
=\int_{0}^1\int_{x}^1 {x(y-x)^2\over2}\,dy\,dx
={1\over2}\int_{0}^1 {x(1-x)^3\over3}\,dx \\
&={1\over6}\int_{0}^1 x-3x^2+3x^3-x^4\,dx
={1\over120}. \\
}$$
We compute moments as before, except now there is a third moment\index{moment}:
$$\eqalign{
M_{xy} &= \int_{0}^1\int_{x}^1\int_{0}^{y-x} xz^2\,dz\,dy\,dx
={1\over360}, \\
M_{xz} &= \int_{0}^1\int_{x}^1\int_{0}^{y-x} xyz\,dz\,dy\,dx
={1\over144}, \\
M_{yz} &= \int_{0}^1\int_{x}^1\int_{0}^{y-x} x^2z\,dz\,dy\,dx
={1\over360}. \\
}$$
Finally, the coordinates of the center of mass are
$\bar x=M_{yz}/M=1/3$, $\bar y=M_{xz}/M=5/6$, and  $\bar
z=M_{xy}/M=1/3$.
\label{example:3d center of mass}
\end{example}

\begin{exercises}

\exercise Evaluate $\ds\int_{0}^{1}\int_{0}^{x}\int_{0}^{x+y}
2x+y-1 \,dz\,dy\,dx$.
\begin{answer} $11/24$
\end{answer}

\exercise Evaluate $\ds\int_{0}^{2}\int_{-1}^{x^2}\int_{1}^{y}
xyz \,dz\,dy\,dx$.
\begin{answer} $623/60$
\end{answer}

\exercise Evaluate $\ds\int_{0}^{1}\int_{0}^{x}\int_{0}^{\ln y}
e^{x+y+z}\,dz\,dy\,dx$.
\begin{answer} $-3e^2/4+2e-3/4$
\end{answer}

\exercise Evaluate
$\ds\int_{0}^{\pi/2}\int_{0}^{\sin\theta}\int_{0}^{r\cos\theta}
r^2\,dz\,dr\,d\theta$.
\begin{answer} $1/20$
\end{answer}

\exercise Evaluate 
$\ds\int_{0}^{\pi}\int_{0}^{\sin\theta}\int_{0}^{r\sin\theta}
r\cos^2\theta\,dz\,dr\,d\theta$.
\begin{answer} $\pi/48$
\end{answer}

\exercise Evaluate $\ds\int_{0}^{1}\int_{0}^{y^2}\int_{0}^{x+y}
x\,dz\,dx\,dy$.
\begin{answer} $11/84$
\end{answer}

\exercise Evaluate $\ds\int_{1}^{2}\int_{y}^{y^2}\int_{0}^{\ln(y+z)}
e^x\,dx\,dz\,dy$.
\begin{answer} $151/60$
\end{answer}

\exercise For each of the integrals in the previous exercises, give a
description of the volume (both algebraic and geometric) that is the
domain of integration.


\exercise Find the mass of a cube with edge length 2 and density equal
to the square of the distance from one corner.
\begin{answer} $32$
\end{answer}

\exercise Find the mass of a cube with edge length 2 and density equal
to the square of the distance from one edge.
\begin{answer} $64/3$
\end{answer}

\exercise An object occupies the volume of the upper hemisphere of 
$x^2+y^2+z^2=4$ and has density $z$ at $(x,y,z)$. Find the center of mass.
\begin{answer} $\bar x=\bar y=0$, $\bar z=16/15$
\end{answer}

\exercise An object occupies the volume of the pyramid with corners at 
$(1,1,0)$, $(1,-1,0)$, $(-1,-1,0)$, $(-1,1,0)$, and $(0,0,2)$ and has
density $x^2+y^2$ at $(x,y,z)$. Find the center of mass.
\begin{answer} $\bar x=\bar y=0$, $\bar z=1/3$
\end{answer}

\exercise Verify the moments $M_{xy}$, $M_{xz}$, and $M_{yz}$
of example~\xrefn{example:3d center of mass} by evaluating the
integrals. 


\exercise Find the region $E$ for which $\ds\tint{E}
  (1-x^2-y^2-z^2) \; dV$ is a maximum.


\end{exercises}

