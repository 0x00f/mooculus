\section{Linear Approximations}{}{}
\nobreak
Newton's method is one example of the usefulness of the tangent line
as an approximation to a curve. Here we explore another such
application.

Recall that the tangent line to $f(x)$ at a point $x=a$ is given by
$L(x) = f'(a) (x-a) + f(a)$.  The tangent line in this context is also
called the {\dfont linear approximation\index{linear approximation}\/}
to $f$ at $a$.

If $f$ is differentiable at $a$ then $L$ is a good approximation of
$f$ so long as $x$ is ``not too far'' from $a$.  Put another way, if
$f$ is differentiable at $a$ then under a microscope $f$ will look
very much like a straight line. Figure~\xrefn{fig:linear
approximation} shows a tangent line to $\ds y=x^2$ at three different
magnifications. 

If we want to approximate $f(b)$,
because computing it exactly is difficult, we can approximate the
value using a linear approximation, provided that we can compute the
tangent line at some $a$ close to $b$.

%% BADBAD
%% \figure
%% \hbox to \hsize{\hss
%% \epsfxsize4.5cm\epsfbox{linear_approx_1.eps}\hfill
%% \epsfxsize4.5cm\epsfbox{linear_approx_2.eps}\hfill
%% \epsfxsize4.5cm\epsfbox{linear_approx_3.eps}\hfill
%% \hss}
%% \figrdef{fig:linear approximation}
%% \endfigure{The linear approximation to $\ds y=x^2$.}

\begin{example} Let $\ds f(x)=\sqrt{x+4}$. Then $\ds f'(x)=1/(2\sqrt{x+4})$.
 The linear approximation to $f$ at $x=5$ is $\ds
L(x)=1/(2\sqrt{5+4})(x-5)+\sqrt{5+4}=(x-5)/6+3$.  As an immediate
application we can approximate square roots of numbers near 9 by hand.
To estimate $\ds\sqrt{10}$, we substitute 6 into the linear
approximation instead of into $f(x)$, so
$\ds \sqrt{6+4}\approx (6-5)/6+3 = 19/6\approx 3.1\overline{6}$.
This rounds to $3.17$ while the square root of 10 is actually
$3.16$ to two decimal places, so this estimate is only accurate to one
decimal place. This is not too surprising, as 10 is really not very
close to 9; on the other hand, for many calculations, $3.2$ would be
accurate enough.
\endexam

With modern calculators and computing software it may not appear
necessary to use linear approximations. But in fact they are quite
useful. In cases requiring an explicit numerical approximation, they
allow us to get a quick rough estimate which can be used as a
``reality check'' on a more complex calculation. In some complex
calculations involving functions, the linear approximation makes an
otherwise intractable calculation possible, without serious loss of
accuracy.

\exam\relax
\label{exam:linear approximation of sine}
Consider the trigonometric function $\sin x$. Its linear approximation
at $x=0$ is simply $L(x)=x$. When $x$ is small this is quite a good
approximation and is used frequently by engineers and scientists
to simplify some calculations.
\endexam

\begin{definition} Let $y=f(x)$ be a differentiable function. We define a new
  independent variable $dx$, and a new dependent variable
  $dy=f'(x)\,dx$. Notice that $dy$ is a function both of $x$ (since
  $f'(x)$ is a function of $x$) and of $dx$.  We say that $dx$ and
  $dy$ are \dfont{differentials\index{differential}}.  
\end{definition}

Let $\Delta x =x-a$ and $\Delta y= f(x)-f(a)$.
If $x$ is near $a$ then $\Delta x$ is small. If we set
$dx=\Delta x$ then 
$$dy = f'(a)\,dx \approx {\Delta y\over\Delta x}\Delta x = \Delta y.$$
Thus, $dy$ can be used to approximate $\Delta y$, the actual change in
the function $f$ between $a$ and $x$. This is exactly the
approximation given by the tangent line:
$$dy = f'(a)(x-a) = f'(a)(x-a)+f(a)-f(a)=L(x)-f(a).$$
While $L(x)$ approximates $f(x)$, $dy$ approximates how $f(x)$ has
changed from $f(a)$.
Figure~\xrefn{fig:differentials} illustrates the relationships.

%% BADBAD
%% \figure
%% \vbox{\beginpicture
%% \normalgraphs
%% \sevenpoint
%% \setcoordinatesystem units <2truecm,2truecm>
%% \setplotarea x from 0 to 4.5, y from 0 to 2.5
%% \axis left /
%% \axis bottom ticks withvalues {$a$} {$x$} / at 1 3 / /
%% \plot 0.500 0.707 0.588 0.766 0.675 0.822 0.762 0.873 0.850 0.922 
%% 0.938 0.968 1.025 1.012 1.112 1.055 1.200 1.095 1.288 1.135 
%% 1.375 1.173 1.462 1.209 1.550 1.245 1.638 1.280 1.725 1.313 
%% 1.812 1.346 1.900 1.378 1.988 1.410 2.075 1.440 2.162 1.471 
%% 2.250 1.500 2.338 1.529 2.425 1.557 2.512 1.585 2.600 1.612 
%% 2.688 1.639 2.775 1.666 2.862 1.692 2.950 1.718 3.038 1.743 
%% 3.125 1.768 3.212 1.792 3.300 1.817 3.388 1.841 3.475 1.864 
%% 3.562 1.887 3.650 1.910 3.738 1.933 3.825 1.956 3.912 1.978 
%% 4.000 2.000 /
%% \setlinear
%% \plot 1 1 3 2 3 1 1 1 /
%% \betweenarrows {$dx=\Delta x$} from 1 0.8 to 3 0.8
%% \betweenarrows {$\Delta y$} from 3.2 1 to 3.2 1.73
%% \betweenarrows {$dy$} from 3.6 1 to 3.6 2
%% \setdashes <2pt>
%% \putrule from 3 2 to 3.6 2
%% \putrule from 3 1 to 3.6 1
%% \putrule from 3 1.73 to 3.2 1.73
%% \endpicture}
%% \figrdef{fig:differentials}
%% \endfigure{Differentials.}

\begin{exercises}

\begin{exercise} Let $\ds f(x) = x^4$. If $a=1$ and $dx= \Delta x =1/2$, 
what are $\Delta y$ and $dy$?
\begin{answer} $\Delta y=65/16$, $dy=2$
\end{answer}\end{exercise}

\begin{exercise} Let $\ds f(x) = \sqrt{x}$. If $a=1$ and $dx= \Delta x
=1/10$, what are $\Delta y$ and $dy$?
\begin{answer} $\ds \Delta y=\sqrt{11/10}-1$, $dy=0.05$
\end{answer}\end{exercise}


\begin{exercise} Let $f(x) = \sin (2x)$. If $a=\pi$ and $dx= \Delta x
=\pi/100$, what are $\Delta y$ and $dy$?
\begin{answer} $\ds \Delta y=\sin(\pi/50)$, $dy=\pi/50$
\end{answer}\end{exercise}

\begin{exercise} Use differentials to estimate the amount of paint needed to
 apply a coat of paint 0.02 cm thick to a sphere with diameter $40$
 meters. (Recall that the volume of a sphere of radius $r$ is $V
 =(4/3)\pi r^3$. Notice that you are given that $dr=0.02$.)
\begin{answer} $dV=8\pi/25$
\end{answer}\end{exercise}

\begin{exercise} Show in detail that the linear approximation of
 $\sin x$ at $x=0$ is $L(x)=x$ and the linear approximation of $\cos x$
 at $x=0$ is $L(x)=1$.

\end{exercises}
