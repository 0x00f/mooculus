\documentclass[12pt]{amsart}
\usepackage{fullpage}
\usepackage{hyperref}
\newcommand{\mooculus}{\textsf{\textbf{mooc}ulus}}
\usepackage{nopageno}

\title{What is Mooculus?}
\author{Jim Fowler \and Bart Snapp}

\begin{document}

\maketitle

On January~7, 2013, the math department launched its first massive
open online course (MOOC); the course is called ``Calculus One'' and
is designed to cover the same content as our local, in-person sections
of Math~1151.  Our online calculus course is available on Coursera at
\begin{center}
\url{https://www.coursera.org/course/calc1}
\end{center}
but since Coursera's platform lacks, for instance, randomly generated
math problems, we built our own MOOC platform here at Ohio State---we
call it \mooculus.  I encourage you to explore the platform we've
built by going to
\begin{center}
\url{https://mooculus.osu.edu/}
\end{center}

\subsection*{How many people are taking the course?}

There are 32,890 students enrolled; a popular concern about MOOCs is
the high attrition rate, but we have had 11,133 students engage with
the course during Week~4, which is quite a healthy showing.  People
are spending upwards of ten hours a week on our content.

\subsection*{How has the course been received?}

Remarkably well; posts on the forum include ``the teaching style is
the best I've seen on Coursera so far'' and ``one of the best
lecturers I've ever seen anywhere, live or online'' and ``this is the
first example I have seen in either Edx or Coursera of using the
medium really well.''  Thousands of people from all over the world are
successfully learning calculus with OSU's help.

\subsection*{How does a student engage with our course?}

For the student, \mooculus\ provides new content each week, in the form of
\begin{itemize}
\item another chapter to our free, open-source calculus textbook;
\item a dozen short lecture videos, between 2 and 15 minutes long; and
\item interactive randomly-generated exercises that emphasize both conceptual and computational learning.
\end{itemize}
Since one learns math by doing math, the interactive exercises are
especially important.  Unlike any other MOOC platform, we use
a Hidden Markov Model to estimate student learning, based on how the
student engages with our website.  Once \mooculus\ believes the
student has mastered a particular problem, the
student is provided with another challenge.

\subsection*{Why build our own platform?}

By building \mooculus, we provide a better student experience than any
other MOOC platform.  For instance, a common complaint with other
systems that don't rely on randomly generated exercises is that a
student eventually ``runs out'' of new exercises to try---\mooculus\
never runs out of problems.

Additionally, now that we've built \mooculus, we can use the same
platform for other courses---even the English MOOC will be built on
our same system.

\end{document}

%%% Local Variables: 
%%% mode: latex
%%% TeX-master: t
%%% End: 
