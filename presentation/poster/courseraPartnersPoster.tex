\documentclass[12pt,serif,mathserif,final]{beamer}
\mode<presentation>{\usetheme{Lankton}}
\usepackage{amsmath,amsfonts,amssymb,pxfonts,eulervm,xspace}
\usepackage{graphicx}
\usepackage{kerkis}
\usepackage[orientation=landscape,size=custom,width=121.92,height=121.92,scale=1.5,debug]{beamerposter}

%-- Header and footer information ----------------------------------
\newcommand{\footleft}{}
\newcommand{\footright}{}
\title{Calculus\&\textsf{MOOCulus}}
\author{Jim Fowler, Steve Gubkin, and Bart Snapp}
\institute{The Ohio State University}
%-------------------------------------------------------------------


%-- Main Document --------------------------------------------------
\begin{document}

\begin{frame}[t]
    \vspace{6cm}
    \huge \textbf{Calculus and Mooculus} \normalsize
    \vspace{5cm}
    %\begin{center}
    %    \includegraphics{figures/banner}
    %\end{center}
  \begin{columns}[t]

    %-- Column 1 ---------------------------------------------------
    \begin{column}{0.32\linewidth}
    
      %-- Block 1-1
      \begin{block}{The No-Three-In-Line Problem}
        Imagine you are given a chess board with a given number of rows
        and columns.  How many chess pieces could you place on the board
        without any three points being on the same line?  Or more generally,
        how many points can be place on a grid with no three points in a line?
        \begin{center}
            %\includegraphics[width=.6\columnwidth]{figures/10x10_wiki_soln.png}

            A solution to the $10\times 10$ case.
        \end{center}
      \end{block}

      %-- Block 1-2
      \begin{block}{Our Approach}
        Using the Hilbert function to count monomials of a given degree,
        we used combinatorial methods and commutative algebra to generate 
        complete lists of solutions to the No-Three-In-Line problem.  We 
        wrote our algorithms in \textit{Macaulay2}, a language designed specifically
        for commutative algebra.
      \end{block}

      %-- Block 1-3
      \begin{block}{Our Big Decision}
        Because of the computational complexity of the problem, we chose not
        to directly pursue the No-Three-In-Line Problem.  Instead, we decided
        to vary the problem slightly, by meeting the edges of the grid into a
        donut shape, or torus.
      \end{block}

    \end{column}%1

    %-- Column 2 ---------------------------------------------------
    \begin{column}{0.32\linewidth}

      %-- Block 2-1
      \begin{block}{Fun with Tori}
        \begin{center}
            %\includegraphics[width=.6\columnwidth]{figures/torusHIRES.pdf}

            An $11\times 11$ torus before points are placed.
        \end{center}
        Despite the fact that the torus is curved, we can still imagine 
        it with grid lines and, therefore, locations to place points.
        In this variant of the problem, we can still place points in the
        same way, but we have to be aware of the fact that the lines now
        wrap around the torus!
      \end{block}

      %-- Block 2-2
      \begin{block}{Lines on a Torus}
        Lines on a torus work just like lines on a grid, they just curve along
        the surface.  Imagine the line bending as your wrapped the edges around
        before.
        \begin{center}
            %\includegraphics[width=.6\columnwidth]{figures/torus_with_lineHIRES.pdf}

            The same torus with a line on it.
        \end{center}
      \end{block}

    \end{column}%2

    %-- Column 3 ---------------------------------------------------
    \begin{column}{0.32\linewidth}

      %-- Block 3-1
      \begin{block}{Our Results}
        We were able to enumerate the unique solutions on the torus for square
        tori up to the $14 \times 14$ case, and for rectangular tori varying from 
        $2 \times 2$ to $7 \times 19$.  We were also able to form a construction 
        for maximal (that is $2n$) for the $p \times p^2$ torus, where $p$ is any
        odd prime.
      \end{block}

      %-- Block 3-2
      \begin{block}{Notable Features of our Research}
        \begin{itemize}
            \item By extending the problem to the torus, we were able to introduce
                group theory to the no-three-in-line problem
            \item By using the Hilbert function, we were able to count the number
                of solutions on both the grid and torus
            \item Our methods also allowed us to view all of the solutions for
                a given problem size
        \end{itemize}
      \end{block}

      \begin{block}{Maximal Solutions}
        In either case, the obvious upper-bound is $2n$.  That is,
        $2$ points in every row (or column).

        The image in the first column is a maximal solution on the grid.

        On the square torus, solutions are much different, and somewhat smaller.
        However, we were able to find completely maximal solutions for the 
        $p \times p^2$ torus, where $p$ is prime.  The following is a maximal 
        solution on the $5\times 25$ torus:
        \begin{center}
            %\includegraphics[width=\columnwidth]{figures/maximal_torusHIRES.pdf}

            A solution to the $5\times 25$ case.
        \end{center}
      \end{block}

    \end{column}%3

  \end{columns}
  \begin{center}
    %\includegraphics{figures/QR_CODE.png}
  \end{center}
\end{frame}
\end{document}
