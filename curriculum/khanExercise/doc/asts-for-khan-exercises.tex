\documentclass{article}
\usepackage{fullpage}
\usepackage{geometry}
\usepackage{fancyhdr}
\usepackage{amsmath, amsthm, amssymb}
\usepackage{graphicx}
\usepackage{hyperref}
\usepackage{parskip}
\newcommand{\var}[1]{\emph{#1}}

\title{Abstract Syntax Trees\\ for Khan Exercises}
\begin{document}
\maketitle
\begin{verbatim}
Authors: Jeff Dyer <jeffdyer@acm.org>
         Mathias Hiron <mathias@khanacademy.org>

History: 04-Jan-2012, second draft, reviewed by Mathias
         03-Jan-2012, first draft, unreviewed
\end{verbatim}

\section{Overview}

In this note we define an abstract syntax tree (AST) form for encoding information used in Khan 
exercises. We also define a mapping from ASTs to LaTeX formatted strings. ASTs and LaTeX have wide 
applicability in the Khan exercise framework, and so by defining and implementing a core library 
for handling them we avoid duplication of effort and gain the benefits normally associated by 
software reuse.

\section{Abstract Syntax Trees}

We use trees to represent expressions, equations, and other structured data that underlies Khan 
exercises. The trees consist of nodes containing an operator string and zero or more child nodes.

The nodes are encoded as JavaScript objects of the form:

\begin{quote}
\verb'{'{\bf op}{\tt :} {\em opstr}{\tt ,} {\bf args}{\tt :} {\tt [}$nd1${\tt ,}\dots $ndN${\tt ]}$\verb'}'$
\end{quote}

where $opstr$ is an \emph{operator string} and $nd1, \dots ndN$ is a sequence of child nodes.

An \emph{operator string} is a string value that, along with the number of arguments, distinguishes 
one node kind from another. AST nodes get their meaning from the parsing, formatting and evaluating 
procedures that construct and interpret them, and otherwise have no special meaning within the AST
definition and implementation.

Primitive (non-object) JavaScript values are used to represent literal numbers and strings 
in the AST. Variables are encoded as nodes with the unary ``var'' operator with a literal string
operand.

We define a core set of operators here, but this set is not fixed. As new exercises require new 
operators, additional strings to be contrived to represent those operators, and interpreters can be 
defined to format and evaluate nodes with those operators.

\section{Predefined Operators}

The following set of operators are implemented by the AST module (see below). Additional operators
may be added to this core set over time, and others may be added by problem specific utility 
modules.

\vspace{20pt}

\begin{tabular}{ | l | l | p{10cm} |}
\hline
{\bf Operator}& {\bf Sample}   & {\bf Comment} \\ \hline
\verb'+'      & $x+y$          & addition, unary plus\\ \hline
\verb'-'      & $x-y$          & subtraction, unary minus\\ \hline
\verb'times'  & $x \times y$   & multiplication\\ \hline
\verb'div'    & $x \div y$     & division\\ \hline
\verb'frac'   & $\frac{x}{y}$  & fraction\\ \hline
\verb'^'      & $x^y$          & exponent\\ \hline
\verb'sqrt'   & $\sqrt{x}$     & unary square root\\ \hline
\verb'sqrt'   & $\sqrt[n]{x}$  & binary square root\\ \hline
\verb'pm'     & $x \pm y$      & plus or minus, unary plus or minus\\ \hline
\verb'sin'    & $\sin{\theta}$ & sine\\ \hline
\verb'cos'    & $\cos{\theta}$ & cosine\\ \hline
\verb'tan'    & $\tan{\theta}$ & tangent\\ \hline
\verb'sec'    & $\sec{\theta}$ & secant\\ \hline
\verb'ln'     & $\ln{x}$       & natural logarithm\\ \hline
\verb'()'     & $(expr)$       & parenthesis\\ \hline
\verb'var'    & $\var{x}$      & variable (unknown)\\ \hline
\end{tabular}


\pagebreak

\section{LaTeX Translation}

In addition to the internal AST form, we define a mapping from ASTs to LaTeX. The following table 
shows the mapping between the AST nodes and LaTeX.

\vspace{20pt}

\begin{tabular}{ | l | l |}
\hline
{\bf AST}                           & {\bf LaTeX} \\ \hline
\verb'{op: "+", args: [1, 2]}'      & \verb'1+2' \\ \hline
\verb'{op: "-", args: [1, 2]}'      & \verb'1-2' \\ \hline
\verb'{op: "times", args: [1, 2]}'  & \verb'1 \times 2' \\ \hline
\verb'{op: "div", args: [1, 2]}'    & \verb'1 \div 2' \\ \hline
\verb'{op: "frac", args: [1, 2]}'   & \verb'\frac{1}{2}' \\ \hline
\verb'{op: "^", args: [1, 2]}'      & \verb'1^2' \\ \hline
\verb'{op: "sqrt", args: [9]}'      & \verb'\sqrt{9}' \\ \hline
\verb'{op: "sqrt", args: [3, 8]}'   & \verb'\sqrt[3]{8}' \\ \hline
\verb'{op: "pm", args: [2, 1]}'     & \verb'2 \pm 1' \\ \hline
\verb'{op: "sin", args: [0]}'       & \verb'\sin{0}' \\ \hline
\verb'{op: "cos", args: [0]}'       & \verb'\cos{0}' \\ \hline
\verb'{op: "tan", args: [0]}'       & \verb'\tan{0}' \\ \hline
\verb'{op: "sec", args: [0]}'       & \verb'\sec{0}' \\ \hline
\verb'{op: "ln", args: [1]}'        & \verb'\ln{1}' \\ \hline
\verb'{op: "()", args: [10]}'       & \verb'(10)' \\ \hline
\verb'{op: "var", args: ["x"]}'     & \verb'\var{x}' \\ \hline
\end{tabular}


\vspace{20pt}

The AST library provides functions for translating between ASTs and LaTeX formatted strings. 
These formatting and parsing algorithms conform to common practice including:

\begin{itemize}
\item Insert \verb'\left' and \verb'\right' to auto size brackets.
\item Rewrite addition of a negative number as subtraction.
\item Elide \verb'\times' when it occurs before a non-numeric symbol during formatting, and construct 
      a \verb'\times' node when a non-numeric symbol occurs after a non-operator symbol.
\item Insert parenthesis when formatting a binary expression of lower precedence that occurs in an 
      expression of higher precedence.
\item (more here)
\end{itemize}

\pagebreak
\section{AST API}

%We provide an API for constructing, formatting, evaluating, and interning ASTs. Some of these
%functions take a optional \emph{model} object which interprets the meaning of operators that are not 
%defined by the AST library, or which override those that are. When the \emph{model} parameter is 
%given, the AST function delegates the call to that object, which may call back to the AST function 
%to handle operators it does not distinguish.

The AST API is implemented as a module that can be loaded into an exercise. All functions are 
instance methods of the {\bf ast} property added to {\bf KhanUtil}, and therefore are called through 
that property (e.g. {\bf KhanUtil.ast.fromLaTeX}(``1+2''), or {\bf ast.fromLaTeX}(``1+2'') when 
called from exercise HTML that loads the AST module).

Below is the list of API methods implemented by {\tt ast.js}.

\begin{tabular}{ | l | p{10cm} |}
\hline
{\bf Name} & {\bf Description} \\ \hline
{\bf fromLaTeX}(\emph{str}) & Return an AST node constructed from a LaTeX formatted string \\ \hline
{\bf toLaTeX}(\emph{node})  & Return the LaTeX formatted string correponding to an AST node \\ \hline
{\bf eval}(\emph{node})     & Return the result of evaluating an AST \\ \hline
{\bf intern}(\emph{node})   & Add an AST node and its child nodes to the node pool and return the root node id (nid) \\ \hline
{\bf node}(\emph{nid})      & Reconstitute an AST from its node id \\ \hline
\end{tabular}


\end{document}
