\documentclass[12pt]{article}

\usepackage[margin=1in]{geometry}
\usepackage{times} %should use 12 pt Times New Roman font.
\usepackage[T1]{fontenc}
\usepackage{mathptmx}
\usepackage{hyperref}

\usepackage[style=authoryear]{biblatex}
\addbibresource{references.bib}

\usepackage{titlesec}
\titleformat*{\section}{\normalsize\bfseries}

\begin{document}

\begin{center}
  \textbf{MOOCulus: Iteratively Improving Calculus Instruction}
\end{center}

% PROPOSAL FORMAT
% 
% 1. Proposal overview
% 2. Literature and existing research that informs their proposal
% 3. Context of research (MOOC taught, designed, university/organizational affiliation, MOOC provider, publisher)
% 4. Research questions to be addressed by the project
% 5. Data sources being considered for the project
% 6. Methodology planned for research activities

\section{Proposal overview}

In January 2013, The Ohio State University Mathematics Department
launched its first massive open online course.  This MOOC was designed
to cover the same content as the local, in-person sections of calculus
at Ohio State.

Part of this MOOC consisted of a home-built platform designed to
deliver randomly-generated interactive mathematics problems to
students.  This home-built adaptive learning platform is usable for
much more than calculus: this same platform has already been re-used
for a English writing course called
WexMOOC \parencite{gates-foundation-grant}.  The American Languages
Program at The Ohio State has also reached out to modify the adaptive
learning platform to teach English grammar for their ESL students.

With tens of thousands of students enrolled in the calculus MOOC,
there have been millions of attempts on homework exercises, with data
on each of these attempts.  Funding is sought to use this data to both
evaluate the success of and improve upon the adaptive learning
platform built at OSU.

\section{Literature}

Hidden Markov models are widely used in educational data mining; one
nice example of such is Shih-Koedinger-Scheines' 2010 paper
presented at the International Conference on Educational Data Mining,
``Discovery of Student Strategies using Hidden Markov Model
Clustering.''  The survey papers of Romero and Ventura provide an
excellent overview.

\section{Context of research}

Coursera was chosen as the MOOC provider, but since Coursera's
platform lacked randomly generated math problems, the team at Ohio
State built their own MOOC platform to complement Coursera's
services---this home-built platform is called MOOCulus, and can be
explored at \url{https://mooculus.osu.edu/}.

MOOCulus logs all student input on homework exercises. This data is
fed into a hidden Markov model to produce an estimate of the student's
understanding. The student receives additional practice problems of
the same sort until it is estimated that their understanding reaches a
threshold, at which point the student is given a new type of exercise
and the process repeats.  The overall goal is to keep providing
exercises that are difficult enough to be fun and educational, but not
so easy as to become boring and repetitive.  In spite of the fact that
many tens of thousands of students used MOOCulus, each experienced a
personalized path through the material.  In short, MOOCulus promises
to enable MOOCs to be massive, but not mass-produced.

In addition to building MOOCulus and running Calculus One, both Fowler
and Snapp have experience using technology and analyzing its
effectiveness.  For example, Fowler built the mobile phone clicker
used to measure student engagement for OSU's technology enhanced
calculus lectures, and Fowler's research training includes
high-dimensional data analysis from a topological perspective; Snapp
taught for Calculus Remote at OSU (CROSU), and worked with OSU's
Center for Enterprise Transformation \& Innovation to build
interactive textbooks for the iPad, with the goal of measuring student
engagement while reading the textbook.

\section{Research questions}

As students interact with MOOCulus, the data needed to refine the
parameters of the hidden Markov model are obtained.  At this point a
number of questions arise. To what degree is a correct answer after
three ``hints'' strong evidence of student understanding?  What prior
probability should be assigned to the likelihood of a student to
submit a correct answer on the first try?  The parameters depend
strongly on the particular homework exercise; some exercises are
conceptually deep but procedurally easy, while some questions are
conceptually light but procedurally quite involved, so even a strong
student may make ``careless'' mistakes which the model shouldn't count
against him/her.  It's easy for an expert to qualitatively make these
judgments, but analysis of big data is required to accurately set
these parameters.

This leads to two basic questions:
\begin{itemize}
\item What parameters in the adaptive
  learning model best align online student performance with in-class
  student performance?
\item To what extent does student use
  of adaptive learning correlate with success in traditional,
  in-person courses?
\end{itemize}
The former question addresses improving the hidden Markov model that
powers MOOCulus; the second question gets deeper, by addressing the
extent to which adaptive learning helps with in-person courses.

\section{Data sources}

A large data set already exists from the first run of the Calculus One
MOOC in Spring 2013; with a total enrollment of 47k~students, the
course still had 11,000 students engaging with the material after four
weeks, and 2,000 students at Week~15.  Altogether, a total of
2,079,428 correct answers were submitted to MOOCulus, with a total of
10.3 person-years having been spent by students working problems.  The
activity logs are stored in a SQL database and this database is
imported for analysis into R.

A second iteration of the Calculus One MOOC has been started at OSU
for Fall 2013.  In addition, Snapp is teaching two on-campus calculus
courses: a course for engineers and a course for teachers.  Fowler and
Snapp have obtained a letter of final determination from the
Institutional Review Board facilitating the use of human subjects in
our research.  The proposed plan is to have each of these groups of
students enroll in MOOCulus so that we can directly compare students'
performance in the online course with their in-class performance.


\section{Methodology}

To set the parameters of the hidden Markov model, the Baum-Welch
algorithm will be used on data gathered from the first run of the
Calculus One MOOC. This will allow the parameters in the MOOCulus
hidden Markov model will be set for each problem separately, for a
more effective implementation in Spring 2013.

To assess the correlation between success in the MOOC and success in
the traditional course, in addition to teaching the online MOOC, Snapp
will be teaching calculus to two, quite different, student populations
in Fall 2013.  The first is a calculus course for middle school
teachers ($n=28$); the second is a calculus course for engineering
students ($n\approx 180$).  By involving his in-person students also
in the online MOOCulus course, the extent to which participation in
the adaptive learning system correlates with success on in-person
assessments can be determined.

\section{Budget needs}

A course buyout provides time for Bart Snapp to perform data analysis,
to write up the results, and to oversee the work of his Master's
student, David Lindberg.  One course buyout for Bart Snapp is \$15k.
To release David Lindberg from his teaching further costs \$1925/month
plus 12.6\% in benefits; for the Spring Semester from January--May,
this costs $\$1925 \cdot 5 \cdot 112.6\% = \$10838$.  Altogether, the
total cost for the two researchers' time is \$25838.

\printbibliography

\end{document}

%%% Local Variables: 
%%% mode: latex
%%% TeX-master: t
%%% End: 
