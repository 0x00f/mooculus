\documentclass[12pt]{article}

\usepackage[margin=1in]{geometry}
\usepackage{times}
\usepackage{hyperref}

\begin{document}

\begin{center}
  \textbf{MOOCulus: Iteratively Improving Calculus Instruction}
\end{center}

\subsubsection*{Proposal overview}

In January 2013, The Ohio State University Mathematics Departement
lauched its first massive open online course. This MOOC was designed
to cover the same content as the local, in-person sections of calculus
at Ohio State. Part of this MOOC consisted of a home-built platform
designed to deliver randomly-generated interactive mathematics
problems to the students. Already having had tens of thousands of
students, we have had millions of attempts on our homework exercises,
with data on each of these attempts.  Funding is sought to use this
data to both evaluate the success of and improve upon our calculus
MOOC and overall teaching at Ohio State.

%Clearly detail research methods, techniques, and tools

\subsubsection*{Context of research}

The Ohio State math department launched its first massive open online
course (MOOC) in January 2013; that course, ``Calculus One,'' was
designed to cover the same content as the local, in-person sections of
calculus at Ohio State.  Coursera was chosen as the MOOC provider, but
since Coursera's platform lacked, for instance, randomly generated
math problems, the course team built their own MOOC platform at Ohio
State to complement Coursera's services---that home-built platform is
called MOOCulus, and can be explored at
\url{https://mooculus.osu.edu/}.

MOOCulus logs student input on all homework exercises, whether the
input is a correct or incorrect response to the randomly generated
question, and how much time it took the student to arrive at that
input.  We also log whenever the student requests a "hint."  This data
is fed into a hidden Markov model to produce an estimate of the
student's understanding; the student receives additional practice
problems of the same sort until our estimate of their understanding
reaches a threshold, at which point they are given a new type of
exercise and the process repeats.  The overall goal is to keep
providing exercises that are difficult enough to be fun and
educational, but not so easy as to become boring and repetitive.

In short, MOOCs are massive, but not mass-produced.  The idea In spite
of the fact that MOOCulus had many tens of thousands of students, they
each experienced a personalized path through the material.

In addition to building MOOCulus and running Calculus One, both Fowler
and Snapp have experience using technology and analyzing its
effectiveness.  For example, Fowler built the mobile phone clicker
used to measure student engagement for OSU's technology enhanced
calculus lectures, and Fowler's research training includes
high-dimensional data analysis from a topological perspective; Snapp
taught for Calculus Remote at OSU (CROSU), and worked with OSU's
Center for Enterprise Transformation \& Innovation to build
interactive textbooks for the iPad, with the goal of measuring student
engagement while reading the textbook.

\subsubsection*{Data sources}

A large data set already exists from the first run of Calculus One in
Spring 2013; with an initial enrollment of 35k, the course still had
11k students engaging with the material after four weeks, and 2k
students at Week~15.  Altogether, a total of 2,079,428 correct answers
were submitted to MOOCulus, with a total of 10.3 person--years having
been spent by students working problems.

In Fall 2013, the Calculus One MOOC will be taught again at OSU.  In
addition Snapp will be teaching two on-campus calculus courses: a
course for engineers and a course for teachers.  Fowler and Snapp have
obtained a letter of final determination from the Institutional Review
Board facilitating the use of human subjects in our research.  The
proposed plan is to have each of these groups of students enroll in
MOOCulus so that we can directly compare students' performance in the
online course with their in-class performance.

The activity logs are stored in a SQL database; this database is
imported for analysis into R.

\subsubsection*{Research question}

As students participate with MOOCulus, we get the data needed to
refine the parameters in the hidden Markov model.  To what degree is a
correct answer after three hints strong evidence of student
understanding?  What prior probability should be assigned to the
likelihood of a student to submit a correct answer on the first try?
The parameters depend strongly on the particular homework exercise:
some exercises are conceptually deep but procedurally easy, while some
questions are conceptually light but procedurally quite involved, so
even a strong student may make ``careless'' mistakes which the model
shouldn't count against him/her.  It's easy for an expert to
qualitatively make these judgments, but big data is needed to make
this quantitative.

There are, then, two basic questions:
\begin{itemize}
\item What parameters in the adaptive
  learning model best align online student performance with in-class
  student performance?
\item To what extent does student use
  of adaptive learning correlate with success in traditional,
  in-person courses?
\end{itemize}
The former question addresses improving the hidden Markov model that
powers MOOCulus; the second question gets deeper, by addressing the
extent to which adaptive learning helps with in-person courses.

\subsubsection*{Literature}

Hidden Markov models are widely used in educational data mining; one
nice example of such is Shih--Koedinger--Scheines' 2010 paper
presented at the International Conference on Educational Data Mining,
``Discovery of Student Strategies using Hidden Markov Model
Clustering.''  The survey papers of Romero and Ventura provide an
excellent overview.

\subsubsection*{Methodology}

First, by using the Baum-Welch algorithm on historic data, the
parameters in the MOOCulus hidden Markov model will be set for each
problem separately.

Then, in addition to teaching the online MOOC, Snapp will be teaching
calculus to two, quite different, student populations in Fall 2013.
The first is a calculus course for middle school teachers ($n=28$);
the second is a calculus course for engineering students ($n\approx
180$).  By involving his in-person students also in the online
MOOCulus course, the extent to which participation in the adaptive
learning system correlates with success on in-person assessments can
be determined.

\end{document}

%%% Local Variables: 
%%% mode: latex
%%% TeX-master: t
%%% End: 
