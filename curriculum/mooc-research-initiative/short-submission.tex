\documentclass[12pt]{article}

\usepackage[margin=1in]{geometry}
\usepackage{times}
\usepackage{hyperref}

\begin{document}

\begin{center}
  \textbf{MOOCulus: Iteratively Improving Calculus Instruction}
\end{center}

\subsubsection*{Proposal overview}

In January 2013, The Ohio State University Mathematics Departement
lauched its first massive open online course. This MOOC was designed
to cover the same content as the local, in-person sections of calculus
at Ohio State. Part of this MOOC consisted of a home-built platform
designed to deliver randomly-generated interactive mathematics
problems to the students. Already having had tens of thousands of
students, we have had millions of attempts on our homework exercises,
with data on each of these attempts.  Funding is sought to use this
data to both evaluate the success of and improve upon our calculus
MOOC and overall teaching at Ohio State.

%Clearly detail research methods, techniques, and tools

\subsubsection*{Context of research}

The Ohio State math department launched its first massive open online
course (MOOC) in January 2013; that course, ``Calculus One,'' was
designed to cover the same content as the local, in-person sections of
calculus at Ohio State.  Coursera was chosen as the MOOC provider, but
since Coursera's platform lacked, for instance, randomly generated
math problems, the course team built their own MOOC platform at Ohio
State to complement Coursera's services---that home-built platform is
called MOOCulus, and can be explored at
\url{https://mooculus.osu.edu/}.

In addition to building MOOCulus and running Calculus One, both Fowler
and Snapp have experience using technology and analyzing its
effectiveness.  For example, Fowler built the mobile phone clicker
used to measure student engagement for OSU's technology enhanced
calculus lectures, and Fowler's research training includes
high-dimensional data analysis from a topological perspective; Snapp
taught for Calculus Remote at OSU (CROSU), and worked with OSU's
Center for Enterprise Transformation \& Innovation to build
interactive textbooks for the iPad, with the goal of measuring student
engagement while reading the textbook.

\subsubsection*{Data sources}

What data sources will be used for the project

Describe how data will be acquired for analysis, including sources,
ethics clearance, tools used for data cleaning, and evaluation
methods

Data from Spring 2013: 10.3 person--years were spent doing exercises
on MOOCulus, with a total of 2,079,428 correct answers submitted.

With an initial enrollment of 35k, the course still had 11k students
engaging with the material after four weeks, and 2k students at
Week~15.

Data from Fall 2013: Calculus One will be running again, so we have an
opportunity to gather even more data.

\subsubsection*{Research questions}

MOOCulus stores student input on all homework exercises, whether the
input is a correct or incorrect response to the randomly generated
question, and how much time it took the student to arrive at that
input.  We also log whenever the student requests a "hint."  This data
is fed into a hidden Markov model to produce an estimate of the
student's understanding; the student receives additional practice
problems of the same sort until our estimate of their understanding
reaches a threshold, at which point they are given a new type of
exercise and the process repeats.  You can see this in action at
http://mooculus.osu.edu/ The overall goal is to keep providing
exercises that are difficult enough to be fun and educational, but not
so easy as to become boring and repetitive.

But there's more.

As students participate, we get the data needed to refine the
parameters in the hidden Markov model; we have grad students working
on this analysis now.  To what degree is a correct answer after three
hints strong evidence of student understanding?  How likely is a
strong student to submit a correct answer on the first try?  The
parameters depend strongly on the particular homework exercise: some
exercises are conceptually deep but procedurally easy, while some
questions are conceptually light but procedurally quite involved, so
even a strong student may make "careless" mistakes which the model
shouldn't count against him/her.  It's easy for an expert to
qualitatively make these judgments, but big data is needed to make
this quantitative.

I like to say "MOOCs are massive, but not mass-produced."  The idea
that students will pay tuition for a human instructor to merely assign
and grade the exact same ten problems from the textbook will be seen
as charmingly anachronistic---a sort of throwback to the 20th
century's mass-produced education.  In spite of the fact that MOOCulus
had many tens of thousands of students, they each experienced a
personalized path through the material.

With tens of thousands of students, we have had millions of attempts
on our homework exercises, so I've already looked at correlations
between student success on earlier problems and later problems.  This
fall when we run the course again, we will emphasize those problems as
effective preparation for what is to come.  When a student earns a "A"
in Calculus One, what we are trying to say is that there is evidence
that this student will succeed in Calculus Two, but the way in which
faculty write final exams isn't yet nearly as data-driven as what
MOOCs provide.

to be addressed by the project

How can MOOCs improve other types of instruction, including blended
learning, face to face and online instructional experiences for
academically weak students?

How can MOOCs help students succeed in remedial and introductory
coursework?  What are the challenges, issues, and barriers to engaging
less academically prepared and less self-motivated students in MOOC
coursework? What are promising approaches?  Can MOOCs effectively
customize/personalize the learning experience for students?  For which
students, disciplines, types of learning and contexts are MOOCs
more/less effective?  What perceptions about MOOCs exist among
students, faculty, administrators, and state/regulatory leaders? How
do these perceptions influence current and future opportunities for
MOOCs to serve the higher education community and, in particular,
low-income and disadvantaged young adult populations?

Is there a realistic business model for MOOCs focused on remedial and introductory coursework?

\subsubsection*{Methodology}

planned for research activities

affective surveys

A/B user testing

\subsubsection*{Literature}

Hidden Markov models are widely used in educational data mining; one
nice example of such is Shih--Koedinger--Scheines' 2010 paper
presented at the International Conference on Educational Data Mining,
``Discovery of Student Strategies using Hidden Markov Model
Clustering.''  The survey papers of Romero and Ventura provide an
excellent overview.

\end{document}

%%% Local Variables: 
%%% mode: latex
%%% TeX-master: t
%%% End: 
